\begin{table*}[]
\centering
	\begin{tabular}{@{}lccc c@{}} \hline
		\textbf{Dataset} & \textbf{Instances} & \textbf{Features} & \textbf{Anomalies}  & \textbf{Contamination}\\ \hline
		AnnThyroid   & 7200                & 6                  & 534        & 7.42\%        \\ 
		Breastw      & 683                 & 9                  & 239        & 35 \%        \\
		Cardio &	1831 & 21 & 176 & 9.6\% \\
		Cover & 286048 & 10 & 2747 & 0.9\% \\
		Ionosphere   & 351                 & 33                 & 126          &      36\%\\
		Letter & 1600 &  32 & 100 & 6.25\% \\
		Mammography  & 11183               & 6                  & 260           &  2.32\%   \\
		Mnist & 7603 & 100 & 700 & 9.2\% \\
		Optdigits	& 5216 & 64 &	150 & 3\% \\
		Pendigits    & 6870                & 16                 & 156          &  2.27\%    \\
		Pima         & 768                 & 8                  & 268          &  35\%    \\
		Satellite & 6435 & 36 & 2036  & 32\% \\
		Satimage-2	&5803&36&	71& 1.2\% \\
		Thyroid	& 3772 & 6 & 93 & 2.5\% \\
		Vertebral	&240&	6	&30 &12.5\% \\
		Vowels& 	1456&12	&50 &3.4\%   \\
		WBC	&278	&30	&21 &5.6\%   \\ 
		Wine & 129	& 13	& 10 & 7.7\% \\ \hline
		
	\end{tabular}
	\caption{Set of data used in the experimental phase. The first column gives the name of the dataset; the second column describes the number of instances contained in each set; the third column defines the total amount of features; the fourth column gives the number of outliers; the last column presents the contamination rates.}
	\label{dataset_used}
	\end{table*}