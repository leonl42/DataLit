\section{Conclusion \& Future works}

In summary,  we present DiffMimic utilizes differentiable physics simulators (DPS) for motion mimicking and outperforms several commonly used RL-based methods in various tasks with high accuracy, stability, and efficiency. We believe DiffMimic can serve as a starting point for motion mimicking with DPS and that our simulation environments provide exciting opportunities for future research.

While there are many exciting directions to explore, there are also various challenges. One such challenge is to enable motion mimicking within minutes given any arbitrary demonstration. With differentiable physics, this would greatly accelerate downstream tasks. In our work, we demonstrate that DiffMimic can learn challenging motions in just 10 minutes. However, the tasks we evaluated were relatively short and did not involve any interactions with other objects. As the dynamic system becomes more complex with multiple objects, we leave these challenges for future work.



\clearpage
\section*{Ethics Statements}
We have carefully reviewed and adhered to the ICLR Code of Ethics. DiffMimic conducts experiments exclusively on humanoid characters, and no experiments involve human subjects. Additionally, all experiments are performed using open-sourced materials and are properly cited.

\section*{Reproducibility Statements}
Our code is easily accessible and openly available for review at \hyperlink{https://github.com/diffmimic/diffmimic}{https://github.com/diffmimic/diffmimic}. Additionally, the reported results can be easily reproduced using the provided code.

\section*{Acknowledgment}
This research is supported by the National Research Foundation, Singapore under its AI Singapore Programme (AISG Award No: AISG2-PhD-2022-01-036T,  AISG2-PhD-2021-08-015T and AISG2-PhD-2021-08-018), NTU NAP, MOE AcRF Tier 2 (T2EP20221-0012), and under the RIE2020 Industry Alignment Fund - Industry Collaboration Projects
(IAF-ICP) Funding Initiative, as well as cash and in-kind contribution from the industry partner(s).