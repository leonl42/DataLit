\section{Related Work}

\textbf{Protein-Conditioned RNA Design.} Early methods for computational design of protein-binding RNAs involved generating a large number of RNA sequences and selecting by desired secondary structure motifs \cite{kim2010computational} or molecular dynamics \cite{zhou2015searching, buglak2020methods}. These approaches are computationally expensive and require specifying design constraints a priori, which are often unknown for new protein targets. More recently, generative sequence-based approaches based on LSTMs and VAEs have been trained on SELEX data \cite{im2019generative, iwano2022generative}. However, these methods can only be trained for one protein at a time and cannot be applied to proteins where SELEX data does not exist.

\textbf{Unconditional RNA Design.} Computational methods have been developed for RNA structure design, including classical algorithmic approaches \cite{yesselman2019computational} and generative modeling methods. In particular, \citet{morehead2023towards} recently proposed an $SE(3)$-discrete diffusion model (MMDiff) for joint generation of nucleic acid sequences and structures. While MMDiff can generate short micro-RNA molecules, it has trouble conditionally generating protein-binding RNAs and samples of longer sequence length.

Deep learning methods for inverse folding generally apply graph-based encoders for a single RNA structure \cite{tan2023hierarchical} or several conformers \cite{joshi2023multi}. We adapt an inverse folding model to accept protein information as a condition and train on a denoising objective.



\textbf{Protein Structure Design.} Deep generative methods for protein design have demonstrated that realistic protein backbones can be generated efficiently via flow matching \cite{yim2023fast, bose2023se}. Prior work has also shown that diffusion models can be conditioned on a target protein to generate functional protein binders \cite{ingraham2023illuminating, watson2023novo}. We apply a similar approach within the flow matching framework. Our work is also related to sequence-structure co-design methods where the outputs are iteratively refined \cite{jin2021iterative, stark2023harmonic}.