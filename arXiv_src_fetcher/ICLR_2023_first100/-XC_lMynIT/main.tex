%%%%%%%% ICML 2024 EXAMPLE LATEX SUBMISSION FILE %%%%%%%%%%%%%%%%%

\documentclass{article}

% Recommended, but optional, packages for figures and better typesetting:
\usepackage{microtype}
\usepackage{graphicx}
\usepackage{subfigure}
\usepackage{booktabs} % for professional tables
\usepackage{multirow}
% \usepackage{algorithmicx}
% \usepackage{algpseudocode}

% hyperref makes hyperlinks in the resulting PDF.
% If your build breaks (sometimes temporarily if a hyperlink spans a page)
% please comment out the following usepackage line and replace
% \usepackage{icml2024} with \usepackage[nohyperref]{icml2024} above.
\usepackage{hyperref}


% Attempt to make hyperref and algorithmic work together better:
% \newcommand{\theHalgorithm}{\arabic{algorithm}}

% Use the following line for the initial blind version submitted for review:
% \usepackage{icml2024}

% If accepted, instead use the following line for the camera-ready submission:
\usepackage[accepted]{icml2024}

% For theorems and such
\usepackage{amsmath}
\usepackage{amssymb}
\usepackage{mathtools}
\usepackage{amsthm}


% if you use cleveref..
\usepackage[capitalize,noabbrev]{cleveref}

%%%%%%%%%%%%%%%%%%%%%%%%%%%%%%%%
% THEOREMS
%%%%%%%%%%%%%%%%%%%%%%%%%%%%%%%%
\theoremstyle{plain}
\newtheorem{theorem}{Theorem}[section]
\newtheorem{proposition}[theorem]{Proposition}
\newtheorem{lemma}[theorem]{Lemma}
\newtheorem{corollary}[theorem]{Corollary}
\theoremstyle{definition}
\newtheorem{definition}[theorem]{Definition}
\newtheorem{assumption}[theorem]{Assumption}
\theoremstyle{remark}
\newtheorem{remark}[theorem]{Remark}

% Todonotes is useful during development; simply uncomment the next line
%    and comment out the line below the next line to turn off comments
%\usepackage[disable,textsize=tiny]{todonotes}
\usepackage[textsize=tiny]{todonotes}


% The \icmltitle you define below is probably too long as a header.
% Therefore, a short form for the running title is supplied here:
\icmltitlerunning{RNAFlow: RNA Structure \& Sequence Design via Inverse Folding-Based Flow Matching}

\begin{document}

\twocolumn[
\icmltitle{RNAFlow: RNA Structure \& Sequence Design via \\ Inverse Folding-Based Flow Matching}

% It is OKAY to include author information, even for blind
% submissions: the style file will automatically remove it for you
% unless you've provided the [accepted] option to the icml2024
% package.

% List of affiliations: The first argument should be a (short)
% identifier you will use later to specify author affiliations
% Academic affiliations should list Department, University, City, Region, Country
% Industry affiliations should list Company, City, Region, Country

% You can specify symbols, otherwise they are numbered in order.
% Ideally, you should not use this facility. Affiliations will be numbered
% in order of appearance and this is the preferred way.
\icmlsetsymbol{equal}{*}

\begin{icmlauthorlist}
\icmlauthor{Divya Nori}{yyy}
\icmlauthor{Wengong Jin}{xxx}
%\icmlauthor{}{sch}
%\icmlauthor{}{sch}
\end{icmlauthorlist}

\icmlaffiliation{yyy}{Department of Electrical Engineering and Computer Science, Massachusetts Institute of Technology, Cambridge, MA, USA}
\icmlaffiliation{xxx}{Broad Institute of MIT and Harvard, Cambridge, MA, USA}

\icmlcorrespondingauthor{Divya Nori}{divnor80@mit.edu}

% You may provide any keywords that you
% find helpful for describing your paper; these are used to populate
% the "keywords" metadata in the PDF but will not be shown in the document
\icmlkeywords{Machine Learning, ICML}

\vskip 0.3in
]

% this must go after the closing bracket ] following \twocolumn[ ...

% This command actually creates the footnote in the first column
% listing the affiliations and the copyright notice.
% The command takes one argument, which is text to display at the start of the footnote.
% The \icmlEqualContribution command is standard text for equal contribution.
% Remove it (just {}) if you do not need this facility.

%\printAffiliationsAndNotice{}  % leave blank if no need to mention equal contribution
\printAffiliationsAndNotice{} % otherwise use the standard text.

\begin{abstract}
The growing significance of RNA engineering in diverse biological applications has spurred interest in developing AI methods for structure-based RNA design. While diffusion models have excelled in protein design, adapting them for RNA presents new challenges due to RNA's conformational flexibility and the computational cost of fine-tuning large structure prediction models. To this end, we propose RNAFlow, a flow matching model for protein-conditioned RNA sequence-structure design. Its denoising network integrates an RNA inverse folding model and a pre-trained RosettaFold2NA network for generation of RNA sequences and structures. The integration of inverse folding in the structure denoising process allows us to simplify training by fixing the structure prediction network. We further enhance the inverse folding model by conditioning it on inferred conformational ensembles to model dynamic RNA conformations. Evaluation on protein-conditioned RNA structure and sequence generation tasks demonstrates RNAFlow's advantage over existing RNA design methods.
\end{abstract}


\section{Introduction}\label{sec:intro}
% Multi-armed bandit (MAB) is a classic sequential decision making problem \citep{auer2002finite}, where a learning agent chooses among competing actions sequentially to maximize its accumulative reward over time. 
% %Despite its simplicity, MAB exemplifies the exploration-and-exploitation dilemma that also exists in more complicated problems. 
% An important extension of MAB, named linear contextual bandit \citep{li2010contextual}, incorporates contextual information in the problem setting, by assuming a linear mapping between the context and expected reward. It has gained popularity in various applications, such as recommender systems \citep{li2010contextual}, display advertisement \citep{li2010exploitation} and clinical trials \citep{durand2018contextual}.
% Most existing linear bandit solutions are designed under a centralized learning setting, i.e., data is readily available at a central server. However, with the increasing public concerns of privacy, especially the bandit algorithms usually directly learn from user data,
% %more and more people are reluctant to provide their own data and strict regulations on data usage like GDPR have also went into effect \cite{voigt2017eu}, which makes 
% there is a growing demand to keep data decentralized and push the learning of bandit models to the client side. 
% % This idea is also made much more feasible due to the growing computational power of edge devices nowadays. 

% Federated learning has recently emerged as a promising setting for decentralized machine learning.
% % , and its effectiveness was first validated at a large scale by training a global model across all mobile devices via the Google Keyboard Android application \cite{konevcny2016federated}. 
% %The term ``federated learning" was first introduced by \citet{mcmahan2017communication} with an emphasis on efficiently training deep models over mobile device applications. As significant amount of later works have applied federated learning to other applications, there may be variations in its meaning for different research communities. 
% Since its debut in \citet{mcmahan2017communication}, there have been variations in its definition for different applications \citep{yang2019federated}.
% In this paper, we follow the general definition by \citet{kairouz2019advances}: multiple clients collaborate in solving a machine learning problem under the coordination of a central server, while keeping each client's raw data local. 
% So far, most existing works in federated learning study offline supervised learning problems \citep{konevcny2016federated,zhao2018federated}, where labeled training instances already sit on the client side. How to perform bandit learning under the federated learning setting remains underexplored.
As a popular online learning problem, linear contextual bandit has been used for a variety of applications, including recommender systems \citep{li2010contextual}, display advertisement \citep{li2010exploitation} and clinical trials \citep{durand2018contextual}. While most existing solutions are designed under a centralized setting (i.e., data is readily available at a central server), in response to the increasing application scale and public concerns of privacy, there is a growing demand to keep data decentralized and push the learning of bandit models to the client side.
% As a classic sequential decision making problem, linear contextual bandit has been widely used for a variety of real-world applications, including recommender systems \citep{li2010contextual}, display advertisement \citep{li2010exploitation} and clinical trials \citep{durand2018contextual}. 
% Most existing solutions are designed under a centralized learning setting, i.e., data is readily available at a central server. However, with the increasing public concerns of privacy, especially the bandit algorithms usually directly learn from user data,
% there is a growing demand to keep data decentralized and push the learning of bandit models to the client side. 
Federated learning has recently emerged as a promising setting for decentralized machine learning \citep{konevcny2016federated}.
% , and its effectiveness was first validated at a large scale by training a global model across all mobile devices via the Google Keyboard Android application \cite{konevcny2016federated}. 
%The term ``federated learning" was first introduced by \citet{mcmahan2017communication} with an emphasis on efficiently training deep models over mobile device applications. As significant amount of later works have applied federated learning to other applications, there may be variations in its meaning for different research communities. 
Since its debut in \citeyear{mcmahan2017communication}, there have been many variations for different applications \citep{yang2019federated}. However, most existing works study offline supervised learning problems \citep{li2019convergence,zhao2018federated}, which only concerns optimization convergence over a fixed dataset. How to perform federated bandit learning remains under-explored, and is the main focus of this paper. 

Analogous to its offline counterpart, the goal of federated bandit learning is to minimize the cumulative regret incurred by $N$ clients during their online interactions with the environment over time horizon $T$,
% $N$ clients in a learning system need to collaborate to minimize the overall cumulative regret over a finite time horizon $T$, 
while keeping each client's raw data local. Take recommender systems as an example, where the clients correspond to the edge devices that directly interact with user by making recommendations and receiving feedbacks. Unlike centralized setting where observations from all clients are immediately transmitted to the server to learn a single model, in federated bandit learning, each client makes recommendations based on its local model, with occasional communication for collaborative model estimation.

% In this paper, we follow the general definition by \citet{kairouz2019advances}: multiple clients collaborate in solving a machine learning problem under the coordination of a central server, while keeping each client's raw data local. 


%Though having potential for wide range of applications, online learning problems like linear bandit in federated learning setting, a.k.a. federated linear bandits \cite{dubey2020differentially}, have not attracted enough attention and still remain an open problem. 

% Therefore, it is a natural idea to study contextual linear bandit in a federated learning paradigm, which is also referred to as federated linear bandits \cite{dubey2020differentially}. In a federated learning paradigm, multiple clients collaborate in solving a machine learning problem, under the coordination of a central server, and each client's raw data is stored locally and not transferred to the server. 
% when linear bandit algorithms are applied to the federated learning paradigm, because these algorithms assume a traditional centralized machine learning system where all the data are collected together and all the computation happens in one machine or data center. 
Several new challenges arise in this problem setting. 
The first is the conflict between the need of timely data/model aggregation for \emph{regret minimization} and the need of \emph{communication efficiency}, since communication is the main bottleneck for many distributed application scenarios, e.g., communication in a network of mobile devices can be slower than local computation by several orders of magnitude \citep{huang2013depth}. A well-designed communication strategy becomes vital to strike the balance. 
In addition, 
% constraints from real-world applications should also be taken into consideration when designing the communication strategy. For example, 
the clients often have various response time and even occasional unavailability in reality, due to the differences in their computational and communication capacities.
% the clients may differ in their computational and communication capacities. This will lead to various response time and even occasional unavailability. 
This hampers global synchronization employed in existing federated bandit solutions \citep{wang2019distributed,dubey2020differentially}, which requires the server to first send a synchronization signal to all clients, wait and collect their returned local updates, and finally send the aggregated update back to every client.
Second, it is very restrictive to only assume homogeneous clients, i.e., they solve the same learning problem. 
% As bandit algorithms are mostly deployed to interact with individual users, studying heterogeneous clients with personalized learning problems has a greater potential.
Studying \emph{heterogeneous clients} with distinct learning problems has a greater potential in practice.
This is referred to as ``\emph{non-IIDness}" of data in the context of federated learning, e.g., the difference in $\mathcal{P}_{i}(\bx,y)=\mathcal{P}_{i}(\bx) \mathcal{P}_{i}(y|\bx)$ is caused by each client $i\in[N]$ serving a particular user or group of users, a particular geographic region, or a particular time period. Apparently, it is also unreasonable to assume every client has equal amount of new observations, which however is assumed in existing works. 

%To be more concrete, due to the time-varying arm set $\cA_{t}$ and the dependence on history data for arm selection in linear bandit, context vector $X$ is non-IID in nature and is not the main concern. 
% It is not a major concern since the performance metric, i.e. regret $r_{t}$, is defined against the best arm in $\cA_{t}$. 

% For example, internet connection and the different computation power of devices.
% \textcolor{red}{reasons we need async algo}

% This naturally leads to the question: how to balance between regret minimization and communication efficiency in the federated linear bandit problem.
To address the first challenge, we propose an asynchronous event-triggered communication framework for federated linear bandit. 
%Our event-triggering mechanism offers a flexible way to balance between the regret-minimization and communication-efficiency dilemma. 
Communication with a client happens only when the last communicated update to the client becomes irrelevant to the latest one; and we prove only by then effective regret reduction can be expected in this client because of the communication. 
Under this asynchronous communication, each client sends local update to and receives aggregated update from the server independently from other clients, with no need for global synchronization. This improves our method's robustness against possible delays and temporary unavailability of clients. It also brings in reduced communication cost when the clients have distinct availability of new observations, because global synchronization requires every client in the learning system to send its local update despite the fact that some clients can have very few new observations since last synchronization.
% make the proposed method more robust and practical against the infrastructure constraints, because the aggregated update sent to each client is asynchronous and  
% This makes our method more robust against possible delays in the communication, and we prove that the client enjoys the same benefit in regret reduction as long as it receives the update before its next interaction with the environment.

To address the second challenge, we design algorithms for federated linear bandit with both ``\emph{IIDness}" and ``\emph{non-IIDness}" based on the proposed communication framework. We consider two different assumptions on the reward functions. First, all the clients share a common reward function i.e., a single model is learned for all clients. Second, each client has a distinct reward function with mutual dependence captured by globally shared components in the unknown parameter, which resembles 
%so one model per client is learned during the interaction with the environment, which in essence is similar to the problem considered in
federated multi-task learning \citep{smith2017federated}.
We rigorously prove the upper bounds of accumulative regret and communication cost for the proposed algorithms in these two settings, and conduct extensive empirical evaluations to demonstrate the effectiveness of our proposed framework.
% especially its flexibility in balancing the trade-off between regret and communication cost.

\noindent \textbf{Compression} \quad According the survey paper \cite{Gupta2020CompressionOD}, model compression methods for NLP currently include: pruning(\citet{Michel2019AreSH},\citet{Voita2019AnalyzingMS},\citet{Prasanna2020WhenBP}), quantization\cite{Cheong2019transformersZ}, knowledge distillation(\citet{Jiao2020TinyBERTDB},\citet{Iandola2020SqueezeBERTWC}), parameter sharing(\citet{Lan2020ALBERTAL},\citet{Lan2020ALBERTAL}), tensor decomposition and sub-quadratic complexity transformers. \\

\noindent \textbf{Fairness} \quad Google Brain \cite{Hooker2020CharacterisingBI} tries to characterize compression's impact on fairness for vision models. They tests quantization and pruning techniques and argue that though compressed models achieve similar overall error rate, but fairness is compromised because performance of samples with under-represented features is sacrificed after compression. Researchers from University of Utah \cite{Joseph2020GoingBC} proposes adding fairness into the compression objective function for vision tasks. However, to the best of our knowledge, no prior work has been done studying Knowledge Distillation method, nor are there any compression fairness studies on NLP models.\\

\noindent \textbf{Compression as regularization} \quad 
\cite{Fan2020ReducingTD} introduces a compression method for transformers named structured dropout, which is shown to achieve higher performance than distillation and weight pruning. The method assumes that transformer models are over-parametrized and sub-structures of the original model could achieve equivalent performances, plus that smaller networks will enjoy the benefit of regularization. Many studies (\citet{Jordo2021OnTE}, \citet{Bartoldson2020TheGT}) also argue that pruning of Convolutional Neural Networks serves as a way of regularization. \\

\noindent \textbf{Compression for robust learning} \quad
The seminal work of \cite{Papernot2016DistillationAA} introduces Knowledge Distillation as a defense against adversarial perturbations. Following works continue to use Knowledge Distillation to improve generalization \cite{Arani2019ImprovingGA} and robustness (\citet{Goldblum2020AdversariallyRD}). Knowledge Distillation is also used to improve models on privacy protection (\citet{Shejwalkar2019ReconcilingUA}, \citet{Zhao2021KnowledgeDW}). Moreover, pruning can improve model robustness according to the following studies (\citet{Jordo2021OnTE}, \citet{Pang2021BagOT},\citet{Hendrycks2019BenchmarkingNN} ). \cite{Kaya2019ShallowDeepNU} shows that stopping at earlier layers during inference can improve model robustness. The intuition is still that smaller and shallower networks are more robust.\\

\noindent \textbf{Compression for fairness} \quad Our experiments demonstrate monotonic reduction of model toxicity and biases as the model size decreases with distillation. The gold question is whether the regularization and robustness effect of model compression incur the toxicity and bias reduction that we observed in distilled generative language models. If yes, can we also develop techniques to improve NLP fairness using model compression? If not, what is the cause of the monotonic toxicity and bias reduction?

\section{Experiment Setup}

\subsection{Model aspect ratio}\label{section:method:ratio}

For our Transformer models we fix the number of embedding features, sequence features, attention features, and the hidden layer dimension in the FFNs for each task.
We vary the number of layers, $L$, and the number of heads per layer, $H$, whilst keeping $L \times H$ constant.
Starting with typical values for $L$ \& $H$, we then move down to a single layer with one to two intermediate model aspect ratios, observing how test accuracy changes for trained models.
See \Cref{fig:wide} for an illustration of our deepest and widest models.

In all of our tasks we do not use pretrained embeddings or pretrained model parameters as this allows us to make a fairer comparisons.
Computing pretrained embeddings and weights that are optimised for each combination of attention and model aspect ratio would be computationally prohibitive, and using ones typically used for deep networks would introduce bias.


\subsection{Datasts and models}\label{section:method:datasets}

\begin{table}[!h]
    \caption{The different tasks and datasets used.}
    \label{table:tasks}
    \begin{center}
        \begin{tabular}{l | l l l l}
            \toprule
            \textbf{Task Name} & \textbf{Classification} & \textbf{Dataset} & \textbf{Input Type} & \textbf{Input Length} \\
            \midrule
            IMDb Token Level & Binary & IMDb Reviews & Review text tokens & 500 \\
            IMDb Byte Level & Binary & IMDb Reviews & Review text bytes & 1000 \\
            Listops & 10-way & LRA Listops & Listop bytes & 2000 \\
            Document Matching & Binary & ACL Anthology & Document bytes & 4000 \\
            \bottomrule
        \end{tabular}
    \end{center}
\end{table}

Primarily we investigate using 4 different text classification tasks, a vision based task is investigated in \Cref{sec:discussion:vit}.
The first two are sentiment analysis (binary classification) on the IMDb dataset.
One uses input embeddings at the token level with an input sequence length of 500, and the other uses input embeddings at the byte level and an input sequence length of 1k.
This second task is taken from LRA \citep{lra}, as are the final two.
The third task is Listops 10-way classification with a sequence length of 2k.
This task involves reasoning about sequences of hierarchical operations to determine a result, and the input is given at the byte level.
The final task used is byte level document matching, a binary classification task with a sequence length of 4k.
This uses the ACL anthology network for related article matching \citep{acl}.
We summarise each task in \Cref{table:tasks}, further details on them can be found in \citet{lra}.

For the text classification and Listops task we try four different model aspect ratios.
In terms of number of layers and heads per layer these are: 6 layers, 8 heads; 3 layers, 16 heads; 2 layers, 24 heads; and finally 1 layer, 48 heads.
As the matching task has an input sequence length of 4k, the models used are smaller to offset the computation size involved.
Thus the combinations we use are: 4 layers, 4 heads; 2 layers, 8 heads; 1 layer, 16 heads.

In order to investigate whether the type of the attention mechanism influences the effects of widening the attention layer, we test on 10 different types of Transformer attention, including the original dot-product attention \citep{tfm}.
The others are: Bigbird \citep{bigbird}, Linear Transformer \citep{linear_tfm}, Linformer \citep{linformer}, Local attention \citep{local_tfm}, Longformer \citep{longformer}, Performer \citep{performer}, Sinkhorn \citep{sinkhorn}, Sparse Transformer \citep{sparse_tfm}, and Synthesizer \citep{synthesizer}.
The implementations and hyper-parameter choices for each attention type are the same as used in LRA.
Unlike LRA, we do not test with Reformer \citep{reformer} due to it requiring the sequence features and attention features to have the same dimension. Training and other Transformer hyperparameters used for each task are given in \Cref{appendix:tasks}.


\section{Experimental Results}
\label{exp}

\begin{figure}
    \centering
    \includegraphics[width=\textwidth]{square_toroid_6fig.pdf}
    \caption{Test of the algorithm on a square toroidal dataset: anomalies lie inside of a box made up of normal instances. This setting is particularly challenging for the Isolation Forest algorithm as it is much easier to quickly separate normal points (depicted in purple) w.r.t anomalies (yellow). The anomaly score on test samples is shown on the second and third panel: the yellow is assigned to the points having the highest anomaly score, the purple viceversa. In the second row of panels the performances of the detector at each iteration is depicted: the \emph{area under the ROC curve} (auc) and the \emph{average precision} (ap) measured on the test set quickly improve. }
    \label{fig:toroid}
\end{figure}

%\begin{figure}
%    \centering
%    \includegraphics[width=\textwidth]{images/anomalous_cluster_6fig.pdf}
%    \caption{ciao ciao ciao}
 %   \label{cluster}
%\end{figure}

%tabella dei dataset
%descrizione dataset
In this section,  we analyse the performance of \approach, matching both the proposed update strategies with the two described query approaches. Doing so, we hope to obtain a full and detailed evaluation of the presented model, with the purpose of analysing the efficiency of each combination as well as giving meaningful guidance on the proper use of the proposed strategies.
%This section describes the achieved results obtained using the two query strategies proposed in the previous section. Specifically, we match the proposed fixing strategy together with the two query strategies, in order to obtain a full and detailed evaluation of the presented model. 

%The proposed model performance is compared with the Isolation Forest as well as with the Random Forest. Our approach, in fact, essentially relies on designing an active learning based modification of the classical Isolation Forest. Therefore, since, even if of a small size, we are now dealing with a labeled set of data, we decided it was fair to compare the proposed approach with a supervised learning model. Specifically, we picked the Random Forest on account of its binary tree based structure.

Firstly, we tested our approach on synthetic datasets like the challenging shape depicted in Figure \ref{fig:toroid}, where the normal data make up a square toroid and the anomalous data lie inside it. In this kind of datasets, the Isolation Forest perform quite badly and there is a lot of room for improvement as it struggles to  separate in few steps the anomalies: in this case it is much easier to wrongly separate normal points w.r.t. anomalies, indeed the first iteration of the model corresponding to the unsupervised training gives the highest anomaly score to the normal bottom left point of the toroid. On the contrary, at the 25-\textit{th} iteration the model has learnt the correct function and is able to perfectly classify all the points. This is also visible in the panels of the second row of Figure \ref{fig:toroid}, where the performance of the detector at each iteration is depicted with lines having different colors. As the model is allowed to query new points, the average detection performance quickly improves.

We decided to test the model on a set of $18$ real data openly available \cite{Rayana,Dua:2019}.
Table \ref{dataset_used} presents the dataset used to test the performance of the proposed model. These come from different domains like medicine, industry and natural sciences and are characterised by various number of points as well as percentage of anomalous data. Defining the contamination as the ratio between the number of anomalies and total number of samples of the dataset, they are characterized by a wide feature range between $6$ to $100$ and a contamination percentage between $0.9\%$ and $36\%$. It is very important to highlight how most of these datasets were built: they are adaptations of multi-class classification datasets to the anomaly detection task where one of the classes is under sampled and labelled as outlier. This means that points considered outliers are not just points randomly scattered in the features space like general anomalies, but they live in specific positions of the space that can be hard to be defined without labels. In this context, weakly supervised methods prove their efficacy, starting from an unsupervised guess, and improving continuously as labels are included in the model.

\begin{table*}[]
\centering
	\begin{tabular}{@{}lccc c@{}} \hline
		\textbf{Dataset} & \textbf{Instances} & \textbf{Features} & \textbf{Anomalies}  & \textbf{Contamination}\\ \hline
		AnnThyroid   & 7200                & 6                  & 534        & 7.42\%        \\ 
		Breastw      & 683                 & 9                  & 239        & 35 \%        \\
		Cardio &	1831 & 21 & 176 & 9.6\% \\
		Cover & 286048 & 10 & 2747 & 0.9\% \\
		Ionosphere   & 351                 & 33                 & 126          &      36\%\\
		Letter & 1600 &  32 & 100 & 6.25\% \\
		Mammography  & 11183               & 6                  & 260           &  2.32\%   \\
		Mnist & 7603 & 100 & 700 & 9.2\% \\
		Optdigits	& 5216 & 64 &	150 & 3\% \\
		Pendigits    & 6870                & 16                 & 156          &  2.27\%    \\
		Pima         & 768                 & 8                  & 268          &  35\%    \\
		Satellite & 6435 & 36 & 2036  & 32\% \\
		Satimage-2	&5803&36&	71& 1.2\% \\
		Thyroid	& 3772 & 6 & 93 & 2.5\% \\
		Vertebral	&240&	6	&30 &12.5\% \\
		Vowels& 	1456&12	&50 &3.4\%   \\
		WBC	&278	&30	&21 &5.6\%   \\ 
		Wine & 129	& 13	& 10 & 7.7\% \\ \hline
		
	\end{tabular}
	\caption{Set of data used in the experimental phase. The first column gives the name of the dataset; the second column describes the number of instances contained in each set; the third column defines the total amount of features; the fourth column gives the number of outliers; the last column presents the contamination rates.}
	\label{dataset_used}
	\end{table*}

We used the average precision score metric to measure the results obtained. Splitting equally the dataset in training and testing set, we carried on a number of $25$ queries, and the experiments were conducted $50$ independent times to study the performances distribution. 
The experiments were performed on equipment with Intel Core i7-6800K CPU and 32 GB RAM. The implementation of \approach is freely available online\footnote{\url{https://github.com/tombarba/ActiveLearningIsolationForest} }. 

First, we tested the four possible combinations of the two proposed update strategies together with the two query strategies so as to determine the best combination with respect to the majority of the considered test sets. Figure \ref{risultati_1} shows the obtained results.

%why credibility has problems
\begin{figure*}
   \centering
    \includegraphics[width=\textwidth]{four_strategies.pdf}
    \caption{Performance of the four combinations of the proposed strategies with the datasets described in Table \ref{dataset_used}. As a general result, it can be noticed that querying the most anomalous point represents the most appropriate choice, overall leading to quicker improvements of the performance. When it comes to the update strategies, both the linear and the logarithmic depths seem to represent a reasonable choice. However, the linear depth appears to moderately be more stable.}
    \label{risultati_1}
\end{figure*}

%\tommi{un commento dei revisori potrebbe essere: mi mostri anche la query randomica?} \gas{concordo}
% 4 approcci
Given that the first point of the curve represents the performance of the fully unsupervised Isolation Forest, it can be observed as, broadly speaking, all four proposed matches represent an improvement with respect to the performance of the Isolation Forest. The two trials of the "most anomalous" query are depicted in solid line, while the "maximum uncertainty" in dashed line. Even if there are some exceptions, in most cases the first policy seems the one having the fastest improvements. This is a very interesting aspect since the "most anomalous" strategy is the cheapest and most natural policy among the two considering the DSS scenario previously described. Concerning the updating strategy, as expected, the piece-wise linear is the one having the most stable improvements in the dataset and among the datasets. As a direct consequence, we decided to use the combination "most anomalous - piecewise linear" for the comparison to the other baseline and state-of-art competitor.

% state-of-art
As our approach \approach essentially relies on designing an active learning based modification of the classical Isolation Forest, we decided to compare it with other tree based models: a fully supervised model, i.e. the Random Forest, and a weakly supervised model, the Isolation Forest - Active Anomaly Detection (IF-AAD).
The RF represents the baseline since it is the most well known but simple classification approach; unfortunately it requires samples from both the classes inlier/outlier and it is computationally expensive as every time the user labels a new data point the forest is retrained. As the RF does not have a default method to query its points, in the following comparison the points given to the RF for training are the points queried by our \approach.
On the other side IF-AAD is a active semi-supervised model  that does not need both class labels and is available online\footnote{\url{https://github.com/shubhomoydas/ad_examples}}. 

\begin{figure}
    \centering
    \includegraphics[width=\textwidth]{anomalous_piecewise.pdf}
    \caption{Comparison of most anomalous - piece-wise linear \approach with IF-AAD and RF. It can be observed that, in general our method represents the best course of action, having the highest performance score usually with a very small amount of labeled data.}
    \label{benchmark}
\end{figure}

Figure \ref{benchmark} and Table \ref{table:final_results} show the benchmark results: it is prominent that \approach obtains the finest results with a very little number of labels, generally outperforming IF-AAD. Due to the strong imbalance between inliers and outliers RF receives quite late both labels from the two classes, leading to poor detection performances: in \emph{breastw}, \emph{satellite}, and \emph{satimage-2} the querying process never got them over 25 iterations and 50 independent repetitions %\tommi{questi sarebbero stati tutti falsi positivi comunque}.



\begin{table}[]
\centering
\begin{adjustbox}{angle=270}
\begin{tabular}{llllllllll}
\toprule
{} &      {}  & \multicolumn{2}{l}{anom-log} & \multicolumn{2}{l}{anom-lin} & \multicolumn{2}{l}{unc-log} & \multicolumn{2}{l}{unc-lin} \\
{} & IF-AAD &       RF &   ALIF &       RF &   ALIF &      RF &   ALIF &      RF &   ALIF \\
\midrule
\textbf{annthyroid } &   0.34 &     0.11 &  0.41 &     0.11 &  0.41 &    0.11 &  0.44 &    0.22 &  \textbf{0.45} \\
\textbf{breastw    } &   0.93 &     0.00 &  0.98 &     0.00 &  0.98 &    0.28 &  \textbf{0.99} &    0.48 &  0.98 \\
\textbf{cardio     } &   0.56 &     0.15 &  0.63 &     0.16 &  0.63 &    0.34 &  0.57 &    0.49 &  \textbf{0.69} \\
\textbf{cover      } &   0.18 &     0.20 &  0.21 &     0.18 &  0.20 &    0.25 &  0.09 &    \textbf{0.46} &  0.19 \\
\textbf{ionosphere } &   0.84 &     0.08 &  \textbf{0.85} &     0.08 &  \textbf{0.85} &    0.47 &  0.70 &    0.52 &  0.82 \\
\textbf{letter     } &   0.12 &     0.08 &  0.11 &     0.07 &  \textbf{0.14} &    0.06 &  0.07 &    0.07 &  0.11 \\
\textbf{mammography} &   0.28 &     0.09 &  0.40 &     0.10 &  \textbf{0.43} &    0.04 &  0.28 &    0.18 &  0.28 \\
\textbf{mnist      } &   0.36 &     0.15 &  0.41 &     0.15 &  \textbf{0.43} &    0.25 &  0.24 &    0.42 &  0.42 \\
\textbf{optdigits  } &   0.11 &     0.46 &  0.49 &     0.38 &  \textbf{0.51} &    0.00 &  0.05 &    0.00 &  0.06 \\
\textbf{pendigits  } &   0.42 &     0.28 &  0.39 &     0.34 &  \textbf{0.59} &    0.10 &  0.19 &    0.24 &  0.42 \\
\textbf{pima       } &   0.50 &     0.44 &  0.49 &     0.43 &  0.56 &    0.34 &  \textbf{0.57} &    0.43 &  0.54 \\
\textbf{satellite  } &   0.65 &     0.01 &  0.69 &     0.01 &  0.68 &    0.42 &  0.64 &    0.49 &  \textbf{0.70} \\
\textbf{satimage-2 } &   0.92 &     0.00 &  \textbf{0.93} &     0.00 &  \textbf{0.93} &    0.25 &  0.48 &    0.53 &  0.88 \\
\textbf{thyroid    } &   0.66 &     0.37 &  0.69 &     0.33 &  \textbf{0.80} &    0.10 &  0.73 &    0.33 &  \textbf{0.80} \\
\textbf{vertebral  } &   0.14 &     0.16 &  \textbf{0.27} &     0.12 &  0.22 &    0.17 &  0.14 &    0.22 &  0.17 \\
\textbf{vowels     } &   0.29 &     0.29 &  0.33 &     0.30 &  \textbf{0.51} &    0.07 &  0.06 &    0.18 &  0.27 \\
\textbf{wbc        } &   0.71 &     0.68 &  0.76 &     0.68 &  \textbf{0.82} &    0.11 &  0.69 &    0.21 &  0.73 \\
\textbf{wine       } &   0.69 &     0.73 &  0.80 &     0.75 &  \textbf{0.85} &    0.28 &  0.38 &    0.47 &  0.64 \\
\bottomrule
\end{tabular}
\end{adjustbox}

\caption{Summary of the obtained results. The reported performances are the mean performance along the 25 iterations and 50 repetitions of the algorithm. \approach consistently beats the other tested approached on all the datasets, in particular the combination of the most anomalous query policy and the piece-wise linear leaf update.}
\label{table:final_results}
\end{table}

%\gas{C'e' altro di sperimentale che possiamo mostrare? La Fig. 4 e 5 sono ottime, ma se vi viene in mente altro (il paper non e' lunghissimo al momento) non guasterebbe}




%\tommi{domani ci lavoro su}
%\begin{figure}
%    \centering
%    \includegraphics[width=\textwidth]{images/anomalous_log.pdf}
%    \caption{Comparison of most anomalous - logarithmic ALBIF with IF-AAD and RF. As for the most anomalous - piece-wise combination shown in Figure \ref{benchmark}, ALBIF often has the highest performance score having a very small amount of labeled data at disposal.}
%    \label{benchmark2}
%\end{figure}



\section{Conclusion}

In this paper, we present RNAFlow, the first protein-conditioned generative model for RNA structure and sequence design. We show that an inverse folding model is an effective score prediction network within the flow matching framework, enabling the design of RNAs that outperform the baselines in terms of structure and sequence accuracy. Additionally, we show that by inverse folding over an inferred conformational ensemble, we can design plausible aptamers for GRK2 binding.

While RNAFlow shows an empirical advantage over existing methods, there is still a large gap to reach the level of accuracy achieved in \textit{de novo} protein design. We observe that currently, RNAFlow tends to achieve lower RMSD and higher recovery when the ground-truth RNA is more structurally stable (i.e. more base pairing). This suggests that RNAFlow is practically useful for aptamer design. On the other hand, when the RNA is very conformationally diverse (i.e. short single-stranded RNAs), RNAFlow often does not outperform a random sequence. In this setting, a pure language approach may be desired as structural supervision is less relevant.

To improve performance on this task, one major bottleneck is accuracy and efficiency of protein-RNA folding and docking models. This would allow us to supervise docked full-atom complexes and better model protein-RNA structural interaction, which is essential for designing RNAs that rely heavily on side chains for their function (i.e. ribozymes).

Code is available at \url{https://github.com/divnori/rnaflow}.

\section*{Impact Statement}

This paper presents work whose goal is to advance machine learning applications for structural biology and drug discovery. There are many potential societal consequences of our work, including application of our model for discovery of an RNA drug candidate. We highlight that molecules designed by RNAFlow have not been tested experimentally which is critical for any biological application. 

\bibliography{main}
\bibliographystyle{icml2024}

%%%%%%%%%%%%%%%%%%%%%%%%%%%%%%%%%%%%%%%%%%%%%%%%%%%%%%%%%%%%%%%%%%%%%%%%%%%%%%%
%%%%%%%%%%%%%%%%%%%%%%%%%%%%%%%%%%%%%%%%%%%%%%%%%%%%%%%%%%%%%%%%%%%%%%%%%%%%%%%
% APPENDIX
%%%%%%%%%%%%%%%%%%%%%%%%%%%%%%%%%%%%%%%%%%%%%%%%%%%%%%%%%%%%%%%%%%%%%%%%%%%%%%%
%%%%%%%%%%%%%%%%%%%%%%%%%%%%%%%%%%%%%%%%%%%%%%%%%%%%%%%%%%%%%%%%%%%%%%%%%%%%%%%
\newpage
\appendix
\onecolumn


\section{Flow Matching on RNA Structures}

We justify the main RNAFlow objective given in Equation \hyperref[loss]{3}, following much of the theory given by \citet{jing2023alphafold}. By the standard formulation of flow matching in Euclidean space $\mathbb{R}^{3L_r}$ ($L_r$ is RNA length), we expect to supervise with the reparameterized denoising objective (MSE).

$$ \mathcal{L} = \mathbb{E}_{R_1 \sim p_1, R_t \sim p_t} [{||\hat R_1 - \vec R_1||}^2] $$

$\hat R_1$ is predicted by neural network $\hat R_1 (\vec R_t; \theta)$ which accepts noised input $\vec R_t$ and a timestep embedding. While $\hat R_1$ is composed of an inverse folding model and pre-trained folding model, the denoised sequence can simply be thought of as an intermediate representation from which we predict the clean structure.

However, $\hat R_1 (\vec R_t; \theta)$ is $SE(3)$-invariant given that RF2NA is trained with $SE(3)$-invariant FAPE loss. We are thus flow matching over the Riemannian manifold given by quotient space $\mathbb{R}^{3L_r} / SE(3)$. To flow match in this new space, we must define a conditional probability path $p_t(\vec R_t | \vec R_1)$. We can generalize linear interpolation in the Euclidean case to interpolant $\vec R_t | \vec R_1 = \psi(R_0 | R_1)$ defined as the geodesic path from $R_0$ to $R_1$, where the geodesic path is Kabsch alignment followed by linear interpolation in $\mathbb{R}^{3L_r}$.

$$ R_0 = \text{Kabsch}(R_0, R_1) $$

$$ \psi(R_0 | R_1) = (1-t) *R_0 + t * R_1 $$

We define a distance metric $d$ on the manifold correspondingly, where $d$ is a valid metric because it is defined for all points on the manifold.

$$ d = ||\hat R_1 - \text{Kabsch}(\vec R_1, \hat R_1)|| $$

We can thus plug this expression into our original flow matching objective,

$$ \mathcal{L} = \mathbb{E}_{\vec R_1 | \vec R_t} [{||\hat R_1 - \text{Kabsch}(\vec R_1, \hat R_1)||}^2] = \text{MSE}(\hat R_1, \text{Kabsch}(\vec R_1, \hat R_1)) $$

which is equivalent to the structure loss expression in Equation \hyperref[loss]{3}. The cross entropy term simply exerts auxiliary sequence supervision.

\begin{figure*}[t!]
    \centering
    \includegraphics[width=0.6\columnwidth]{rna_sequence_lengths_histogram.png}
    \label{fig:lengths}
    \caption{Distribution of RNA lengths in processed PDBBind dataset.}
    \centering
    \includegraphics[width=0.6\columnwidth]{rnasolo_histogram.png}
    \label{fig:lengths2}
    \caption{Distribution of RNA lengths in processed RNASolo dataset.}
\end{figure*}

\section{Dataset Details}

\subsection{PDBBind}

We filter PDBBind to complexes where at least one protein $C_\alpha$ atom and RNA $C_{4}'$ atom were  within $7$ \AA, a threshold that has been used to perform alanine scans for protein-RNA interaction site analysis \cite{kruger2018protein}. For complexes containing many protein-RNA interaction sites, we use the interaction with least distance between the protein $C_\alpha$ atom and RNA $C_{4}'$ atom. We filter to RNA chains of length $\geq 6$ and $\leq 96$, and protein chains are contiguously cropped to length $50$. As shown in Figure \hyperref[fig:lengths]{5}, while there are many examples of short RNAs, the model is also exposed to several longer RNA examples. 

In the sequence similarity split, there are $1015$ complexes in train, $105$ in validation, and $72$ in test. The RF2NA split has $1059$ complexes in train, $117$ in validation, and $16$ in test. 



\subsection{RNASolo}

As mentioned, we use the RNASolo dataset for Traj-to-Seq training. The dataset consists of RNAs from apo RNA structures, protein-RNA complexes, and RNA-DNA complexes. As shown in Figure \hyperref[fig:lengths2]{6}, the distribution of RNA lengths in the dataset is somewhat bimodal, with many sequences around length $10$ and around length $75$. In the sequence similarity split, there are $2314$ distinct RNA sequences in train, $106$ in validation, and $78$ in test. In the RF2NA split, there are $2370$ distinct RNA sequences in train, $110$ in validation, and $16$ in test. 

\section{Training and Architecture Details}

\subsection{Noise-to-Seq}

For both splits, the Noise-to-Seq encoder and decoder GVP layers use a node scalar feature dimension of $128$, node vector feature dimension of $16$, edge scalar feature dimension of $32$, and edge vector feature dimension of $1$. On both splits, the model was pre-trained for $100$ epochs using an Adam optimizer with a learning rate of $0.001$, which takes a few hours on an NVIDIA A5000-24GB GPU.

\subsection{Traj-to-Seq}

For both splits, the Traj-to-Seq encoder and decoder GVP layers use a node scalar feature dimension of $128$, node vector feature dimension of $16$, edge scalar feature dimension of $64$, and edge vector feature dimension of $4$. The model was trained for $100$ epochs by an Adam optimizer with a learning rate of $0.001$.

\subsection{RNAFlow}

We fine-tune RNAFlow for $100$ epochs which takes one day on an NVIDIA A5000-24GB GPU, and we use the Adam optimizer with a learning rate of $0.001$.

\subsection{LSTM}

For both splits, the encoder's hidden dimension is $128$ which was selected by a hyperparameter sweep, and the model was trained by an Adam optimizer with a learning rate of $0.001$.


\subsection{Output Rescoring Model}

Following Noise-to-Seq, the encoder and decoder GVP layers use a node scalar feature dimension of $128$, node vector feature dimension of $16$, edge scalar feature dimension of $32$, and edge vector feature dimension of $1$. The node output feature dimension is $256$, and the final feedforward network consists of three fully connected layers with ReLU activation. The model was trained for $50$ epochs using an Adam optimizer with a learning rate of $0.01$.


%%%%%%%%%%%%%%%%%%%%%%%%%%%%%%%%%%%%%%%%%%%%%%%%%%%%%%%%%%%%%%%%%%%%%%%%%%%%%%%
%%%%%%%%%%%%%%%%%%%%%%%%%%%%%%%%%%%%%%%%%%%%%%%%%%%%%%%%%%%%%%%%%%%%%%%%%%%%%%%


\end{document}


% This document was modified from the file originally made available by
% Pat Langley and Andrea Danyluk for ICML-2K. This version was created
% by Iain Murray in 2018, and modified by Alexandre Bouchard in
% 2019 and 2021 and by Csaba Szepesvari, Gang Niu and Sivan Sabato in 2022.
% Modified again in 2023 and 2024 by Sivan Sabato and Jonathan Scarlett.
% Previous contributors include Dan Roy, Lise Getoor and Tobias
% Scheffer, which was slightly modified from the 2010 version by
% Thorsten Joachims & Johannes Fuernkranz, slightly modified from the
% 2009 version by Kiri Wagstaff and Sam Roweis's 2008 version, which is
% slightly modified from Prasad Tadepalli's 2007 version which is a
% lightly changed version of the previous year's version by Andrew
% Moore, which was in turn edited from those of Kristian Kersting and
% Codrina Lauth. Alex Smola contributed to the algorithmic style files.
