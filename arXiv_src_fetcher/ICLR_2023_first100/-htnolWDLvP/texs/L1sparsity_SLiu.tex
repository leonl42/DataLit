
%\noindent \textbf{\JC{Ablation study on parameter scheduler in \MUSparse.}}
% However, as \citep{bach2012optimization} mentioned, the downside of $\ell_1$ regulation term will affect the is its loss  in {\RA} and {\TA} compared to {\FT} and {\retrain}. Therefore, we conducted a comprehensive study of the scheduler of $\lambda$ in {\MUSparse}. In \textbf{Tab.\,\ref{tab: ablation_l1_scheduler}}, we present the results of unlearning performance on different parameter schedulers: constant scheduler, linear growing scheduler, and linear decaying scheduler.
% It shows that the decaying $\gamma$ scheduler performs the best among all the schedulers. If we directly apply a constant $\lambda$ to {\MUSparse}, it will either get a low {\UA} with lower $\lambda$ (for $\lambda = 0$, the method reduces to {\FT}) or worse {\RA} and {\TA} 
%  under higher $\lambda$. In Tab.\,\ref{tab: ablation_l1_scheduler}, we picked a sweet point to get a balance between them. If we use the linear growing scheduler, which means the method focuses on the unlearning term first then moves the focus to sparsity. If we use the linear decaying scheduler, the method focuses on the unlearning term first then moves the focus to sparsity.
\iffalse
\begin{wraptable}{R}{80mm}
\centering
\vspace*{-4.25mm}
\caption{\footnotesize{{\MU} performance  comparison of using {\MUSparse} with different   sparsity schedulers of $\gamma$  in \eqref{eq: MUSparse} and using {\retrain}.  
The unlearning scenario is given by random data forgetting (10\% data points across all classes) on (ResNet-18, CIFAR-10).
%considering various unlearning scenarios on CIFAR10 dataset. The content format follows Tab.\,\ref{tab: overall_perfoamnce}.
%Bold numbers indicate the closest performance to {\retrain}, and
A performance gap  against \textcolor{blue}{{\retrain}} is provided 
%. The relative drop or improvement represented 
in (\textcolor{blue}{$\bullet$}).
}}
\label{tab: ablation_l1_scheduler}
% \vspace*{0.1in} % Requirements, do not delete.
\resizebox{0.57\textwidth}{!}{
\begin{tabular}{c|c|c|c|c|c}
\toprule[1pt]
\midrule
  {\MU}& {\UA} & {{\MIAF}}& {{\RA}} & {{\TA}} & {{RTE} (min)} \\ 

% \cline{3-10}

% \midrule
% \rowcolor{Gray}
% \multicolumn{6}{c}{Class-wise forgetting} \\
% \midrule
% {\retrain} & \textcolor{blue}{100.00} & \textcolor{blue}{100.00} & \textcolor{blue}{100.00} & \textcolor{blue}{94.83} & 43.23
% \\
% {\MUSparse} + constant $\gamma$ & 100.00 (\textcolor{blue}{0.00}) & 100.00 (\textcolor{blue}{0.00}) & 91.69 (\textcolor{blue}{8.31})	& 87.3 (\textcolor{blue}{7.53}) & 2.61
% \\
% {\MUSparse} +   growing $\gamma$  & 100.00 (\textcolor{blue}{0.00}) & 100.00 (\textcolor{blue}{0.00}) & 94.43 (\textcolor{blue}{5.57})	& 88.43 (\textcolor{blue}{6.40}) & 2.61
% \\
% {\MUSparse} +   decaying $\gamma$ & 100.00 (\textcolor{blue}{0.00}) & 100.00 (\textcolor{blue}{0.00}) & \textbf{98.99} (\textcolor{blue}{\textbf{1.01}})	& \textbf{93.40} (\textcolor{blue}{\textbf{1.43}}) & 2.61
% \\
% \midrule
% \rowcolor{Gray}
% \multicolumn{6}{c}{Random data forgetting (all classes)} \\
\midrule
{\retrain} & \textcolor{blue}{5.41} & \textcolor{blue}{13.12} & \textcolor{blue}{100.00} & \textcolor{blue}{94.42} & 42.15
\\
%  \FT & $6.83$ (\textcolor{blue}{$1.42$}) & $14.97$ (\textcolor{blue}{$1.85$})& $96.61$ (\textcolor{blue}{$3.39$})& $90.13$ (\textcolor{blue}{$4.29$})  & 2.33 
% \\		
% \IU & $2.03$ (\textcolor{blue}{$3.38$})&  $5.07$ (\textcolor{blue}{$8.05$})& $98.26$ (\textcolor{blue}{$1.74$})& $\textbf{91.33}$ (\textcolor{blue}{$\textbf{3.09}$}) & 3.22
% \\
{\MUSparse} + constant $\gamma$ & 6.60 (\textcolor{blue}{1.19}) & 14.64 (\textcolor{blue}{1.52}) & 96.51 (\textcolor{blue}{3.49})	& 87.30 (\textcolor{blue}{7.53}) & 2.53
\\

{\MUSparse} + linear growing $\gamma$  & 3.80 (\textcolor{blue}{1.61}) & 8.75 (\textcolor{blue}{4.37}) & 97.13 (\textcolor{blue}{2.87})	& 90.63 (\textcolor{blue}{3.79}) & 2.53
\\
{\MUSparse} + linear decaying $\gamma$ & \textbf{5.35} (\textcolor{blue}{\textbf{0.06}}) & \textbf{12.71} (\textcolor{blue}{\textbf{0.41}}) & \textbf{97.39} (\textcolor{blue}{\textbf{2.61}})	& {\textbf{91.26}} (\textcolor{blue}{\textbf{3.16}}) & 2.53
\\
\midrule
\bottomrule[1pt]
\end{tabular}
}
\vspace*{-4mm}
\end{wraptable}
\fi
\vspace{-5mm}
\begin{table}[htb!]
\centering
\caption{\footnotesize{{\MU} performance  comparison of using {\MUSparse} with different   sparsity schedulers of $\gamma$  in \eqref{eq: MUSparse} and using {\retrain}.  
The unlearning scenario is given by random data forgetting (10\% data points across all classes) on (ResNet-18, CIFAR-10).
%considering various unlearning scenarios on CIFAR10 dataset. The content format follows Tab.\,\ref{tab: overall_perfoamnce}.
%Bold numbers indicate the closest performance to {\retrain}, and
A performance gap  against \textcolor{blue}{{\retrain}} is provided 
%. The relative drop or improvement represented 
in (\textcolor{blue}{$\bullet$}).
}}
\label{tab: ablation_l1_scheduler}
\vspace*{1mm}
% \vspace*{0.1in} % Requirements, do not delete.
\resizebox{0.85\textwidth}{!}{
\begin{tabular}{c|c|c|c|c|c}
\toprule[1pt]
\midrule
  {\MU}& {\UA} & {{\MIAF}}& {{\RA}} & {{\TA}} & {{RTE} (min)} \\ 

% \cline{3-10}

% \midrule
% \rowcolor{Gray}
% \multicolumn{6}{c}{Class-wise forgetting} \\
% \midrule
% {\retrain} & \textcolor{blue}{100.00} & \textcolor{blue}{100.00} & \textcolor{blue}{100.00} & \textcolor{blue}{94.83} & 43.23
% \\
% {\MUSparse} + constant $\gamma$ & 100.00 (\textcolor{blue}{0.00}) & 100.00 (\textcolor{blue}{0.00}) & 91.69 (\textcolor{blue}{8.31})	& 87.3 (\textcolor{blue}{7.53}) & 2.61
% \\
% {\MUSparse} +   growing $\gamma$  & 100.00 (\textcolor{blue}{0.00}) & 100.00 (\textcolor{blue}{0.00}) & 94.43 (\textcolor{blue}{5.57})	& 88.43 (\textcolor{blue}{6.40}) & 2.61
% \\
% {\MUSparse} +   decaying $\gamma$ & 100.00 (\textcolor{blue}{0.00}) & 100.00 (\textcolor{blue}{0.00}) & \textbf{98.99} (\textcolor{blue}{\textbf{1.01}})	& \textbf{93.40} (\textcolor{blue}{\textbf{1.43}}) & 2.61
% \\
% \midrule
% \rowcolor{Gray}
% \multicolumn{6}{c}{Random data forgetting (all classes)} \\
\midrule
{\retrain} & \textcolor{blue}{5.41} & \textcolor{blue}{13.12} & \textcolor{blue}{100.00} & \textcolor{blue}{94.42} & 42.15
\\
%  \FT & $6.83$ (\textcolor{blue}{$1.42$}) & $14.97$ (\textcolor{blue}{$1.85$})& $96.61$ (\textcolor{blue}{$3.39$})& $90.13$ (\textcolor{blue}{$4.29$})  & 2.33 
% \\		
% \IU & $2.03$ (\textcolor{blue}{$3.38$})&  $5.07$ (\textcolor{blue}{$8.05$})& $98.26$ (\textcolor{blue}{$1.74$})& $\textbf{91.33}$ (\textcolor{blue}{$\textbf{3.09}$}) & 3.22
% \\
{\MUSparse} + constant $\gamma$ & 6.60 (\textcolor{blue}{1.19}) & 14.64 (\textcolor{blue}{1.52}) & 96.51 (\textcolor{blue}{3.49})	& 87.30 (\textcolor{blue}{7.12}) & 2.53
\\

{\MUSparse} + linear growing $\gamma$  & 3.80 (\textcolor{blue}{1.61}) & 8.75 (\textcolor{blue}{4.37}) & 97.13 (\textcolor{blue}{2.87})	& 90.63 (\textcolor{blue}{3.79}) & 2.53
\\
{\MUSparse} + linear decaying $\gamma$ & \textbf{5.35} (\textcolor{blue}{\textbf{0.06}}) & \textbf{12.71} (\textcolor{blue}{\textbf{0.41}}) & \textbf{97.39} (\textcolor{blue}{\textbf{2.61}})	& {\textbf{91.26}} (\textcolor{blue}{\textbf{3.16}}) & 2.53
\\
\midrule
\bottomrule[1pt]
\end{tabular}
}
\end{table}
In practice,  the unlearning performance could be sensitive to the choice of the sparse regularization parameter $\gamma$. To address this limitation, we propose the design of a sparse regularization scheduler. Specifically, we explore three schemes: (1) constant $\gamma$, (2) linearly growing $\gamma$ and (3) linearly decaying $\gamma$; see Sec.\,\ref{sec: exp_setup} for detailed implementations. Our empirical evaluation presented in \textbf{Tab.\,\ref{tab: ablation_l1_scheduler}} shows that the use of a linearly decreasing $\gamma$ scheduler outperforms   other schemes.  
This scheduler not only minimizes the gap in unlearning efficacy compared to {\retrain}, but also improves the preservation of {\RA} and {\TA} after unlearning. 
These findings suggest that it is advantageous to prioritize promoting sparsity during the early stages of unlearning and then gradually shift the focus towards enhancing fine-tuning accuracy on the remaining dataset $\Dr$.

% As Sec.\,\ref{sec: sparsity_MU_alg} pointed out, the downside of the $\ell_1$ regularization term is its suppression on {\RA} and {\TA} compared to {\FT} and {\retrain}. To facilitate this deficiency, we introduced a well-designed scheduler to $\gamma$, the parameter of the regularization term, and conducted a comprehensive ablation study on designing the scheduler. The results of unlearning performance with different parameter schedulers, constant, linearly increasing, and linearly decreasing schedulers, are presented in \textbf{Tab.\,\ref{tab: ablation_l1_scheduler}}.

% Directly applying a constant $\gamma$ to {\MUSparse} would either yield a low unlearning efficacy with lower $\gamma$ (for $\gamma = 0$, the method reduces to {\FT}) or degraded generalization performance under higher $\gamma$. In \textbf{Tab.\,\ref{tab: ablation_l1_scheduler}}, we have identified an optimal point that balances these metrics. Still, it cannot achieve as good performance as scheduled $\gamma$. The linearly increasing scheduler implies that the method initially emphasizes the unlearning term, then gradually shifts its focus towards sparsity. Conversely, the linearly decaying scheduler suggests that the focus initially lands on the sparsity, then gradually shifts to the unlearning term. The results reveal that the decaying scheduler outperforms all the others on both class-wise forgetting and random data forgetting, which aligned with the inspiration of the `prune first, then unlearn' paradigm. Therefore, the decaying scheduler will use in successive experiments except specified otherwise.
