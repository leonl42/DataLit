\section{Discussion}\label{sec:discussion}

Whether neural networks can generalise compositionally is often studied using artificial tasks that assume strictly \emph{local} interpretations of compositionality.
We argued that such interpretations exclude large parts of language and that to move towards human-like productive usage of language, tests are needed that assess how compositional models trained on \emph{natural data} are.\footnote{\citet{dupoux2018cognitive} makes a similar point for models of language acquisition, providing several concrete examples where using less than fully complex data proved problematic.}
We laid out reformulations of three compositional generalisation tests -- systematicity, substitutivity and overgeneralisation -- for NMT models trained on natural corpora, and assessed models trained on different amounts of data.
Our work provides an empirical contribution but also highlights vital hurdles to overcome when considering what it means for models of natural language to be compositional.
Below, we reflect on these hurdles and our results.

\paragraph{The proxy-to-meaning problem}
Compositionality is a property of the mapping between the form and meaning of an expression.
Since translation is a \emph{meaning-preserving} mapping from form in one language to form in another, it is an attractive task to evaluate compositionality: the translation of its sentence can be seen as a proxy to its meaning.
However, while expressions are assumed to have only one meaning, translation is a \emph{many-to-many} mapping: the same sentence can have multiple correct translations.
This does not only complicate evaluation -- MT systems are typically evaluated with BLEU because accuracy is not a suitable option -- it also raises questions about how compositional the desired behaviour of an MT model should be.
On the one hand, one could argue that for optimal generalisation, robustness, and accountability, we like models to behave systematically and consistently: we expect the translations of expressions to be independent of unrelated contextual changes that do not affect their meaning (e.g.\ swapping out a synonym in a nearby sentence).
Additionally, model performance could be improved if small changes do not introduce errors in unrelated parts of the translation.
On the other hand, non-compositional behaviour is not always incorrect -- it is one of the main arguments in our plea to test compositionality `in the wild' -- and we observe that indeed, not all non-compositional changes alter the correctness of the resulting translations.
Changing a translation from ``atleet'' (``athlete'') to ``sporter'' (``sportsman'') based on an unrelated word somewhat far away may not be (locally) compositional, but is it a problem?
And how do we separate such `harmful' mistakes from helpful ones?

\paragraph{The locality problem}
Inextricably linked to the proxy-to-meaning problem is the locality problem.
In our tests we see that \emph{small, local source changes} elicit \emph{global changes in translations}.
For instance, in our systematicity tests, changing one noun in a sentence elicited changes in the translation of a sentence that it was conjoined with.
In our substitutivity test, even synonyms that merely differed in spelling (e.g. ``doughnut'' and ``donut'') elicited changes to the remainder of the sentence.
This counters the idea of compositionality as a means of productively reusing language: if a phrase's translation depends on (unrelated) context that is not in its direct vicinity, this suggests that more evidence is required to acquire the translation of this phrase.

Tests involving synthetic data present the models with sentences in which maximally local behaviour is possible, and we argue that it is, therefore, also desirable.
Our experiments show that even in such setups, models do not translate in a local fashion: with varying degrees of correctness, they frequently change their translation when we slightly adapt the input.
On the one hand, this well-known \emph{volatility} \citep[see also][]{fadaee2020unreasonable} might be essential for coping with ambiguities for which meanings are context-dependent.
On the other hand, our manual analysis shows that the observed non-compositional behaviour does not reflect the incorporation of necessary contextual information and that oftentimes it is even altering the correctness of the translations.
Furthermore, this erratic behaviour highlights a lack of default reasoning, which can, in some cases, be problematic or even harmful, especially if faithfulness \citep{parthasarathi2021sometimes} or consistency is important.

In linguistics, it has been discussed how to extend the syntax and semantics such that `problem cases’ can be a part of a compositional language \citep{westerstaahl2002compositionality,pagin2010compositionality}.
In such formalisations, global information is used to disambiguate the problem cases, while other parts of the language are still treated locally.
In our models, global behaviour appears in situations where a local treatment would be perfectly suitable and where there is no clear evidence for ambiguity.
We follow \citet{baggio2021compositionality} in suggesting that we should learn from strategies employed by humans, who can assign compositional interpretations to expressions but can for some inputs also derive non-compositional meanings.
For \textit{human-like} linguistic generalisation, it is vital to investigate how models can represent both these types of processing, providing a locally compositional treatment when possible and deviating from that when needed.

\paragraph{Conclusion}
In conclusion, with this work, we contribute to the question of how compositional models trained on \emph{natural} data are, and we argue that MT is a suitable and relevant testing ground to ask this question.
Focusing on the balance between \emph{local} and \emph{global} forms of compositionality, we formulate three different compositionality tests and discuss the issues and considerations that come up when considering compositionality in the context of natural data.
Our tests indicate that models show both local and global processing, but not necessarily for the right samples.
Furthermore, they underscore the difficulty of separating helpful and harmful types of non-compositionality, stressing the need to rethink the evaluation of compositionality using natural language, where composing meaning is not as straightforward as doing the math.
