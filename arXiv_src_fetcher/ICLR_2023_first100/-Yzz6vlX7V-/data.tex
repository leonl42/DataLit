\subsection{Evaluation data}\label{sec:data}

While all our models are trained on fully natural data, for evaluation we use different types of data: synthetic, semi-natural and natural data.

\paragraph{Synthetic data}
For our \textbf{synthetic} evaluation data, we consider the data generated by \citet{lakretz2019emergence},
previously used to probe for hierarchical structure in neural language models.
This data consist of sentences with a fixed syntactic structure and diverse lexical material.
We extend the vocabulary and the templates used to generate the data and generate 3000 sentences for each of the resulting 10 templates (see Table~\ref{tab:synthetic_data}).

\paragraph{Semi-natural data}
In the synthetic data, we have full control over the sentence structure and lexical items, but the sentences are shorter (9 tokens vs 16 in \textsc{OPUS}) and simpler than typical in NMT data.
To obtain more complex yet plausible test sentences, we employ a data-driven approach to generate \textbf{semi-natural} data. 
Using the tree substitution grammar Double DOP \citep{vancranenburgh2016disc}, we obtain noun and verb phrases (NP, VP) whose structures frequently occur in \textsc{OPUS}.
We then embed these NPs and VPs in ten synthetic templates with 3000 samples each (see Table~\ref{tab:semi_natural}).
See Appendix~\ref{ap:ddop} for details on the data generation.

\paragraph{Natural data}
Lastly, we extract \textbf{natural} data directly from \textsc{OPUS}, as detailed in the subsections of the individual tests (\S\ref{sec:experiments}).
