\documentclass{article}

\usepackage{iclr2023_conference,times}

\pdfoutput=1

\usepackage{PRIMEarxiv}

\usepackage[utf8]{inputenc} % allow utf-8 input
\usepackage[T1]{fontenc}    % use 8-bit T1 fonts
\usepackage{hyperref}       % hyperlinks
\usepackage{url}            % simple URL typesetting
\usepackage{booktabs}       % professional-quality tables
\usepackage{amsfonts}       % blackboard math symbols
\usepackage{nicefrac}       % compact symbols for 1/2, etc.
\usepackage{microtype}      % microtypography
\usepackage{lipsum}
\usepackage{fancyhdr}       % header
\usepackage{graphicx}       % graphics
\graphicspath{{media/}}     % organize your images and other figures under media/ folder

\usepackage{hyperref}
\usepackage{url}
\usepackage{amsmath}
\usepackage{multirow}
\usepackage{cleveref}
\usepackage{adjustbox}
\usepackage{booktabs}
\usepackage{algorithm2e}
\usepackage{graphicx}

%Header
\pagestyle{fancy}
\thispagestyle{empty}
\rhead{ \textit{ }} 

% Update your Headers here
% \fancyhead[RE]{Firstauthor and Secondauthor} % Firstauthor et al. if more than 2 - must use \documentclass[twoside]{article}



  
%% Title
\title{Augmentation Backdoors}

\author{
  Joseph Rance \\
  University of Cambridge \\
  \texttt{jr879@cam.ac.uk} \\
  %% examples of more authors
   \And
  Yiren Zhao \\
  University of Cambridge \&\\ Imperial College London \\
  \texttt{yiren.zhao@cl.cam.ac.uk} \\
   \And
  Ilia Shumailov \\
  University of Oxford \\
  \texttt{ilia.shumailov@cl.cam.ac.uk} \\
   \And
  Robert Mullins \\
  University of Cambridge \\
  \texttt{Robert.Mullins@cl.cam.ac.uk} \\
  %% \AND
  %% Coauthor \\
  %% Affiliation \\
  %% Address \\
  %% \texttt{email} \\
  %% \And
  %% Coauthor \\
  %% Affiliation \\
  %% Address \\
  %% \texttt{email} \\
  %% \And
  %% Coauthor \\
  %% Affiliation \\
  %% Address \\
  %% \texttt{email} \\
}


\begin{document}
\maketitle
\setcitestyle{authoryear, aysep={,}}

A \gls{np} estimates a stochastic process implicitly defined with neural networks given a stream of data, rather than pre-specifying priors already known, such as Gaussian processes. An ideal \gls{np} would learn everything from data without any inductive biases, but in practice, we often restrict the class of stochastic processes for the ease of estimation. One such restriction is the use of a finite-dimensional latent variable accounting for the uncertainty in the functions drawn from \glspl{np}. Some recent works show that this can be improved with more ``data-driven’’ source of uncertainty such as bootstrapping. In this work, we take a different approach based on the martingale posterior, a recently developed alternative to Bayesian inference. For the martingale posterior, instead of specifying prior-likelihood pairs, a predictive distribution for future data is specified. Under specific conditions on the predictive distribution, it can be shown that the uncertainty in the generated future data actually corresponds to the uncertainty of the implicitly defined Bayesian posteriors. Based on this result, instead of assuming any form of the latent variables, we equip a \gls{np} with a predictive distribution implicitly defined with neural networks and use the corresponding martingale posteriors as the source of uncertainty. The resulting model, which we name as \gls{mpnp}, is demonstrated to outperform baselines on various tasks.

\section{Introduction}
\label{intro}

% very general intro on ML
Machine Learning (ML) gained a huge success in the last decades, becoming one of the most popular and studied branches of artificial intelligence \cite{jordan2015machine}. ML methods are widely used in many fields of research, with the aim of obtaining a general working learning rule from input data, namely a prediction function, to be used for future predictions of never-seen-before data.
Specifically, ML algorithms have been widely exploited in industrial processes, playing a relevant role in a wide range of applications: Industry 4.0 \cite{angelopoulos2019tacklin}, healthcare \cite{kourou2015machine}, transportation \cite{hamner2010predicting}, natural science \cite{yao2008quantitative}, social media \cite{balaji2021machine}, fraud detection \cite{awoyemi2017credit} and so on.

% General intro on AD
Anomaly detection represents an important and widely used ML task, broadly applied in various domains and applications where the issue of monitoring unexpected data behaviour is essential. This task defines the process of identifying anomalous data, i.e., data being characterised by a different behaviour with respect to other data distinguishing itself from the rest of the dataset. To date, there is no unanimously accepted definition but broadly speaking, an anomalous point, often named equivalently anomaly or outlier, is defined \textit{as an observation that deviates so much from other observations as to arouse suspicion that it was generated by a different mechanism} \cite{hawkins1980identification}. Anomaly detection is commonly tackled in many industrial scenarios, such as credit card fraud detection \cite{ghosh1994credit}, insurance fraud detection \cite{fawcett1999activity}, insider trading detection \cite{donoho2004early}, medical anomaly detection \cite{wong2003bayesian}. In these dynamic and often complex contexts, the problem of detecting anomalies is crucial in order to predict and avoid failures as well as to perform fault detection. In many industrial processes in fact, data-driven approaches for smart monitoring (for example predictive maintenance) have a key role, allowing to identify and isolate faults and to prevent future sudden failures. To solve this problem, anomaly detection represents an efficient solution.
Generally, in this scenario a great amount of collected data are available but, since labelling is an expensive and time consuming process, there is a lack of ground truth labels, undoubtedly stating whether or not a point is anomalous. The learning problem therefore is unsupervised and the algorithm can just blindly look at the structure of the dataset, without a clear definition of what is an anomaly from the user perspective. Therefore unsupervised algorithms can only detect samples that exhibit some general property different to the rest of the dataset, for example some approaches look for points far from the majority, or detect points living in low density areas.

% Anomalies are domain specific
Unfortunately, anomalies are strongly domain specific \cite{foorthuis2021nature}: as stated above, since no official definition is given, the concept of what an anomaly is entirely relies on the application in question. Specifically, it may happen that a set of data has different anomalies based on the given application domain and that the same data may be considered anomalous in one domain but normal in another \cite{sejr2021explainable}. For instance looking at data acquired by a measuring instrument, the manufacturer might define anomalies as events where the instrument has a faulty behaviour while the end-user might be more interested in events where the measured process behaves in a previously unseen way \cite{barbariol2020self}. As a direct consequence, training a domain specific anomaly detector would require a full set of labeled data to capture the user definition of anomaly. 

% full set of labels are too expensive
In real world applications, assigning labels to input data pose a considerable challenge to take into account \cite{zhu2009introduction}. In order to train reliable models, a large amount of labeled data is needed but, in practical scenarios, labeled examples are limited or often too expensive and time-consuming to collect, leading to a huge issue to face. Obtaining labels requires an often too expensive cost to take care of since the labeling procedure is usually carried on by a human domain expert who manually labels each point with a time-consuming and demanding routine. Moreover, by definition anomalous points are rare and difficult to spot, making the problem a difficult challenge to be solved. 
%as well as extremely unbalanced one. 
%\gas{Forse toglierei la roba dell'unbalanced qua o lo scriverei diversamente} 

% unsupervised AD is used in practice, but has no precise anomaly definition
Due to the difficulty of finding labeled points, in practical contexts anomaly detection is often treated as an unsupervised learning task. For the classical unsupervised anomaly detection problem, the purpose is to detect outliers with no use of labeled data based on the fact that normal data greatly outnumbers anomalous data, and anomalies are very different with respect to inliers. Unsupervised anomaly detection models are not tuned for the precise domain of application but are generally based on identifying rules based on specific data characteristics, such as density based algorithms, distance based methods etc. \cite{hochenbaum2017automatic, hill2010anomaly, knorr1998algorithms, breunig2000lof}. 
However recent literature \cite{das2016incorporating,sejr2021explainable} distinguishes the outliers to the anomalies: the first are the points highlighted by an unsupervised model, while the second are the ones the user actually sees as anomalous. As the unsupervised model is not directly tuned to the detection of the anomalies, the outliers might weakly correlate with them. 

% outliers do not coincide with anomalies
Therefore, running unsupervised anomaly detection algorithms may be risky and often misleading: as stated above anomalies are strongly domain dependent and as a direct consequence, an unsupervised detector might not identify anomalous data which should be considered as such, as well as could wrongly detect as anomalies points which are normal based on the context taken under consideration \cite{das2016incorporating}. Figure \ref{anomalyvsoutlier} presents a visual representation of the strong connection between anomalous data points and context domain. Specifically, the plot shows the two-dimensional projection of the \textit{vowels} dataset \cite{Rayana}. As it can be seen, anomalies are not defined just as data points lying far or in low-density regions, but they form a class with a specific pattern defined by domain-experts, making complicated for the automatic detector to correctly identify them.

\begin{figure}
    \centering
    \includegraphics[scale=0.6]{vowelsPCA.pdf}
    \caption{Projection of the \emph{vowels} dataset on 2 dimensions using Principal Component Analysis. Purple data points are normal data, yellow points are anomalous data. Two aspects are clearly visible: i) it is impossible to separate anomalies from normal data with only two features and ii) anomalies tend to form a different class and might be quite different from general outliers. Not all the data points lying in low density areas or far from the majority are defined as anomalies, but just the ones lying in a specific part of the space. 
    %As shown, anomalies do not tend to have any peculiar discrepancies with respect to normal points. On the contrary, they lie on the same portion of space occupied by normal points.
   As a consequence, in the considered scenario, identifying anomalies might be a challenging task for an unsupervised detector.}
    \label{anomalyvsoutlier}
\end{figure}


% presentazione di IF
Among the unsupervised models, a very popular anomaly detection algorithm is the Isolation Forest (IF) \cite{liu2008isolation, liu2012isolation}, which presents a very different approach w.r.t. the majority of models: instead of creating a profile for normal data, it explicitly tries to isolate anomalies. To do it, IF relies on two assumptions: anomalies are fewer in number and they have very different attributes compared to normal data. 


%\gas{Manca imho il discorso: AD pervasiva -> ci sono i sistemi di supporto alle decisioni -> abbiamo ora la possibilità in tante applicazioni di ottenere una taggatura non costosa dopo aver sviluppato un primo sistema di Anomaly Detection}

In Decision Support Systems (DSS) \cite{keen1980decision}, data streams are analysed in order to quickly extract strategic decisions on complex problems. Such process is monitored by users who frequently interact with the system and represent the actual decision maker of the whole process. In such framework, \approach represents an extremely appealing approach. Specifically, if a DSS is present, as a direct consequence, a user is already overseeing the process and inspecting data points: considering an unsupervised anomaly detection problem, inexpensive labels may be obtained in a fast way and using \approach the model may be inexpensively updated. 


% novelty
In this paper we describe a procedure able to tune the detector model on domain specific anomalies by interacting with a human expert. To perform the proposed tuning method, not every training data are presented and labeled but a subset is automatically selected so that the number of interactions between the system and the human is minimised. The core idea is to ask labels corresponding to the most significant points to reduce the labeling cost and at the same time to maximize the detection performance. As a direct consequence, the proposed procedure may be regarded as an Active Learning (AL) based model \cite{kumar2020active}. 

\begin{figure}
\centering
\usetikzlibrary{automata, arrows.meta, positioning}
 \begin{tikzpicture} [node distance = 4cm, on grid, auto]
 
\node (q0) [draw,lightgray, rounded corners, text width=1.5cm,yshift=-1.2cm, align =center] {\textcolor{black}{unlabeled \\dataset}};
\node (q1) [draw,lightgray, text width=1.5cm, rounded corners,above right = of q0,  yshift=-1.2cm,align =center] {\textcolor{black}{active \\selection}};
\node (q2) [draw,lightgray,rounded corners, text width=1.5cm, below right = of q1,yshift=1.2cm,align =center] {\textcolor{black}{labeled \\dataset}};
\node (q3) [draw,lightgray,rounded corners, text width=1.5cm, below left = of q2,yshift=1.2cm,align =center] {\textcolor{black}{learning \\model}};
 
\path [-stealth, thick]
    (q0) edge [lightgray, bend left]  (q1)
    (q1) edge [lightgray,bend left]  (q2)
    (q3) edge [lightgray,bend left]  (q0)
    (q2) edge [lightgray,bend left]  (q3);
\end{tikzpicture}
\caption{Active learning core structure. At each iteration a novel point is actively selected from the unlabeled set of data and the corresponding label is requested. Based on the received information, the model is modified.} \label{al}
\end{figure}

Indeed AL represents a training approach particularly suitable when labeled samples are too expensive or difficult to obtain. Specifically, AL is a particular ML algorithm based on a key idea: despite the shortage of labeled data, high accuracy results may be obtained if the training algorithm is allowed to choose the points to be labeled and learn from them \cite{settles1995active}. An AL algorithm asks an oracle to label the data considered most informative with an iterative approach. Doing so, since the queried points are directly selected by the learning algorithm, the amount of necessary labeled data is much smaller than that required for classical supervised ML approach. Figure \ref{al} shows the core structure of any AL algorithm: at each iteration the model is updated using the labelled dataset, and is allowed to ask for a new label in the unlabelled dataset. This process repeats until the model reaches sufficient performances or when the number of iterations reaches the maximum budget.

%\\In recent years, some active learning-based anomaly detection algorithms have been proposed.
%The idea of incorporating expert feedback in unsupervised anomaly detection algorithms aims at improving the achieved performance adding a relatively small computational cost. Active anomaly detection (AAD) algorithm \cite{das2016incorporating, das2017incorporating} proposes an active learning anomaly detection approach where points are ranked based on their anomalous behavior. The main goal is to maximize the number of true anomalies presented to the domain expert. An inclusion of AAD algorithm in the One Class Support Vector Machine (OCSVM) framework is tackled in \cite{lesouple2021incorporating}.
%Always using OCSVM, an expert feedback inclusion has been recently proposed %\cite{lesouple2021introduce}: to solve the problem, the paper combines together the $\mu$-SVM for the %labeled data and the OCSVM for the unlabeled set of data.

This paper focuses on the Isolation Forest detector, and suggests a strategy to tune it towards the user definition of anomaly. In this work the authors compare two AL query policies to ask the user new labels, and other two policies to update the internal structure with minimal computational effort. The goal is to increase the performance of the detector as much as possible, keeping very low both the labelling effort and the updating procedure. Moreover this method has two key advantages over the supervised and computationally expensive models: as it relies on an initial unsupervised training, it can start to work when there are no labels, but more importantly it can work even if instances from only one class are labelled. This is particularly useful when obtaining labels from the anomalous class is very uncommon or expensive.

%\gas{The rest of the papers is organized as follows: ...}
The rest of the paper is organized as follows. Initially, in Section \ref{rw} we outline the Isolation Forest in detail, and we indicate an existing active learning-based anomaly detection algorithm that will be used as a benchmark in thos work. Then, in Section \ref{pm} we illustrate the proposed model \approach: namely, we describe the strategies suggested to query the points as well as the approaches employed to update the model. In Section \ref{exp} we test \approach, comparing it with other models in relation to multiple real set of data. Finally, in Section \ref{conclusions} we draw conclusions for the present work.

%This paper compares two AL query strategies for anomaly detection using Isolation Forest. The algorithm key idea is to modify the internal structure of the unsupervised model based on a small amount of labeled data queried to the domain expert, trying to minimize the labelling effort but at the same time maximising the detection performance. Specifically, the presented model is based on the Isolation Forest model, i.e., the basic concept used to classify anomalies remains isolating data points from the rest of the dataset. Anyway, an innovative approach is used to query the labels and to update the model: novel information is achieved with the use of an active learning strategy by which the inner framework of the model is modified in order to match with the information obtained. Once the information is achieved, a user friendly iterative process allows to adapt the model by adjusting its structure on the basis of the novel input.
%The proposed method relies onto two distinguished yet essential issues: a process to select the most significant points and a tuning procedure to modify the structure of the model based on the information achieved. 
%This method has two key advantages over the supervised computationally expensive models: it is extremely computationally efficient and works even if instances from only one class are labelled.
%Based on the treated framework and on the data in hand, active learning algorithms are characterized by many possible query strategies \cite{settles1995active}. In this way, based on the considered purpose, the suitable query strategy may be selected and the corresponding most informative point may be labeled. 
%In the same way, we propose two possible approaches to address the selection of points to be labeled, producing two different query strategies with respect to both their benefits as well as their computational costs.
%\tommi{bisognerebbe scrivere che testiamo il nostro metodo su dataset reali e pubblichiamo il codice bla bla bla.}
%The importance of the algorithm...
%As we will see later the proposed model is based on...
%Section \ref{} ecc..
%list of the notation - simboli usati fare tabella





%\tommi{monitoring unexpected behaviour?} is essential. 
 %Even if, to date, there is no clear or official definition, broadly speaking, an anomalous point, also known as anomaly or outlier, is defined \textit{as an observation that deviates so much from other observations as to arouse suspicion that it was generated by a different mechanism} \cite{hawkins1980identification}. In general, an anomaly is identified as a variation from the norm, a data being characterised by a different behaviour with respect to other data that distinguishes itself from the rest of the dataset. Consequently, anomaly detection algorithms aim at detecting or identifying data that seem not to conduct themselves with a standard trend.
%\tommi{non sembra che parliamo di distribuzione gaussiana?}. 
%Anomaly detection techniques may be divided into three groups based on the available data in use: supervised, unsupervised and semi-supervised. 
%If the dataset contains labeled data, any technique for binary classification may be used, leading to highly accurate results. 
%Unfortunately, a critical challenge of anomaly detection is the lack of labeled data \cite{chandola2009anomaly}. Obtaining labels requires an often too expensive cost to take care of, since the labeling procedure is usually carried on by a human expert domain who, by hand labels each point with a time consuming and demanding routine. To solve this matter, semi-supervised or unsupervised techniques are employed. In the first scenario, a labeled portion of the original dataset is required, usually belonging to the most representative normal class, and, based on such labeled data, a model representing the normal behaviour is produced. \tommi{dici?} Generally, in fact, labeling normal data requires less efforts: normal points tend to have a common and more static behavior, making it easier to  be identified.
%For the classical unsupervised anomaly detection problem, the purpose is to isolate outliers with no use of labeled data based on the fact that normal data greatly outnumbers anomalous data. As a rule, unsupervised anomaly detection models are not tuned for the precise domain of application but are generally based on identifying rules based on specific data characteristics. 
%Based on the context and on the problem in question, different popular anomaly detection techniques exist. 
%\tommi{tornarci} A large class of unsupervised anomaly detectors are those based on statistical methods. These techniques produce a statistical distribution based on the given set of data and classify points according to where data falls: points that are not consistent or that simply lie in the tails of the computed distribution are consider anomalies \cite{hochenbaum2017automatic, hill2010anomaly}. 
%Some anomaly detection methods are based on the traditional classification problem. Based on the classical Support Vector Machine framework several methods have been proposed: the one-class support vector machine \cite{scholkopf1999support} computes an hyperplane that separates the data points from the origin and at the same time maximizes its distance with respect to the origin; the support vector data description \cite{tax2004support} looks for the smallest hypersphere containing the considered dataset. Distance-based \cite{knorr1998algorithms} and density-based \cite{breunig2000lof} outlier detection methods try to solve the problem by finding a profile for normal data based on inner data characteristics: the first uses the average distance of points, since, by nature, outliers have a higher value with respect to normal points; the latter is based on density area, due to the fact that outliers tend to stay in low density areas compared to normal points, which usually assemble in the same area. 

% tommi: andrei a capo in modo da evidenziare IF che è il metodo che andremmo ad analizzare
%A very popular anomaly detection algorithm is the Isolation Forest algorithm \cite{liu2008isolation, liu2012isolation}, which presents a brand-new approach: instead of creating a profile for normal data, it explicitly tries to isolate anomalies. To do it, Isolation Forest relies on two anomaly inner characteristics: they are fewer in number and they have very different attributes compared to normal data. 

%To cope with the lack of labeled data, in recent years some active learning-based anomaly detection algorithms have been proposed.
%The idea of incorporating expert feedback in unsupervised anomaly detection algorithms aims at improving the achieved performance adding a relatively small computational cost. Active anomaly detection (AAD) algorithm \cite{das2016incorporating, das2017incorporating} propose an active learning anomaly detection approach where points are ranked based on their anomalous behavior. The main goal is to maximize the number of true anomalies presented to the domain expert. An inclusion of AAD algorithm in the One Class Support Vector Machine (OCSVM) framework is tackled in \cite{lesouple2021incorporating}.
%Always using OCSVM, an expert feedback inclusion has been recently proposed \cite{lesouple2021introduce}. To solve the problem, the paper combines together the $\mu$-SVM for the labeled data and the OCSVM for the unlabeled set of data. 
%\cite{vercruyssen2018semi}. 


%This paper presents a novel active learning strategy for anomaly detection using Isolation Forest. The algorithm key idea is to modify the internal structure of the unsupervised model based on a small amount of labeled data queried to the domain expert, trying to minimize the labelling effort but maximising the detection performance. Specifically, the presented model is based on the Isolation Forest model, i.e., the basic concept used to classify anomalies remains isolating data points from the rest of the dataset. Anyway, an innovative approach is used to query the labels and to update the model: novel information is achieved with the use of an active learning strategy by which the inner framework of the model is modified in order to match with the information obtained. Once the information is achieved, a user friendly iterative process allows to adapt the model by adjusting its structure on the basis of the novel input.
%Based on the treated framework and on the data in hand, active learning algorithms are characterized by many possible query strategies \cite{settles1995active}. In this way, based on the considered purpose, the suitable query strategy may be selected and the corresponding most informative point may be labeled. 
%In the same way, we propose two possible approaches to address the selection of points to be labeled, producing two different query strategies with respect to both their benefits as well as their computational costs.
%\tommi{bisognerebbe scrivere che testiamo il nostro metodo su dataset reali e pubblichiamo il codice bla bla bla.}
%The importance of the algorithm...
%As we will see later the proposed model is based on...
%Section \ref{} ecc..
%list of the notation - simboli usati fare tabella

%
% vars
%
\newcommand{\X}{\ensuremath{\mathbf{X}}}
\newcommand{\B}{\ensuremath{\mathbf{B}}}
\newcommand{\Y}{\ensuremath{\mathbf{Y}}}
\newcommand{\Z}{\ensuremath{\mathbf{Z}}}
\newcommand{\Q}{\ensuremath{\mathbf{Q}}}
\newcommand{\Cb}{\ensuremath{\mathbf{C}}}
\newcommand{\V}{\ensuremath{\mathbf{V}}}
\newcommand{\A}{\ensuremath{\mathbf{A}}}
\newcommand{\x}{\ensuremath{\boldsymbol{x}}}
\newcommand{\bo}{\ensuremath{\boldsymbol{b}}}
\newcommand{\y}{\ensuremath{\boldsymbol{y}}}
\newcommand{\z}{\ensuremath{\boldsymbol{z}}}
\newcommand{\e}{\ensuremath{\boldsymbol{e}}}
\newcommand{\PC}{\ensuremath{\mathcal{C}}}
\newcommand{\PCaug}{\ensuremath{\mathcal{A}}}
\newcommand{\LC}{\ensuremath{\mathcal{L}}}
\newcommand{\WMCC}{\ensuremath{\mathcal{C}}}
\newcommand{\PSDD}{\ensuremath{\mathcal{C}}}
\newcommand{\AOMDD}{\ensuremath{\mathcal{C}}}
\newcommand{\PDG}{\ensuremath{\mathcal{C}}}
\newcommand{\primep}{\ensuremath{\mathsf{P}}}
\newcommand{\sub}{\ensuremath{\mathsf{S}}}
\newcommand{\CNET}{\ensuremath{\mathcal{C}}}
\newcommand{\C}{\ensuremath{\mathcal{C}}}
\newcommand{\CLT}{\ensuremath{\mathcal{T}}}
\newcommand{\model}{\ensuremath{\mathsf{m}}}
\newcommand{\modelfam}{\ensuremath{\mathcal{M}}}
\newcommand{\val}{\ensuremath{\mathsf{val}}}
\newcommand{\supp}{\ensuremath{\mathsf{supp}}}
\newcommand{\structure}{\ensuremath{\mathcal{G}}}
\newcommand{\tree}{\ensuremath{\mathcal{T}}}
\newcommand{\graph}{\ensuremath{\mathcal{G}}}
\newcommand{\params}{\ensuremath{\boldsymbol{\theta}}}
\newcommand{\sumparams}{\ensuremath{\boldsymbol{\theta}_{\mathsf{S}}}}
% \newcommand{\sumparams}{\ensuremath{\boldsymbol{\omega}}}
\newcommand{\leafparams}{\ensuremath{\boldsymbol{\theta}_{\mathsf{L}}}}
%\newcommand{\leafparams}{\ensuremath{\boldsymbol{\lambda}}}
\newcommand{\scope}{\ensuremath{{\phi}}}
\newcommand{\mixture}{\ensuremath{\mathcal{M}}}
\newcommand{\vtree}{\ensuremath{\mathcal{V}}}
% \newcommand{\p}{\ensuremath{\mathsf{Pr}}}
\newcommand{\p}{{p}}
\newcommand{\q}{{q}}
\newcommand{\m}{{m}}
\newcommand{\ch}{\ensuremath{\mathsf{in}}}
\newcommand{\pa}{\ensuremath{\mathsf{pa}}}
\newcommand{\leftn}{\ensuremath{\mathsf{L}}}
\newcommand{\rightn}{\ensuremath{\mathsf{R}}}% \newcommand{\f}{\ensuremath{f}}
\newcommand{\f}{\ensuremath{\Delta}}
\newcommand{\vtreenode}{\ensuremath{\mathcal{v}}}
\newcommand{\Ent}{\ensuremath{\mathbb{H}}}
\newcommand{\Mom}{\ensuremath{\mathbb{M}}}
\newcommand{\Ex}{\ensuremath{\mathbb{E}}}
\newcommand{\interval}{\ensuremath{\mathcal{I}}}
\newcommand{\Le}{\ensuremath{\mathsf{L}}}
\newcommand{\Ri}{\ensuremath{\mathsf{R}}}

\newcommand{\poly}[1]{\texttt{poly}(#1)}

%
% misc, writing
%
\newcommand{\eg}{e.g.,\ }
\newcommand{\wrt}{w.r.t.\ }
\newcommand{\ie}{i.e.,\ }
\newcommand{\cf}{cf.\ }
\newcommand{\aka}{a.k.a.\ }
\newcommand{\iid}{i.i.d.\ }

%
% symbols, concepts
\newcommand{\prob}{\ensuremath{\mathsf{Pr}}}
\newcommand{\uprob}{\ensuremath{{\bwidetilde{\mathsf{P}}\mathsf{r}}}}

%
% operators
\newcommand{\vars}{\ensuremath{\mathsf{vars}}}
\newcommand{\id}[1]{\llbracket{#1}\rrbracket}
\newcommand{\neigh}{\ensuremath{\mathsf{neigh}}}

\newcommand{\mathL}{\mathcal{L}}
\newcommand{\mathP}{\mathcal{P}}
% \newcommand{\E}{\mathcal{E}}
\newcommand{\data}{\mathcal{D}}
\newcommand{\F}{\mathcal{F}}
\newcommand{\mi}{\text{MI}}
\newcommand{\true}[0]{\texttt{true}}
\newcommand{\false}[0]{\texttt{false}}
\newcommand{\oplusl}{\operatornamewithlimits{\oplus}}
\newcommand{\otimesl}{\operatornamewithlimits{\otimes}}
\newcommand{\landl}{\operatornamewithlimits{\land}}

%
\newcommand{\flow}{\mathrm{F}}
\newcommand{\expflow}{\mathrm{EF}}
\newcommand{\context}{\gamma}
\newcommand{\pseudocount}{\alpha}
\newcommand{\bigO}{\mathcal{O}}
\newcommand{\bigOmega}{\Omega}
\newcommand{\bigTheta}{\Theta}
\newcommand{\indicator}[1]{\mathbbm{1}[#1]}
\newcommand{\weight}[2]{\mathtt{weight}(#1,#2)}
\newcommand{\entropy}{\mathtt{ENT}}
\newcommand{\expectation}{\mathbb{E}}
\newcommand{\LL}{\mathtt{LL}}
\newcommand{\lit}{\mathtt{Lit}}

\newcommand{\pluseq}{\mathrel{+}=}
\newcommand{\minuseq}{\mathrel{-}=}

\newcommand{\w}{\mathbf{w}}
\newcommand{\W}{\mathbf{W}}

\newcommand{\bp}{\mathbf{p}}

\newcommand{\SL}{\mathrm{L}^{\mathrm{s}}}
\newcommand{\WSL}{\mathrm{L}^{\mathrm{ws}}}
\newcommand{\MC}{\mathtt{MC}}

%
% queries
\newcommand{\nlquery}[2]{\begin{minipage}{.07\textwidth}$q_{#1}:$\end{minipage}\begin{minipage}{.88\textwidth}\raggedright\emph{#2}\end{minipage}}

\newcommand{\SPLIT}{\mathtt{SPLIT}}

\newcommand{\given}{\vert}
\section{Related work}
\label{sec:relatedwork}

We now discuss how our results relate to phenomena that have been observed or proven in the literature before.

\paragraph{Robust and non-robust useful features}
In the words of \citet{ilyas19, springer21}, for
directed attacks, all robust features become less useful, but adversarial
training uses robust features more.  In the small sample-size regime
$n<d-1$ in particular, robust learning assigns so much weight
on the robust (possibly non-useful) features, that the signal in the non-robust
features is drowned. This leads to an unavoidable and large increase
in standard error that dominates the decrease in susceptibility and
hence ultimately leads to an increase of the robust error.

\paragraph{Small sample size and robustness}
A direct consequence of Theorem~\ref{thm:linlinf} is that in order to
achieve the same robust error as standard training, adversarial
training requires more samples. This statement might remind the reader
of sample complexity results for robust generalization in
\citet{schmidt18, Yin19, Khim18}. While those results compare sample
complexity bounds for standard vs. robust error, our theorem
statement compares two algorithms, standard vs. adversarial training,
with respect to the robust error.


\paragraph{Trade-off between standard and robust error} 

Many papers observed that even though adversarial training decreases robust error compared to standard training, it may lead
to an increase in standard test error \cite{madry18, zhang19}.  
%This
%trade-off phenomenon has been theoretically studied under different
%assumptions: \cite{nakkiran19, madry18} show that small model capacity
%could result in the poor generalization performance. 
For example, \citet{tsipras19, zhang19, javanmard20, dobriban20, chen20} study settings where the Bayes optimal robust classifier is not equal to the Bayes optimal (standard)
classifier (i.e. the perturbations are inconsistent or the dataset is non-separable).
%% However, these settings inherently assume inconsistent3
%% $\ell_p$-ball perturbations, i.e. where perturbations may also change
%% the ground truth label.
%% By definition however, adversarial attacks
%% are “only useful” when they do not change the class for the ground
%% truth classifier such as the human (like for traffic sign stickers or
%% imperceptible perturbation).  Also, neural networks are known to be
%% complex enough to even fit random labels and inputs.
\cite{raghunathan20} study consistent perturbations, as in our paper,
and prove that for small sample size, fitting adversarial
examples can increase standard error even in the absence of
noise. In contrast to aforementioned works, which do not refute that
adversarial training decreases robust error, we prove that for
\nameofattacks perturbations, in the small sample regime adversarial training may also increase \emph{robust error}.

%As a reaction to the trade-off phenomenon,
\paragraph{Mitigation of the trade-off} 
A long line of work has proposed procedures to 
mitigate the trade-off phenomenon.  For example \citet{alayrac19,
  Carmon19, zhai20, raghunathan20} study robust self training, which
leverages a large set of unlabelled data, while \citet{lee20, lamb19,
  xu20} use data augmentation by interpolation. \citet{Ding20,
  balaji19, Cheng20} on the other hand propose to use adaptive
perturbation budgets $\epstrain$ that vary across inputs. 
%Their
%approach is closest to the story of this paper, although it is
%motivated by inconsistent perturbations. 
%Some of these mitigation approaches also result in an increase in robust accuracy against
%$\ell_p$-perturbations and hence might be candidates to successfully increase robust accuracy for \nameofattacks perturbations in the small sample size regime.
Our intuition from the theoretical analysis suggests that the standard
mitigation procedures for imperceptible perturbations may not work for
perceptible \nameofattacks, because all relevant features are non-robust.
%% robustness inherently requires
%% using the signal component in the data less. 
%% less weight on
%% the signal component in the data.
We leave a thorough empirical study
as interesting future work.

%\paragraph{Perceptible adversarial perturbations} \jc{doubting}
%% Multiple works have considered perceptible perturbations in the form of adversarial perceptible perturbations or common image corruptions. For example stickers on traffic signs \cite{Eykholt18}, adversarially coloured glasses or face-stickers \cite{Wu20} and adversarial watermarks \cite{Hayes18}. The common defences against these attacks are data-augmentation techniques such as adversarial training and variants thereof. 

 %adaptive perturbation budgets $\epstrain$ are proposed.
%% In many cases we observe that adversarial training, while gaining some robust accuracy, can cause a drop in clean test accuracy \cite{madry18, zhang19}. In \cite{tsipras19, zhang19} they advocate that the drop in standard accuracy stems from a general trade-off between the standard and robust accuracy objectives. In contrast, \cite{yang20} shows that there exist solutions in commonly used function spaces that achieve both objectives on common datasets. We show that adversarial training with signal-attacking perturbations can also hurt robust generalization. \jc{would need to get this in there aboout sample complexity} In \cite{schmidt18}, they show that finding robust classifiers may require more samples whereas in \cite{nakkiran19, madry18} they find that robustness may require a larger model capacity.

%% Recent work has tried to come up with different approaches to mitigate
%% the drop in standard accuracy. For example \cite{alayrac19, Carmon19, zhai20, raghunathan20} propose robust self training, which leverages unlabelled data, the works \cite{lee20, lamb19, xu20} use data augmentation by interpolation and in the works \cite{Ding20, balaji19, Cheng20} adaptive perturbation budgets $\epstrain$ are proposed. We expect the methods that are able to increase standard accuracy or fully mitigate the drop, to also help adversarial training in the small sample size regime. In particular, robust self-training may help, but is because of the large unlabelled data set recourse and time expensive. 
%% In \cite{ilyas19, springer21} they characterize the existence of robust useful features, useful features and non-useful features. Our work aligns with this perspective. In the low sample regime, adversarial training causes the classifier to rely heavier on non-useful features to classify the dataset. This in contrast to the high sample regime where adversarial training gains robust accuracy by causing the classifier to comparatively rely less on non-robust useful features instead then more on non-useful features. 

%% In \cite{bhattacharjee20} they note that adversarial training can hurt robust generalisation one sample at the time. We characterize the phenomenon and give statistical bounds. In \cite{Bornschein20, Brigato21} they consider deep learning in the low sample regime. Both works find that smaller models may outperform larger commonly used models in the low sample regime. Our results comply with the results published there, but we note the high dependence on the specific dataset.



%% \paragraph{Non-perfect adversarial training}
%% On the other hand, recent empirical and theoretical papers have noted that test
%% robust accuracy might decrease with robust training accuracy and hence regularizing model complexity could benefit robust generalization \cite{rice20, Sanyal20, Donhauser}, a phenomenon referred to as robust overfitting. 
%% \fy{whats this adversarial feature overfitting - are there other overfitting things}
%% \fy{dunno if that stuff on non-robust features is any relevant}

%% Here we have perfectly certified models and still have that it hurts.
%% \fy{certification is in some sense uber perfect fitting attack models - should be even worse than just interpolating on adversarial examples?}

%% Attack-model overfitting
%% We want to emphasize that usually

%% Recent work has tried to come up with different approaches to mitigate
%% the drop in standard accuracy 
%% some of which show final improvement in ....
%% We expect them to work well 

% Section 4.1: borrow justification from MC-BRL (and two similar works) and MoREL

\section{Methodology}

In this work, we propose a novel offline MBRL algorithm based on Bayes Adaptive MCTS. The core challenge is to design a Bayes Adaptive planning method that is efficient in large stochastic MDPs. In this case, we propose Continuous BAMCP in Section \ref{ContBAMCP}, which can be applied to continuous control tasks with high dynamics stochasticity. Then, in Section \ref{SBPI}, we present a search-based policy iteration framework, where the search results are distilled into policy and value networks for policy improvement and policy evaluation, respectively, at each iteration. In this way, we integrate offline MBRL with Bayes Adaptive MCTS. Both components require the use of an ensemble of world models for practical implementation and uncertainty quantification, which is detailed in Section \ref{DeepEns}.


\subsection{The Key Role of Deep Ensembles} \label{DeepEns}

% why bayes adaptive
% why using ensembles

Offline MBRL methods estimate world models $\mathcal{M}_\theta$ from a static dataset $\mathcal{D}_\mu$, which would inevitably induce epistemic uncertainty about the identity of the real MDP $\mathcal{M}^*$. Specifically, there could be various potential MDPs that behave identically on the limited set of states and actions in $\mathcal{D}_\mu$, but their dynamics and reward functions may differ, especially on out-of-sample states and actions. Thus, we are actually dealing with a distribution of world models that follow a prior distribution $b_0(\theta) \triangleq P(\mathcal{M}_\theta | \mathcal{D}_\mu)$. As introduced in Section \ref{back}, Bayesian RL based on BAMDP is a principled framework for handling model uncertainty by explicitly including the belief over the models in its state representation.
% by transforming the uncertainty about the world model into certainty about the current state inside an augmented state space $\mathcal{S} \times \mathcal{B}$
Essentially, the belief is updated with experience, providing a measure of how the models' uncertainty has changed since the beginning of the episode. As a result, the agent can adjust its behavior upon receiving new information that reduces the epistemic uncertainty. Such an adaptive policy is necessary to act optimally in offline RL, as demonstrated in (\cite{DBLP:conf/icml/GhoshAAL22}).

The idea of deep ensembles (\cite{DBLP:conf/nips/Lakshminarayanan17}) is to train multiple deep neural networks as approximations of a function, each using a different weight initialization and optimized with a different mini-batch sequence. For offline MBRL, we can learn an ensemble of dynamics models $\{\mathcal{P}_{\theta}^1, \cdots, \mathcal{P}_{\theta}^K\}$ and reward models $\{\mathcal{R}_{\theta}^1, \cdots, \mathcal{R}_{\theta}^K\}$\footnote{In some MBRL scenarios, a certain reward function is available, for instance, as defined by domain experts. Otherwise, the reward and dynamics function $(\mathcal{R}_\theta^i, \mathcal{P}_\theta^i)$ are usually trained as a unified probabilistic model $\mathcal{N}(\mu_{\theta}^i, \sigma_{\theta}^i)$, since the reward $r$ can be viewed as an element of the next state $s'$.} from the dataset $\mathcal{D}_\mu$ by minimizing the following supervised learning loss: ($i = 1, \cdots, K$)
\begin{equation}
    \mathcal{L}(\mathcal{P}_{\theta}^i) = -\mathbb{E}_{(s, a, s') \sim \mathcal{D}_\mu} \left[\log \mathcal{P}_{\theta}^i(s'|s, a)\right],\ \mathcal{L}(\mathcal{R}_{\theta}^i) = -\mathbb{E}_{(s, a, r) \sim \mathcal{D}_\mu} \left[\log \mathcal{R}_{\theta}^i(r|s, a)\right]
\end{equation}
$\{(\mathcal{P}_{\theta}^i, \mathcal{R}_{\theta}^i)_{i=1}^K\}$ can be viewed as a set of independent and identically distributed (IID) samples from the prior $P(\mathcal{M}_\theta | \mathcal{D}_\mu)$ and constitute a finite approximation of the space of world models. With such an ensemble, the belief over the world models can be converted to a mass function over a set of $K$ items, where the $i$-th element denotes the probability of being in the MDP $(\mathcal{P}_{\theta}^i, \mathcal{R}_{\theta}^i)$. In this case, a reasonable prior distribution is $b_0(\theta) = [1/K, \cdots, 1/K]$, since these models are IID prior samples. After receiving a transition $(s, a, r, s')$, the belief can be updated as follows:
\begin{equation} \label{b'}
    b'(\theta)(i) = x^i / \sum_{j=1}^K x^j,\ x^i = b(\theta)(i)\mathcal{P}_\theta^i(s' | s, a) \mathcal{R}_\theta^i(r | s, a)
\end{equation}
This update requires a single inference from each ensemble member, but can be parallelized for computational efficiency. Equation (\ref{b'}) is a practical implementation of the Bayesian posterior update based on deep ensembles, and the simplification of the definition of $b(\theta)$ makes the transitions in Bayesian RL (i.e., Equation (\ref{bamdp-mdp})) both practical and efficient.

% justification for using such a finite approximation

The ensemble can also be used for uncertainty quantification. As aforementioned, our algorithm relies on thorough search on the learned world models. Without any constraints on the search process, the learned policy may overfit to an inaccurate model (by overestimating the expected return) and fail in the true MDP.
Although the agent could adapt its belief and follow more reliable ensemble members in the Bayesian RL framework, there could be regions in the state-action space where none of the members generalize well, as they are all learned from a static offline dataset.
A typical solution is to construct a P-MDP (see Section \ref{rel_works}), which lower-bounds the true MDP and discourages the policy from regions where there is large discrepancy between the true and learned world models. 
% This type of methods provides a theoretical guarantee of improvement over cloning the behavior policy, and can be implemented by using a reward function penalized by the uncertainty in the world model. 
We construct the P-MDP by modifying each reward estimation $r$ into $\tilde{r}$: ($\mu_\theta(s, a) = \sum_{i=1}^K b(\theta)(i) \mu_{\theta}^i(s, a)$)
\begin{equation} \label{p-rwd}
    \Tilde{r}(s, a, r, b(\theta)) = r - \lambda \sqrt{\sum_{i=1}^K b(\theta)(i) (\sigma_{\theta}^i(s, a)^2 + \mu_{\theta}^i(s, a)^2) - \mu_\theta(s, a)^2}
\end{equation}
The reward penalty is weighted by a hyperparameter $\lambda > 0$ and corresponds to the standard deviation (std) of the mixture of Gaussian dynamics models, where $\mu_{\theta}^i$ and $\sigma_{\theta}^i$ are the mean and std from the ensemble member $i$. 
% We choose to use \(b(\theta)\) rather than \(b'(\theta)\) in Eq. (\ref{p-rwd}) since \(b(\theta)\) provides the mixture weights of the ensemble members (i.e., $(\mu_\theta^i, \sigma_\theta^i)_{i=1}^K$) when determining $r$ and $s'$. 
This penalty design combines epistemic and aleatoric model uncertainty and has been shown to be successful at capturing errors in predicted dynamics (\cite{DBLP:conf/iclr/LuBPOR22})\footnote{In the original literature (\cite{DBLP:conf/nips/Lakshminarayanan17, DBLP:conf/iclr/LuBPOR22}), the ensemble is treated as a uniformly-weighted mixture model, i.e., $b(\theta)(i) = 1/K,\ (i=1, \cdots K)$, since their belief is not adaptive. Equation (\ref{p-rwd}) fits into the Bayesian RL framework by adapting the belief $b(\theta)$, which is part of our novelty.}.

% justify that the idea of P-MDP and its theoretical derivation can still be applied when the main algorithm component is MCTS

% possible replacement for deep ensembles

\begin{algorithm}[t]
  \caption{Continuous BAMCP}
  \label{alg:2}
  \begin{multicols}{2}
    \begin{algorithmic}
      \State \textbf{Input:} $\pi$, $V$, $E$, $d_{\text{max}}$, $\gamma$, $\alpha$, $\beta$, $\mathcal{P}_\theta^{1:K}$, $\mathcal{R}_\theta^{1:K}$
      \Procedure{Search}{$(s, h), b(\theta)$}
        \For{$e=1 \cdots E$}
        \State \textproc{Simulate}($(s, h), b(\theta), d_{\text{max}}$)
        \EndFor
        \State $v_{\text{ret}} = \sum_{a \in C((s, h))} \frac{N((s, h), a)}{N((s, h))} Q((s, h), a)$
        \State {\color{blue} \Return $\pi_{\text{ret}}, v_{\text{ret}}$} 
      \EndProcedure
      \Procedure{Simulate}{$(s, h), b(\theta), d$}
        \IfThen{$d==0$}{{\color{blue}\Return $V((s, h))$}}
        \State $a \leftarrow \textproc{ActionPW}((s, h))$
        \State $r, s', b'(\theta) \leftarrow \textproc{StatePW}((s, h), b(\theta), a)$
        \State $N((s, h)) \mathrel{{+}{=}} 1, N((s, h), a) \mathrel{{+}{=}} 1$
        \If{{\color{blue}$N((s, h)) > 1$}}
        \State {\small $R \leftarrow \textproc{Simulate}((s', hars'), b'(\theta), d-1)$}
        \Else
        \State {\color{blue}{\small $R \leftarrow V((s', hars'))$}}
        \EndIf
        \State Access $\tilde{r}$ or calculate $\tilde{r}$ using Eq. (\ref{p-rwd})
        \State $R \leftarrow \tilde{r} + \gamma R$, cache $\tilde{r}$
        \State $Q((s, h), a) \mathrel{{+}{=}} \frac{R - Q((s, h), a)}{N((s, h), a)}$
        \State \Return $R$ 
      \EndProcedure
    \end{algorithmic}
    \columnbreak
    \begin{algorithmic}
      \Procedure{ActionPW}{$(s, h)$}
        \IfThen{first visit}{$C((s, h)) \leftarrow \emptyset$}
        \If{{\color{blue}$\left \lfloor {N((s, h))^\alpha}\right \rfloor \geq |C((s, h))|$}}
        \State $a \sim \pi(\cdot|(s, h))$
        \State $C((s, h)) \leftarrow C((s, h)) \cup \{a\}$
        \State $N((s, h), a), Q((s, h), a) \leftarrow 0, 0$
        \Else
        \State {\color{blue}$a \leftarrow \arg \max_{x \in C((s, h))} \tilde{Q}((s, h), x)$}
        \EndIf
        \State \Return $a$
      \EndProcedure
      \Procedure{StatePW}{$(s, h), b(\theta), a$}
        \IfThen{first visit}{$C((s, h), a) \leftarrow \emptyset$}
        \If{{\color{blue}$\left \lfloor {N((s, h), a)^\beta}\right \rfloor \geq |C((s, h), a)|$}}
        \State $r \sim \sum_{i=1}^K b(\theta)(i) \mathcal{R}_\theta^i(\cdot|s, a)$
        \State $s' \sim \sum_{i=1}^K b(\theta)(i) \mathcal{P}_\theta^i(\cdot|s, a)$
        \State Update $b(\theta)$ to $b'(\theta)$ using Eq. (\ref{b'})
        \State {\scriptsize $C((s, h), a) \leftarrow C((s, h), a) \cup \{(r, s', b'(\theta))\}$}
        \State $N((s', hars')) \leftarrow 0$
        \State \Return $r, s', b'(\theta)$
        \EndIf
        \State {\color{blue}\Return the least visited node in {\small $C((s, h), a)$}}
      \EndProcedure
    \end{algorithmic}
  \end{multicols}
\end{algorithm}

\subsection{Bayes Adaptive MCTS in Continuous State and Action Spaces} \label{ContBAMCP}

% decision-making at the node (s, b) via planning

% according to MoREL, any online methods can be applied on top of the P-MDP

% less desired than running more simulations with worse estimates

BAMCP (\cite{DBLP:journals/jair/GuezSD13}) has been successful in solving large-scale BAMDPs, as detailed in Appendix \ref{bamcp}, but it is limited to scenarios with discrete state and action spaces.
In this subsection, we introduce an online planning method to approximate the Bayes-optimal policy at a decision point $(s, h)$. ($h$ denotes the transition history that ends at $s$.) In particular, this method can be used to solve BAMDPs with continuous states and actions. 
% Checking Algorithm \ref{alg:1}, we can see that, when the action or observation space is continuous, the tree generation degenerates and does not extend beyond a single layer of nodes since each simulation procedure would produce a new branch from the root node. 
Specifically, we adopt double progressive widening (DPW, \cite{DBLP:conf/lion/CouetouxHSTB11, DBLP:conf/pkdd/AugerCT13}), which maintains a finite list of chance nodes to be searched at each decision point and incrementally adds a new chance node to the list based on the visitation counts. For a node $(s, h)$, a new action $a$ is sampled (with the current policy) and added to its children set $C((s, h))$, if $\left \lfloor {N((s, h))^\alpha}\right \rfloor \geq |C((s, h))|$, where $\alpha \in (0, 1)$ is a hyperparameter that controls the growth rate and $N$ denotes the visitation counts. Otherwise, an action will be sampled from $C((s, h))$ according to the UCT (\cite{DBLP:conf/ecml/KocsisS06}) rule. Similarly, to handle the infinitely many possible transitions, a new next state \(s'\) is added to the children set \(C((s, h), a)\) only if \(\left \lfloor {N((s, h), a)^\beta}\right \rfloor \geq |C((s, h), a)|\) (\(\beta \in (0, 1)\)). Otherwise, the least visited child in \(C((s, h), a)\) will be selected as the next state. With DPW, the sets of possible actions or next states for search are all finite, allowing deep search as in discrete scenarios, and the more promising states and actions (with higher $N$) have more subsequent branches, thereby reducing corresponding estimation uncertainty. 
% However, it also introduces bias; for example, the state transition \(\tilde{\mathcal{P}}_\theta(s'|s, a)\) differs from the true distribution \(\mathcal{P}_\theta(s'|s, a)\) due to the DPW rule.
However, directly integrating DPW and BAMCP (i.e., Algorithm \ref{alg:1}) likely cannot solve BAMDPs with continuous state and action spaces. As introduced in Appendix \ref{bamcp}, BAMCP relies on root sampling, which samples dynamics functions (from $b(\theta)$) only at the root node and follows a specific dynamics function throughout a simulation procedure. However, the rationale of root sampling (i.e., Lemma \ref{lem:1}) does not hold when applying DPW\footnote{The last equality of Eq. (\ref{root_sampling}) does not hold, since $\Tilde{b}(\theta|has') \propto \tilde{b}(\theta|ha) \widetilde{\mathcal{P}}_\theta(s'|s, a) \neq \tilde{b}(\theta|ha) \mathcal{P}_\theta(s'|s, a)$. \(\widetilde{\mathcal{P}}_\theta(s'|s, a)\) represents the state transition distribution when applying DPW, which differs from the true distribution \(\mathcal{P}_\theta(s'|s, a)\), as dictated by the DPW rule.}. Also, BAMCP assumes that the reward function is certain rather than a function of $\theta$. 

On the other hand, Polynomial Upper Confidence Trees (PUCT, \cite{DBLP:conf/pkdd/AugerCT13}), based on DPW, is a provably consistent planning method for solving MDPs with infinite-scale state/action spaces and arbitrarily stochastic transition kernels.
% be more specific about the guarantee
In this case, we propose casting BAMDPs into MDPs (i.e., $\mathcal{M}^+$ defined in Section \ref{back}) and solving them with PUCT. The pseudo code is shown as Algorithm \ref{alg:2}. Ideally, as the number of simulations $E \rightarrow \infty$, PUCT can find a near-optimal solution of $\mathcal{M}^+$, which is also a near Bayes-optimal solution for the true environment. Each simulation follows a path that begins at the root node and ends at a leaf node, utilizing progressive widening for sampling both actions and next states, as illustrated in the \textproc{ActionPW} and \textproc{StatePW} procedures. Compared to PUCT, the significant modifications include: (1) replacing \(\langle \mathcal{S}, \mathcal{P}, \mathcal{R} \rangle\) with their extended definitions in BAMDPs, \(\langle \mathcal{S}^+, \mathcal{P}^+, \mathcal{R}^+ \rangle\), and (2) applying reward penalties to account for model uncertainty. To be specific, in \textproc{StatePW}, $r$ and $s'$ are sampled from a mixture of ensemble members, which is a practical implementation of sampling from $\mathcal{R}^+$ and $\mathcal{P}^+$ as shown in Eq. (\ref{bamdp-mdp}). After receiving the transition $(s, a, r, s')$, the belief vector $b(\theta)$ is updated to $b'(\theta)$ following Eq. (\ref{b'}), finishing the transition in $\mathcal{S}^+$ from $(s, b(\theta))$ to $(s, b'(\theta))$. Moreover, the reward \(r\) is adjusted with a penalty term as defined in Eq. (\ref{p-rwd}), which is then used to calculate the return \(R\). As aforementioned, applying such a reward penalty can effectively mitigate the issue of model exploitation.
% Note that the penalties are not part of the BAMDP and won't influence the Bayesian posterior update.
% Note that, in BAMDPs, $s$ and the corresponding belief $b(\theta)$ constitute an extended state, and $b(\theta)$ is a function of the prior $b_0(\theta)$ and history $h$ (according to its recursive updating rule).
% The use of ensembling enables these processes to be parallelized and computationally feasible. Moreover, in \textproc{Simulate}, the reward \(r\) is adjusted with a penalty term as defined in Eq. (\ref{p-rwd}), which is then used to calculate the return \(R\). The penalty comes from the aleatoric and epistemic uncertainty of the learned models (i.e., $\mathcal{P}_\theta^{1:K}$, $\mathcal{R}_\theta^{1:K}$) and can effectively mitigate the model exploitation issue. Note that the penalties are not part of the BAMDP and won't influence the Bayesian posterior update.
% Without these penalties, Algorithm \ref{alg:2} is PUCT applied to $\mathcal{M}^+$. However, the world models are learned from offline data, which may provide poor coverage in certain regions of the state-action space. In these regions, none of the ensemble members are reliable, which violates the Bayesian RL assumption that the true MDP lies within the prior distribution (with high probability), so simply adapting the belief vector over them cannot resolve the issues brought by model uncertainty. However, the reward penalties can effectively indicate these regions and prevent the agent from overly exploiting experiences from them.

The proposed algorithm can be used to approximate the Bayes-optimal policy at $(s, h)$, which is $\pi_{\text{ret}}(a|(s, h)) \propto N((s, h), a), a \in C((s, h))$ (\cite{DBLP:conf/pkdd/AugerCT13}). However, our objective is to solve the entire BAMDP offline, eliminating the need for anything beyond simple lookup or inference using the learned policy function during execution. This necessitates a well-learned policy function at each state, but we cannot execute Algorithm \ref{alg:2} at every $(s, h)$ due to the scale of the state space. Therefore, we integrate the planning algorithm into a policy iteration framework, which is introduced in the next subsection. In this case, $\pi$ and $V$ in Algorithm \ref{alg:2} denote the policy and value functions from the previous learning iteration.\footnote{\(\pi\) and \(V\) are implemented as functions of \((s, h)\) because the states in BAMDPs consist of both \(s\) and the corresponding belief \(b(\theta)\), with \(b(\theta)\) being a function of the history \(h\) (according to its recursive updating rule).} As further details, multiple terms (labeled in blue) in Algorithm \ref{alg:2} have alternative design choices across different literatures, which we elaborate on in Appendix \ref{alg1_details}.

%  We also notice POMCPOW



\subsection{The Overall Framework: Search-based Policy Iteration} \label{SBPI}

% The algorithm introduced in the previous subsection is an online method, while the overall algorithm should be an offline one so that only policy inference is required during the execution.

In this subsection, we present a framework (inspired by MuZero\footnote{The novelty of our algorithm in comparison to MuZero is detailed in Section \ref{rel_works}.}) that integrates continuous BAMCP (i.e., Algorithm \ref{alg:2}) into policy improvement and policy evaluation. By iteratively running these procedures, we can approach a near Bayes-optimal policy, i.e., $\pi$, that can be directly referred to during execution in the true environment. The pseudo code of the overall framework is shown as Algorithm \ref{alg:3}. For efficiency, a learner and a number of actors execute in parallel, reading from and sending data to the replay buffer $\mathcal{D}$ respectively. The actors update their copies of policy and value functions every $E_l$ learner steps.

In particular, each actor would interact with the learned world models to sample trajectories $\tau$.\footnote{In Algorithm \ref{alg:3}, $\hat{\rho}(s)$ denotes the empirical distribution of initial states in the offline dataset $\mathcal{D}_\mu$.} At each time step of the trajectory, a \textproc{Search} procedure (defined in Algorithm \ref{alg:2}) is executed using the actor's current copy of $\pi$ and $V$ for planning at the current decision point. The search result $\pi_{\text{ret}}$ is then used to indicate the action choice, i.e., $a \sim \pi_{\text{ret}}(\cdot|(s, h))$. The subsequent transition process follows a BAMDP, where $r \sim \mathcal{R}^+(\cdot|(s, h), a)$, $s' \sim \mathcal{P}^+(\cdot|(s, h), a)$, and the belief $b({\theta})$ adapts with experience, as described in \textproc{StatePW}. The collected trajectories are used in the learning process, where $\pi$ is trained to mimic the search result $\pi_{\text{ret}}$ by minimizing a cross-entropy loss (i.e., $\mathcal{L}(\pi, \{\tau_i\}_{i=1}^{B})$), while $V$ is trained to predict the $n$-step pessimistic return $z$.\footnote{In the definition of $z$, $\tilde{r}^{(0)}=\tilde{r}$ is the pessimistic reward associated with $(s, h)$ and the subscript $(i)$ denotes that the corresponding quantity is collected in $i$ time steps within that trajectory.} As noted in (\cite{DBLP:conf/icml/HubertSABSS21}), $\pi_{\text{ret}}$ improves $\pi$ at each decision point, so repeatedly applying continuous BAMCP to obtain $\pi_{\text{ret}}$ and projecting the search results to the parameter space of $\pi$ (by minimizing $\mathcal{L}(\pi, \{\tau_i\}_{i=1}^{B})$) represents the policy improvement process. Meanwhile, value backups to update the Q estimates (i.e., $R \leftarrow V((s', hars'))$ in Algorithm \ref{alg:2}) and the temporal difference learning for the value function $V$ constitute the policy evaluation process. In this way, the search algorithm is used as a powerful improvement operator to iteratively enhance the performance of the learned policy \(\pi\).

% using SAC instead of 

\begin{algorithm}[t]
  \caption{Search-based Policy Iteration}
  \label{alg:3}
  \begin{multicols}{2}
    \begin{algorithmic}
      \State \textbf{Input:} $T$, $E_l$, $\mathcal{P}_\theta^{1:K}$, $\mathcal{R}_\theta^{1:K}$
      \State
      \State Initialize $\pi$ and $V$, $\mathcal{D} \leftarrow \emptyset$
      \Procedure{Learner}{}
      \State $e \leftarrow 0$
        \While{true}
            \State $\{\tau_i\}_{i=1}^{B} \sim \mathcal{D}$
            \State $\min_\pi \mathcal{L}(\pi, \{\tau_i\}_{i=1}^{B})$, $\min_V \mathcal{L}(V, \{\tau_i\}_{i=1}^{B})$
            \State $e \mathrel{{+}{=}} 1$
            \State Update $\pi, V$ in \textproc{Actor} if $e \% E_l == 0$
        \EndWhile
      \EndProcedure
      \State {\scriptsize $\mathcal{L}(\pi, \{\tau_i\}_{i=1}^{B}) = -\sum_{((s, h), \pi_{\text{ret}})} \pi_{\text{ret}}^T \log \pi(\cdot|(s, h)) / (BT)$}
      \State {\scriptsize $\mathcal{L}(V, \{\tau_i\}_{i=1}^{B}) = \sum_{((s, h), z)} (V(s, h) - z)^2 / (BT)$}, 
      \State {\scriptsize $z = \tilde{r}^{(0)} + \gamma \tilde{r}^{(1)} + \cdots + \gamma^{n-1} \tilde{r}^{(n-1)} + \gamma^{n} v_{\text{ret}}^{(n)}$}
      
    \end{algorithmic}
    \columnbreak
    \begin{algorithmic}
      \Procedure{Actor}{}
        \While{true}
            \State $s \leftarrow \hat{\rho}_0(\cdot)$, $h \leftarrow s$, $\tau \leftarrow []$
            \State $b(\theta) \leftarrow [1/K, \cdots, 1/K]$
            \For{$t=1 \cdots T$}
            \State $\pi_{\text{ret}}, v_{\text{ret}} \leftarrow \textproc{Search}((s, h), b(\theta))$
            \State $a \sim \pi_{\text{ret}}(\cdot|(s, h))$
            \State Acquire $r, s', b'(\theta)$ as in {\small \textproc{StatePW}}
            \State Calculate $\tilde{r}$ using Eq. (\ref{p-rwd})
            \State Append $\tau$ with $((s, h), \tilde{r}, \pi_{\text{ret}}, v_{\text{ret}})$
            \State $s, h, b(\theta) \leftarrow s', hars', b'(\theta)$
            \EndFor
            \State $\mathcal{D} \leftarrow \mathcal{D} \cup \{\tau\}$
        \EndWhile
      \EndProcedure
    \end{algorithmic}
  \end{multicols}
\end{algorithm}
\section{Evaluation} \label{eval}

% To evaluate the effectiveness of each component in our algorithm design, we introduce three variants of our algorithm: BA-MBRL, BA-MCTS, and BA-MCTS-SL. (1) Existing offline MBRL methods leverage learned world models as surrogate simulators, applying reward penalties to collected transitions and employing standard online RL algorithms (e.g., SAC (\cite{DBLP:conf/icml/HaarnojaZAL18})) to learn a policy. BA-MBRL follows this approach but models the problem as a BAMDP, with environment transitions defined by Equations (\ref{bamdp-mdp}) and (\ref{b'}) and the reward penalty given by Equation (\ref{p-rwd}). BA-MBRL is designed to evaluate the effectiveness of Bayesian RL. (2) Building on BA-MBRL, BA-MCTS introduces Continuous BAMCP (Algorithm \ref{alg:2}) to plan at decision points, rather than inferring directly from the policy, to generate trajectories for downstream SAC. BA-MCTS demonstrates the impact of deep search on policy learning. (3) BA-MCTS-SL, described in Algorithm \ref{alg:3}, replaces the policy learning algorithm in BA-MCTS from policy gradient methods (as in SAC) with supervised learning (SL). By comparing these two approaches, we aim to determine which method offers a more efficient policy update mechanism, particularly for continuous control tasks.

To evaluate the effectiveness of each component in our algorithm design, we introduce three variants: (1) \textbf{BA-MBRL} leverages learned world models as surrogate simulators, applying reward penalties to collected transitions and using standard online RL algorithms (e.g., SAC (\cite{DBLP:conf/icml/HaarnojaZAL18})) to learn a policy. While following existing offline MBRL methods, it models the problem as a BAMDP (rather than an MDP), with environment transitions defined by Equations (\ref{bamdp-mdp}) and (\ref{b'}) and the reward penalty by Equation (\ref{p-rwd}), and is designed to evaluate the effectiveness of Bayesian RL. (2) \textbf{BA-MCTS} builds on BA-MBRL by introducing Continuous BAMCP (Algorithm \ref{alg:2}) to plan at decision points, rather than inferring directly from the policy, to generate trajectories for downstream SAC, demonstrating the impact of deep search on policy learning. (3) \textbf{BA-MCTS-SL}, described in Algorithm \ref{alg:3}, replaces the policy learning algorithm in BA-MCTS from policy gradient methods (as in SAC) with supervised learning (SL), allowing us to compare which approach offers a more efficient policy update mechanism, particularly for continuous control tasks.
 
\begin{table}[t]
\scriptsize
\centering
\begin{tabular}{c|c|c|c|c|c|c|c|c}
\hline
{\makecell{Data Type}} & {Environment} & {\makecell{BA-MCTS \\ -SL (ours)}} & {\makecell{BA-MCTS \\ (ours)}} & {\makecell{BA-MBRL \\ (ours)}} & {Optimized} & {COMBO} & {MOReL} & {MOPO}\\
\hline 
\hline
{random} & {HalfCheetah} & {29.20 $\pm$ 2.00} & {36.23 $\pm$ 1.04} & {32.76 $\pm$ 1.16} & {31.7} & {\textbf{38.8}} & {25.6} & {35.4} \\
{random} & {Hopper} & {33.83 $\pm$ 0.10} & {31.56 $\pm$ 0.12} & {31.47 $\pm$ 0.03} & {12.1} & {17.9} & {\textbf{53.6}} & {11.7} \\
{random} & {Walker2d} & {21.89 $\pm$ 0.07} & {21.59 $\pm$ 0.32} & {21.45 $\pm$ 0.53} & {21.7} & {7.0} & {\textbf{37.3}} & {13.6} \\
\hline
{medium} & {HalfCheetah} & {70.47 $\pm$ 3.52} & {\textbf{75.84} $\pm$ 3.81} & {56.54 $\pm$ 5.20} & {45.7} & {54.2} & {42.1} & {42.3} \\
{medium} & {Hopper} & {97.75 $\pm$ 7.09} & {96.70 $\pm$ 14.0} & {\textbf{98.25} $\pm$ 3.42} & {69.3} & {97.2} & {95.4} & {28.0} \\
{medium} & {Walker2d} & {\textbf{82.24} $\pm$ 1.85} & {74.73 $\pm$ 3.25} & {75.41 $\pm$ 4.17} & {79.7} & {81.9} & {77.8} & {17.8} \\
\hline 
{med-replay} & {HalfCheetah} & {61.16 $\pm$ 1.60} & {\textbf{65.45} $\pm$ 0.81} & {62.50 $\pm$ 0.18} & {58.0} & {55.1} & {40.2} & {53.1} \\
{med-replay} & {Hopper} & {\textbf{106.3} $\pm$ 0.13} & {101.8 $\pm$ 3.46} & {93.91 $\pm$ 4.25} & {90.8} & {89.5} & {93.6} & {67.5} \\
{med-replay} & {Walker2d} & {92.13 $\pm$ 5.13} & {95.06 $\pm$ 2.11} & {\textbf{97.54}$\pm$ 1.93} & {65.8} & {56.0} & {49.8} & {39.0} \\
\hline 
{med-expert} & {HalfCheetah} & {80.53 $\pm$ 6.63} & {76.16 $\pm$ 10.3} & {90.52 $\pm$ 4.13} & {\textbf{104.2}} & {90.0} & {53.3} & {63.3} \\
{med-expert} & {Hopper} & {\textbf{112.2} $\pm$ 0.29} & {108.3 $\pm$ 0.22} & {107.8 $\pm$ 0.37} & {105.8} & {111.1} & {108.7} & {23.7} \\
{med-expert} & {Walker2d} & {107.7 $\pm$ 0.82} & {\textbf{110.0} $\pm$ 1.74} & {84.71 $\pm$ 0.87} & {97.1} & {103.3} & {95.6} & {44.6} \\
\hline
\hline
\multicolumn{2}{c|}{Average Score} & {\textbf{74.62}} & {74.45} & {71.06} & {65.16} & {66.83} & {64.42} & {36.67} \\
\hline 
\end{tabular}
\caption{Comparisons between the proposed algorithms and SOTA offline model-based RL methods on the D4RL benchmark suite. Each value represents the normalized score, as proposed in (\cite{DBLP:journals/corr/abs-2004-07219}), of the policy trained by the corresponding algorithm. These scores are undiscounted returns normalized to approximately range between 0 and 100, where a score of 0 corresponds to a random policy and a score of 100 corresponds to an expert-level policy. For our algorithms, we report the average score of the final ten policy learning epochs and its standard deviation across three random seeds. Results in the last four columns are taken from the original papers (\cite{DBLP:conf/iclr/LuBPOR22, DBLP:conf/nips/YuKRRLF21, DBLP:conf/nips/KidambiRNJ20, DBLP:conf/nips/YuTYEZLFM20}), respectively.} 
\label{table:1}
\end{table}

We first evaluate our algorithms on a widely-used continuous control benchmark for offline RL methods -- D4RL MuJoCo (\cite{DBLP:journals/corr/abs-2004-07219}). The evaluation results for three types of robotic agents, each with offline datasets of four different qualities, are presented in Table \ref{table:1}. (1) Compared to SOTA offline MBRL methods, our algorithms achieve superior performance on nine out of twelve tasks. In terms of average performance, BA-MBRL significantly outperforms the baselines, demonstrating the effectiveness of using BAMDPs to handle model uncertainties in offline MBRL. (2) Both BA-MCTS and BA-MCTS-SL further improve upon BA-MBRL, highlighting the enhancement brought by deep search in policy learning. Notably, we apply Continuous BAMCP to only 10\% of states when collecting training trajectories, while for the remaining states, we sample actions directly from the policy, i.e., $a \sim \pi(\cdot|s)$. Increasing the search ratio could further enhance policy performance at the cost of increased computation. (3) BA-MCTS-SL performs similarly to BA-MCTS, validating the effectiveness of both policy update mechanisms. However, BA-MCTS-SL struggles on Walker2d, where a warm-up training phase (using BA-MBRL) is required to establish a better initial policy. On the other hand, the advantage of the SL-based policy update is evident in the training plots of our algorithms in Figure \ref{fig:3}, where BA-MCTS-SL exhibits much smoother learning curves compared to the other two algorithms, indicating greater robustness in model selection. (4) We further compare our algorithms with model-free offline policy learning methods, as shown in Appendix \ref{ComMF}. The performance improvement is even greater than that over model-based methods, highlighting the necessity of model-based learning. Particularly, when data quality is low, merely mimicking or staying close to the behavior policy would result in an underperforming policy. (5) We provide an ablation study in Appendix \ref{ASRP} to demonstrate the necessity of incorporating the reward penalty in offline MBRL to prevent the overexploitation of inaccurate world models. Additionally, we find that the SL-based policy update is less sensitive to model inaccuracies.

\begin{figure*}[t]
\centering
\subfigure[hc-med-expert]{
\label{fig:1(a)} 
\includegraphics[width=1.5in, height=0.8in]{hc-m-exp-mz.pdf}}
\subfigure[hc-med-replay]{
\label{fig:1(b)} 
\includegraphics[width=1.5in, height=0.8in]{hc-m-rep-mz.pdf}}
\subfigure[hc-medium]{
\label{fig:1(c)} 
\includegraphics[width=1.5in, height=0.8in]{hc-med-mz.pdf}}
\subfigure[hc-random]{
\label{fig:1(d)} 
\includegraphics[width=1.5in, height=0.8in]{hc-rnd-mz.pdf}}
\subfigure[hp-med-expert]{
\label{fig:1(e)} 
\includegraphics[width=1.5in, height=0.8in]{hp-m-exp-mz.pdf}}
\subfigure[hp-med-replay]{
\label{fig:1(f)} 
\includegraphics[width=1.5in, height=0.8in]{hp-m-rep-mz.pdf}}
\subfigure[hp-medium]{
\label{fig:1(g)} 
\includegraphics[width=1.5in, height=0.8in]{hp-med-mz.pdf}}
\subfigure[hp-random]{
\label{fig:1(h)} 
\includegraphics[width=1.5in, height=0.8in]{hp-rnd-mz.pdf}}
\subfigure[wk-med-expert]{
\label{fig:1(i)} 
\includegraphics[width=1.5in, height=0.8in]{wk-m-exp-mz.pdf}}
\subfigure[wk-med-replay]{
\label{fig:1(j)} 
\includegraphics[width=1.5in, height=0.8in]{wk-m-rep-mz.pdf}}
\subfigure[wk-medium]{
\label{fig:1(k)} 
\includegraphics[width=1.5in, height=0.8in]{wk-med-mz.pdf}}
\subfigure[wk-random]{
\label{fig:1(l)} 
\includegraphics[width=1.5in, height=0.8in]{wk-rnd-mz.pdf}}
\caption{Performance of Sampled EfficientZero on D4RL MuJoCo tasks. The results for HalfCheetah, Hopper, and Walker2d are presented in the three rows, respectively. Each subfigure depicts the change in undiscounted episodic return as a function of the number of training samples. Experiments are repeated three times with different random seeds, with the solid line representing the mean and the shaded area indicating the 95\% confidence interval. For reference, the expert-level episodic returns for HalfCheetah, Hopper, and Walker2d are 12135, 3234.3, and 4592.3, respectively.}
\label{fig:1} 
\end{figure*}


\begin{figure}[t]
    \centering
    \includegraphics[width=\linewidth, height=1.5in]{nf_return-eps-converted-to.pdf} % or use height
    \caption{Evaluation results for the tokamak control tasks. The figure shows the change in episodic returns over training epochs for the proposed algorithms and baselines across three target tracking tasks in the nuclear fusion scenario. Solid lines represent the average performance, while shaded areas indicate the 95\% confidence intervals.}
    \label{fig:2}
\end{figure}

For fair comparisons and real-time execution, we do not perform test-time search and adopt the same policy architecture as the baselines, i.e., a feedforward neural network, rather than an RNN that incorporates transition history as input. These alternative designs have the potential to further improve our algorithms. For implementation, we build on the codebase of Optimized (\cite{DBLP:conf/iclr/LuBPOR22}), which thoroughly explores design choices in offline MBRL, making minimal changes to the code and hyperparameter settings\footnote{The detailed hyperparameter setups of our algorithms are provided in Appendix \ref{KeyPara}.}. Therefore, we believe the performance improvements stem from the Bayesian RL framework and deep search component. Further, both components can be seamlessly integrated with other advancements in offline MBRL, such as more accurate world model learning and improved uncertainty quantification for constructing pessimistic MDPs. 

MuZero also applies deep search to MBRL. To evaluate its performance on D4RL MuJoCo tasks, we use the open-source implementation and hyperparameter configurations of Sampled EfficientZero (\cite{DBLP:conf/nips/YeLKAG21}) provided by LightZero (\cite{DBLP:conf/nips/NiuPYLZRHLL23}). Benchmarking results from LightZero indicate that Sampled EfficientZero, equipped with a Gaussian policy, achieves the best performance on (online) MuJoCo locomotion tasks compared to other MuZero variants. To adapt Sampled EfficientZero for offline learning, we employ the reanalyse technique proposed by (\cite{DBLP:conf/nips/SchrittwieserHM21}). The evaluation results are presented in Figure \ref{fig:1}. For reference, the expert-level episodic returns (corresponding to scores of 100) for HalfCheetah, Hopper, and Walker2d are 12135, 3234.3, and 4592.3, respectively. As shown, the results are significantly worse compared to the performance of offline RL methods listed in Table \ref{table:1}, despite Sampled EfficientZero's higher computational cost (detailed in Appendix \ref{CompCost}). Notably, both Sampled EfficientZero and BA-MCTS-SL rely on supervised learning for policy improvement. However, for continuous control tasks, the agent can only sample a finite number of actions at a decision point, and the search result (e.g., $\pi_{\text{ret}}$ in Algorithm \ref{alg:2}) is a distribution over this finite set, which could be a poor approximation of the optimal action distribution. Thus, purely mimicking the search result may be less sample-efficient than policy gradient methods, as it fails to account for the continuous nature of the action space. Furthermore, world model learning is the foundation of MBRL and can be particularly challenging in \textbf{continuous control and offline learning settings}, where the state-action space is vast but training data is limited. Sampled EfficientZero integrates model learning and policy training into a single stage, which significantly increases the learning difficulty (compared to BA-MCTS-SL).

Finally, we evaluate our algorithms on three target tracking tasks in tokamak control. The tokamak is one of the most promising confinement devices for achieving controllable nuclear fusion, where the primary challenge lies in confining the plasma, i.e., an ionized gas of hydrogen isotopes, while heating it and increasing its pressure to initiate and sustain fusion reactions (\cite{1512794}). Tokamak control involves applying a series of direct actuators (e.g., neutral beam, ECH power, magnetic field) and indirect actuators (e.g., setting targets for the plasma shape and density) to confine the plasma to achieve a desired state or track a given target. This sophisticated physical process is an ideal test bed for our algorithms. Specifically, we use a well-trained data-driven dynamics model provided by \cite{DBLP:journals/corr/abs-2404-12416} as a ``ground truth" simulator for the nuclear fusion process during evaluation, and generate a dataset containing 725270 transitions using this model for offline RL. We select a reference shot (i.e., an episode of a fusion process) from DIII-D\footnote{DIII-D is a tokamak device located in San Diego, California, operated by General Atomics.}, and use its trajectories of Ion Rotation, Electron Temperature, and $\beta_n$ as targets for three tracking tasks. These are critical quantities in tokamak control, particularly $\beta_n$, which serves as an economic indicator of the efficiency of nuclear fusion. The tracking tasks have a 28-dimensional state space and a 14-dimensional action space, both continuous. Moreover, these tasks are \textbf{highly stochastic}, as the underlying dynamics model is a probabilistic neural network and each state transition is a sample from this model. For details on the simulator, and the design of the state/action spaces and reward functions, please refer to Appendix \ref{DetTCT}. We compare our algorithms with SOTA model-free and model-based offline RL methods, specifically CQL and Optimized. The learning performance on the three tracking tasks is shown in Figure \ref{fig:2}, where the x-axis and y-axis represent the training epochs and (negative) full-shot tracking errors, respectively. Our algorithms consistently outperform the baselines. Notably, the offline dataset does not include the reference shot or any similar, nearby shots. Therefore, restricting the policy to stay close to the behavior policy, as done in model-free methods, can be problematic. Also, learning dynamics models for MBRL is quite challenging in this nuclear fusion scenario. Our algorithms share the same ensemble of dynamics models with ``Optimized" for policy learning, and the comparisons can demonstrate the superiority of Bayesian RL and deep search. Figure \ref{fig:2} has been smoothed for visualization\footnote{The episodic return is plotted every 10 training epochs, with the y-axis representing the average value of a sliding window of length 5.}. We further report the average return over the final 10 training epochs in Table \ref{table:6}, and the conclusions align with those from the D4RL MuJoCo tasks, showing the robustness of our proposed algorithms.

\begin{table}[t]
\small
\centering
\begin{tabular}{c|c|c|c|c|c}
\hline
{Task} & {\makecell{BA-MCTS \\ -SL (ours)}} & {\makecell{BA-MCTS \\ (ours)}} & {\makecell{BA-MBRL \\ (ours)}} & {CQL} & {Optimized}\\
\hline 
\hline
{Temperature} & {\textbf{-21.16} $\pm$ 5.00} & {-23.83 $\pm$ 9.66} & {-29.35 $\pm$ 4.72} & {-59.62 $\pm$ 1.57} & {-83.55 $\pm$ 10.56} \\
{Rotation} & {\textbf{-14.14} $\pm$ 1.88} & {-19.07 $\pm$ 5.85} & {-31.33 $\pm$ 11.54} & {-85.48 $\pm$ 2.72} & {-71.54 $\pm$ 9.88} \\
{$\beta_{n}$} & {-37.03 $\pm$ 17.98} & {\textbf{-18.93} $\pm$ 1.75} & {-23.4 $\pm$ 10.77} & {-36.37 $\pm$ 1.17} & {-57.84 $\pm$ 10.27} \\
\hline
\hline
{Average} & {-24.11} & {\textbf{-20.61}} & {-28.03} & {-60.49} & {-70.98} \\
\hline 
\end{tabular}
\caption{Comparisons between the proposed algorithms and offline RL baselines on the target tracking tasks. For each algorithm, we report the average return of the final ten policy learning epochs and its standard deviation across three different random seeds.}
\label{table:6}
\end{table}

 % get rid of implementation tricks, do minimal changes, see the improvement brought by the search component. 

 % muzero fails, learning dyn and policy at the same stage can be very challenging, especially for complex continuous control tasks

\section{Future work}


This paper aims to caution the practitioner against blindly following
current widespread practices to increase the robust
performance of machine learning models.
% Specifically, we study how
Specifically, adversarial training is currently recognized to be one
of the most effective defense mechanisms for $\ell_p$-perturbations,
%\fy{but also others like manifold adv. ex?}
significantly outperforming robust performance of standard training.  However, we prove that
this common wisdom is not applicable for directed attacks -- that are perceptible (albeit consistent) but efficiently focus their
attack budget to target ground truth class information -- in the low-sample size regime.
In particular, in such settings adversarial training can in fact yield worse accuracy than standard training.

% On a high level, our paper reveals fundamental and provable
% differences in robustness behavior between perceptible and imperceptible 
% perturbations. In particular, it underlines the necessity
% of future work that targets general understanding of adversarial robustness, to study a broader scope of perturbation
%types.  %(both empirical and theoretically) 

%% In particular, we show
%% theoretically and experimentally that it is critical to consider the
%% relationship between the attack transformation type and the believed
%% ground truth signal direction.


%detrimental to finding the ground truth, since the structural bias is
%actively worsened.

%% In the overparameterized and small-sample regime, many
%% classifiers can fit the training data perfectly.  Key is to have the
%% right inductive bias.  If the perturbation attacks the signal,
%% adversarial training is very detrimental to finding the ground truth,
%% since the structural bias is actively worsened.  Hence standard
%% error increases severely.

In terms of follow-up work on directed attacks in the low-sample
regime, there are some concrete questions that would be interesting to
explore.  For example, as discussed in Section~\ref{sec:relatedwork},
it would be useful to test whether some methods to mitigate the
standard accuracy vs. robustness trade-off would also relieve the
perils of adversarial training for directed attacks. Further, we
hypothesize, independent of the attack during test time, it is
important in the small sample-size regime to choose perturbation sets
during training that align with
the ground truth signal (such as rotations for data with inherent
rotation). If this hypothesis were to be confirmed, it would break
with yet another general rule that the best defense perturbation type
should always match the attack during evaluation.  The insights from
this study might also be helpful in the context of searching for
good defense perturbations.

%% help even when
%% the type of robustness during evaluation is a \nameofattack .  In
%% other words, in the overparameterized small sample regime, different
%% sets may be better.



\section{Ethics statement}

This paper explores backdoor attacks that can be inserted through data augmentation. For critical applications such as self driving cars, backdoors inserted by malicious attackers could have serious consequences. Therefore, in this paper we aim to encourage people to inspect their augmentation functions to ensure that any external code is clean.
\section{Reproducibility}

All hyperparameters used to produce our results are provided under each table or in Appendices B, C, and D. Additionally, our PyTorch code used to achieve the results for all three backdoors can be found at at \href{https://github.com/slkdfjslkjfd/augmentation\_backdoors}{https://github.com/slkdfjslkjfd/augmentation\_backdoors}.

\bibliographystyle{iclr2023_conference}
\bibliography{references}

\newpage
\section*{Appendix}
\appendix
\section{AugMix backdoor algorithm}

\begin{algorithm}
\caption{AugMix backdoor}\label{alg:two}
\SetKwInOut{Input}{input}
\SetKwFunction{gb}{getbatnetbatch}
\SetKwFunction{SGD}{SGD}
\Input{batch $B$, transforms $T$, iterations $n$, surrogate model $M$, loss function $L$}
\texttt{\\}
$w\gets$\text{ random samples from Dirichlet(}$1$\text{) in shape (len(}$B$\text{), len(}$T$))\;
$m\gets$\text{ random samples from Beta(}$\alpha, \alpha$\text{) in shape (len}($T$)\;
\texttt{\\}
$U\gets$ apply BADNET backdoor to $B$\;
$l_u\gets L(M(U$.inputs), $U$.labels)\;
$g_u\gets$ backpropagate gradients from $l_u$ to weights of $M$
\texttt{\\}
\For{$n$ iterations}{
    $V\gets$ apply AugMix to $B$.inputs, using weights $w$[i], $m$[i] for $B$.inputs[i]\;
    $l_v\gets L(M(V$.inputs), $V$.labels)\;
    $g_v\gets$ backpropagate gradients from $l_v$ to weights of $M$\;
    \texttt{\\}
    $E\gets||g_u-g_v||^p$\;
    $g_E\gets$ backpropagate gradients from $E$ to $w$ and $m$\;
    \texttt{\\}
    $w,m\gets$\SGD($[w, m],g_E$)\;
}
\Return{$V$}\;
\end{algorithm}

\section{Datasets}

\textbf{MNIST} The MNIST dataset \citep{mnist} consists of 60000 train images and 10000 test images. Each 28x28 pixel greyscale image displays a single digit between 0 and 9 inclusive. The class of the image is the digit it contains.

\textbf{Omniglot} The Omniglot dataset \citep{omniglot} consists of 1623 classes of handwritten characters from 50 different alphabets, with each class containing 20 samples. We downscale the dataset to 28x28 greyscale images and reduce the number of classes to 50. We split each class into 15 train images and 5 test images.

\textbf{CIFAR-10} The CIFAR-10 dataset \citep{cifar} consists of 50000 train images and 10000 test images, both equally split into 10 classes. Each 32x32 pixel colour image displays a subject from one of the 10 classes.

\textbf{CIFAR-100} The CIFAR-100 dataset \citep{cifar} is similar to the CIFAR-10 dataset, but with 100 classes of 500 train and 100 test images.

\section{Models}

\textbf{ResNet} We use a ResNet-50 classifier for the CIFAR-10 dataset \citep{resnet}, and the WideResNet variant implementation at \href{https://github.com/meliketoy/wide-resnet.pytorch}{https://github.com/meliketoy/wide-resnet.pytorch} to train our CIFAR-100 classifier.

\textbf{DenseNet} We use the DenseNet \citep{densenet} implementation at \href{https://github.com/amurthy1/dagan\_torch}{https://github.com/amurthy1/dagan\_torch} to train our Omniglot classifier.

\textbf{CNN} We use a CNN with two convolutional layers for our MNIST classifiers. The architecture of our classifiers is detailed in Table 4.

\section{Hardware systems}

The testing of our GAN and AugMix backdoors was carried out on a hardware system with 4x NVIDIA GeForce GTX 1080 Ti. The simple transform backdoor training was carried out on NVIDIA T4 GPUs.

% idk what this does but it makes the table go to the top
\makeatletter
\setlength{\@fptop}{0pt}
\makeatother

\begin{table}[ht!]
\caption{Architecture of the classifier we trained on the MNIST dataset}
\centering
\adjustbox{scale=1.0}{\begin{tabular}{c|ccccc}
\hline
& input & filter shape & stride & output & activation \\
\hline
Conv0 & (1, 28, 28) & (8, 1, 5, 5) & 1 & (8, 24, 24) & ReLU \\
Pool0 & (8, 28, 28) & Max, (2, 2) & 2 & (8, 12, 12) & \\
Conv1 & (8, 12, 12) & (16, 8, 5, 5) & 1 & (16, 8, 8) & ReLU \\
Pool1 & (16, 8, 8) & Max, (2, 2) & 2 & (16, 4, 4) & \\
Dense0 & (16, 4, 4) & & & (128) & ReLU \\
Dense1 & (128) & & & (96) & ReLU \\
Dense2 & (96) & & & (10) & \\
\hline
\end{tabular}}
\end{table}


\end{document}
