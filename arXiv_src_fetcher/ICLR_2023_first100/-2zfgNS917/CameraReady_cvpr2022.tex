% CVPR 2022 Paper Template
% based on the CVPR template provided by Ming-Ming Cheng (https://github.com/MCG-NKU/CVPR_Template)
% modified and extended by Stefan Roth (stefan.roth@NOSPAMtu-darmstadt.de)

\documentclass[10pt,twocolumn,letterpaper]{article}

%%%%%%%%% PAPER TYPE  - PLEASE UPDATE FOR FINAL VERSION
% \usepackage[review]{cvpr}      % To produce the REVIEW version
\usepackage{cvpr}              % To produce the CAMERA-READY version
% \usepackage[pagenumbers]{cvpr} % To force page numbers, e.g. for an arXiv version

% Include other packages here, before hyperref.
\usepackage{graphicx}
\usepackage{threeparttable}
\usepackage{amsmath}
\usepackage{xcolor}
\usepackage{color, colortbl}
\usepackage{amssymb}
\usepackage{booktabs}
\usepackage{multirow}
\usepackage{enumitem}
\usepackage[accsupp]{axessibility}
\newcommand{\lsh}[1]{\textcolor{magenta}{ (#1)}}
\newcommand{\tablestyle}[2]{\setlength{\tabcolsep}{#1}\renewcommand{\arraystretch}{#2}\centering\footnotesize}
\newlength\savewidth\newcommand\shline{\noalign{\global\savewidth\arrayrulewidth
  \global\arrayrulewidth 1pt}\hline\noalign{\global\arrayrulewidth\savewidth}}
% \newcommand{\myplus}[1]{\color{green}{\tiny{$+$#1}}}
% \newcommand{\myminus}[1]{\color{red}{\tiny{$-$#1}}}
\newcommand{\myplus}[1]{\color{green}{\tiny{}}}
\newcommand{\myminus}[1]{\color{red}{\tiny{}}}
\newcommand{\xd}[1]{\color{orange}{\tiny{$-$}#1}}

\newcommand\mypara[1]{\vspace{1mm}\noindent\textbf{#1}}

% It is strongly recommended to use hyperref, especially for the review version.
% hyperref with option pagebackref eases the reviewers' job.
% Please disable hyperref *only* if you encounter grave issues, e.g. with the
% file validation for the camera-ready version.
%
% If you comment hyperref and then uncomment it, you should delete
% ReviewTempalte.aux before re-running LaTeX.
% (Or just hit 'q' on the first LaTeX run, let it finish, and you
%  should be clear).
\usepackage[pagebackref,breaklinks,colorlinks]{hyperref}
\renewcommand*{\thefootnote}{\fnsymbol{footnote}}

% Support for easy cross-referencing
\usepackage[capitalize]{cleveref}
\crefname{section}{Sec.}{Secs.}
\Crefname{section}{Section}{Sections}
\Crefname{table}{Table}{Tables}
\crefname{table}{Tab.}{Tabs.}


%%%%%%%%% PAPER ID  - PLEASE UPDATE
% \def\cvprPaperID{7432} % *** Enter the CVPR Paper ID here
\def\confName{CVPR}
\def\confYear{2022}


\newcommand{\ky}[1]{{\color{blue}{#1}}}
\newcommand{\KY}[1]{{\color{blue}{\bf #1}}}

% \newcommand{\sf}[1]{{\color{red}{#1}}}
\newcommand{\SF}[1]{{\color{red}{\bf #1}}}


\begin{document}

%%%%%%%%% TITLE - PLEASE UPDATE
%\title{Self-assembling Knowledge Distillation via Semantic Position Encoding}

\title{TiG-BEV: Multi-view BEV 3D Object Detection via\\Target Inner-Geometry Learning}

%\title{Self-assembling Knowledge Distillation with Semantic Alignment}



\author{Peixiang Huang$^{1}$\footnotemark[2]\;,\;
Li Liu$^{2}$\footnotemark[2]\, \footnotemark[4]\;,\;
Renrui Zhang$^{3}$\footnotemark[2]\;,\;
Song Zhang$^{4}$\;,\;
Xinli Xu$^{2}$\;\;\\
Baichao Wang$^{2}$\;\;
Guoyi Liu$^{2}$\vspace{0.2cm}\\
%Baichao Wang$^{2}$,\;
%Guoyi Liu$^{2}$\\
$^1$Peking University\;
$^2$NIO\;
$^3$The Chinese University of Hong Kong\;\\
$^4$University of Chinese Academy of Sciences\\%\;
%$^5$Beijing Institute of Technology\\
%{\tt\small \{linsihao6, hongwei.xie.90, Kaicheng.yu.yt, xdliang328\}@gmail.com}\\
{\tt\small liuli.ll9412@gmail.com,}\;
{\tt\small huangpx@stu.pku.edu.cn,}\;
{\tt\small 1155186671@link.cuhk.edu.hk,}\;\\
{\tt\small zhangsong20@mails.ucas.ac.cn,}\;
{\tt\small xxlbigbrother@gmail.com,}\;
{\tt\small \{baichao.wang,gary.liu\}@nio.com}\\
% For a paper whose authors are all at the same institution,
% omit the following lines up until the closing ``}''.
% Additional authors and addresses can be added with ``\and'',
% just like the second author.
% To save space, use either the email address or home page, not both
% \and
% Second Author\\
% Institution2\\
% First line of institution2 address\\
% {\tt\small secondauthor@i2.org}
}
\maketitle

A \gls{np} estimates a stochastic process implicitly defined with neural networks given a stream of data, rather than pre-specifying priors already known, such as Gaussian processes. An ideal \gls{np} would learn everything from data without any inductive biases, but in practice, we often restrict the class of stochastic processes for the ease of estimation. One such restriction is the use of a finite-dimensional latent variable accounting for the uncertainty in the functions drawn from \glspl{np}. Some recent works show that this can be improved with more ``data-driven’’ source of uncertainty such as bootstrapping. In this work, we take a different approach based on the martingale posterior, a recently developed alternative to Bayesian inference. For the martingale posterior, instead of specifying prior-likelihood pairs, a predictive distribution for future data is specified. Under specific conditions on the predictive distribution, it can be shown that the uncertainty in the generated future data actually corresponds to the uncertainty of the implicitly defined Bayesian posteriors. Based on this result, instead of assuming any form of the latent variables, we equip a \gls{np} with a predictive distribution implicitly defined with neural networks and use the corresponding martingale posteriors as the source of uncertainty. The resulting model, which we name as \gls{mpnp}, is demonstrated to outperform baselines on various tasks.

\footnotetext[2]{Equal Contribution.}
\footnotetext[4]{Corresponding Author.}
%\footnotetext[3]{Part of the work done when as an intern in NIO.}


\section{Introduction}\label{sec:intro}
% Multi-armed bandit (MAB) is a classic sequential decision making problem \citep{auer2002finite}, where a learning agent chooses among competing actions sequentially to maximize its accumulative reward over time. 
% %Despite its simplicity, MAB exemplifies the exploration-and-exploitation dilemma that also exists in more complicated problems. 
% An important extension of MAB, named linear contextual bandit \citep{li2010contextual}, incorporates contextual information in the problem setting, by assuming a linear mapping between the context and expected reward. It has gained popularity in various applications, such as recommender systems \citep{li2010contextual}, display advertisement \citep{li2010exploitation} and clinical trials \citep{durand2018contextual}.
% Most existing linear bandit solutions are designed under a centralized learning setting, i.e., data is readily available at a central server. However, with the increasing public concerns of privacy, especially the bandit algorithms usually directly learn from user data,
% %more and more people are reluctant to provide their own data and strict regulations on data usage like GDPR have also went into effect \cite{voigt2017eu}, which makes 
% there is a growing demand to keep data decentralized and push the learning of bandit models to the client side. 
% % This idea is also made much more feasible due to the growing computational power of edge devices nowadays. 

% Federated learning has recently emerged as a promising setting for decentralized machine learning.
% % , and its effectiveness was first validated at a large scale by training a global model across all mobile devices via the Google Keyboard Android application \cite{konevcny2016federated}. 
% %The term ``federated learning" was first introduced by \citet{mcmahan2017communication} with an emphasis on efficiently training deep models over mobile device applications. As significant amount of later works have applied federated learning to other applications, there may be variations in its meaning for different research communities. 
% Since its debut in \citet{mcmahan2017communication}, there have been variations in its definition for different applications \citep{yang2019federated}.
% In this paper, we follow the general definition by \citet{kairouz2019advances}: multiple clients collaborate in solving a machine learning problem under the coordination of a central server, while keeping each client's raw data local. 
% So far, most existing works in federated learning study offline supervised learning problems \citep{konevcny2016federated,zhao2018federated}, where labeled training instances already sit on the client side. How to perform bandit learning under the federated learning setting remains underexplored.
As a popular online learning problem, linear contextual bandit has been used for a variety of applications, including recommender systems \citep{li2010contextual}, display advertisement \citep{li2010exploitation} and clinical trials \citep{durand2018contextual}. While most existing solutions are designed under a centralized setting (i.e., data is readily available at a central server), in response to the increasing application scale and public concerns of privacy, there is a growing demand to keep data decentralized and push the learning of bandit models to the client side.
% As a classic sequential decision making problem, linear contextual bandit has been widely used for a variety of real-world applications, including recommender systems \citep{li2010contextual}, display advertisement \citep{li2010exploitation} and clinical trials \citep{durand2018contextual}. 
% Most existing solutions are designed under a centralized learning setting, i.e., data is readily available at a central server. However, with the increasing public concerns of privacy, especially the bandit algorithms usually directly learn from user data,
% there is a growing demand to keep data decentralized and push the learning of bandit models to the client side. 
Federated learning has recently emerged as a promising setting for decentralized machine learning \citep{konevcny2016federated}.
% , and its effectiveness was first validated at a large scale by training a global model across all mobile devices via the Google Keyboard Android application \cite{konevcny2016federated}. 
%The term ``federated learning" was first introduced by \citet{mcmahan2017communication} with an emphasis on efficiently training deep models over mobile device applications. As significant amount of later works have applied federated learning to other applications, there may be variations in its meaning for different research communities. 
Since its debut in \citeyear{mcmahan2017communication}, there have been many variations for different applications \citep{yang2019federated}. However, most existing works study offline supervised learning problems \citep{li2019convergence,zhao2018federated}, which only concerns optimization convergence over a fixed dataset. How to perform federated bandit learning remains under-explored, and is the main focus of this paper. 

Analogous to its offline counterpart, the goal of federated bandit learning is to minimize the cumulative regret incurred by $N$ clients during their online interactions with the environment over time horizon $T$,
% $N$ clients in a learning system need to collaborate to minimize the overall cumulative regret over a finite time horizon $T$, 
while keeping each client's raw data local. Take recommender systems as an example, where the clients correspond to the edge devices that directly interact with user by making recommendations and receiving feedbacks. Unlike centralized setting where observations from all clients are immediately transmitted to the server to learn a single model, in federated bandit learning, each client makes recommendations based on its local model, with occasional communication for collaborative model estimation.

% In this paper, we follow the general definition by \citet{kairouz2019advances}: multiple clients collaborate in solving a machine learning problem under the coordination of a central server, while keeping each client's raw data local. 


%Though having potential for wide range of applications, online learning problems like linear bandit in federated learning setting, a.k.a. federated linear bandits \cite{dubey2020differentially}, have not attracted enough attention and still remain an open problem. 

% Therefore, it is a natural idea to study contextual linear bandit in a federated learning paradigm, which is also referred to as federated linear bandits \cite{dubey2020differentially}. In a federated learning paradigm, multiple clients collaborate in solving a machine learning problem, under the coordination of a central server, and each client's raw data is stored locally and not transferred to the server. 
% when linear bandit algorithms are applied to the federated learning paradigm, because these algorithms assume a traditional centralized machine learning system where all the data are collected together and all the computation happens in one machine or data center. 
Several new challenges arise in this problem setting. 
The first is the conflict between the need of timely data/model aggregation for \emph{regret minimization} and the need of \emph{communication efficiency}, since communication is the main bottleneck for many distributed application scenarios, e.g., communication in a network of mobile devices can be slower than local computation by several orders of magnitude \citep{huang2013depth}. A well-designed communication strategy becomes vital to strike the balance. 
In addition, 
% constraints from real-world applications should also be taken into consideration when designing the communication strategy. For example, 
the clients often have various response time and even occasional unavailability in reality, due to the differences in their computational and communication capacities.
% the clients may differ in their computational and communication capacities. This will lead to various response time and even occasional unavailability. 
This hampers global synchronization employed in existing federated bandit solutions \citep{wang2019distributed,dubey2020differentially}, which requires the server to first send a synchronization signal to all clients, wait and collect their returned local updates, and finally send the aggregated update back to every client.
Second, it is very restrictive to only assume homogeneous clients, i.e., they solve the same learning problem. 
% As bandit algorithms are mostly deployed to interact with individual users, studying heterogeneous clients with personalized learning problems has a greater potential.
Studying \emph{heterogeneous clients} with distinct learning problems has a greater potential in practice.
This is referred to as ``\emph{non-IIDness}" of data in the context of federated learning, e.g., the difference in $\mathcal{P}_{i}(\bx,y)=\mathcal{P}_{i}(\bx) \mathcal{P}_{i}(y|\bx)$ is caused by each client $i\in[N]$ serving a particular user or group of users, a particular geographic region, or a particular time period. Apparently, it is also unreasonable to assume every client has equal amount of new observations, which however is assumed in existing works. 

%To be more concrete, due to the time-varying arm set $\cA_{t}$ and the dependence on history data for arm selection in linear bandit, context vector $X$ is non-IID in nature and is not the main concern. 
% It is not a major concern since the performance metric, i.e. regret $r_{t}$, is defined against the best arm in $\cA_{t}$. 

% For example, internet connection and the different computation power of devices.
% \textcolor{red}{reasons we need async algo}

% This naturally leads to the question: how to balance between regret minimization and communication efficiency in the federated linear bandit problem.
To address the first challenge, we propose an asynchronous event-triggered communication framework for federated linear bandit. 
%Our event-triggering mechanism offers a flexible way to balance between the regret-minimization and communication-efficiency dilemma. 
Communication with a client happens only when the last communicated update to the client becomes irrelevant to the latest one; and we prove only by then effective regret reduction can be expected in this client because of the communication. 
Under this asynchronous communication, each client sends local update to and receives aggregated update from the server independently from other clients, with no need for global synchronization. This improves our method's robustness against possible delays and temporary unavailability of clients. It also brings in reduced communication cost when the clients have distinct availability of new observations, because global synchronization requires every client in the learning system to send its local update despite the fact that some clients can have very few new observations since last synchronization.
% make the proposed method more robust and practical against the infrastructure constraints, because the aggregated update sent to each client is asynchronous and  
% This makes our method more robust against possible delays in the communication, and we prove that the client enjoys the same benefit in regret reduction as long as it receives the update before its next interaction with the environment.

To address the second challenge, we design algorithms for federated linear bandit with both ``\emph{IIDness}" and ``\emph{non-IIDness}" based on the proposed communication framework. We consider two different assumptions on the reward functions. First, all the clients share a common reward function i.e., a single model is learned for all clients. Second, each client has a distinct reward function with mutual dependence captured by globally shared components in the unknown parameter, which resembles 
%so one model per client is learned during the interaction with the environment, which in essence is similar to the problem considered in
federated multi-task learning \citep{smith2017federated}.
We rigorously prove the upper bounds of accumulative regret and communication cost for the proposed algorithms in these two settings, and conduct extensive empirical evaluations to demonstrate the effectiveness of our proposed framework.
% especially its flexibility in balancing the trade-off between regret and communication cost.


\section{Related Works}
\paragraph{Camera-based 3D Object Detection.} Camera-based 3D object detection has been widely used for applications like autonomous driving since its low cost compared with LiDAR-based detectors. FCOS3D \cite{b12} first predicts the 3D attributes of objects through the features around 2D centers and PGD \cite{b17} utilizes the relational graphs to improve the depth estimation for 3D monocular object detection. Further, MonoDETR~\cite{b47} introduces DETR-like~\cite{b18} architectures without complex post-processing.
Recently, Bird’s-Eye-View~(BEV), as a unified representation of surrounding views same as LiDAR-based detector, has attracted much attention. DETR3D \cite{b13} follows the DETR \cite{b18} to adopt the 3D reference points in BEV space by using object queries. BEVDet \cite{b19} utilizes the Lift-Splat-Shoot~(LSS) operation \cite{b20} to transform 2D image features into 3D Ego-car coordinate to generate 3D BEV feature. PETR \cite{b21} obtain the 3D position-aware ability by 3D positional embedding. Inspired by the recently developed attention mechanism, BEVFormer \cite{b11} and PolarFormer \cite{b22} automates the camera-to-BEV process with learnable attention modules and queries a BEV feature according to its position in 3D space. To further improve the detection performance, the temporal information has been introduced in BEVDet4D \cite{b23} and PETRv2 \cite{b24}, which achieve significant performance enhancement. Moreover, BEVDepth \cite{b7} observes that accurate depth estimation is essential for BEV 3D object detection supervised by projected LiDAR points. MonoDETR-MV~\cite{b48} proposes a depth-guided transformer for multi-view geometric cues, but predicts only foreground depth map without dense depth supervision. As a LiDAR-to-camera learning scehme, our TiG-BEV leverages the pre-trained LiDAR-based detector to improve the performance of camera-based detectors for multi-view BEV 3D object detection.

\paragraph{Depth Estimation.} Depth estimation is a classical problem in computer vision. These method can be divided into single-view depth estimation and multi-view depth estimation. Single-view depth estimation is either regarded as a regression problem of a dense depth map or a classification problem of the depth distribution. \cite{b26, b27, b28, b29, b30} generally build an encoder-decoder architecture to regress the depth map from contextual features. Multi-view depth estimation methods usually construct a cost volume to regress disparities based on photometric consistency \cite{b31, b32, b33, b34, b35, b62}. For 3D object detection, previous methods~\cite{b61,b51,b52} also introduce additional networks for depth estimation to improve the localization accuracy in 3D space.
Notably, MonoDETR \cite{b47,b48} proposes to only predict the foreground depth maps instead of the dense depth values, but cannot leverage the advanced geometries provided by LiDAR modality. Different from them, our TiG-BEV conducts inner-depth supervision that captures local sptial structures of different foreground targets.

\paragraph{Knowledge Distillation.} Knowledge Distillation has shown very promising ability in transferring learned representation from the larger model (teacher) to the smaller one (student). Prior works \cite{b37, b38, b39, b40} have been proposed to help the student network learn the structural representation for better generalization ability. These methods generally utilize the correlation of the instances to describe the geometry, similarity, or dissimilarity in the feature space. The following methods extend the teacher-student paradigm to many vision task, demonstrating its effectiveness including action recognition \cite{b41}, video caption \cite{b42}, 3D representation learning~\cite{b59,b49,b50,b60}, object detection \cite{b43, b44} and semantic segmentation \cite{b45, b46}. However, only a few of works consider the multi-modal setting between different sensor sources. For 3D representation learning, I2P-MAE~\cite{b49} leverages masked autoencoders to distill 2D pre-trained knowledge into 3D transformers. UVTR \cite{b25} presents a simple approach by directly regularizing the voxel representations between the student and teacher models. BEVDistill \cite{b9} transfer knowledge from LiDAR feature to the cam feature by dense feature distillation and sparse instance distillation. Our TiG-BEV also follows such teacher-student paradigm and effectively distills knowledge from the LiDAR modality into the camera modality,


%However, they overestimated the prior of spatial order while neglected the issues of semantic mismatch, \ie, the pixels of teacher feature map often contains richer semantic compared to that of student on the same spatial location. We found that some works~\cite{Park2019RelationalKD, Passalis2018LearningDR,Peng2019CorrelationCF,Tung2019SimilarityPreservingKD,Yim2017AGF,huang2017like,Liu2021ICKD}, though unintended, have been proposed to relax the spatial constrain during feature transfer. Typically, they defined the relational graph, and similarity matrix in the feature space of teacher network and transferred it to the student network. For instances, Tung and Mori~\cite{Tung2019SimilarityPreservingKD} calculated the similarity matrix where each entry encoded the similarity between two instances. Liu \etal~\cite{Liu2021ICKD} measured the correlation between channels by inner-product. They condensed and compressed the entire feature to some properties (often scalar) and thus collapsed the spatial information. On the other hand, such process damaged the original teacher feature and may lead to sub-optimal solution. 

%The spread of KD has also driven some methods designed for specific vision tasks including video captioning~\cite{pan2020spatio}, action recognition~\cite{wang2019progressive,cui2020knowledge}, object detection~\cite{chen2017learning,zhang2020improve,dai2021general} and semantic segmentation~\cite{liu2019structured,he2019knowledge,Wang2020IntraclassFV}. Regarding the semantic segmentation, these methods are indeed related to relation knowledge distillation which computes similarity matrix~\cite{Tung2019SimilarityPreservingKD}. To investigate the potential of our method, we also adapt the method to semantic segmentation with hierarchical distillation.



\section{Experiment Setup}

\subsection{Model aspect ratio}\label{section:method:ratio}

For our Transformer models we fix the number of embedding features, sequence features, attention features, and the hidden layer dimension in the FFNs for each task.
We vary the number of layers, $L$, and the number of heads per layer, $H$, whilst keeping $L \times H$ constant.
Starting with typical values for $L$ \& $H$, we then move down to a single layer with one to two intermediate model aspect ratios, observing how test accuracy changes for trained models.
See \Cref{fig:wide} for an illustration of our deepest and widest models.

In all of our tasks we do not use pretrained embeddings or pretrained model parameters as this allows us to make a fairer comparisons.
Computing pretrained embeddings and weights that are optimised for each combination of attention and model aspect ratio would be computationally prohibitive, and using ones typically used for deep networks would introduce bias.


\subsection{Datasts and models}\label{section:method:datasets}

\begin{table}[!h]
    \caption{The different tasks and datasets used.}
    \label{table:tasks}
    \begin{center}
        \begin{tabular}{l | l l l l}
            \toprule
            \textbf{Task Name} & \textbf{Classification} & \textbf{Dataset} & \textbf{Input Type} & \textbf{Input Length} \\
            \midrule
            IMDb Token Level & Binary & IMDb Reviews & Review text tokens & 500 \\
            IMDb Byte Level & Binary & IMDb Reviews & Review text bytes & 1000 \\
            Listops & 10-way & LRA Listops & Listop bytes & 2000 \\
            Document Matching & Binary & ACL Anthology & Document bytes & 4000 \\
            \bottomrule
        \end{tabular}
    \end{center}
\end{table}

Primarily we investigate using 4 different text classification tasks, a vision based task is investigated in \Cref{sec:discussion:vit}.
The first two are sentiment analysis (binary classification) on the IMDb dataset.
One uses input embeddings at the token level with an input sequence length of 500, and the other uses input embeddings at the byte level and an input sequence length of 1k.
This second task is taken from LRA \citep{lra}, as are the final two.
The third task is Listops 10-way classification with a sequence length of 2k.
This task involves reasoning about sequences of hierarchical operations to determine a result, and the input is given at the byte level.
The final task used is byte level document matching, a binary classification task with a sequence length of 4k.
This uses the ACL anthology network for related article matching \citep{acl}.
We summarise each task in \Cref{table:tasks}, further details on them can be found in \citet{lra}.

For the text classification and Listops task we try four different model aspect ratios.
In terms of number of layers and heads per layer these are: 6 layers, 8 heads; 3 layers, 16 heads; 2 layers, 24 heads; and finally 1 layer, 48 heads.
As the matching task has an input sequence length of 4k, the models used are smaller to offset the computation size involved.
Thus the combinations we use are: 4 layers, 4 heads; 2 layers, 8 heads; 1 layer, 16 heads.

In order to investigate whether the type of the attention mechanism influences the effects of widening the attention layer, we test on 10 different types of Transformer attention, including the original dot-product attention \citep{tfm}.
The others are: Bigbird \citep{bigbird}, Linear Transformer \citep{linear_tfm}, Linformer \citep{linformer}, Local attention \citep{local_tfm}, Longformer \citep{longformer}, Performer \citep{performer}, Sinkhorn \citep{sinkhorn}, Sparse Transformer \citep{sparse_tfm}, and Synthesizer \citep{synthesizer}.
The implementations and hyper-parameter choices for each attention type are the same as used in LRA.
Unlike LRA, we do not test with Reformer \citep{reformer} due to it requiring the sequence features and attention features to have the same dimension. Training and other Transformer hyperparameters used for each task are given in \Cref{appendix:tasks}.




% \begin{figure*}[ht]
% % \vspace{-4mm}
% \centering     %%% not \center
% \subfigure[Homogeneous (uniform)]{\label{fig:a}\includegraphics[width=0.32\textwidth]{imgs/regretVScommCost_uniform_noclutter.png}}
% \subfigure[Homogeneous (non-uniform)]{\label{fig:b}\includegraphics[width=0.32\textwidth]{imgs/regretVScommCost_nonuniform_noclutter.png}}
% \subfigure[Heterogeneous clients]{\label{fig:c}\includegraphics[width=0.31\textwidth]{imgs/sim_hetero_30000.png}}
% % \vspace{-3mm}
% \medskip
% % \vspace{-2.5mm}
% \subfigure[LastFM ($N=1892$)]{\label{fig:d}\includegraphics[width=0.32\textwidth]{imgs/regretVScommCost_lastfm_noclutter.png}}
% \subfigure[Delicious ($N=1867$)]{\label{fig:e}\includegraphics[width=0.32\textwidth]{imgs/regretVScommCost_delicious_noclutter.png}}
% \subfigure[MovieLens ($N=54$)]{\label{fig:f}\includegraphics[width=0.32\textwidth]{imgs/regretVScommCost_movielens_noclutter.png}}
% % \vspace{-2.5mm}
% \caption{Experiment results on synthetic and real-world recommendation datasets.}
% % \vspace{-1mm}
% \end{figure*}

% \begin{figure*}[ht]
% \vspace{-2mm}
% \centering
% \begin{tabular}{c c c}
% \includegraphics[width=0.32\textwidth]{imgs/regretVScommCost_uniform_noclutter.png} &
% \includegraphics[width=0.32\textwidth]{imgs/regretVScommCost_nonuniform_noclutter.png} &
% \includegraphics[width=0.31\textwidth]{imgs/sim_hetero_30000.png}\\
% \small (a) Homogeneous (uniform) \normalsize & \small (b) Homogeneous (non-uniform) \normalsize & \small (c) Heterogeneous clients \normalsize\\
% % \includegraphics[width=0.3\textwidth]{imgs/real_lastfm_legend.png} &
% \includegraphics[width=0.31\textwidth]{imgs/regretVScommCost_lastfm_noclutter.png} &
% % \includegraphics[width=0.3\textwidth]{imgs/real_delicious_legend.png} &
% \includegraphics[width=0.31\textwidth]{imgs/regretVScommCost_delicious_noclutter.png} &
% % \includegraphics[width=0.3\textwidth]{imgs/real_movielens_legend.png}\\
% \includegraphics[width=0.31\textwidth]{imgs/regretVScommCost_movielens_noclutter.png}\\
% \multicolumn{1}{c}{\small(d) LastFM ($N=1892$)\normalsize} & \multicolumn{1}{c}{\small(e) Delicious ($N=1867$)\normalsize} & \multicolumn{1}{c}{\small(f) MovieLens ($N=54$)\normalsize}
% \end{tabular}
% \caption{Experiment results on synthetic and real-world recommendation datasets.} \label{fig:exp_result}
% % \vspace{-2mm}
% \end{figure*}

% \begin{figure*}[ht]
% \centering
% \begin{tabular}{c c c}
% \includegraphics[width=5.2cm]{imgs/real_lastfm.png} &
% \includegraphics[width=5.2cm]{imgs/real_delicious.png} &
% \includegraphics[width=5.2cm]{imgs/real_movielens.png} \\
% \small (a) LastFM ($T=96733, N=1892$)  & (b) Delicious ($T=104799,N=1867$) & \small (b) MovieLens ($T=214729,N=54$)
% \normalsize
% \end{tabular}
% \caption{Normalized accumulative reward and communication cost on real-world datasets.} \label{fig:realworld}
% \end{figure*}



% \begin{figure*}[ht]
% \centering
% \begin{tabular}{c c c}
%  \multicolumn{1}{c}{\includegraphics[width=0.3\textwidth]{imgs/sim_homo_50000_marker_repeat_legend.png}} &
%  \multicolumn{1}{c}{\includegraphics[width=3cm]{imgs/legend.png}} &
%  \multicolumn{1}{c}{\includegraphics[width=0.3\textwidth]{imgs/sim_hetero_50000_marker_repeat_legend.png}}\\
% \multicolumn{1}{c}{\small(a) Homogeneous clients ($T=50000,N=1000$)\normalsize} &
% % & \multicolumn{2}{c}{$\quad $} 
% & \multicolumn{1}{c}{ \small(b) Heterogeneous clients ($T=50000,N=1000$)\normalsize} \\
% \multicolumn{1}{c}{\includegraphics[width=5.3cm]{imgs/real_lastfm.png}} & \multicolumn{1}{c}{\includegraphics[width=5.4cm]{imgs/real_delicious.png}} & \multicolumn{1}{c}{\includegraphics[width=5.3cm]{imgs/real_movielens.png}} \\
% \multicolumn{1}{c}{\small(c) LastFM ($T=96733, N=1892$)\normalsize} & \multicolumn{1}{c}{\small(d) Delicious ($T=104799,N=1867$)\normalsize} & \multicolumn{1}{c}{\small(e) MovieLens ($T=214729,N=54$)\normalsize} \\
% \end{tabular}
% \caption{Experiment results on synthetic and real-world recommendation datasets.} \label{fig:exp_result}
% \end{figure*}

\section{Experiments}\label{sec:exp}
We performed extensive empirical evaluations of \modelone{} and \modeltwo{} on both synthetic and real-world datasets (we set $\gamma_{U}=\gamma_{D}=\gamma$ in all experiments), and included \modelbaseline{} \citep{wang2019distributed} as baseline. 
% Due to the space limit, we provide a brief summary of the experiment setup and results. Detailed descriptions and discussions are presented in the appendix.
\subsection{Experiments on Synthetic Dataset}
To validate our theoretical analysis in Section \ref{subsec:async_LinUCB} and Section \ref{subsec:async_LinUCB_AM}, two sets of simulation experiments were conducted.
We first conducted simulation experiment in homogeneous client setting to
% compare how well the algorithms can balance regret $R_{T}$ and communication cost $C_{T}$ under uniform and non-uniform client distributions, as well as 
validate our theoretical comparison between \modelone{} and \modelbaseline{} (see Section \ref{subsec:async_LinUCB} and Section \ref{sec:tb_theretical} in appendix), i.e., how well the algorithms balance regret $R_{T}$ and communication cost $C_{T}$ under uniform and non-uniform client distributions. Then we conducted simulation experiment in heterogeneous setting to validate our regret upper bound for \modeltwo{} (see Section \ref{subsec:async_LinUCB_AM}), i.e., how the portion of global components $\frac{d_{g}}{d_{g}+d_{i}}$ in the bandit parameter affects the regret of \modeltwo{}.
\subsubsection{Synthetic dataset.} We simulated the federated linear bandit problem setting in Section \ref{subsec:problem_formulation}, with $T=30000, N=1000$, and $\cA_{t}$ ($K=25$) uniformly sampled from a $\ell_2$ ball.
% is sampled from a pool of $1000$ randomly generated arms. 
% Two sets of simulation experiments were performed, with 
(1) Homogeneous clients: 
To compare how the algorithms balance $R_{T}$ and $C_{T}$ under uniform ($P(i_{t}=i)=\frac{1}{N},\forall i,t$) and non-uniform client distributions ($P(i_{t})$ is an arbitrary point on probability simplex), we fixed $d=25$, and ran \modelone{} and \modelbaseline{} with a large range of threshold values (logarithmically spaced between $10^{-2}$ and $10^{3}$).
% To compare algorithms' regret and communication cost under different trade-off settings over time, we fix $d=25$, $L=S=1$, and run \modelone{} with $\gamma \in \{1,2,5,8,+\infty\}$, \modelbaseline{} with threshold $D \in \{\frac{T}{d N^{2} \log{T}}, \frac{T}{d N^{1.5} \log{T}}\}$ (see Remark \ref{rmk:regret}) with both uniformly and non-uniformly sampled clients.
(2) Heterogeneous clients: To see how $R_{T}$ and $C_{T}$ of \modeltwo{} changes as the portion of global components change, we set the local components for all clients to have equal dimension, i.e., $d_{i}=d_{l},\forall i \in [N]$, fixed $d_{g}+d_{l}=25$, $\forall i\in [N]$, and then ran \modeltwo{} (with $\gamma=5$) under varying $d_{g} \in \{4,8,12,16,20,24\}$.

\subsubsection{Experiment results.}
Experiment results (averaged over $10$ runs) on synthetic dataset are shown in Figure
% \ref{fig:exp_result}(a)-(c):
\ref{fig:a}-\ref{fig:c}. 
Note that in the scatter plots, 
% the x-axis is the total communication cost at iteration $T$ and the y-axis is the corresponding accumulative regret/reward at iteration $T$. 
each dot denotes the cumulative communication cost (x-axis) and regret (y-axis) that an algorithm (\modelone{} or \modelbaseline{}) with certain threshold value (labeled next to the dot) has obtained at iteration $T$.

\begin{figure}
\centering     %%% not \center
\subfigure[Homogeneous (uniform client distribution)]{\label{fig:a}\includegraphics[width=0.55\textwidth]{imgs/regretVScommCost_uniform_noclutter.png}}
\vspace{-1mm}
\subfigure[Homogeneous (non-uniform client distribution)]{\label{fig:b}\includegraphics[width=0.55\textwidth]{imgs/regretVScommCost_nonuniform_noclutter.png}}
\vspace{-1mm}
%\medskip
\subfigure[Heterogeneous clients]{\label{fig:c}\includegraphics[width=0.55\textwidth]{imgs/sim_hetero_30000.png}}
% \vspace{-1mm}
% \subfigure[LastFM ($N=1892$)]{\label{fig:d}\includegraphics[width=0.48\textwidth]{imgs/regretVScommCost_lastfm_noclutter.png}}
% \medskip
% \subfigure[Delicious ($N=1867$)]{\label{fig:e}\includegraphics[width=0.48\textwidth]{imgs/regretVScommCost_delicious_noclutter.png}}
% \subfigure[MovieLens ($N=54$)]{\label{fig:f}\includegraphics[width=0.48\textwidth]{imgs/regretVScommCost_movielens_noclutter.png}}
\vspace{-1mm}
\caption{Experiment results on synthetic dataset.}
\end{figure}

(1) Homogeneous clients (Figure \ref{fig:a}-\ref{fig:b}): 
From both Figure \ref{fig:a} and Figure \ref{fig:b}, we can see that as the threshold value increases, $C_{T}$ decreases and $R_{T}$ increases, and that the use of event-triggered communication significantly reduces $C_{T}$ while attaining low $R_{T}$, compared with synchronizing all the clients at each time step (\modelone{} with $\gamma=1$). In Figure \ref{fig:a}, \modelbaseline{} has lower $C_{T}$ than \modelone{} under the same $R_{T}$, and in Figure \ref{fig:b}, \modelone{} has lower $C_{T}$ than \modelbaseline{} under the same $R_{T}$, which conform with our theoretical results that \modelbaseline{} has inefficient communication under non-uniform client distribution.
% Among all the algorithms compared, \modelone{} with $\gamma=5$ and $\gamma=8$ and \modelbaseline{} with $D=\frac{T}{d \log{T}}$ strike a good balance between regret and communication cost. The other algorithms either incur a very high regret, i.e. \modelone{} with $\gamma=+\infty$ (as no communication can be triggered), or incur too much communication cost, i.e., \modelone{} with $\gamma=1$ and \modelbaseline{} with $D=\frac{T}{d N \log{T}}$ (as they always communicate). And almost in all results, both synthetic and real-world datasets, always communicating (i.e., \modelone{} with $\gamma=1$) costs significant overhead in communication, while it does not necessarily lead to an obvious advantage in regret. This suggests the necessity and benefit of our event-triggered communication control. In addition, the results for \modelbaseline{} with $D=\frac{T}{d N \log{T}}$ and $D=\frac{T}{d \log{T}}$ conform with our theoretical analysis in Remark \ref{rmk:regret_comm}, e.g., it incurs higher regret when matching the communication cost with \modelone{} or higher communication cost when matching the regret with \modelone{}. 
% \modelone{} with $\gamma=1$ and \modelbaseline{} with $D=\frac{T}{d N \log{T}}$ attain the lowest regret, but also incur much higher communication costs compared with other algorithms.

(2) Heterogeneous clients (Figure \ref{fig:c}): By increasing the portion of global components $\theta^{g}$ in the bandit parameter, we can observe a clear trend in both regret and communication cost, i.e., the regret keeps decreasing while the communication cost keeps increasing. This validates our theoretical analysis about $R_{T}$ and $C_{T}$ in Section \ref{subsec:async_LinUCB_AM}. With $d_{g}$ increases and $d_{l}$ decreases, the first term in the upper bound of $R_{T}$ dominates (which grows slower w.r.t. $N$ compared with the second term), leading to the decreased regret, but the communication cost would increase since $C_{T}=O(d_{g} N \log{T})$.

\subsection{Experiments on Real-world Dataset}
% From the experiments on synthetic dataset, we have validated our upper bounds on the regret and communication cost of \modelone{} and \modeltwo{}. 
%However, we should note that our theoretical results are based on a slightly stronger assumption on the context vectors (see Assumption \ref{assump:context_diversity}), compared with \modelbaseline{}. 
%Therefore, we want to validate whether such an assumption is reasonable in practice, i.e., whether our algorithms can still work properly on the real-world datasets that contain TF-IDF feature vectors extracted from the tags and metadata of the items. 
We continue investigating the effectiveness of our proposed solution on real-world datasets. Note that these real-world datasets do not necessarily satisfy the assumption that all the clients are homogeneous, in other words not all the users have the same preference, we pay special attention to \modeltwo{} in the comparison, by setting $x_{g} \equiv x_{i}, \forall i \in [N]$, as mentioned in Section \ref{subsec:problem_formulation}. This allows the clients to learn a global model collaboratively, and in the meantime each learns a personalized model independently. Intuitively, this should make \modeltwo{} more robust to different settings, i.e., the clients are either homogeneous or heterogeneous.
\subsubsection{Real-world dataset.}
We compared \modelone{}, \modeltwo{} and \modelbaseline{} on three public recommendation datasets: LastFM, Delicious and MovieLens \citep{Cantador:RecSys2011,harper2015movielens}, with various threshold values (logarithmically spaced between $10^{-2}$ and $10^{3}$). 
The LastFM dataset contains $N=1892$ users, 17632 items (artists), and $T=96733$ interactions. We consider the ``\textit{listened artists}'' in each user as positive feedback. The Delicious dataset contains $N=1861$ users, 69226 items (URLs), and $T=104799$ interactions. We treat the bookmarked URLs in each user as positive feedback. 
The MovieLens dataset used in the experiment is extracted from the MovieLens 20M dataset by keeping users with over $3000$ observations, which results in a dataset with $N=54$ users, 26567 items (movies), and $T=214729$ interactions. We consider all items with non-zero ratings as positive feedback. The datasets were preprocessed following the procedure in \cite{cesa2013gang} to fit the linear bandit setting (with TF-IDF feature $d=25$ and arm set $K=25$).
% When applying \modeltwo{}, we assumed $x_{g} \equiv x_{i}, \forall i \in [N]$, as mentioned in Section \ref{subsec:problem_formulation}.
% We compare \modelone{} (with $\gamma \in \{1,5,+\infty\}$), \modeltwo{} (with $\gamma=5$) and \modelbaseline{} (with $D \in \{\frac{T}{d N \log{T}}, \frac{T}{d \log{T}}\}$) on three public recommendation datasets: LastFM, Delicious and MovieLens \cite{Cantador:RecSys2011,harper2015movielens}, with normalized reward (by a random strategy) and communication cost reported in Figure \ref{fig:exp_result}(c)-(e). When applying \modeltwo{}, we assume $x_{g} \equiv x_{i}, \forall i \in [N]$, as mentioned in Section \ref{subsec:problem_formulation}. The datasets are preprocessed following the procedure in \cite{cesa2013gang} to fit the linear bandit setting (with TF-IDF feature $d=25$ and arm set $K=25$).

% \subsubsection{Real-world dataset.}
% The three real-world datasets used in our experiment, i.e., LastFM, Delicious and MovieLens, are public recommendation datasets. Specifically, the LastFM dataset was extracted from the music streaming service Last.fm, and the Delicious dataset was extracted from the social bookmark sharing service Delicious. They were made availalbe by the HetRec 2011 workshop. The LastFM dataset contains $N=1892$ users, 17632 items (artists), and $T=96733$ interactions. We consider the ``\textit{listened artists}'' in each user as positive feedback. The Delicious dataset contains $N=1861$ users, 69226 items (URLs), and $T=104799$ interactions. We treat the bookmarked URLs in each user as positive feedback. 
% The MovieLens dataset used in the experiment is extracted from the MovieLens 20M dataset by keeping users with over $3000$ observations, which results in a dataset with $N=54$ users, 26567 items (movies), and $T=214729$ interactions.
% % that contains 20 million ratings with 27,000 movies and 138,000 users.
% % the MovieLens 20M dataset by keeping users with denser observations, which results in 54 users and 26567 items (movies).
% We consider all items with non-zero ratings as positive feedback in this dataset.
% On the LastFM and Delicious datasets, we extracted TF-IDF feature vectors using tags associated with each item; and on the MovieLens dataset, we also used information from items' metadata, such as movie titles, genres, etc., in addition to the tags, to construct the context vector. We then applied PCA to the resulting TF-IDF feature vectors, and retained the first 25 principle components as the context vectors, i.e., $d=25$. Then we normalized all features to have a zero mean and unit variance in each dimension. To generate the interaction sequence for each user, at each time step when a particular user $i_{t}$ is served, the candidate arm pool for user $i_{t}$ is generated by keeping the item with positive feedback at this time step and sampling another 24 unrated/non-interacted items from this user, i.e., $K=25$.

\subsubsection{Experiment results.}
% \textbf{Results on real-world datasets}:
%First, from the results on all three datasets (Figure \ref{fig:d}-\ref{fig:f}), the performance of \modelone{} is as good as that of \modelbaseline{}, which indicates Assumption \ref{assump:context_diversity} is still reasonable with the TF-IDF feature vectors. Especially in the results on MovieLens dataset (Figure \ref{fig:f}), whose data conforms with our homogeneous clients assumption as we will see below, we can see the dots corresponding to \modelone{} and \modelbaseline{} have very similar patterns as that in the ideal simulation environment (Figure \ref{fig:a}).

Experiment results on the three real-world datasets are shown in Figure \ref{fig:d}-\ref{fig:f}. 
In the scatter plots, each dot denotes the cumulative communication cost (x-axis) and normalized reward by a random strategy (y-axis) that an algorithm (\modelone{}, \modeltwo{}, or \modelbaseline{}) with certain threshold value (labeled next to the dot) has obtained at iteration $T$.
To understand the results of these algorithms on the three real-world datasets, we can first look at how well the two extreme cases, \modelone{} with $\gamma=1$ (as the communication cost of this algorithm is outside of the figure, its result is illustrated as text label) and \modelone{} with $\gamma=+\infty$ perform. 

\begin{figure}
\centering     %%% not \center
% \subfigure[Homogeneous (uniform)]{\label{fig:a}\includegraphics[width=0.48\textwidth]{imgs/regretVScommCost_uniform_noclutter.png}}
% \subfigure[Homogeneous (non-uniform)]{\label{fig:b}\includegraphics[width=0.48\textwidth]{imgs/regretVScommCost_nonuniform_noclutter.png}}
% % \vspace{-1mm}
% \medskip
% \subfigure[Heterogeneous clients]{\label{fig:c}\includegraphics[width=0.48\textwidth]{imgs/sim_hetero_30000.png}}
\vspace{-2mm}
\subfigure[LastFM ($N=1892$)]{\label{fig:d}\includegraphics[width=0.55\textwidth]{imgs/regretVScommCost_lastfm_noclutter.png}}
\vspace{-2mm}
\subfigure[Delicious ($N=1867$)]{\label{fig:e}\includegraphics[width=0.55\textwidth]{imgs/regretVScommCost_delicious_noclutter.png}}
\vspace{-2mm}
\subfigure[MovieLens ($N=54$)]{\label{fig:f}\includegraphics[width=0.55\textwidth]{imgs/regretVScommCost_movielens_noclutter.png}}
% \vspace{-1mm}
\caption{Experiment results on real-world recommendation datasets.}
\end{figure}

(1) LastFM \& Delicious (Figure \ref{fig:d}-\ref{fig:e}): 
On both LastFM and Delicious datasets, \modelone{} with $\gamma=+\infty$ (illustrated as the red dot) attains very high reward, which suggests users in these two datasets have very diverse preferences, such that aggregating their data has a negative impact on the performance. 
Since the homogeneous clients assumption does not hold in this case, both \modelone{} and \modelbaseline{} perform as badly as the extreme case of \modelone{} with $\gamma=1$, which is especially true when the clients frequently communicate with each other, i.e., with lower threshold values.
% , and the more the clients communicate with each other, the lower rewards they will obtain, due to the mistakenly aggregated heterogeneous data. 
In comparison, \modeltwo{} attains relatively good performance even when the clients frequently communicate with each other, as it allows personalized models to be learned on each client. Note that on Delicious dataset, in the low communication/high threshold region (top left corner of Figure \ref{fig:e}), the reward of \modelone{} actually increases as communication increases. Our hypothesis is that, with high threshold, only the most active users contribute to global data sharing, and when the other less active clients download these data, the benefit from reduced variance outweighs the harm caused by the increased bias (due to user heterogeneity). However, with the threshold further reduced, many more clients are able to contribute to global data sharing, such that the global data would become so heterogeneous that it starts to hurt the overall performance. Additional experiment and visualization are given in appendix (Section \ref{sec:additional_exp}) to validate this hypothesis.

(2) MovieLens (Figure \ref{fig:f}): 
Note that on this dataset, \modelone{} with $\gamma=1$ attains very high reward, which indicates that the users share similar preferences, so that data aggregation over different users becomes vital for good performance. 
In this case, learning a personalized model on each client becomes unnecessary and slows down the convergence of model estimation, which leads to the lower accumulative reward of \modeltwo{} compared with the other two algorithms.
% And as \modeltwo{} requires each client to learn a personalized model, which becomes unnecessary in this case, 
% it slows down convergence of the model estimation and thus leads a lower reward compared with the other two algorithms.
However, we can see that \modeltwo{} can still benefit from collaborative model estimation, as it has a much higher accumulative reward than the extreme case of \modelone{} with $\gamma=+\infty$.
% \vspace{-2mm}
\section{Conclusion}
% \vspace{-2mm}
In this paper, we propose a novel target inner-geometry learning framework that enables the camera-based detector to inherit the effective foreground geometric semantics from the LiDAR modality. We first introduce an inner-depth supervision with target-adaptive depth reference to help the student learn better local geometric structures. Then, we conduct inner-feature distillation in BEV space for both channel-wise and keypoint-wise, which contributes to high-level inner-geometry semantics learning from the LiDAR modality. Extensive experiments are implemented to illustrate the significance of TiG-BEV for multi-view BEV 3D object detection. For future works, we will focus on exploring multi-modal learning strategy that can boost both camera and LiDAR modalities for unified real-world perception.
%In this paper, we propose a novel cross-modality geometry-aware distillation framework that enables the camera-based detector to aggregate the useful geometry component information transferred from lidar modality to predict more accurate 3D bboxes for multi-view 3d object detection. Our inner-geometry relative attention depth supervision transfer the structured spatial position knowledge of target to effectively improve the Network’s perception of the inner-geometry of each object. Our structured attention feature supervision can transfer more reliable relative spatial feature relationships from LiDAR-based detectors to improve the structured awareness ability of camera-based detectors. Our method is successfully involve the target inner-geometry aware to cross-modality distillation. We conduct thorough experiments to validate the effectiveness of the method and advance the state-of-the-art.

%\vspace{-1mm}
%\section{Discussion}
%\vspace{-2mm}
%\textbf{Potential negative societal impact.} Our method has no ethical risk on dataset usage and privacy violation as all the benchmarks are public and transparent.

%\textbf{Limitations.} There are some issues of interest that we would like to explore in the future: (1) Currently, we only select the last layer of the backbone network for distillation. It would be interesting to see the efficacy when multiple layers are get involved with distillation which has been explored by some works \cite{Zagoruyko2017PayingMA,chen2021distilling}. (2) Also, we didn't investigate the effectiveness on other applications like object detection, which may need to design the new objective to fit the nature of specific application. 

%\vspace{-3mm}
%\section*{Acknowledgement}
%\vspace{-2mm}
% This work was supported in part by Australian Research Council (ARC) Discovery Early Career Researcher Award (DECRA) under DE190100626, National Natural Science Foundation of China (NSFC) under Grant No.61976233, and ”Leading Innovation
% Team of the Zhejiang Province” (2018R01017).
% National Key Research and Development Program of China (Grant NO. 2020AAA0108104) and Alibaba Innovative Research (AIR) Program.
%This work was supported in part by National Natural Science Foundation of China (NSFC) under Grant No.61976233, National Key Research and Development Program of China (Grant NO. 2020AAA0108104), Australian Research Council (ARC) Discovery Early Career Researcher Award (DECRA) under DE190100626, and Alibaba Innovative Research (AIR) Program.
%%%%%%%%% REFERENCES
% \clearpage
{\small
\bibliographystyle{ieee_fullname}
\bibliography{egbib}
}

\end{document}
