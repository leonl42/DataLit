
% \begin{figure*}[t]
%   \centering
%   \includegraphics[width=\textwidth]{cvpr_2022/framework_4.pdf}
%   \vspace{-5mm}
%   \caption{Illustration of our framework. (a) \textbf{Target-aware Transformer}. Conditioned on the teacher feature and the student feature, the transformation map Corre. is computed and then applied on the student feature to reconfigure itself, which is then asked to minimize the L$_2$ loss with the corresponding teacher feature. (b) \textbf{Patch-group Distillation}. Both teacher and student features are to be sliced and rearranged as groups for distillation. By concatenating the patches within a group, we explicitly introduce the spatial correlation among the patches beyond the patches themselves. (c) \textbf{Anchor-point Distillation}. Each color indicates a region. We use average pooling to extract the \textit{anchor} within a local area of the given feature map, forming the new feature map of a smaller size. The generated anchor-point features will participate in the distillation.}
%   \label{fig:framework}
%   \vspace{-5mm}
% \end{figure*}

%-------------------------------------------------------------------------
%\usepackage{amsmath}
%\usepackage{autobreak}
\section{Method}
\label{sec:method}
% There is a temptation, though, to let the student perfectly mimic the teacher, which does not appear to be viable in reality because student is inferior to teacher in learning capacities. Rather, we introduce the cross-attention to assist the feature matching between student and teacher. The cross-attention is conditioned by a pair of teacher feature and student feature that reflects the context of two given features. In practice \cite{Cho2019OnTE,Ji2021ShowAA,Chen2020CrossLayerDW}, it's hard for the student to directly match (\ie reconstruct) a teacher layer. Prior methods used an attention \textit{weight} to regulate the loss function or decide the allocation of the teacher layers to student, which is a global manner. In contrast, our method allows the student to perform self-assessment through the cross-attention and be aware of the difference and similarity compared to teacher feature, which can facilitate the knowledge transfer. (See Figure~\ref{fig:framework} (a)).


% In this section, we will provide a detailed description of the proposed cross-modality distillation method. The overall architecture is illustrated by Fig. 1. Our architecture consists of three branch: Lidar-based detector branch, Multi-view based detector branch and the cross-modal supervision. We will introduce cross-modal supervision components in detail in the following sections. 

\begin{figure*}[t!]
  \centering
  % \vspace{0.2cm}
  \includegraphics[width=\textwidth]{cvpr_2022/iccv_fig4.drawio.png}
  % \vspace{-5mm}
  \caption{\textbf{Overall Framework of TiG-BEV,} which contains a pre-trained LiDAR-based detector as teacher, a camera-based detector as student, and a target inner-geometry scheme for cross-model learning. Our proposed learning paradigm effectively transfers the inner-geometry semantics of the LiDAR modality via two components, an inner-depth supervision (Section~\ref{sec:Inner-depth Supervision}) for foreground relative depth, and an inner-feature BEV distillation (Section~\ref{sec:Inner-feature BEV Distillation}) from both channel-wise and keypoint-wise.}
  \label{fig:framework}
  % \vspace{0.3cm}
  % \vspace{-5mm}
\end{figure*}

The overall architecture of TiG-BEV is shown in Figure~\ref{fig:framework}, which consists of three components: the student camera-based detector, the teacher LiDAR-based detector, and our proposed target inner-geometry learning scheme. In Section~\ref{sec:Baseline Models}, we first introduce the adopted baseline models. Then, we specifically illustrate the designs of TiG-BEV for inner-depth supervision in Section~\ref{sec:Inner-depth Supervision} and inner-BEV feature distillation in Section~\ref{sec:Inner-feature BEV Distillation}. Finally in Section~\ref{sec:overall_loss}, we present the overall loss of our TiG-BEV for LiDAR-to-camera learning.

\subsection{Baseline Models}
\label{sec:Baseline Models}
\paragraph{Student Camera-based Detector.}
By default, we adopt BEVDepth~\cite{b7} as our student camera-based detector for multi-view 3D object detection. 
Given the input multi-view images (normally 6 views for a scene), the student model first utilizes a shared 2D backbone and FPN module~\cite{b54} to extract the $C$-channel visual features $\{F_i\}_{i=1}^6$, where $F_i \in {\mathbb{R}^{C\times H_v\times W_v}}$, and $H_v, W_v$ denote the size of feature maps. These features are fed into a shared depth network to generate the categorical depth map~\cite{b51}, $\{D_i\}_{i=1}^6$, where ${D_i}\in {\mathbb{R}^{D\times H_v\times W_v}}$, where D denotes the pre-defined number of depth bins. During training, BEVDepth adopts dense depth supervision for the predicted depth maps, which projects the paired LiDAR input onto multi-view image planes to construct pixel-by-pixel absolute depth ground truth, $\{D_i^{gt}\}_{i=1}^6$, where ${D_i^{gt}}\in {\mathbb{R}^{1\times H_v\times W_v}}$. 
Then, following~\cite{b20}, the multi-view visual features are projected into a unified BEV representation via the predicted depth maps, which is further encoded by a BEV encoder, denoted as $F^{2d}_{\rm bev}\in {\mathbb{R}^{C\times H_{\rm bev}\times W_{\rm bev}}}$. Finally, the detection heads are applied on top to predict objects in 3D space. We represent the two basic losses of the student model as $\mathcal{L}_{\rm{depth}}^{A}$ and $\mathcal{L}_{\rm{det}}$, respectively denoting the Binary Cross Entropy loss for dense absolute depth values and the 3D detection loss.

\paragraph{Teacher LiDAR-based Detector.}
We select the popular LiDAR detector CenterPoint~\cite{b53} as the teacher for target inner-geometry learning. Given the input point cloud data, CenterPoint voxelizes into grid-based data and utilizes a 3D backbone to obtain the $C$-channel LiDAR BEV feature $F^{3d}_{\rm bev}\in {\mathbb{R}^{C\times H_{\rm bev}\times W_{\rm bev}}}$, which has the same feature size as $F^{2d}_{\rm bev}$ from the student detector. As the CenterPoint has been well pre-trained, $F^{3d}_{bev}$ can provide the student BEV feature with sufficient geometric and semantic knowledge, espeically in the target foreground areas. Note that the LiDAR-based teacher is merely required during training for cross-modal learning, and for inference, only multi-view images are token as input for the camera-based detector.


% To adapt the method to semantic segmentation, we introduce the hierarchical distillation, which is more computationally practical, to transfer the feature and long-range dependency, respectively. 

%This section first presents the general formulation of the proposed method in the scope of image classification. 
% We then explain the insights of the model. 
%In order to adapt to semantic segmentation, we introduce the hierarchical distillation consisting of patch-group and anchor-point distillation to transfer the  feature and global dependency respectively.

% \subsection{Absolute Depth Supervision}
% \label{sec:formulation}
% % Suppose that teacher and student are two \ky{differentiable} functions, which are parameterized by CNNs in this work and denoted by $T$ and $S$. 
% % Suppose the teacher and the student are two convolutional neural networks, denoted by $T$ and $S$.
% % $F^T\in {\mathbb{R}^{H\times W\times C}}$ and ${F^S}\in \mathbb{R}^{H\times W\times C^{'}}$ denote the teacher feature and student feature respectively, where $H$ and $W$ are the height and width of the feature map, and $C$ represents the channel numbers. In the pioneer work~\cite{Hinton2015DistillingTK}, the distillation loss is formulated by a distance of features that come from the last layer of the networks. For example, in the image classification domain, it refers to the  ``logits'' before going in the softmax layer and cross-entropy loss. 
% Following early work BEVDepth, we propose to supervise the intermediate depth prediction ${D_i}^{pred}$ using ground-truth ${D_i}^{gt}$ generated from point clouds data $P$. Specifically, we create the depth maps by the Depth Generator and detail the depth generation process below.

% Donated $R_i\in{\mathbb{R}^{3\times 3}}$ and $t_i\in{\mathbb{R}^{3}}$ as the rotation and translation matrix from the ego coordinate to the camera coordinate of the $i^{th}$ view, and donated $K_i\in{\mathbb{R}^{3\times 3}}$ as the intrinsic parameter of the $i^{th}$ camera. Then we can obtain ${D_i}^{gt}$ by:
% \begin{equation}
%     \hat{P_i^{img}}(u{D_i}^{gt}, v{D_i}^{gt}, {D_i}^{gt}) = K_{i}(R_i P + t_i),
% \end{equation}
% where $u$ and $v$ denote coordinates in pixel coordinate. 

% Then we apply one hot encoding to ${D_i}^{gt}$ considering the depth estimation as a classification problem of the depth distribution. For the defined depth bins $b_i$, we interpret the $D$ Softmax scores, $p^{k}_i$, $k=1,..,D$, and choose the final prediction depth $b^{k}_i$ at highest prediction confidence bin. Finally we adopt Binary Cross Entropy as absolute depth supervision loss to optimize our predicted depth distribution ${D_i}^{pred}$ as shown in Eq.~\ref{eq:BCE},
% \begin{equation}
%     \mathcal{L}_{\rm{Adepth}} = -\sum_{k=1}^D{(y^k\log(p^k) + (1 - y^k)\log(1 - p^k))}
%     \label{eq:BCE}
% \end{equation}
% where $y^k$ is either zero or one indicating whether $k$ is the discrete ground truth depth.

% With the help of absolute depth supervision, camera branch will be able to produce reliable ${D_i}^{pred}$. 
 %\noindent we finally can get GT depth map in image coordinates $P_i^{img}(u, v,  d)$, where $u$ and $v$ denote coordinates in pixel coordinate. If the 2.5D projection of a certain point cloud does not fall into the $i^{th}$ view, we simply discard it. See Fig.~\ref{fig:bevdepth} for an example of the projection result. Then, to align the shape between the projected point clouds and the predicted depth, a \textit{min pooling} and a \textit{one hot} are adopted on $P_i^{img}$. We jointly define these two operations as $\phi$, the resulting $D^{gt}$ can thus be written in Eq.~\ref{dgt}. As for the depth loss $L_{depth}$, we simply adopt Binary Cross Entropy. 


\vspace{0.1cm}
\subsection{Inner-depth Supervision}
\label{sec:Inner-depth Supervision}

In addition to the dense absolute depth supervision, we propose to guide the student model to learn the inner-depth geometries in different target foreground areas. As shown in Figure~\ref{fig:relative_depth} (a), for the instance level, the existing absolute depth supervision with categorical representation ignores the relative structural information inside each object and provide no explicit fine-grained depth signals. Therefore, we propose to additionally conduct inner-depth supervision with continuous values from the LiDAR projected depth maps shown in Figure~\ref{fig:relative_depth} (b), which effectively boosts the network to capture the inner-geometry of object targets.

\paragraph{Foreground Target Localization.}
To accurately obtain the inner-depth values, we first localize the foreground pixels for each object targets in the depth maps. Given the ground-truth 3D bounding boxes, we extract the corresponding 3D LiDAR points inside the box for each object target, and project them onto different image planes. In this way, we can attain the pixels within foreground object areas on both the predicted and ground-truth depth maps, $\{D_i, D_i^{gt}\}_{i=1}^6$. The foreground pixels can roughly depict the geometric contour of different target objects and well improve the subsequent inner-depth learning. We taking the $i$-th view as an example and omit the index $i$ in the following texts for simplicity. Suppose there exist $M$ target objects on the image, we denote the foreground depth-value set for the $M$ objects as $\{S_j, S_j^{gt}\}_{j=1}^M$, where each $\{S_j, S_j^{gt}\}$ includes the foreground categorical depth prediction and ground-truth depth values for the $j$-th target.

\paragraph{Continuous Depth Representation.}
Different from the categorical representation of absolute depth values, we represent the predicted inner depth of foreground targets by continuous values, which reflects more fine-grained geometric variations. For pixel $(x, y)$ of the $j$-th target object $S_j$, the predicted possibility of $k$-th depth bin is denoted as $S_j(x, y)[k]$, where $1\le k\le D$. Referring to MonoDETR~\cite{b47,b48}, we calculate the continuous depth value $d_j(x, y)$ for the pixel $(x, y)$ as
 \begin{equation}
 \label{eq:FM}
 \begin{aligned}
    d_j(x, y) = {\sum_{k=1}^D({d[k]\cdot S_j(x, y)[k]})},
 \end{aligned}
 \end{equation}
where $d[k]$ denotes the depth value of the $k$-th bin center. By this, we convert the categorical depth prediction of different target objects, $\{S_j\}_{j=1}^M$, into continuous representations, denoted as $\{\hat{S_j}\}_{j=1}^M$.

\begin{figure}[!t]
% \vspace{-0.5cm}
    \centering
    \includegraphics[scale=0.042]{cvpr_2022/iccv_fig5.drawio.png}
    \caption{\textbf{Comparison of Categorical Absolute Depth and Continuous Inner Depth.} We adopt the inner-depth supervision with continuous depth values to guide the camera-based student to learn local spatial structures of foreground object targets.
    }
    \label{fig:relative_depth}
    % \vspace{-0.8cm}
\end{figure}

%Start with $N$ ground truth 3D bounding boxes $\bm{B_{gt}}= \{\bm{b}_{i}\}_{i=1}^{N}$, which contains box center position, size and heading angle. We then obtain points in each 3D bounding boxes $\bm{P_{gt}}= \{\bm{P}^{pc}_i\}_{i=1}^{N}$, ${P}^{pc}_i$ means point clouds $P$ within 3d bounding box ${b}_{i}$. Take an object instance as example, we project ${P}^{pc}_i$ on image coordinate. 

%\begin{equation}
%    \hat{P_i^{}}(ud, vd, d) = K_{i}(R_i P_{i} + t_i),
%\end{equation}
 %In the section 3.1, we consider the depth estimation as a classification problem of the depth distribution. For the defined depth bins $b_i$, we interpret the $D$ Softmax scores, $p_k$, $k=1,..,D$, and choose the final prediction depth $b^{k}_i$ at highest prediction confidence bin. 

% In the absolute depth supervision, we apply one hot encoding to ${D_i}^{gt}$ considering the depth estimation as a classification problem of the depth distribution, first adopted in BEVDepth \cite{b7}. We divided the depth distance into D depth bins, the depth-bin-centers set denoted as ${\mathbf{B_i}=\{b^1_i, b^2_i,..., b^k_i,... b^D_i\}}$ , and we interpret the $D$ Softmax scores, $p^{k}_i$, $k=1,..,D$ at each pixel as probabilities over the ${\mathbf{B_i}}$ vector, and choose the final prediction depth $b^{k}_i$ at highest prediction confidence bin. 
%  \begin{equation}
%  \label{eq:FM}
%  \begin{aligned}
%     D_i = b^{k}_i {\max_{k=1}^D {{p^{k}_i}}}.
%  \end{aligned}
%  \end{equation}
% Here, we want to calculate the exact relative depth for different pixels within the target, the discrete depth prediction like above is not enough to express the tiny depth difference. Thus, for the further accurate prediction depth and can be used for the relative depth relation among different pixels within the certain ROI, we calculate the final depth value $\hat{D_i}$ by the hybrid mean regression as follow:
%  \begin{equation}
%  \label{eq:FM}
%  \begin{aligned}
%     {\hat{D_i}} = {\sum_{k=1}^D {{b^{k}_i}{p^{k}_i}}}.
%  \end{aligned}
%  \end{equation}
 % Compared the discrete depth prediction, we do not predict the depth as the chosen likely bin. This enables us to predict smooth depth without the discriminative artifacts, which helps to obtain the complete representation of the geometry depth.  
 
 \paragraph{Adaptive Depth Reference.}
 To calculate the relative depth values, we propose to utilize an adaptive depth reference for different foreground targets.
 Specifically, according to the predicted continuous depth values in $\{\hat{S_j}\}_{j=1}^M$, we select the pixel with the smallest depth prediction error as the reference point for each target, and correspondingly set its depth value as the depth reference, as shown in Figure~\ref{fig:relative_depth}. For the $j$-th target with the ground-truth inner-depth $\{\hat{S_j}, \hat{S^{gt}_j}\}_{j=1}^M$, we calculate the depth reference point $(x_r, y_r)$ by
  \begin{equation}
 \label{eq:FM}
 \begin{aligned}
    (x_r, y_r) = \mathop{\text{Argmin}}_{(x, y)\in \hat{S_j}} \left ({S^{gt}_j}(x, y) - {\hat{S_j}}(x, y)\right ).
 \end{aligned}
 \end{equation}
 Then, the predicted and ground-truth reference depth values are denoted as $d_j(x_r, y_r)$ and $d^{gt}_j(x_r, y_r)$, respectively. By adaptively selecting the reference point with the smallest error, the inner-depth distribution can dynamically adapt to objects with different shapes and appearances, which stabilizes the network learning for some truncated and occluded objects.

\paragraph{Inner-depth Calculation.}
On top of the reference depth value, we calculate the relative depth values within the foreground area of each target object. For pixel $(x, y)$ of the $j$-th target $\{\hat{S_j}, S_j^{gt}\}$, the predicted and ground-truth inner-depth values are formulated as
\begin{equation}
 \label{eq:FM}
 \begin{aligned}
    rd_j(x, y) &= d_j(x, y) - d_j(x_r, y_r),\\
    rd^{gt}_j(x, y) &= d^{gt}_j(x, y) - d^{gt}_j(x_r, y_r).
 \end{aligned}
 \end{equation}
 We denote the obtained relative depth-value sets for $M$ target objects as $\{\hat{R_j}, R_j^{gt}\}_{j=1}^M$. Finally, we supervise the inner-depth prediction of the student detector by an L2 loss, formulated as
 \begin{equation}
\label{eq:FM}
    \mathcal{L}_{\rm{depth}}^{R} = \sum_{j=1}^M ||\hat{R}_{j}-R^{gt}_{j}||_2.
\end{equation}
 
%  When we set the inner point and the reference point within the ROI, we can define the two relative depth sets ${\mathbf{\hat{D}}^{gt}}$ ${\mathbf{\hat{D}}^{pred}}$ of the pixels with cardinality ${M}$. 
% \begin{equation}
%     {\mathbf{\hat{D}}^{gt}} = 
% \left[{\hat{D}^{gt}_{1r},\hat{D}^{gt}_{2r},\cdots,\hat{D}^{gt}_{ir},\hat{D}^{gt}_{Mr}}\right].
% \end{equation}
% \begin{equation}
%     {\mathbf{\hat{D}}^{pred}} = 
% \left[{\hat{D}^{pred}_{1r},\hat{D}^{pred}_{2r},\cdots,\hat{D}^{pred}_{ir},\hat{D}^{pred}_{Mr}}\right].
% \end{equation}
%  where the ${\hat{D}_{ir}^{gt}}=\hat{D}_i^{gt}-\hat{D}_r^{gt}$ and ${\hat{D}_{ir}^{pred}=\hat{D}_i^{pred}-\hat{D}_r^{pred}}$ denotes the relative depth distance of the lidar pixels and cam pixels. 
 
%  To obtain the inner-geomtry depth relation from the lidar points, we minimize the discrepancy between the above sets in a one-to-one relative spatial matching manner. 
% \begin{equation}
% \label{eq:FM}
%     \mathcal{L}_{\rm{Rdepth}} = ||{\mathbf{\hat{D}}^{gt}}-{\mathbf{\hat{D}}^{pred}}||_2 = \sum_{i=1}^M ||\hat{D}^{gt}_{ir}-\hat{D}^{pred}_{ir}||_2.
% \end{equation}
%% \KY{focus on semantic mismatch}
%\noindent This formulation assumes that the semantic distributions of the teacher and the student match exactly. 
% However, a recent study~\cite{Cho2019OnTE} observes that small students are inefficient to mimic large teachers. 
%However, as mentioned earlier, for the feature maps of the teacher network, which usually encompasses more layers and larger feature channels, the spatial information of the same pixel location contains a richer semantic information compare to the student network. Directly regressing the features in a pixel-wise manner may lead to suboptimal distillation results. 
% has a stronger learning capability (\eg larger receptive field) and richer representation. That means the semantic of spatial components of teacher and student usually varies. Directly linking the teacher and student by spatial order may trigger the issues of semantic mismatch and lead to sub-optimal results.
% \KY{this is not helping the story, size difference is one reason that KD is hard. but our approach does not help in this way} 
% As the teacher grows in capacity and accuracy, the student often finds it difficult to emulate the teacher. 
% To this end, we propose to guide the whole student to mimic each spatial component of the teacher respectively. In this way, we can increase the matching capability and subsequently improve the knowledge distillation performance.

%This formulation does not consider the gap of expressivity and the exact semantic distance between $f^s_i$ and $f^t_i$, which may introduce bias. To address the limitation of Eq. \ref{eq:FM}, we reconfigure each elements in set $f^s$ by a target-aware transformer. 
%A straightforward solution is to measure the semantic distance between the elements of two sets and then assign the $f^t_i$ with the most semantic-related one from $f^s$. As we will demonstrate, this is the special case of our proposed method.
%To this end, we propose a one-to-all spatial matching knowledge distillation pipeline that allows the each feature location of the teacher to teach the entire student features in a dynamic manner.
%To make the whole student mimic a spatial component of the teacher, we propose the \textbf{T}arget-\textbf{a}ware \textbf{T}ransformer (\textbf{TaT}) to pixel-wisely reconfigure the semantic of student feature in the certain position.
%We propose the Target-aware Transformer (\textbf{TaT}) to pixel-wisely reconfigure the semantic of student feature in the certain position. 
%Given a spatial component (alignment target) of the teacher, we use \textbf{TaT} to guide the whole student to reconstruct the feature in its corresponding location. Conditioned on the alignment target,  \textbf{TaT} should reflect the semantic similarity with the components of the student feature. We use a linear operator to avoid changing the distribution of student semantics. The formulation of transformation operator $W^i$ can be defined as:
%Given the alignment target $f^t_i$, \textbf{TaT} is to find the weights $W^i$ that controls the flow of semantic aggregation across the student feature w.r.t the $i$-th pixel of student feature. Conditioned on the alignment target, \textbf{TaT} should reflect the semantic similarity with the components of the student feature. Also, it should be a linear operator otherwise it changes the distribution of student semantics. The formulation of $W^i$ can be defined as:
%\begin{equation}
%\label{eq:spe}
%\begin{aligned}
%   W^i&= \sigma(\langle {f^s_1},{f^t_i}\rangle,\langle {f^s_2},{f^t_i}\rangle,\dots,\langle {f^s_N},{f^t_i}\rangle)\\
%   &=[{w^i_1},{w^i_2},\dots,{w^i_N}],
%\end{aligned}
%\end{equation}
%\noindent where $f^t_i$ and $f^s_i$ denote the corresponding $i$-th components of teacher and student, $\langle \cdot,\cdot \rangle$ represents the inner-product and  $\|W^{i}\|=1$. We use inner-product to measure the semantic distance and softmax function for normalization. 
%Note that if we only reserve the entry of the maximum of $W^{'}$, it degrades to the nearest-neighbor. 
% Here $W^{i}$ is the gate that guides the semantic flow to the reconfigured point ${f^s_i}^{'}$. 



%Note this is the simple non-parametric method that only depends on the original features. To facilitate the training, we introduce the parametric method with the extra linear transformation applied on the student feature and teacher feature. We observe that parametric version performs better than non-parametric one in ablation study. Guided by the target-aware transformer, the reconfigured student feature can be formulated as: 
% \KY{why we need parameteric formulation? does non-param work? do we have the experiment? if not, consider to do this in supplementary}
% Also, the issue of semantic mismatching may occur in the channel dimension. To address this issue, we propose to partition the feature tensor along the channel dimension and performs the self-assembling in parallel:
%\begin{equation}
%\label{eq:mul-self-essem}
%    {f^s}^{'}=\sigma(\gamma(f^{s})\cdot \theta({f^t})^{\top})\cdot \phi(f^{s}),
%\end{equation}
%\noindent where $\theta(\cdot)$, $\gamma(\cdot)$ and $\phi(\cdot)$ are the linear functions consisting of $3\times 3$ conv layer plus the BN layer \cite{ioffe2015batch}. We compare the parametric \textbf{TaT} to non-parametric one to analyse the effectiveness brought by these linear functions in the Section~\ref{sec:ablation}. 
%In the case that the channel numbers of $F^S$ do not match with that of $F^T$, $\gamma(\cdot)$ can help with alignment.

% \KY{where? point the section} 
%The resulting \textbf{TaT} map ($\gamma(f^{s})\cdot \theta({f^t})^{\top}$) is of size $\mathbb{R}^{HW \times HW}$, which is acceptable considering that most classification networks have small feature map size on the top layers. On ResNet18, the spatial size of feature map in the 4-th block is, for example, $7\times7$.  \KY{discuss the complexity in the next section...}

%After reconfiguration, each component of ${f^s}^{'}$ aggregates the meaningful semantic from the original feature, which enhances the expressivity. We do not require the student to reconstruct the teacher feature in a pixel-to-pixel manner. Indeed, our model allows the student to act as a whole to mimic the teacher. The resulting ${f^s}^{'}$ is lately asked to minimize the L$_2$ loss with the teacher feature. The objective for \textbf{TaT} knowledge distillation can be given by:
%\begin{equation}
%    \mathcal{L}_{\rm{TaT}}= ||{f^s}^{'}-f^t||_2.
%    \label{eq:fm}
%\end{equation}

% \KY{this only apply to Cls? how about other loss? consider change cls $\rightarrow$ task. L_T -> $L_{TaT}$ }
%Finally, the total loss of our proposed method can be defined by: 

%\begin{equation}
%\label{eq:objective}
    %\mathcal{L}=\alpha\mathcal{L}_{\rm{Task}}+\beta\mathcal{L}_{\rm{KL}}+\epsilon\mathcal{L}_{\rm{TaT}},
%\end{equation}
%\noindent Here $\mathcal{L}_{\rm{Task}}$ can be any loss on the generic machine learning tasks. $\alpha$, $\beta$ and $\epsilon$ are the weight factors to balance the loss. 
%Empirically, we find that our model benefits from $\mathcal{L}_{\rm{KL}}$. However, the model can achieve state-of-the-art without the help of $\mathcal{L}_{\rm{KL}}$.
% , \ie, $\beta$ is set to 0. \KY{this is strange... why mentioning this if we set beta = 0??}

% \subsection{Empirical \& Theoretical Analysis}
% This section provides some intuition to the formulation discussed above. Without loss of generality, let's remove the linear functions $\theta(\cdot)$ and $\gamma(\cdot)$ and consider only one attention head. The Eq. \ref{eq:fm} can be expressed in another way:
% \begin{equation}
%     softmax(X\cdot Y^{\top})\cdot X=Y.
% \end{equation}
% \noindent Here $softmax(X\cdot Y)$ is the cross-attention matrix which is applied to the student feature $X$. The objective for the student is to reconstruct the teahcer feature. Denote the optimum solution to $X$ as $\hat{X}$. The non-trivial solution indeed requires that $softmax(\hat{X}\cdot Y^{\top})=I$ and $\hat{X}=Y$.

% Recall that each raw of $X$ and $Y$ corresponds to a pixel in the original feature tensor. By means of matrix multiplication, it calculates the inner-product of each paired pixels between student and teacher, resulting the cross-attention matrix. The inner-product between two pixels measures the similarity against difference, and it's normalized by the softmax function. Since the cross-attention matrix is required to be the identity matrix, this can be interpreted that the distance, reflected by the inner-product, of the associated positions between $f_s$ and $f_t$ should be as close as possible, otherwise distant. This is the necessary condition if $\hat{X}=Y$ holds.

% We now begin to give a theoretical analyse to the existence of the solution $\hat{X}$. Because student feature is expected to match the teacher feature, we have $\hat{X}=Y$. Thus, we need to prove that $softmax(Y,Y^{\top})=I$. We presume that the elements of $Y$ is Gaussian distribution and each raw is not linearly dependent from each other. Here $Y$ can be represented as:
% \begin{equation}
% \begin{aligned}
%   Y=[{y_1}^{\top},{y_2}^{\top},{y_3}^{\top},\dots,{y_N}^{\top}]^{\top},\\
% \end{aligned}
% \end{equation}
% \noindent where $Y$ has $N=H\cdot W$ raw vectors. Suppose that $y_i$ and $y_j$ are two distinct vectors, they can be described as:
% \begin{equation}
% \begin{aligned}
%   y_i=[y_{i,1},y_{i,2},y_{i,3},\dots,y_{i,C}],\\
%   y_j=[y_{j,1},y_{j,2},y_{j,3},\dots,y_{j,C}],\\
% \end{aligned}
% \end{equation}
% where $(1\leq i,j\leq N)$ and each vector is of length $C$. The expected value of the inner-product of two vectors can be given by:
% \begin{equation}
% \label{eq:in_prd_same}
%     \begin{aligned}
%       \mathbb{E}\langle y_i,y_i\rangle &= \mathbb{E}(y_{i,1}^2+y_{i,2}^2+y_{i,3}^2+\dots+y_{i,C}^2) \\
%       &=\sum_{k=1}^{C}\mathbb{E}y_{i,k}^2= \sum_{k=1}^{C}(\mu_{i,k}^2+\sigma_{i,k}^2)=C ,
%     \end{aligned}
% \end{equation}

% \begin{equation}
% \label{eq:in_prd_diff}
%     \begin{aligned}
%       \mathbb{E}\langle y_i,y_j\rangle 
%       &= \mathbb{E}(y_{i,1}\cdot y_{j,1}+y_{i,2}\cdot y_{j,2}+\dots+y_{i,C}\cdot y_{j,C}) \\
%       &=\sum_{k=1}^{C}\mathbb{E}(y_{i,k}\cdot y_{j,k})\\
%       &=\sum_{k=1}^C \left[ \mathbb{E}y_{i,k}\cdot \mathbb{E}y_{j,k}+Cov(y_{i,k},y_{j,k}) \right]  \\
%       &\leq \sum_{k=1}^{C}(\mu_{i,k}\cdot \mu_{j,k}+|\sigma_{i,k}\cdot \sigma_{j,k}|)\\
%       &=\rho \cdot C .
%     \end{aligned}
% \end{equation}
% Here $\mu=0$ and $\sigma=1$ is the mean and standard deviation of Gaussian distribution. The Eq. \ref{eq:in_prd_same} indicates the expected value of the inner-product between a vector and itself, while Eq. \ref{eq:in_prd_diff} illustrates the inner-product of two distinct vectors, which is derived by Cauchy–Schwarz inequality a.k.a covariance inequality. Here $0\leq \rho \leq 1$ and it equals to 1 if and only if two vectors are linearly dependent. Since we presume that vectors are not linearly dependent in matrix $Y$, we have $\rho<1$.

% Given Eq. \ref{eq:in_prd_same} and Eq. \ref{eq:in_prd_diff}, we consider the diagonal of the cross-attention matrix. Normalized with softmax function, the limiting condition of the $i$-th position of the $i$-th raw can be described by:
% \begin{equation}
% \label{eq:lim}
%     \lim_{C\to \infty} \frac{e^C}{(N-1)\cdot e^{\rho\cdot C}+e^C}=1.
% \end{equation}

% The limiting condition presented in Eq. \ref{eq:lim} means the $i$-th raw of the cross-attention map is the one-hot vector where $i$-th position is 1 as long as the feature channel is deep enough. Thus the resulting cross-attention matrix is an identity matrix. In our experiment setting, the feature tensor of 4-th layer in ResNet18 is of size $7\times 7 \times 512$, \ie $N=49$ and $C=512$. Even though $\rho$ reaches 0.98, Eq. \ref{eq:lim} can return 0.998 that is very close to 1.  

%--------------------------------------------------------------------------------

%--------------------------------------------------------------------------------
\subsection{Inner-feature BEV Distillation}
\label{sec:Inner-feature BEV Distillation}
%In this section we introduce the adaption of the model discussed previously and show its application on semantic segmentation. 
% \KY{inductive bias? do we really want to talk about this?}
% \ky{Although our one-to-all distillation approach can address the semantic mismatch, it has one limitation about the computational complexity. As the resulting correlation mapping }
% The proposed \textbf{TaT} lift the limitation of previous one-to-one spatial matching fashion. 
%For example, features in the neighborhood are more relevant to themselves, on the contrary, features that are farther away are less relevant. The student must figure out all of these in the learning process, which may be very challenging when the feature map is large.

Besides the depth supervision for low-level spatial information, our TiG-BEV also adopts the inner-geometry learning for high-level BEV semantics from pre-trained LiDAR-based detectors. 
Previous works~\cite{b9,b52} for BEV distillation directly force the student to imitate the teacher's features point-to-point in the BEV space. In spite of the performance improvement, such strategies are constrained by the following two aspects. On the one hand, due to the sparsity of scanned point clouds, the LiDAR-based BEV features might contain redundant and noisy information in the background areas. Although BEVDistill~\cite{b9} utilizes foreground masks to alleviate this issue, such dense feature distillation still cannot provide focused and effective guidance to the student network. On the other hand, the camera-based and LiDAR-based BEV features depict different characteristics of the scene, respectively, visual appearances and spatial structures. Therefore, forcing the BEV features to be completely consistent between two modalities is sub-optimal considering the semantic gap. In our TiG-BEV, we propose an inner-feature BEV distillation (Figure~\ref{fig:structure_attn}) consisting of inter-channel and inter-keypoint learning schemes, which conducts attentive target features distillation and relieve the cross-modal semantic gap.

% There are two main observations, on the one hand, the feature distribution of point clouds and images are not consistent due to the sparsity of point clouds, directly distilling on all region of features is not reasonable. On the other hand, even features of two modalities can hold meaningful information in the region of foreground, the representation of the features from different modalities are diverse in channel and spatial wise. Forcing students to imitate the foreground feature of the teacher is sub-optimal. Therefore, we propose a foreground structured attention feature supervision module to transfer more reliable relative spatial feature relationships from LiDAR-based teacher to multi-view based student. It contains of two distillation modules: 1) Inner Target-Aware distillation, 2) Local Target-Aware distillation,which..


\paragraph{Target Keypoint Extraction.}
To distill the knowledge of LiDAR-based detectors only within sparse foreground regions, we extract the BEV area of each object target and represent it by a series of keypoint features.
Given the ground-truth 3D bounding box for each target, we first enlarge the box size for a little bit in the BEV space to cover the entire foreground area, e.g., object contours and edges. Then, we uniformly sample its BEV bounding box by $N$ keypoints, and adopt bilinear interpolation to obtain the keypoint features from the encoded BEV representations. From both camera-based $F^{2d}_{\rm bev}$ and LiDAR-based $F^{3d}_{\rm bev}$, we respectively extract the keypoint features for all $M$ object targets as $\{f_j^{2d}, f_j^{3d}\}_{j=1}^M$, where $f_j^{2d}, f_j^{3d} \in {\mathbb{R}^{N\times C}}$. By the uniform sampling, such BEV keypoints can well represent the part-wise features and the inner-geometry semantics of foreground targets.

\begin{figure}[!t]
% \vspace{-0.5cm}
    \centering
    \includegraphics[scale=0.17]{cvpr_2022/iccv_fig6_2.drawio.png}
    \caption{\textbf{Detials of Innter-feature BEV Distillation.} For each foreground area in BEV space, we represent rach target feature by a set of keypoints and conduct feature distillation in both inter-channel and inter-keypoint manners.
    }
    \label{fig:structure_attn}
    % \vspace{-0.8cm}
\end{figure}

% and the corresponding background regions and divide into small grids with spatial resolution of $G_i\times G_i \times G_i$, which can summarize each foreground area into the certain number of feature keypoints.  For the bird-view feature maps, we project the keypoint $p_i$ to the 2D bird-view coordinate system and utilize bilinear interpolation to obtain the features $f^{bev}_i$ from the bird-view feature maps. Hence, the roi feature can be represented as follow:

% \begin{equation}
%     {F_i^{ROI}} = {f^{bev}_1,...,f^{bev}_i,f^{bev}_n}, 
%  i=1,..N
% \end{equation}
% which have the strong capability of preserving 3D geometry information of the foreground scene and can also boost the marginal awareness performance of the certain target.

\paragraph{Inter-channel BEV Distillation.}
% \label{sec:seq}
Taking the $j$-th object target as an example, we first apply an inter-channel BEV distillation, which guides the student keypoint features to mimic the channel-wise relationships of the teacher's. Such inter-channel signals imply the overall geometric semantics of each object target. Compared with the previous channel-by-channel supervision, our inter-channel distillation can preserve the distinctive aspects of the two modalities, while effectively transfer the well pre-trained knowledge of LiDAR-based detectors. Specifically, we calculate the inter-channel similarities of both camera-based and LiDAR-based keypoint features, formulated as
\begin{equation}
    A_j^{2d} = f_j^{2d} {f_j^{2d}}^{\top};\ \ \ A_j^{3d} = f_j^{3d} {f_j^{3d}}^{\top},
\end{equation}
where $A_j^{2d}, A_j^{3d} \in {\mathbb{R}^{C\times C}}$ denote the feature relationships between different $C$ channels for the two modalities. For all $M$ objects in a scene, we adopt L2 loss between the two inter-channel similarities for feature distillation, formulated as
\begin{equation}
    \mathcal{L}_{\rm{bev}}^{{IC}}= \sum_{j=1}^{M} ||A_j^{3d}-A_j^{2d}||_2.
    \label{eq:fm}
\end{equation}

% Given the RoI feature of each box proposal, the lidar-based bev feature and cam-based bev feature can be represented as $F_i^{Lidar}\in {\mathbb{R}^{N\times C}}$, $F_i^{cam}\in {\mathbb{R}^{N\times C}}$ respectively, where ${N}$ represents the number of keypoints and ${C}$ represents the channels numbers. For the certain sampled feature of the keypoint, it reflects the certain information within the receptive field. As mentioned before, the lidar-based detector has the better detection performance since the spatial information of the same inner roi location contains the richer semantic information compare to the cam-based detector. Thus, we propose a inner target-aware (\textbf{ITA}) distillation that performs distillation within channel wise, which allows student to learn the context of feature from RoI patches and retain the correlation among different channels. For the bev feature ${i}$th ROI, the channel-wise information matrix is defined as: 
% \begin{equation}
%     {G_i^{ROI}} = {F^{Lidar}_i\cdot {F^{Lidar}_i}}^{\top}
% \end{equation}
% The information matrix has a size of ${C\times C}$ regardless of the spatial dimension ${N}$. ${G_{i,(m,n)}^{ROI}}$ denotes the inner channel correlation between $m$th channel and $n$th channel for the same location of the $i$th ROI.

% \begin{equation}
% \begin{aligned}
% %L_{task} = &\lambda_1 L_{per}(G_s(x),G_t(x))  +\lambda_1  L_{CE}(y,\delta(z_s))\\ & + \lambda_1L_{Focal}(y,y_{out})
% %{G_{i,(m,n)}^{ROI}} = &{f^_{1,m}}{f^_{1,n}}^{\top}  +{f^_{2,m}}{f^_{2,n}}^{\top} + \dots + \\&{f^_{N,m}}\cdot {f^_{N,n}}^{\top}
% {G_{i,(m,n)}^{ROI}} = &{f^_{1,m}}\cdot {f^_{1,n}}^{\top} + {f^_{2,m}}\cdot {f^_{2,n}}^{\top}\\ & + \dots + \\{f^_{N,m}}\cdot {f^_{N,n}}^{\top}
% \end{aligned}
% \end{equation}


% For the same roi area, to make the whole cam-based detector mimic
% a spatial component of the lidar-based detector, we use the Inner-target aware operator ${G_{i,(m,n)}^{ROI}}$ to pixel-wisely reconfigure the
% semantic of cam-based feature in the certain position and transfer the channel correlation of the lidar-based feature to the cam-based feature in the certain position. Conditioned on the same roi, the ${G_{i,(m,n)}^{ROI}}$ should reflect the semantic similarity with between the components. 

% We penalize the $L_2$ distance between the channel-wise information matrix of the lidar-based detector and the cam-based detector, allowing the cam-based to obtain the similar feature diversity for the same foreground area.
% \begin{equation}
%     \mathcal{L}_{\rm{ITA}}= ||G^{Lidar}_{i,cha}-G^{cam}_{i,cha}||_2.
%     \label{eq:fm}
% \end{equation}


\begin{table*}[ht]
\centering
\caption{\textbf{Performance Comparison on nuScenes~\cite{b6} Val Set.} 'C' and 'L' denote the camera-based and LiDAR-based methods, which refer to the input data during inference. * denotes our implementation using BEVDet\cite{b19} codebase.
%using their official codes.
% We have at least 1 absolute point of performance gain against KD \cite{Hinton2015DistillingTK} on 5 out of 7 experimental settings.
}
%\vspace{-0.2cm}
\resizebox{2\columnwidth}{!}{
\tablestyle{5pt}{1.2}
\begin{tabular}{c|c|c|c|cc|ccccc}
\toprule[1.2pt]
Method          & Modality & Backbone & Resolution & mAP↑  &NDS↑& mATE↓ & mASE↓ & mAOE↓ & mAVE↓ & mAAE↓   \\ \midrule
FCOS3D\cite{b12}          & C  &ResNet-101 & 900 $\times$ 1600 & 0.343& 0.415 & 0.725 & 0.263 & 0.422 & 1.292 & 0.153  \\
PGD\cite{b17}             & C  &ResNet-101 & 900 $\times$ 1600 & 0.369& 0.428 & 0.683 & 0.260 & 0.439 & 1.268 & 0.185  \\
MonoDETR\cite{b47}             & C  &ResNet-101 & 900 $\times$ 1600 & 0.372& 0.434 & 0.676 & 0.258 & 0.429 & 1.253 & 0.176  \\
%BEVDepth-R50        & C &ResNet-50   & 256 \times 704  & 0.351 & 0.639 & 0.267 & 0.479 & 0.428 & 0.198 & 0.475 \\
% BEVDepth\cite{b7}         & C &ResNet-101  & 512 \times 1408 & 0.412 & 0.565 & 0.266 & 0.358 & 0.331 & 0.190 & 0.535 \\
DETR3D\cite{b13}          & C &ResNet-101  & 900 $\times$ 1600 & 0.303& 0.374 & 0.860 & 0.278 & 0.437 & 0.967 & 0.235  \\
PETR\cite{b21}            & C  &ResNet-101    & 512 $\times$ 1408 & 0.357 & 0.421& 0.710 & 0.270 & 0.490 & 0.885 & 0.224  \\
BEVFormer\cite{b11}       & C  &ResNet-101   & 900 $\times$ 1600 & 0.416 & 0.517& 0.673 & 0.274 & 0.372 & 0.394 & 0.198   \\
PETRv2\cite{b24}            & C  &ResNet-101    & 640 $\times$ 1600 & 0.421& 0.524 & 0.681 & 0.267 & 0.357 & 0.377 & 0.186  \\
MonoDETR-MV\cite{b48}            & C  &ResNet-101    & 640 $\times$ 1600 & 0.428 & 0.531 & 0.676 & 0.268 & 0.352 & 0.380 & 0.169  \\ \midrule
CenterPoint~\cite{b53} (Teacher)   & L &VoxelNet   & -          & 0.564 & 0.646& 0.299 & 0.254 & 0.330 & 0.286 & 0.191  \\ \midrule
BEVDet$^*$ \cite{b19} & C &ResNet-50   & 256 $\times$ 704 & 0.298& 0.379 & 0.725 & 0.279 & 0.589 & 0.860 & 0.245  \\ 
\rowcolor{gray!12} \textbf{+ TiG-BEV}     &C  &ResNet-50  & 256 $\times$ 704 & \textbf{0.331} & \textbf{0.411}& 0.678 & 0.271 & 0.589 & 0.784 & 0.218  \\ 
\rowcolor{gray!12}& &&& \textbf{\textcolor{blue}{+3.3$\%$} }& \textbf{\textcolor{blue}{+3.2$\%$}}&\textcolor{blue}{-4.7$\%$} & \textcolor{blue}{-0.8$\%$} & \textcolor{blue}{-0.0$\%$} & \textcolor{blue}{-7.6$\%$} & \textcolor{blue}{-2.7$\%$}   \\ 
\midrule
BEVDet4D$^*$ \cite{b23} & C &ResNet-50   & 256 $\times$ 704 & 0.322 & 0.451& 0.724& 0.277& 0.520 &0.366 &0.212  \\
\rowcolor{gray!12} \textbf{+ TiG-BEV}     & C &ResNet-50  & 256 $\times$ 704 & \textbf{0.356}& \textbf{0.477} & 0.648 & 0.273 & 0.517 & 0.364 & 0.210  \\ 
\rowcolor{gray!12}& &&& \textbf{\textcolor{blue}{+3.4$\%$} }& \textbf{\textcolor{blue}{+2.6$\%$}}&\textcolor{blue}{-7.6$\%$} & \textcolor{blue}{-0.4$\%$} & \textcolor{blue}{-0.3$\%$} & \textcolor{blue}{-0.2$\%$} & \textcolor{blue}{-0.2$\%$}   \\ 
\midrule
BEVDepth$^*$ \cite{b7} & C &ResNet-101   & 512 $\times$ 1408 & 0.416 & 0.521& 0.605 & 0.268 & 0.455 & 0.333 & 0.203  \\
\rowcolor{gray!12}   \textbf{+ TiG-BEV}   & C & ResNet-101  & 512 $\times$ 1408  & \textbf{0.440}& \textbf{0.544} & 0.570 & 0.267 & 0.392 & 0.331 & 0.201  \\ 
\rowcolor{gray!12}& &&& \textbf{\textcolor{blue}{+2.4$\%$} }& \textbf{\textcolor{blue}{+2.3$\%$}}&\textcolor{blue}{-3.5$\%$} & \textcolor{blue}{-0.1$\%$} & \textcolor{blue}{-6.3$\%$} & \textcolor{blue}{-0.2$\%$} & \textcolor{blue}{-0.2$\%$}   \\ 
\bottomrule[1.2pt]
\end{tabular}
}

\label{tab:nus_val_sota}
%\vspace{-3mm}
\end{table*}
\paragraph{Inter-keypoint BEV Distillation.}
\label{sec:anchor}
The inter-channel distillation guides the camera-based detector to learn the channel-wise diversity from the LiDAR-based teacher. However, it is conducted without considering the inner correlation of different keypoints within each object target, which is not capable of capturing the local geometries among different foreground parts, e.g., the front and rear of cars. To this end, we propose to utilize the inter-keypoint correlations of LiDAR-based BEV features and transfer such inner-geometry semantics into camera-based detectors. Analogous to the aforementioned inter-channel module, for the $j$-th target object, we calculate the inter-keypoint similarities in a transposed manner for the two modalities as
\begin{equation}
    B_j^{2d} = {f_j^{2d}}^{\top} {f_j^{2d}};\ \ \ B_j^{3d} = {f_j^{3d}}^{\top} {f_j^{3d}},
\end{equation}
where $B_j^{2d}, B_j^{3d} \in {\mathbb{R}^{N\times N}}$ denote the feature relationships between different $N$ keypoints respectively for camera and LiDAR. We also adopt L2 loss for all $M$ targets as
\begin{equation}
    \mathcal{L}_{\rm{bev}}^{{IK}}= \sum_{j=1}^{M} ||B_j^{3d}-B_j^{2d}||_2.
    \label{eq:fm}
\end{equation}
Then, the distillation loss for inter-channel and inter-keypoint features in BEV space is formulated as
\begin{equation}
    \mathcal{L}_{\rm{bev}}=
    \mathcal{L}_{\rm{bev}}^{{IC}}+
    \mathcal{L}_{\rm{bev}}^{{IK}},
    \label{eq:seg}
\end{equation}
where the two terms are orthogonal respectively for the channel-wise feature diversity and keypoint-wise semantic correlations.

% The attempt to preserve the local correlation through concatenating all the position would fail. 
% We hope that detector can capture the context from the one position to another within the target area. In this case, the cam-based detector will, however, be distracted from the locality since the \textbf{ITA} encodes all the related semantic over the whole feature among different positions. In other words, the \textbf{ITA} will aggregates redundant local semantic.
% Furthermore, a large feature map will hinder the inductive bias since it may encourage the student to integrate the less relevant semantic from remote positions by mistake, which may deteriorate the subsequent distillation performance.
% For complex scenes and similar targets, the postion wise dependency is important to capture the relation (\eg layout) of different components within the target.

% We address the conundrum by the proposed local target-aware(\textbf{LTA}) distillation. Like the the channel-wise information matrix defined in \textbf{ITA}, the position-wise matrix is defined as follow:
% \begin{equation}
%     {G_{i}^{pos}} = {{F^{Lidar}_i}^{\top}\cdot F^{Lidar}_i}
% \end{equation}

% The information matrix has the size of ${N\times N}$ regardless of the channel dimension ${C}$. ${G_{i,(p,q)}^{pos}}$ denotes the inner spatial correlation between $p$th keypoint and $q$th keypoint for the $i$th ROI. Similarly, the formulation of the operator can be defined as:

% \begin{equation}
% \begin{aligned}
%     {G_{i,(p,q)}^{pos}} = {f^_{p,1}}\cdot {f^_{q,1}}^{\top} + {f^_{p,2}}\cdot {f^_{q,2}}^{\top} + \dots + \\{f^_{p,C}}\cdot {f^_{q,C}}^{\top}
% \end{aligned}
% \end{equation}

% After reconfiguration, the spatial relationship within the target can be extracted and summerized, which is complementary to the channel relationship. Therefore, as the cam-based detector, we enhance its gemetry expressivity by asking them to mimic the teacher. Thus, the objective for (\textbf{LTA}) knowledge distillation can be given by:

% \begin{equation}
%     \mathcal{L}_{\rm{LTA}}= ||G^{Lidar}_{i, pos}-G^{Cam}_{i, pos}||_2.
%     \label{eq:fm}
% \end{equation}

\subsection{Overall Loss}
\label{sec:overall_loss}
To sum up, we benefit the student camera-based detector by target inner-geometry from two complementary aspects, i.e., an inner-depth supervision for low-level signals and an inner-feature BEV distillation for high-level semantics. They produce two losses as $\mathcal{L}^R_{\rm{depth}}$ and $\mathcal{L}_{\rm{bev}}$. Together with the original two losses, i.e., dense absolute depth supervision $\mathcal{L}^A_{\rm{depth}}$, and 3D detection $\mathcal{L}_{\rm{det}}$, the overall loss of our TiG-BEV is formulated as
\begin{equation}
    \mathcal{L}_{\rm{TiG}}=
    \mathcal{L}_{\rm{det}}+
    \mathcal{L}^A_{\rm{depth}}+
    \mathcal{L}^R_{\rm{depth}}+
    \mathcal{L}^{IC}_{\rm{bev}}+
    \mathcal{L}^{IK}_{\rm{bev}}.
    \label{eq:seg}
\end{equation}

% The inner target-aware distillation enables the detector to mimic the inner feature diversity while the local target-aware distillation allows it to learn the local representation over the spatial feature, which are complementary to each other. Therefore, the combination of these two objectives can bring the best of two worlds. Our objective designed for structured attention feature supervision can be written by:
% \begin{equation}
%     \mathcal{L}_{\rm{SAF}}=
%     \delta\mathcal{L}_{\rm{ITA}}+
%     \zeta\mathcal{L}_{\rm{LTA}}
%     \label{eq:seg}
% \end{equation}



% Finally, the total loss of our proposed method can be defined by:
% \begin{equation}
%     \mathcal{L}_{\rm{TiG}}=
%     \mathcal{L}_{\rm{det}}+
%     \mathcal{L}^A_{\rm{depth}}+
%     \mathcal{L}^R_{\rm{depth}}+
%     \mathcal{L}^{IC}_{\rm{bev}}+
%     \mathcal{L}^{IK}_{\rm{bev}}
%     \label{eq:seg}
% \end{equation}
% Here $L_{task}$ is the loss of the 3D detection task. $\alpha$, $\beta$, $\gamma$, $\delta$ and $\zeta$ are the weight factors to balance the loss.
