\section{Methodology}
\label{sec:Methodology}

{\bf Network architecture:}
We apply DeepReDuce to standard ResNet18/34 architectures as defined in~\cite{he2016deep}, and also, on the non-residual networks VGGNet \cite{simonyan2014very} and MobileNets \cite{howard2017mobilenets}. 
%Since the datasets we use (CIFAR-100 and TinyImageNet) have smaller resolution images than ImageNet, we do not perform down sampling after the Conv1 layer or the first stage .

To show DeepReDuce networks outperform shallower ResNets we trained
ResNet10 and ResNet9 in Table~\ref{tab:ShallowNetComp}.
In these networks we removed half of the residual Blocks in each stage of ResNet18
(each stage now has only one residual Block).
Furthermore, for ResNet9, there is only one $3\times3$ convolution layer in the first residual Block of $S_1$.  
All other comparisons with state-of-the-art use reported results from respective papers.
 
\begin{table} [t] \centering
\caption{Optimizations applied (Culling, Thinning, and Reshaping) to Pareto points in Figure \ref{fig:ParetoFrontier}. Stages with ``$^\star$'' have only one Block inside the ReLU-stages. Acc. is top-1 accuracy for ResNet18 on CIFAR-100, and Lat. is inference time in seconds.}
%\fixme{Lines 8 and 10 are the same.
%it seems like we're just searching and not following the method.}}
\label{tab:ParetoPoints} 
\resizebox{0.5\textwidth}{!}{
\begin{tabular}{cccccccc}\toprule
\multirow{2}{*}{ Culled } & \multirow{2}{*}{ Thinning } & \multicolumn{2}{c}{ ReLU Reshaping } & \multirow{2}{*}{ \#ReLUs } & \multirow{2}{*}{ Acc.(\%) } & \multirow{2}{*}{ Lat.(s) } & \multirow{2}{*}{ Acc./ReLU }  \\
\cmidrule(lr{0.5em}){3-4}
& & Ch. & Fmap & & & & \\ \toprule 
\multicolumn{8}{c} {ResNet18 baseline model: \#ReLUs = 557.06K, top-1 accuracy (W/o KD) = 74.46\%} \\ \midrule
$S_1$ & NA & NA & NA & 229.38K & 76.22        &      4.61           & 0.332 \\
$S_1$+$S_4$ & NA & NA & NA & 196.61K & 75.51   &      3.94             & 0.384 \\
$S_1$ & $S_2$+$S_3$+$S_4$ & NA & NA & 114.69K & 74.72  & 2.38                & 0.651 \\
%$S_1$+$S_2$ & NA & NA & NA & 98.31K & 73.16        &     2.18          & 0.744 \\
$S_1$ & $S_2$+$S_3$+$S_4$ & 0.5$\times$ & NA & 57.34K & 72.68     & 1.37     & 1.27 \\ 
$S_1$+$S_4$  & $S_2$+$S_3$ & 0.5$\times$ & NA & 49.15K & 69.50          &   1.19       & 1.45 \\
$S_1$ & $S_2$+$S_3$+$S_4$ & NA & 0.5$\times$ & 28.67K & 68.68  &  0.74        & 2.40 \\
$S_1$+$S_4$ & $S_2$+$S_3$ & 0.5$\times$ & NA & 24.57K & 68.41  &   0.56      & 2.78 \\ 
$S_1$ & $S_2$+$S_3$+$S_4$ & 0.5$\times$ & 0.5$\times$ & 14.33K & 65.36& 0.52 & 4.56 \\
$S_1$+$S_4$ & $S_2$+$S_3$ & 0.5$\times$ & 0.5$\times$ & 12.28K & 64.97 & 0.45 & 5.29 \\
$S_1$ & $S_2^\star$+$S_3^\star$+$S_4^\star$ & 0.5$\times$ & 0.5$\times$ & 7.17K & 62.30 & 0.21 & 8.69 \\
\bottomrule
\end{tabular}}
\end{table}


  

{\bf Training process:}
Networks are trained using the following parameters:
an initial learning rate of $0.1$, 
mini-batch size of 128,
the momentum of $0.9$ (fixed), and 
$0.0004$ weight decay factor. 
We train networks for 120 epochs on both CIFAR-100 and TinyImageNet datasets. 
%\fixme{we train all tiny image net models for 120 epochs?}. 
The learning rate is reduced by a factor of $10$ every $30^{th}$ epoch. For training on CIFAR-10, we use cosine learning and train the networks for 150 epochs. 

When using knowledge distillation, we set the hyper-parameters,
temperature and relative weight to cross-entropy loss on hard targets 
as $4$ and $0.9$, respectively~\cite{hinton2015distilling,zagoruyko2016paying,cho2019efficacy}. 
For a fair comparison, we train all the networks with the same hyper-parameters 
and use the baseline model (without any ReLU dropping) as the teacher during KD.
For example, all the DeepReDuce-optimized ResNet18 networks and smaller ResNets,
such as ResNet10 and ResNet9, are trained with the baseline Full-ReLU ResNet18 as a teacher.

\begin{table} [t] \centering
\caption{Optimizations steps (Culling, Thinning, and Reshaping) for ReLU-optimized ResNet18 networks on TinyImageNet. Stages with ``$^\star$'' have only one Block inside the ReLU-stages. Acc. is top-1 accuracy, and Lat. is inference time in seconds.}
\label{tab:R18OnTinyImageNet} 
\resizebox{0.5\textwidth}{!}{
\begin{tabular}{cccccccc}\toprule
\multirow{2}{*}{ Culled } & \multirow{2}{*}{ Thinning } & \multicolumn{2}{c}{ ReLU Reshaping } & \multirow{2}{*}{ \#ReLUs } & \multirow{2}{*}{ Acc.(\%) } & \multirow{2}{*}{ Lat.(s) } & \multirow{2}{*}{ Acc./ReLU }  \\
\cmidrule(lr{0.5em}){3-4}
& & Ch. & Fmap & & & & \\ \toprule 
\multicolumn{8}{c} {ResNet18 baseline model: \#ReLUs = 2228.24K, top-1 accuracy (W/o KD) = 61.28\%} \\ \midrule
$S_1$ & NA & NA & NA & 917.52K & 64.66 & 17.16 & 0.070  \\
$S_1$ & $S_2$+$S_3$+$S_4$ & NA & NA & 458.76K & 62.26 & 8.87 & 0.136  \\
$S_1$+$S_2$ & NA & NA & NA & 393.24K & 61.65 & 7.77 & 0.157  \\
$S_1$ & $S_2$+$S_3$+$S_4$ & 0.5$\times$ & NA & 229.38K & 59.18 & 4.61 & 0.258  \\
$S_1$+$S_2$ & $S_3$+$S_4$ & NA & NA & 196.62K & 57.51 & 4.16 & 0.292  \\
$S_1$ & $S_2$+$S_3$+$S_4$ & NA & 0.5$\times$ & 114.69K & 56.18 & 2.47 & 0.490  \\
$S_1$+$S_2$ & $S_3$+$S_4$ & 0.5$\times$ & NA & 98.31K & 55.67 & 2.64 & 0.566  \\
$S_1$ & $S_2$+$S_3$+$S_4$ & 0.5$\times$ & 0.5$\times$ & 57.35K & 53.75 &  1.85 & 0.937 \\
$S_1$+$S_2$ & $S_3$+$S_4$ & NA & 0.5$\times$ & 49.16K & 49.00 & 1.325 & 0.997  \\
$S_1$ & $S_2^\star$+$S_3^\star$+$S_4^\star$ & 0.5$\times$ & 0.5$\times$ & 28.67K & 47.55 & 0.678 & 1.658  \\
$S_1$+$S_2$ & $S_3$+$S_4$ & 0.5$\times$ & 0.5$\times$ & 24.58K & 47.01 & 0.579 & 1.913  \\
$S_1$+$S_2$ & $S_3^\star$+$S_4^\star$ & 0.5$\times$ & 0.5$\times$ & 12.29K & 41.95 & 0.455 & 3.414  \\ 
\bottomrule
\end{tabular}}
\end{table}



{\bf Dataset:}
We perform our experiments on the CIFAR-100 \cite{2012KrizhevskyCIFAR} and TinyImageNet \cite{le2015tiny,yao2015tiny} datasets.  
CIFAR-100 has 100 output classes with 100 training and test images (resolution $32\times32$) per class.
TinyImageNet has 200 output classes with 500 training and 50 test/validation images (resolution $64\times64$) per class. 
We note that prior work on private inference~\cite{mishra2020delphi} has largely used smaller datasets like MNIST and CIFAR-10 in their evaluations, largely because the high costs of private inference make evaluations on large-scale images difficult.
%TinyImagenet is the most complex dataset used for private inference evaluations to date.

\textbf{Private inference protocol:}
We use the DELPHI~\cite{mishra2020delphi} protocol for private inference. 
DELPHI uses a secret sharing for linear layers and garbled circuits (GC) for ReLU layers. 
DELPHI optimizes the linear layer computations by moving cryptographic operations to an offline (preprocessing) phase. 
The protocol creates secret shares of the model weights during the offline phase (known before the client's input is available) and performs all linear operations over secret-shared data during the online phase.

\textbf{Threat model:}
We assume the same system setup and threat model as used by DELPHI~\cite{mishra2020delphi}, MiniOnn~\cite{liu2017oblivious}, and CryptoNAS~\cite{ghodsi2020cryptonas}.
This model assumes an honest-but-curious adversary.
We refer the interested reader to the referenced work for more details 
as we make no protocol changes and provide the exact same security guarantees.



