\documentclass{article}


\usepackage{arxiv}

\usepackage[utf8]{inputenc} % allow utf-8 input
\usepackage[T1]{fontenc}    % use 8-bit T1 fonts
\usepackage{hyperref}       % hyperlinks
\usepackage{url}            % simple URL typesetting
\usepackage{booktabs}       % professional-quality tables
\usepackage{amsfonts}       % blackboard math symbols
\usepackage{nicefrac}       % compact symbols for 1/2, etc.
\usepackage{microtype}      % microtypography
\usepackage{lipsum}
\usepackage{graphicx}

\usepackage{algorithm}
\usepackage{algorithmic}
\usepackage{amsmath}
\usepackage{amssymb}
\usepackage{multirow}
\usepackage{tabularx}
\usepackage{boldline}

\graphicspath{ {./images/} }


\title{Self-Supervised Contrastive Learning for Videos using Differentiable Local Alignment}

\author{
 Keyne Oei \\
  Universität des Saarlandes\\
  Saarbrücken, Germany \\
  \texttt{s8keoeii@uni-saarland.de} \\
  %% examples of more authors
   \And
 Amr Gomaa \\
  German Research Center for Artificial Intelligence (DFKI)\\
  Saarbrücken, Germany\\
  \texttt{amr.gomaa@dfki.de} \\
  \And
 Anna Maria Feit \\
  Universität des Saarlandes\\
  Saarbrücken, Germany \\
  \texttt{feit@cs.uni-saarland.de} \\
  \And
 João Belo \\
  Universität des Saarlandes\\
  Saarbrücken, Germany \\
  \texttt{jbelo@cs.uni-saarland.de} \\
}

\begin{document}
\maketitle
\begin{abstract}
    Robust frame-wise embeddings are essential to perform video analysis and understanding tasks. We present a self-supervised method for representation learning based on aligning temporal video sequences. Our framework uses a transformer-based encoder to extract frame-level features and leverages them to find the optimal alignment path between video sequences. We introduce the novel Local-Alignment Contrastive (LAC) loss, which combines a differentiable local alignment loss to capture local temporal dependencies with a contrastive loss to enhance discriminative learning. Prior works on video alignment have focused on using global temporal ordering across sequence pairs, whereas our loss encourages  identifying the best-scoring subsequence alignment. LAC uses the differentiable Smith-Waterman (SW) affine method, which features a flexible parameterization learned through the training phase, enabling the model to adjust the temporal gap penalty length dynamically. Evaluations show that our learned representations outperform existing state-of-the-art approaches on action recognition tasks.
\end{abstract}


% keywords can be removed
%\keywords{First keyword \and Second keyword \and More}

\section{Introduction}



Motion mimicking aims to find a policy to generate control signals for recovering demonstrated motion trajectories, which plays a fundamental role in physics-based character animation, and also serves as a prerequisite for many applications such as control stylization and skill composition. 
Although tremendous progress in motion mimicking has been witnessed in recent years, existing methods~\citep{peng2018deepmimic, peng2021amp} mostly adopt reinforcement learning (RL) schemes, which require alternatively learning a reward function and a control policy.
Consequently, RL-based methods often take tens of hours or even days to imitate one single motion sequence, making their scalability notoriously challenging.
In addition, RL-based motion mimicking highly relies on the quality of its designed ~\citep{peng2018deepmimic} or learned~\citep{peng2021amp} reward functions, which further burdens its generalization for complex real-world applications.




Recently, differential physics simulator (DPS) has achieved impressive results in many research fields, such as robot control~\citep{xu2022accelerated} and graphics~\citep{li2022diffcloth}.
Specifically, DPS treats physics operators as differentiable computational graphs, and therefore gradients from objectives (\textit{i.e.}, rewards) can be directly propagated through the environment dynamics to control policy functions.
Instead of alternatively learning between reward functions and control policies, the control policy learning tasks can be resolved in a straightforward and efficient optimization manner with the help of DPS. 
However, despite their analytical environment gradients, optimization with DPS could easily get into local optima, particularly in contact-rich physical systems that often yield stiff and discontinuous gradients~\citep{freeman2021brax, suh2022does, zhong2022differentiable}. 
Besides, numerical gradients could also vanish/explode along the backward path for long trajectories.






In this work, we propose DiffMimic, a fast and stable motion mimicking method with the help of DPS.
Different from RL-based methods that require heavy reward engineering and poor sample efficiency, DiffMimic reformulates motion mimicking as a state matching problem, which could directly minimize the distance between a rollout trajectory generated by the current learning policy and the demonstrated trajectory.
Thanks to the differentiable DPS dynamics, gradients of the trajectory distance can be directly propagated to optimize the control policy.
As a result, DiffMimic could significantly improve the sample efficiency with the first-order gradients.

However, simply utilizing DPS could not guarantee global optimal solutions.
In particular, the rollout trajectory tends to gradually deviate from the expert demonstration and could produce a large accumulative error for long motion sequences, due to the distributional shift between the learning policy and expert policy.
To address these problems, we introduce the \textit{\ourmethod{}} training strategy, which randomly inserts reference states into the rollout trajectory as anchor states to guide the exploration of the policy. 
Empirically, \ourmethod{} gives a smoother gradient estimation, which significantly stabilizes the policy learning of DiffMimic.


To the best of our knowledge, DiffMimic is the first to utilize DPS for motion mimicking. We show that DiffMimic outperforms several commonly used RL-based methods for motion mimicking on a variety of tasks with high accuracy, stability, and efficiency. In particular, DiffMimic allows learning a challenging \textit{Backflip} motion in only 10 minutes on a single V100 GPU.
In addition, we release the DiffMimic simulator as a standard benchmark to encourage future research for motion mimicking.

\begin{figure}[t]
    \centering
    \includegraphics[width=\textwidth]{figures/teaser.pdf}
    \captionof{figure}{Overview of our method. \textbf{Left:} DiffMimic formulates motion mimicking as a straightforward state matching problem and uses analytical gradients to optimize it with off-the-shelf differentiable physics simulators. The formulation results in a simple optimization objective compared to heavy reward engineering in RL-based methods. \textbf{Middle:} DiffMimic is able to mimic highly dynamic skills, \eg, Side-Flip. \textbf{Right:} DiffMimic has a significantly better sample efficiency and time efficiency than state-of-the-art motion mimicking methods. Our approach usually achieves high-quality motion (pose error $<$ 0.15 meter) using less than $2\times10^7$ samples. }
    \label{fig:teaser}
\end{figure}

\section{Related work}

\textbf{Generative Adversarial Networks. }
Generative Adversarial Networks (GAN) is a generative models which is trained by adversarial learning. 
In the early days, unconditional GAN \cite{karras2019style,karras2020analyzing} recovered images from random noise. Developed to the present, conditional GAN \cite{pan2023drag} utilizes text and images for guidance to generate images. GAN has performed strongly on tasks of generating static data such as image generation \cite{karras2019style, karras2020analyzing}, image editing \cite{vinker2021image, patashnik2021styleclip}, and image translation \cite{shao2021spatchgan}.
The generation of dynamic data, such as videos and action sequences, has also been studied. Carl et al. \cite{vondrick2016generating} proposed a video generation network with a spatial-temporal two-stream convolutional architecture based on DCGAN \cite{radford2015unsupervised}. This work is the first application of GAN to video generation. 
TGAN \cite{saito2017temporal} followed, which first generates a set of latent vectors from noise vectors, then generates pictures and synthesizes videos separately.
RNN-GAN \cite{mogren2016c} is based on the temporal modeling capability of RNNs to predict video from a single frame. It has a more robust motion prediction capability compared to the work of Carl et al. However, these impressive results are mainly attributed to the support of many training samples. With limited data, GANs are prone to overfitting, leading to a lack of diversity in the generated data. 

% \textbf{Few-shot Generation. }

\textbf{Motion Style Transfer. }
Image Style Transfer \cite{gatys2016image, saito2020coco} combines style and content features from two images to form a new image. Motion Style Transfer refers to Image Style Transfer to form a new action by transferring one action's style features to another that contains only content features. Early motion style transfer was done by manually defining style features and inferring them through machine learning \cite{xia2015realtime, yumer2016spectral}. This method is effective only for the actions in the training data with limited scope of usefulness. 
Deep learning-based methods have greatly improved the quality and application of motion style transfer. Both Holden et al. \cite{2016A} and Du et al. \cite{du2019stylistic} applied the Gram matrix method to convey motion styles through the distribution of actions in the hidden space. These methods are time-consuming and have limited the quality of action generation for relatively significant motion differences.
Recently, Aberman et al. \cite{aberman2020unpaired} proposed a motion transfer network that combines GAN and AdaIN. 
The method can learn from unpaired data with different styles to migrate model unseen actions. 
Park et al. \cite{ParkSoomin2021Diverse} used a spatio-temporal graph convolutional network to model actions. The method adds random noise in the decoder to enhance action diversity. 
Jang et al. \cite{jang2022motion} proposed a novel motion style transfer network called Motion Puzzle. 
Motion Puzzle divides the human skeleton into five parts, allowing flexible control over the migration of specified parts during generation. This approach is effective for single-action generation tasks.
However, it is usually time-consuming to control parts for generation when generating many actions. In addition, Motion Puzzle's target motion encoder is connected to the decoder at multiple scales, which may constrain the diversity of action.

\textbf{Active Learning. }
Existing active learning methods are categorized into pool-based and synthetic methods \cite{gal2016dropout,beluch2018power,gorriz2017cost,yang2017suggestive,nguyen2004active}. Pool-based methods use different sampling strategies to determine how to select the most informative samples, with uncertainty sampling methods being the most common.
Ebrahimi et al. \cite{ebrahimi2019uncertainty} used a Bayesian neural network for uncertainty evaluation. Gal \cite{gal2016dropout} and Gharamani \cite{gal2017deep} also showed the relationship between uncertainty and dropout to estimate uncertainty in neural network prediction.
Pool-based methods select samples conditional on a large amount of unlabeled data. In the case of scarcity of data, synthetic methods are more suitable than pool-based methods. Synthetic methods use a generative model to generate samples, then sample based on the uncertainty of the model.
The work of Zhu et al.  \cite{zhu2017generative}, Mahapatra et al.  \cite{mahapatra2018efficient}, and Mayer et al. \cite{mayer2020adversarial} uses GAN to generate a sample and then query using the uncertainty principle. Our work uses this same strategy to guide human action generation using the amount of sample information.
\begin{figure*}
\begin{minipage}{0.55\linewidth}
  \centerline{\includegraphics[width=1.0\textwidth]{figures/sdf_loss_a.png}}
\end{minipage}
\hfill
\begin{minipage}{0.51\linewidth}
  \centerline{\includegraphics[width=1.0\textwidth]{figures/sdf_loss_b.png}}
\end{minipage}
\vspace{-5mm}
\caption{\textbf{Self-supervised SDF photometric loss between source and target views}. \textbf{Left:} Voxel center is inside of the surface, SDF is negative. Orange arrows show projecting rays from voxel centers to 2D pixels (P1, P2) on each camera plane, blue arrows show reprojection of surface 3D points (S1, S2) to 2D pixels (P1', P2') in each camera plane. Surface points are estimated by SDF estimation. \textbf{Right:} Voxel center is outside of the surface, SDF is positive. \textbf{The loss is extended to all n views in a fragment}.}
\label{fig:sdf photometric loss}
\vspace{-5mm}
\end{figure*}

\vspace{-2mm}
\section{Method}
\label{sec:method}
\vspace{-2mm}
\subsection{MonoSelfRecon Framework}
\quad Figure \ref{fig:pipeline} shows our MonoSelfRecon framework. In both training and testing, we take the input monocular RGB sequence and camera poses, and reconstruct 3D mesh of the whole scene. We select key frames following the process in \cite{key_select}, where we consider a valid frame to be ``greater than 15 degree in rotation or 0.3 meter in translation'' to the previous frame. Every $n$ consecutive key frames form a scene fragment. Fragments are fed as inputs, from which, the network extracts 2D features per key frame, creates 3D feature volume and jointly estimates SDF and NeRF using separate decoders at three pyramid scales. In SDF decoder, 2D features are fused to 3D voxel features and regress to voxel-SDF values at each level with 3D sparse convolution. Every time the network only estimates SDF corresponding to the fragment in a $[N,N,N]$ 3D voxel region, the Gated Recurrent Unit (GRU) module at each level updates SDF, fuses reconstruction from previous fragments, and completes the whole scene. During training, self-supervised losses are implemented between SDF-input, NeRF-input, and SDF-NeRF, with detailed discussion in \ref{sec:losses}. During testing, 3D mesh can be obtained from SDF through marching cube\cite{marchingcube}.  

\noindent
\textbf{Attentional View Fusion}.
For each fragment, 3D voxel features can be simply obtained by projecting the 3D voxel to each 2D view in the fragment, searching for visible corresponding pixels, and averaging 2D features. However, since each view have different distance, angle, and occlusion to voxels, 2D features from different views should not contribute the same to 3D features. Inspired by recent works \cite{vortx}, we use an attentional view fusion module by adding a light transformer before averaging features. The transformer takes unordered sequence of 2D features and updates weighted features before average to 3D, which enables more flexibility to adjust the contribution of each frame to the fragment. Although simple, this module achieves significant improvement as shown in the ablation study in Table \ref{table:scannet_ablation}.

\noindent
\textbf{GRU}. 
We adapt the GRU module from \cite{neucon}. With camera poses, it is simple to concatenate fragments and replace the overlapping voxels of latter fragment to the previous one, but it ignores the effect of the latter views to the previous ones. Once fusing 3D voxel features from 2D and before regressing voxel-SDF at each level, the GRU module takes 3D voxel features of both previous and current fragments as input to update the current features. GRU fusion makes an obvious improvement as shown in the ablation study in Table \ref{table:scannet_ablation}. 


\noindent
\textbf{NeRF.} We adopt the generalizable MPI (MultiPlane-Images)-NeRF \cite{mpi, mpi-nerf, mononerf}, introduce it to the framework and jointly train with SDF to boost SDF estimation. The core idea is to use an explicit encoder on top of standard positional encoding to enable implicit NeRF with generalization ability. The NeRF estimation also further boosts our SDF performance as shown in the ablation study in Table \ref{table:scannet_ablation}.

\subsection{Self-supervised Losses}
\label{sec:losses}

\noindent
\textbf{SDF Photometric Loss}. Figure \ref{fig:sdf photometric loss} shows a simplified version of the loss implementation between two camera views, where the corresponding 2D coordinates (P1 and P2) can be found by tracing rays (orange arrows) to camera planes. SDF is the distance between a point to its nearest surface, where the value is negative when the point is inside of the surface, and positive when it is outside of the surface. The model estimates SDF per voxel. The depth $\hat{D}_{cam}$ can be estimated by Eq. \ref{eq:get_depth}, where $V_{world}$ is 3D world coordinate of the pre-defined voxel center, $T_{world\xrightarrow{}cam}$ is camera extrinsic, $\hat{SDF}$ is the estimated voxel-SDF from the model. With depth and voxel center in the camera coordinate, the surface points S1 and S2 can be estimated by Eq. \ref{eq:get_surface}, where $\vec{ray}$ is the unit vector at ray direction (orange arrows), and $\hat{S}_{cam}$ is the 3D coordinate of a surface point in the camera view. Finally, the reconstructed pixels are obtained by reprojecting (blue arrows) surface points S1, S2 to camera 2 and 1 as Eq. \ref{eq:point_proj}, where K is camera intrinsic, $T_{cam\xrightarrow{}cam'}$ is camera pose from cam to cam'. P-P' is a pixel reconstruction pair with same photometric intensity, where a photometric consistency loss can be derived as Eq. \ref{eq:pts_loss}. The exact pixel intensities $I_{cam}(P)$ and $I_{cam'}(P')$ are obtained with bilinear interpolation from projected 2D points lying between integer coordinates, and the loss is only traced to the points lying within the camera planes.
\vspace{-3mm}
\begin{equation}
\begin{split}
    V_{cam} = T_{world\xrightarrow{}cam}V_{world} \\
    \hat{D}_{cam} = V_{cam} + \hat{SDF}
    \label{eq:get_depth}
\end{split}
\end{equation}
\vspace{-6mm}

\vspace{-5mm}
\begin{equation}
\begin{split}
    \hat{S}_{cam}(x, y, z) = V_{cam}(x, y, z) + |\hat{SDF}| \Vec{ray}(x, y, z)
    \label{eq:get_surface}
\end{split}
\end{equation}
\vspace{-6mm}

\vspace{-5mm}
\begin{equation}
\begin{split}
    P = Interp(KV_{cam}) \\
    P' = Interp(K'T_{cam\xrightarrow{}cam'}\hat{S}_{cam})
\end{split}
\label{eq:point_proj}
\end{equation}
\vspace{-6mm}

\vspace{-4mm}
\begin{equation}
    L_{sdf} = \sum_{P \in cam}\sum_{P'\in cam'}{|(I_{cam}(P) - I_{cam'}(P')|}
\label{eq:pts_loss}
\end{equation}
\vspace{-4mm}

\iffalse
We make an assumption to set up this point-wise self-supervised SDF loss. Although the distance from the voxel center to different surface points varies - only the nearest distance is SDF. Since the network estimates one SDF value per voxel, we use the same estimated SDF value corresponding to the same voxel center to estimate surface points for different camera views. However, previous supervised works made the same assumption to implement TSDF fusion  \cite{atlas, neucon} and get TSDF ground truth from the depth map. More specifically, instead of using the nearest distance, they take the average distances from all views. Similarly, we assign this task to the network, when the resolution is large enough (voxel size is small enough), the network is trained to get to the average distance that optimizes the consistency loss between all camera views.
\fi

In practice, we implement the loss across all views in the scene fragment and take the weighted average loss, where the weight is in direct proportion to the number of the candidate P-P' pair. If P' lies outside of the other camera's plane, we ignore this P-P' pair. For SFM-based self-supervised depth works, they start from 2D pixel and ends up at 2D pixel to jointly regress depth and camera pose. However, we start from 3D voxel center and end up at 2D pixel to only regress the depth model while taking camera pose as prior, which is why SFM-based self-supervised depth estimation has scale ambiguity, while our SDF estimation is directly in real scale. 

\noindent
\textbf{SDF Co-Planar Loss}. Photometric constraints are insufficient for indoors scenes due to large non-textured regions and in-plane rotations. Thus, we take advantage of the special geometric constraints in indoor scenes. Most indoor scenes have large planes such as walls, floors, and ceilings, where textures within such planes are mostly similar. Inspired by \cite{p2net, planercnn, planenet, piece-wise} that implement planar constraints in 2D depth maps, we extend it to 3D SDF. Specifically, we adopted `Felzenszwalb superpixel segmentation' \cite{plane_seg} to extract `super-pixels', which covers piece-wise large group of regions that have low pixel intensity gradients, which are considered as a planar region. The algorithm uses greedy search to extract super-pixels and is free of learning. Based on the planar segmentation and the depth planar constraints from \cite{structdepth}, we propose a voxel-SDF driven co-planar loss.

Our goal is to derive plane parameters under planar constraints, and learn the plane parameters in a self-supervised manner. Specifically, the plane segmentation extracts $n$ super-pixels from a 2D image, with each super-pixel corresponding to a continuous plane. For the 2D projected voxel center point P (as shown in Figure \ref{fig:sdf photometric loss}), if it belongs to super-pixel $SP_m$ in the 2D plane, then the surface 3D point $S$ corresponding to $P$ also belongs to the surface plane of class $m$ in 3D space. Using the surface point $S$, the plane $m$ can be defined as Eq. \ref{eq:plane_onepoint}, where $\hat{s}_{0}$ is an estimated surface point in the plane, and $A_m$ is the plane parameter.

\vspace{-3mm}
\begin{equation}
    A_{m}^T \hat{s}_{0} = 1
    \label{eq:plane_onepoint}
\end{equation}
\vspace{-5mm}

While using only one 3D point to simulate a plane is ill-posed, a large number of estimated 3D surface points are obtained by projecting voxel centers to different camera views. With $n$ 2D projected points $p_1$, $p_2$ ...... $p_n$ belonging to super-pixel $SP_m$, there are $n$ 3D surface points $s_1$, $s_2$, ......, $s_n$ belonging to 3D surface plane $m$. Eq. \ref{eq:plane_onepoint} is extended to Eq. \ref{eq:plane_npoint}, where $\hat{S}_{n} = [\hat{s}_{1}, \hat{s}_{2}, ......, \hat{s}_{n}]$, and $Y_m = \Vec{1} = [1,1,...,1]$.

\vspace{-3mm}
\begin{equation}
    \hat{S}_{n} A_{m}^T  = Y_{m}
    \label{eq:plane_npoint}
\end{equation}
\vspace{-5mm}

The plane parameter $A_{m}$ is then estimated by least-square method as Eq. \ref{eq:least_square}, where $\epsilon$ is a small scalar for stability, and $I$ is an identity matrix.

\vspace{-3mm}
\begin{equation}
    A_m = (\hat{S}_n^T\hat{S}_n+\epsilon I)^{-1}\hat{S}_n^TY_m
    \label{eq:least_square}
\end{equation}
\vspace{-5mm}

With the estimated plane parameter, the pseudo surface points can be retrieve  by $\hat{S_n}' = (A_m^T \hat{S_n})^{-1}$. The pseudo surface and estimated surface are expected to align together, and we implement such a co-planar geometric constraint as the self-supervised co-planar SDF loss as Eq. \ref{eq:plane_loss}.

\vspace{-3mm}
\begin{equation}
    L_{plane} = \sum_{M}\sum_{N}{|\hat{S_n}-\hat{S_n}'|}
\label{eq:plane_loss}
\end{equation}
\vspace{-3mm}

\noindent
\textbf{Depth Consistency Loss.} We also propose depth consistency loss to further boost SDF from NeRF. Specifically, we estimate sparse Pseudo-SDF depth for target views from estimated SDF (as Figure \ref{fig:sdf photometric loss}), and render NeRF-depth for corresponding target views. Since Pseudo-SDF depth is in real scale, we first use it to recover NeRF-depth's scale, and enforce consistency between the two estimated depths. 

\vspace{-3mm}
\begin{equation}
    L_{depth} = \sum_{N}\sum_{D \in cam}{|\hat{D_{sdf}}-\hat{D_{NeRF}}|}
\label{eq:depth_loss}
\end{equation}
\vspace{-3mm}

\noindent
where $\hat{D_{sdf}}$ and $\hat{D_{NeRF}}$ are Pseudo-SDF depth and the scale-recoverd NeRF-depth, respectively.

\noindent
\textbf{Total Loss.} We implement standard NeRF losses for NeRF encoder, including RGB consistency with input images, SSIM, and smooth loss, and we jointly train everything end-to-end in pure self-supervision.

\vspace{-8mm}
\begin{equation}
    L_{NeRF} = L_{rgb} + L_{smooth} + (1 - SSIM)
\label{eq:nerf_loss}
\end{equation}
\vspace{-6mm}

\noindent
The total loss is the weighted sum of all losses, where $\lambda$s are the weights,

\vspace{-3mm}
\begin{multline}
    L_{total} = \lambda_{sdf}L_{sdf} + \lambda_{plane}L_{plane} \\ + 
    \lambda_{depth}L_{depth} + \lambda_{NeRF}L_{NeRF}
\label{eq:total_loss}
\end{multline}
\vspace{-5mm}
% \newpage

\section{Experiments}

\textbf{Datasets.} We evaluate the performance of the LAC model on two datasets using a variety of evaluation metrics. The Pouring dataset \cite{2017_tcn}, which focuses on the action of pouring liquids, includes 70 training and 14 testing videos. The PennAction dataset \cite{2013_pennaction}, featuring 13 human actions, contains 1140 training and 966 testing videos. We utilize the key events and phases for the videos in both datasets as proposed by TCC \cite{2019_tcc}.


\noindent \textbf{Implementation Details.} 
Our encoder, \(f: \mathbb{R}^{TxCxWxH} \to Z\), maps video inputs $V$ into an embedding space \(Z\). We use a ResNet50-v2 \cite{2015_resnet} as our backbone to extract features from the Conv4c layer with an output size of \(10x10x512\).
These features are then processed through adaptive max pooling, followed by two fully connected layers with ReLU activation.
A subsequent linear layer projects the features into a 256-dimensional space. 
To enhance the model’s capacity to capture long-range dependencies, we integrate sine-cosine positional encoding and employ a two-layer Transformer encoder. 
 To improve the model’s ability to capture long-range dependencies, we incorporate sine-cosine positional encoding and apply a two-layer Transformer encoder. 
 The final embedding layer reduces the dimensionality to 128 for the frame-wise representations.

\begin{table}[]
\small
\centering
\begin{tabularx}{\textwidth}{X|ccc|ccc|c|c}
\multirow{2}{*}{\textbf{Method}} & \multicolumn{3}{c|}{\textbf{Class}} & \multicolumn{3}{c|}{\textbf{AP@K}} & \multirow{2}{*}{\textbf{Progress}} & \multirow{2}{*}{\textbf{$\tau$}} \\ 
\cline{2-7}
 & \textbf{10} & \textbf{50} & \textbf{100} & \textbf{K=5} & \textbf{K=10} & \textbf{K=15} & & \\ 
\hline
\hline
TCN \cite{2017_tcn}        & 80.32 & 81.44 & 83.56 & 76.26 & 76.71 & 77.26 &  82.30 & 83.51 \\
TCC \cite{2019_tcc}        & 86.60 & 86.78 & 86.86 & 81.84 & 80.94 & 81.69 &  83.36 & 85.26 \\
LAV \cite{2021_lav}        & 89.77 & 90.35 & 91.77 & 87.48 & 88.36 & 88.40 &  85.20 & 88.75 \\
GTA \cite{2021_gta}        & 89.34 & 90.20 & 90.22 & 87.79 & 87.48 & 87.82 &  88.67 & 92.47 \\
SCL \cite{2022_carl}       & \underline{92.76} & 92.80 & 93.05 & 88.75 & 88.51 & 88.97 & 91.26 & \underline{98.20} \\
LRPROP \cite{2024_lrprop}  & 92.70 & \underline{94.44} & \underline{94.36} & \underline{92.41} & \underline{90.33} & \underline{90.86} & \underline{94.09} & \textbf{99.46} \\
\hline
LAC & \textbf{95.87} & \textbf{95.78} & \textbf{95.16} & \textbf{92.76} & \textbf{91.07} & \textbf{91.37} & \textbf{94.24} & 97.50 \\
\hline
\hline
\end{tabularx}
\caption{Performance comparison of state-of-the-art methods on the Pouring Dataset \cite{2017_tcn}}
% \caption{Performance comparison of LAC and SOTA on the Pouring Dataset \cite{2017_tcn}}
\label{tab: p_class}
\end{table}



\noindent \textbf{Evaluation Metrics.}
% Following related work \cite{2019_tcc, 2021_lav, 2021_gta, 2022_carl, 2024_lrprop}, we freeze the model's weights and evaluate it using the following metrics: 
Following related work \cite{2019_tcc, 2021_lav, 2021_gta, 2022_carl, 2024_lrprop}, we evaluate our model using the following metrics:
(i) \textit{Phase Classification}, which assesses the accuracy of action phase predictions by training an SVM classifier on our embeddings;
(ii) \textit{Phase Progression}, which evaluates how accurately our embeddings predict action progress using a linear regression model's average R-squared value, based on normalized timestamp differences;
(iii) \textit{Average Precision@K (AP@K)}, which evaluates fine-grained frame retrieval accuracy by calculating the proportion of correctly matched phase labels within the K closest frames;
(iv) \textit{Kendall's Tau}, which quantifies the temporal alignment between sequences by comparing the ratio of concordant to discordant frame pairs.

\subsection{Results}

We apply the same four metrics to the Pouring dataset \cite{2017_tcn}. For the PennAction dataset \cite{2013_pennaction}, we evaluate each metric across all action categories and report the average results. 
To ensure a fair comparison, we replicated the evaluations of previous approaches using the GitHub repositories \footnote{github.com/google-research/google-research/tcc, github.com/trquhuytin/LAV-CVPR21,\\github.com/hadjisma/VideoAlignment, github.com/minghchen/CARL\_code} provided by the original authors. 
Each model was trained using its respective pre-trained backbone. 
An exception was made for LRPROP, as their GitHub repository is not available.

\noindent \textbf{Results on Pouring Dataset.} 
Table~\ref{tab: p_class} presents a comparison of our method's performance against state-of-the-art approaches on the Pouring dataset. Bold and underlined text denote the best and second-best results.
% Our method significantly outperforms prior work when using our embeddings for action recognition tasks (Table~\ref{tab: p_class}).
Notably, it achieves a +3.11\% improvement on Phase Classification using only 10\% of the labels. Additionally, our model excels in AP@K and Progress metrics.
However, we observe lower performance in Kendall's Tau. We hypothesize this is due to how the SW's algorithm encourages skipping unnecessary segments of the sequence. The introduction of gap open and extend penalties could disrupt the continuity needed for high Kendall's Tau scores.

\begin{table}[]
\small
\centering
\begin{tabularx}{\textwidth}{X|ccc|ccc|c|c}
\multirow{2}{*}{\textbf{Method}} & \multicolumn{3}{c|}{\textbf{Class}} & \multicolumn{3}{c|}{\textbf{AP@K}} & \multirow{2}{*}{\textbf{Progress}} & \multirow{2}{*}{\textbf{$\tau$}} \\ 
\cline{2-7}
 & \textbf{10} & \textbf{50} & \textbf{100} & \textbf{K=5} & \textbf{K=10} & \textbf{K=15} & & \\ 
\hline
\hline
TCN \cite{2017_tcn}        & 69.73 & 70.26 & 70.01 & 60.92 & 61.57 & 61.52 &  76.37 & 63.72 \\
TCC \cite{2019_tcc}        & 86.60 & 86.78 & 86.86 & 81.84 & 80.94 & 81.69 &  83.36 & 85.26 \\
LAV \cite{2021_lav}        & 88.51 & 88.72 & 88.97 & 73.47 & 73.13 & 74.27 &  92.52 & 93.06 \\
GTA \cite{2021_gta}        & 84.21 & 84.68 & 85.28 & 71.72 & 72.17 & 71.52 &  90.51 & 83.35 \\
SCL \cite{2022_carl}       & 87.85 & 87.52 & 88.15 & 91.70 & 90.61 & 90.58 &  92.89 & \underline{98.14}\\
LRPROP \cite{2024_lrprop}  & \underline{91.90} & \underline{92.96} & \underline{93.25} & \underline{92.46} & \underline{92.2} & \underline{92.03} & \underline{93.03} & \textbf{99.09} \\
\hline
LAC & \textbf{95.57} & \textbf{93.79} & \textbf{93.40} & \textbf{93.87} & \textbf{93.41} & \textbf{92.65} & \textbf{94.21} & 94.10 \\
\hline
\hline
\end{tabularx}
\caption{Performance comparison of state-of-the-art methods on the PennAction Dataset \cite{2013_pennaction}}
% \caption{Performance comparison of LAC and SOTA on the PennAction Dataset \cite{2013_pennaction}}
\label{tab: pn_class}
\end{table}

\begin{figure}[t]
\begin{tabular}{cc}
\includegraphics[width=0.195\textwidth]{images/v1.png}&\includegraphics[width=0.764\textwidth]{images/v2.png}
% \\
% (a)&(b)
\end{tabular}
\caption{Similarity matrix (left) shows video alignment using an optimal path and the respective video aligned frame-by-frame (right).}
\label{fig: vis_align}
\end{figure}

\noindent \textbf{Results on PennAction Dataset.} 
Table \ref{tab: pn_class} compares the performance of our method with state-of-the-art approaches on the PennAction dataset. Bold and underlined text denote the best and second-best results. 
LAC consistently outperforms previous methods across most metrics, with the exception of Kendall's Tau. 
Notably, the improvement in AP@K is more pronounced on PennAction than on Pouring, likely due to the fewer number of frames in PennAction dataset, which may further impact alignment performance.

\begin{figure}[t]
\begin{tabular}{cc}
\includegraphics[width=0.45\textwidth]{images/v3.png}&\includegraphics[width=0.45\textwidth]{images/v4.png}
% \\
% (a)&(b)
\end{tabular}
\caption{Fine-grained frame retrieval for Pouring (left) and PennAction (right) achieved using nearest neighbors within our embedding space.}
\label{fig: vis_retr}
\end{figure}

\noindent \textbf{Qualitative analysis of results.} 
Figure \ref{fig: vis_align} illustrates the alignment process, with the optimal path depicted on the left and the frame-by-frame alignment on the right. 
This visualization demonstrates the synchronization of the two videos despite differences in their temporal progression and duration. 
Such alignment visualizations enable in-depth analysis of deviations or inefficiencies within specific actions in video understanding.
Additionally, Figure \ref{fig: vis_retr} showcases our embedding-based retrieval system's ability to accurately identify and retrieve frames corresponding to specific action sequences across different videos.

\subsection{Ablation Study}

This section presents multiple experiments on the \textit{Pouring} dataset that analyze the different components of our framework. 

\begin{table}[H]
\small
\centering
\begin{tabularx}{0.99\linewidth}{|X|ccc|}
\hline
\textbf{Architecture} & \textbf{Class} & \textbf{Progress} & \textbf{$\tau$} \\
\hlineB{3}
ResNet-50 + Convolutional 3D & 88.04 & 73.26 & 71.37 \\
\hline
ResNet-50 + Tranformer 
(w/o pretrained weights) & 90.06 & 89.32 & 94.87 \\
\hline
ResNet-50 + Transformer
 & \textbf{95.16} & \textbf{94.24} & \textbf{97.50} \\
\hline
\end{tabularx}
\caption{Different Encoder Architecture on Model Performance.}
\label{tab_lac_arch}
\end{table}

\begin{table}[H]
\parbox{.4\linewidth}{
\centering
\begin{tabularx}{0.99\linewidth}{|c|ccc|}
\hline
\textbf{$\gamma$} & \textbf{Class} & \textbf{Progress} & \textbf{$\tau$} \\
\hlineB{3}
0.6 & 93.44 & 92.12 & 94.93 \\
\hline
0.7 & 94.87 & 92.82 & 97.29 \\
\hline
0.8 &  \textbf{95.16} & \textbf{94.24} & \textbf{97.50} \\
\hline
0.9 & 93.93 & 92.02 & 94.91 \\
\hline
\end{tabularx}
\caption{Different $\gamma$ Values on Model Performance}
\label{tab_lac_y}
}
\hfill
\parbox{.59\linewidth}{
\centering
\begin{tabularx}{0.99\linewidth}{|X|ccc|}
\hline
\textbf{Loss $\mathcal{L}$} & \textbf{Class} & \textbf{Progress} & \textbf{$\tau$} \\
\hlineB{3}
$\mathcal{L}_c $ & 93.44 & 92.12 & 94.93 \\
\hline
$\mathcal{L}_c + \alpha \cdot \mathcal{L}_l$ & 93.71 & 92.16 & 95.01 \\
\hline
$\mathcal{L}_c + \alpha \cdot (\mathcal{L}_l + \beta \cdot (\mathcal{L}_{sw12} + \mathcal{L}_{sw21}))$ & \textbf{95.16} & \textbf{94.24} & \textbf{97.50} \\
\hline
\end{tabularx}
\caption{Different LAC Loss Formulations on Model Performance}
\label{tab_lac_loss}
}
\end{table}

\noindent \textbf{Network Architecture.} Table \ref{tab_lac_arch} demonstrates that our performance evaluation across various network architectures highlights the superiority of the ResNet-50 model with pretrained weights combined with a Transformer.

\noindent \textbf{Hyperparameter of LAC Loss.} Table \ref{tab_lac_y} presents the optimal performance results obtained at the smoothing parameter $\gamma = 0.8$. 
Table~\ref{tab_lac_loss} further demonstrates that integrating local alignment loss into the total loss significantly enhances performance and achieves the highest scores across all metrics compared to using contrastive loss alone.


\section{Conclusion}
The paper introduces a novel approach to representation learning through a Local-Alignment Contrastive (LAC) loss that integrates a differentiable local alignment loss with a contrastive loss, all within a self-supervised framework. 
The method employs a differentiable affine Smith-Waterman algorithm to enable temporal alignment that dynamically adjusts to variations in action sequences. 
This approach is distinct in its focus on capturing local temporal dependencies and enhancing the discriminative learning of video embeddings, accommodating differences in action lengths and sequences.
Experimental results on the Pouring and PennAction datasets showcase the method's superior performance over existing state-of-the-art approaches.

% \newpage

% \section{Introduction}
% Our project is a competition on Kaggle (Predict Future Sales). We are provided with daily historical sales data (including each products’ sale date, block ,shop price and amount). And we will use it to forecast the total amount of each product sold next month. Because of the list of shops and products slightly changes every month. We need to create a robust model that can handle such situations.


% \section{Task description and data construction}
% \label{sec:headings}
% We are provided with five datasets from Kaggle: Sales train, Sale test, items, item categories and shops. In the Sales train dataset, it provides the information about the sales’ number of an item in a shop within a day. In the Sales test dataset, it provides the shop id and item id which are the items and shops we need to predict. In the other three datasets, we can get the information about item’s name and its category, and the shops’ name.
% \paragraph{Task modeling.}
% We approach this task as a regression problem. For every item and shop pair, we need to predict its next month sales(a number).
% \paragraph{Construct train and test data.}
% In the Sales train dataset, it only provides the sale within one day, but we need to predict the sale of next month. So we sum the day's sale into month's sale group by item, shop, date(within a month).
% In the Sales train dataset, it only contains two columns(item id and shop id). Because we need to provide the sales of next month, we add a date column for it, which stand for the date information of next month.

% \subsection{Headings: second level}
% \lipsum[5]
% \begin{equation}
% \xi _{ij}(t)=P(x_{t}=i,x_{t+1}=j|y,v,w;\theta)= {\frac {\alpha _{i}(t)a^{w_t}_{ij}\beta _{j}(t+1)b^{v_{t+1}}_{j}(y_{t+1})}{\sum _{i=1}^{N} \sum _{j=1}^{N} \alpha _{i}(t)a^{w_t}_{ij}\beta _{j}(t+1)b^{v_{t+1}}_{j}(y_{t+1})}}
% \end{equation}

% \subsubsection{Headings: third level}
% \lipsum[6]

% \paragraph{Paragraph}
% \lipsum[7]

% \section{Examples of citations, figures, tables, references}
% \label{sec:others}
% \lipsum[8] \cite{kour2014real,kour2014fast} and see \cite{hadash2018estimate}.

% The documentation for \verb+natbib+ may be found at
% \begin{center}
%   \url{http://mirrors.ctan.org/macros/latex/contrib/natbib/natnotes.pdf}
% \end{center}
% Of note is the command \verb+\citet+, which produces citations
% appropriate for use in inline text.  For example,
% \begin{verbatim}
%    \citet{hasselmo} investigated\dots
% \end{verbatim}
% produces
% \begin{quote}
%   Hasselmo, et al.\ (1995) investigated\dots
% \end{quote}

% \begin{center}
%   \url{https://www.ctan.org/pkg/booktabs}
% \end{center}


% \subsection{Figures}
% \lipsum[10] 
% See Figure \ref{fig:fig1}. Here is how you add footnotes. \footnote{Sample of the first footnote.}
% \lipsum[11] 

% \begin{figure}
%   \centering
%   \fbox{\rule[-.5cm]{4cm}{4cm} \rule[-.5cm]{4cm}{0cm}}
%   \caption{Sample figure caption.}
%   \label{fig:fig1}
% \end{figure}

% \begin{figure} % picture
%     \centering
%     \includegraphics{test.png}
% \end{figure}

% \subsection{Tables}
% \lipsum[12]
% See awesome Table~\ref{tab:table}.

% \begin{table}
%  \caption{Sample table title}
%   \centering
%   \begin{tabular}{lll}
%     \toprule
%     \multicolumn{2}{c}{Part}                   \\
%     \cmidrule(r){1-2}
%     Name     & Description     & Size ($\mu$m) \\
%     \midrule
%     Dendrite & Input terminal  & $\sim$100     \\
%     Axon     & Output terminal & $\sim$10      \\
%     Soma     & Cell body       & up to $10^6$  \\
%     \bottomrule
%   \end{tabular}
%   \label{tab:table}
% \end{table}

% \subsection{Lists}
% \begin{itemize}
% \item Lorem ipsum dolor sit amet
% \item consectetur adipiscing elit. 
% \item Aliquam dignissim blandit est, in dictum tortor gravida eget. In ac rutrum magna.
% \end{itemize}


\bibliographystyle{unsrt}  
\bibliography{references}  %%% Remove comment to use the external .bib file (using bibtex).
%%% and comment out the ``thebibliography'' section.


%%% Comment out this section when you \bibliography{references} is enabled.
% \begin{thebibliography}{1}

% \bibitem{kour2014real}
% George Kour and Raid Saabne.
% \newblock Real-time segmentation of on-line handwritten arabic script.
% \newblock In {\em Frontiers in Handwriting Recognition (ICFHR), 2014 14th
%   International Conference on}, pages 417--422. IEEE, 2014.

% \bibitem{kour2014fast}
% George Kour and Raid Saabne.
% \newblock Fast classification of handwritten on-line arabic characters.
% \newblock In {\em Soft Computing and Pattern Recognition (SoCPaR), 2014 6th
%   International Conference of}, pages 312--318. IEEE, 2014.

% \bibitem{hadash2018estimate}
% Guy Hadash, Einat Kermany, Boaz Carmeli, Ofer Lavi, George Kour, and Alon
%   Jacovi.
% \newblock Estimate and replace: A novel approach to integrating deep neural
%   networks with existing applications.
% \newblock {\em arXiv preprint arXiv:1804.09028}, 2018.

% \end{thebibliography}


\end{document}
