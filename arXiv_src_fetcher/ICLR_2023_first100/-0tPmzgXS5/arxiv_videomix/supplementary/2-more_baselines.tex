\section{Additional Experiments}
\label{appendix:more_baseline}

\subsection{HMDB-51 and UCF-101}

We evaluated VideoMix on HMDB-51~\cite{kuehne2011hmdb} and UCF-101~\cite{soomro2012ucf101} benchmarks. Table~\ref{hmdb51} and Table\ref{ucf101} shows the results.
Our method consistently boosts the top-1 accuracy against the baseline models.

\begin{table}[h]
\centering
\begin{tabular}{@{}ll@{}}
\toprule
Methods              & top1 acc.  \\ \midrule
I3D (Baseline)	& 66.0 \\
I3D + VideoMix	&\textbf{66.9 (+0.9)}   \\ \midrule
\end{tabular}
\caption{HMDB-51 benchmark results.}
\label{hmdb51}
\end{table}

\begin{table}[h]
\centering
\begin{tabular}{@{}ll@{}}
\toprule
Methods              & top1 acc.   \\ \midrule
VGG16 (Baseline)	& 79.8 \\
VGG16 + VideoMix & \textbf{81.7 (+2.1)}\\ 
I3D (Baseline)	& 93.3 \\
I3D + VideoMix	& \textbf{93.4 (+0.1)} \\ \midrule
\end{tabular}
\caption{UCF-101 benchmark results.}
\label{ucf101}
\end{table}

\subsection{Additional ablation study}





We subsample the mini-Kinetics dataset (10$\%$, 25$\%$, 50$\%$, and 100$\%$) and train the SlowOnly-34 model on the sampled datasets. 
Table~\ref{ablation:datasize} shows the results. 
Results show that VideoMix consistently improves accuracies against the baseline for all the subsets of the Mini-Kinetics dataset.

\begin{table}[h]
\centering
\begin{tabular}{@{}lcccc@{}}
\toprule
Dataset size &	10$\%$	& 25$\%$	& 50$\%$	& 100$\%$ \\ \midrule
Baseline	& 31.2 &	47.2	& 67.6 & 	75.2 \\
VideoMix	& \textbf{33.3}	& \textbf{49.6}	& \textbf{68.0}	& \textbf{77.6} \\ \midrule 
\end{tabular}
\caption{\textbf{Impact of dataset sizes.} Top-1 accuracy is reported.}
\label{ablation:datasize}
\end{table}


To see the complementarity among the data augmentation methods (e.g., Cutout, Mixup, RandAug, and VideoMix), we conduct various combinations of data augmentation methods.
Starting from the standard augmentation (flip and random resize), we stack up VideoMix, Mixup, Cutout, and RandAugment.
Table~\ref{ablation:adding} shows the results. 
VideoMix boosts performance against the standard augmentation, a relatively weak augmentation. 
Combining VideoMix with other strong augmentations (Cutout or RandAugment) degrades performance since the combination leads to an excessive amount of regularization.
Mixup also shows similar tendency that combining with other augmentations leads to degraded performance. 

\begin{table}[h]
\tabcolsep=0.05cm
\centering
\begin{tabular}{@{}cccccc@{}}
\toprule
Standard Aug. &	VideoMix	& Mixup & Cutout	& RandAug	& top1  \\ \midrule
\checkmark	& & & &	&		75.2 \\
\checkmark	& \checkmark & &	& &		\textbf{77.6} \\
\checkmark	& \checkmark & & \checkmark & &		76.3 \\
\checkmark	& \checkmark & &	& \checkmark &	76.4 \\ 
\checkmark	& \checkmark & & \checkmark & \checkmark &	73.5 \\
\checkmark	& & \checkmark &  &	&	77.0 \\
\checkmark	& & \checkmark & \checkmark &	&	74.2 \\
\checkmark	& & \checkmark & \checkmark & \checkmark &	76.7  \\
 \midrule
\end{tabular}
\caption{\textbf{Complementarity among data augmentations.}}
\label{ablation:adding}
\end{table}