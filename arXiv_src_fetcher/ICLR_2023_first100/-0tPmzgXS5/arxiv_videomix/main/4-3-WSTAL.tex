
\subsection{Weakly Supervised Temporal Action Localization}
\label{section:experiments:wstal}

The goal of weakly supervised temporal action localization (WSTAL) is to localize actions in untrimmed videos with a classifier trained using video-wise class labels only. 
Given a video input sequence, WSTAL model predicts the sequence's class label and also generates one-dimensional temporal proposals to localize actions in the video. 
WSTAL models do not exploit temporal action annotations during training and the generated temporal proposals are evaluated on validation videos with validation ground truth annotations. 
To localize the action instances well, a video model recognizes action categories from full video sequences and not focus on small discriminant frames of the action.
Through the WSTAL experiments, we verify that VideoMix improves the temporal localization ability of an action recognition models by guiding them to attend on wider frames of action. 
To evaluate the temporal localizability, we apply VideoMix over the baseline WSTAL methods.






\paragraph{Dataset.}
We conduct weakly supervised temporal action localization (WSTAL) task on THUMOS’14 dataset~\cite{THUMOS14}. 
THUMOS’14 dataset originally consists of 13,320 trimmed videos for training and 2,584 untrimmed videos for validation with 101 action categories.
We follow previous WSTAL methods' setting~\cite{lee2020bas,nguyen2018weakly,shou2018autoloc}.
We train WSTAL models with the 20 class subset of the untrimmed videos, which consists of 200 untrimmed videos without temporal annotations. 
The temporal localization performance of a model is evaluated by 212 untrimmed videos with temporal annotations. 
WSTAL on THUMOS’14 dataset is a challenging task since the length of untrimmed video could be quite long (up to 26 minute) and multiple actions could exist in a video. 

\paragraph{Training and inference.}
For training, we first extract I3D~\cite{carreira2017quo} features from the training videos as done in~\cite{lee2020bas}.
We sample $750$ video segments from a training video and RGB frames and optical flows are separately fed into dual-path I3D network. 
Each RGB and optical flow frame results in $1024$-dimensional feature, thus the dimension of extracted feature for a video is $750\times1024$ for RGB input, and $750\times2048$ for both using RGB and optical flow input (RGB+flow).
The WSTAL model, which consists of two $3\times3$ convolutional layers followed with LeakyReLU activation and a $1\times1$ convolutional layer, takes the extracted features and predicts its class label.
Since the network is trained with pre-computed features, we do not consider spatial VideoMix, but utilize \textbf{temporal VideoMix} on the extracted features.
Implementation details are in Appendix~\ref{appendix:wstal}.
For evaluation, temporal class activation mapping (T-CAM)~\cite{zhou2016CAM,nguyen2018weakly} is utilized to localize action instances along the temporal dimension. 
We threshold T-CAM below 50$\%$ of the highest value, and all 1-dimensional continuous segments are considered as action instance proposals as in~\cite{singh2017hide}. 
Evaluation metric is average precision (AP) of intersection over union (IoU) thresholds from $0.1$ to $0.9$, and we report their mean value (mAP) as in~\cite{singh2017hide,nguyen2018weakly,lee2020bas}.


\begin{table}[t]
\centering
\begin{tabular}{@{}lcc@{}}
\toprule
Method & mAP \\ \midrule
I3D RGB (Baseline) & 13.3 \\
+ Hide-and-Seek~\cite{singh2017hide} & 13.6 \\
+ Mixup~\cite{zhang2017mixup} & 14.0 \\
+ VideoMix & \textbf{14.2} \\ \midrule
I3D RGB+Flow (Baseline) & 17.8 \\
+ Hide-and-Seek~\cite{singh2017hide} & 18.2 \\
+ Mixup~\cite{zhang2017mixup} & 18.2 \\ 
+ VideoMix & \textbf{19.3} \\ \midrule
\end{tabular}
\caption{\textbf{Weakly supervised temporal action localization.} Results are on THUMOS14.
}
\label{table:experiment:wstal}
\end{table}



\paragraph{Results.}
Table~\ref{table:experiment:wstal} shows the WSTAL performances with and without optical flow features.
We compare VideoMix against the baselines including Hide-and-Seek~\cite{singh2017hide}, which has been reported to improve weakly supervised object localization on static images and WSTAL on videos.
We also compare against Mixup~\cite{zhang2017mixup}. 
We observe that VideoMix improves the temporal localization accuracy mAP of baseline by $+\mathbf{0.9}\%$ and $+\mathbf{1.5}\%$ with and without optical flows, respectively.
VideoMix also outperforms the Hide-and-Seek and Mixup in both scenarios showing its superior temporal localization ability.
We also conducted VideoMix with a stronger baseline, W-TALC~\cite{paul2018w}, and confirmed that VideoMix improves the performance of W-TALC from 31.1 to \textbf{32.3 (+1.2)} mAP, which is at the state-of-the-art level. 









