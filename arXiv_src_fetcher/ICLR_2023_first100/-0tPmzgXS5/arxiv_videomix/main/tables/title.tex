\begin{table*}[t]
\centering
\tabcolsep=0.07cm
\begin{tabular}{@{}lccccc@{}}
\toprule
Task     & \multicolumn{3}{c}{Action recognition}                                                                                                                                             & Localization  & Detection     \\ \midrule
Dataset  & \begin{tabular}[c]{@{}c@{}}Kinetics-400\\ (acc. \%) \end{tabular} & \begin{tabular}[c]{@{}c@{}} Mini-Kinetics\\ (acc. \%) \end{tabular} & \begin{tabular}[c]{@{}c@{}} Something-V2\\ (acc. \%) \end{tabular} &  \begin{tabular}[c]{@{}c@{}} THUMOS'14\\ (mAP)  \end{tabular}       & \begin{tabular}[c]{@{}c@{}} AVA \\ (mAP)  \end{tabular}           \\ \midrule
Model    & SlowOnly-50                                               & SlowFast-50                                                  & SlowFast-50                                                        & I3D T-CAM         & SlowFast-50  \\ \midrule
Baseline & 73.6                                                   & 79.5                                                    & 61.5                                                            & 17.8          & 23.2          \\
VideoMix & \textbf{74.9}                                          & \textbf{81.9}                                           & \textbf{62.3}                                                   & \textbf{19.3} & \textbf{24.9} \\
Improve. $\Delta$ & ($\mathbf{+1.3}$) & ($\mathbf{+2.4}$) & ($\mathbf{+0.8}$) & ($\mathbf{+1.5}$) & ($\mathbf{+1.7}$) \\ \midrule
\end{tabular}
\caption{\textbf{Overview of VideoMix performances.} 
We compare our VideoMix against the vanilla training strategy (Baseline) on various tasks.
VideoMix consistently improves action recognition, localization, and detection performances without any added parameter or a significant amount of computational overhead.
}
\label{table:intro}
\end{table*}