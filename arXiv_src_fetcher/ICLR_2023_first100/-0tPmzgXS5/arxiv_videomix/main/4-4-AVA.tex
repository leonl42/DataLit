
\subsection{AVA Action Detection}
\label{section:experiments:ava}

Kinetics pretraining is a widely-used practice for many video recognition tasks \cite{gu2018ava,sun2018actor,feichtenhofer2019slowfast}. 
We validate whether VideoMix pretrained models bring better performance on the downstream task of detecting actions in videos. 

\paragraph{Dataset.}
AVA v2.2 dataset~\cite{gu2018ava} consists of 235 training and 64 validation videos of human actions in videos. 
Each video is 15 minutes long, and the action locations are densely annotated as bounding boxes in space and time.
We follow the protocol in~\cite{gu2018ava} to train and evaluate the detection of 60 human action classes.
We use the mean average precision (mAP) metric to measure the performance of video action detection using a frame-level intersection-over-union (IoU) threshold $0.5$.

\paragraph{Detection framework.}
Our detector is based on the Faster R-CNN~\cite{ren2015faster} architecture, which is modified as in~\cite{feichtenhofer2019slowfast} to adapt to the video action detection task.
The spatial stride of the final convolutional block is reduced from $2$ to $1$ to increase the feature map size. The 2D RoIAlign layer~\cite{he2017mask} is replaced by the 3D RoIAlign. 
SlowOnly-50 ($8\times8$) and SlowFast-50 ($8\times8$) have been used as the backbone network for the detection framework. 
We use the human bounding box proposals provided by~\cite{feichtenhofer2019slowfast} computed by an off-the-shelf human detector fine-tuned on AVA persons. 

\paragraph{Training and inference.}

We initialize two detectors with the weights pretrained on Kinetics-400 with or without VidoeMix. We apply the same fine-tuning strategy afterwards to separate the effect of VideoMix on pretraining. 
We train detectors for $20$ epochs using the SGD optimizer with initial learning rate $0.1$ decayed by factor $0.1$ at $10$ and $15$ epoch.
The spatial dimension of the shorter side is resized to 256 pixels while maintaining the aspect ratio. 
$64$ consecutive frames are extracted for training.
Further details are in Appendix~\ref{appendix:ava}. 

\begin{table}[t]
\centering
\begin{tabular}{@{}lcc@{}}
\toprule
\multirow{2}{*}{Backbone}             & VideoMix & \multirow{2}{*}{val mAP} \\
 & Pretrained &  \\ \bottomrule
SlowOnly-50 ($8\times8$)        &       &  19.1 \\
SlowOnly-50 ($8\times8$)        &   \checkmark    &  \textbf{ 20.4  }  \\
\bottomrule
SlowFast-50 ($8\times8$)        &        &    23.2  \\
SlowFast-50 ($8\times8$)        &   \checkmark    & \textbf{ 24.9 } \\ \toprule
\end{tabular}
\caption{\textbf{AVA action detection.} Impact of VideoMix-pretraining on the transfer learning of AVA action detection.
}
\label{table:experiment:ava}
\end{table}

\paragraph{Results.}
Table~\ref{table:experiment:ava} shows the performances of our detector on the AVA benchmark. 
Pretraining the detector with VideoMix improves the performance of SlowOnly-50 and SlowFast-50 to $\mathbf{20.4}$ ($+\mathbf{1.3}$) and $\mathbf{24.9}$ ($+\mathbf{1.7}$) mAP, respectively. Switching the pretrained weights to the VideoMix version introduces the gain in detection performance for free. 
The weights will be published in the future.
