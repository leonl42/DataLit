\usepackage[utf8]{inputenc}
\usepackage[T2A]{fontenc}
%\usepackage[margin=1cm]{geometry}
%\pagestyle{empty}

\usepackage{intcalc, calc}
\usepackage[nomessages]{fp}
\usepackage{graphicx}
\usepackage{wrapfig}
\usepackage{float}


\def\lf{\left\lfloor}   
\def\rf{\right\rfloor}

\def\lc{\left\lceil}   
\def\rc{\right\rceil}


% для красивых раскрасок досок с помощью таблиц
%\usepackage[table,dvipsnames]{xcolor}

\usepackage{tikz}
\usepackage{color}
\usepackage{framed}
\usepackage{ragged2e}
\usepackage{amsmath, amssymb, amsfonts, amsthm}
\usepackage{wasysym}
\usepackage{soul, soulutf8}
\usepackage{calc}
\usepackage{mathtools,mathcomp}
\usepackage{pifont}
\usepackage{enumitem}

\usepackage{ifthen}

\DeclareMathOperator{\nodd}{\operatorname{\text{\textsc{нод}}}}
\DeclareMathOperator{\nokk}{\operatorname{\text{\textsc{нок}}}}
\DeclareMathOperator{\XOR}{\operatorname{\text{\textsc{xor}}}}
\DeclareMathOperator{\Var}{\operatorname{Var}}
\DeclareMathOperator{\spp}{\operatorname{sp}}


% команды \divby и \ndivby для трёхточечной делимости
\def\divdots{\rlap{\raisebox{-1pt}{.}}{\rlap{\raisebox{2pt}{.}}\raisebox{5pt}{.}}}
\def\divby{\mathrel{\divdots}}
\def\ndivby{\mathrel{
    \divdots
    \kern-0.35em\raise0.22ex\hbox{/}
}}


\renewcommand{\leq}{\leqslant}
\renewcommand{\geq}{\geqslant}

% нумерация
\def\fs{\kern 0.5em}
\newcounter{prcnt}
\newcounter{pucnt}
\newcommand{\prmain}[1]{ 
    \medskip%
    \setcounter{pucnt}{0}%
    \stepcounter{prcnt}%
    \noindent\textbf{%
    \theprcnt%
    \ifthenelse{\equal{#1}{1}}{*}{}%
    .}\fs%
}
\def\pr{\prmain{0}}
\def\prh{\prmain{1}}
\newcommand{\newproblems}{\setcounter{prcnt}{0}}

\newcommand{\pumain}[1]{
    \stepcounter{pucnt}{%
    \noindent\bf(\alph{pucnt}%
    \ifthenelse{\equal{#1}{1}}{*}{}%
    )}\fs%
}
\def\pu{\pumain{0}}
\def\puh{\pumain{1}}

%\setlength{\parindent}{0pt}

% вставка изображений
\newcommand{\insertpicture}[2]
{
    \begin{wrapfigure}{R}{#2+10pt}\flushright
        \vspace{-15pt}
        \includegraphics[width=#2]{#1}
        \vspace{-15pt}
    \end{wrapfigure}
}

% перенос символов типа
% a \hm+ b это
% a +
% + b
\newcommand*{\hm}[1]{
    #1\nobreak\discretionary{}%
    {\hbox{$\mathsurround=0pt #1$}}{}
}


\newtheorem{problem}{Problem}
\newtheorem*{exercise}{Упражнение}
\newtheorem{theorem}{Theorem}
\newtheorem{proposition}{Proposition}
\newtheorem{lemma}{Lemma}
\newtheorem{remark}{Remark}
\newtheorem*{theorem*}{Теорема}
\newtheorem{corollary}{Corollary}
\theoremstyle{definition}
\newtheorem{solution}{Решение}
\newtheorem*{definition}{Definition}

% теперь значки (1), (2), ... будут вставляться только к тем уравнениям, на которые есть ссылка
%\mathtoolsset{showonlyrefs}

% теперь заголовки (section, subsection, ...) не будут нумероваться
%\setcounter{secnumdepth}{0}