\documentclass{article}

% if you need to pass options to natbib, use, e.g.:
\PassOptionsToPackage{numbers, compress}{natbib}
% before loading neurips_2021

% ready for submission
\usepackage[preprint]{neurips_2021}

% to compile a preprint version, e.g., for submission to arXiv, add add the
% [preprint] option:
%     \usepackage[preprint]{neurips_2021}

% to compile a camera-ready version, add the [final] option, e.g.:
%     \usepackage[final]{neurips_2021}

% to avoid loading the natbib package, add option nonatbib:
%\usepackage[nonatbib]{neurips_2021}
\input{math_commands.tex}

\usepackage[utf8]{inputenc} % allow utf-8 input
\usepackage[T1]{fontenc}    % use 8-bit T1 fonts
\usepackage{hyperref}       % hyperlinks
\usepackage{url}            % simple URL typesetting
\usepackage{booktabs}       % professional-quality tables
\usepackage{amsfonts}       % blackboard math symbols
\usepackage{nicefrac}       % compact symbols for 1/2, etc.
\usepackage{microtype}      % microtypography
\usepackage{xcolor}         % colors
\usepackage{graphicx}
\usepackage{mathtools}% Loads amsmath
\usepackage[english]{babel}
\newtheorem{theorem}{Theorem}
%\usepackage{subfig}
\usepackage{setspace}
\usepackage{wrapfig}
\usepackage{algorithm}
\usepackage{algorithmic}
\usepackage{comment}
\usepackage{csquotes}
\usepackage{subcaption}

\newcommand{\han}[1]{\textcolor{red}{#1}}
\newcommand{\xr}[1]{\textcolor{red}{#1}}
\newcommand{\yx}[1]{\textcolor{purple}{#1}}
\newcommand{\jt}[1]{\textcolor{brown}{#1}}


\renewcommand{\eqref}[1]{(\ref{#1})}  %%%  Use  Eq.~\eqref{
\newtheorem{lemma}{Lemma}
\newtheorem{proof}{Proof}
\newtheorem{corollary}{Corollary}


\title{Towards the Memorization Effect of Neural Networks in Adversarial Training}

% The \author macro works with any number of authors. There are two commands
% used to separate the names and addresses of multiple authors: \And and \AND.
%
% Using \And between authors leaves it to LaTeX to determine where to break the
% lines. Using \AND forces a line break at that point. So, if LaTeX puts 3 of 4
% authors names on the first line, and the last on the second line, try using
% \AND instead of \And before the third author name.

\author{%
  Han Xu, Xiaorui Liu, Wentao Wang, Anil K. Jain. Jiliang Tang\\ 
  Department of Computer Science and Engineering\\
  Michigan State University\\
  \And Wenbiao Ding, Zhongqin Wu, Zitao Liu\\
  TAL Education Group\\
  
}



\begin{document}

\maketitle

\begin{abstract}
Recent studies suggest that ``memorization'' is one important factor for overparameterized deep neural networks (DNNs) to achieve optimal performance. Specifically, the perfectly fitted DNNs can memorize the labels of many atypical samples, generalize their memorization to correctly classify test atypical samples and enjoy better test performance. While, DNNs which are optimized via adversarial training algorithms can also achieve perfect training performance by memorizing the labels of atypical samples, as well as the adversarially perturbed atypical samples. However, adversarially trained models always suffer from poor generalization, with both relatively low clean accuracy and robustness on the test set. In this work, we study the effect of memorization in adversarial trained DNNs and disclose two important findings: \textbf{(a)} Memorizing atypical samples is only effective to improve DNN's accuracy on clean atypical samples, but hardly improve their adversarial robustness and \textbf{(b)} Memorizing certain atypical samples will even hurt the DNN's performance on typical samples. Based on these two findings, we propose \textit{Benign Adversarial Training (BAT)} which can facilitate adversarial training to avoid fitting ``harmful'' atypical samples and fit as more ``benign'' atypical samples as possible. In our experiments, we validate the effectiveness of BAT, and show it can achieve better clean accuracy vs. robustness trade-off than baseline methods, in benchmark datasets such as CIFAR100 and Tiny~ImageNet.
%\jt{we need to discuss some experimental conclusions and findings}


\end{abstract}

\section{Introduction}\label{sec:intro}
% Multi-armed bandit (MAB) is a classic sequential decision making problem \citep{auer2002finite}, where a learning agent chooses among competing actions sequentially to maximize its accumulative reward over time. 
% %Despite its simplicity, MAB exemplifies the exploration-and-exploitation dilemma that also exists in more complicated problems. 
% An important extension of MAB, named linear contextual bandit \citep{li2010contextual}, incorporates contextual information in the problem setting, by assuming a linear mapping between the context and expected reward. It has gained popularity in various applications, such as recommender systems \citep{li2010contextual}, display advertisement \citep{li2010exploitation} and clinical trials \citep{durand2018contextual}.
% Most existing linear bandit solutions are designed under a centralized learning setting, i.e., data is readily available at a central server. However, with the increasing public concerns of privacy, especially the bandit algorithms usually directly learn from user data,
% %more and more people are reluctant to provide their own data and strict regulations on data usage like GDPR have also went into effect \cite{voigt2017eu}, which makes 
% there is a growing demand to keep data decentralized and push the learning of bandit models to the client side. 
% % This idea is also made much more feasible due to the growing computational power of edge devices nowadays. 

% Federated learning has recently emerged as a promising setting for decentralized machine learning.
% % , and its effectiveness was first validated at a large scale by training a global model across all mobile devices via the Google Keyboard Android application \cite{konevcny2016federated}. 
% %The term ``federated learning" was first introduced by \citet{mcmahan2017communication} with an emphasis on efficiently training deep models over mobile device applications. As significant amount of later works have applied federated learning to other applications, there may be variations in its meaning for different research communities. 
% Since its debut in \citet{mcmahan2017communication}, there have been variations in its definition for different applications \citep{yang2019federated}.
% In this paper, we follow the general definition by \citet{kairouz2019advances}: multiple clients collaborate in solving a machine learning problem under the coordination of a central server, while keeping each client's raw data local. 
% So far, most existing works in federated learning study offline supervised learning problems \citep{konevcny2016federated,zhao2018federated}, where labeled training instances already sit on the client side. How to perform bandit learning under the federated learning setting remains underexplored.
As a popular online learning problem, linear contextual bandit has been used for a variety of applications, including recommender systems \citep{li2010contextual}, display advertisement \citep{li2010exploitation} and clinical trials \citep{durand2018contextual}. While most existing solutions are designed under a centralized setting (i.e., data is readily available at a central server), in response to the increasing application scale and public concerns of privacy, there is a growing demand to keep data decentralized and push the learning of bandit models to the client side.
% As a classic sequential decision making problem, linear contextual bandit has been widely used for a variety of real-world applications, including recommender systems \citep{li2010contextual}, display advertisement \citep{li2010exploitation} and clinical trials \citep{durand2018contextual}. 
% Most existing solutions are designed under a centralized learning setting, i.e., data is readily available at a central server. However, with the increasing public concerns of privacy, especially the bandit algorithms usually directly learn from user data,
% there is a growing demand to keep data decentralized and push the learning of bandit models to the client side. 
Federated learning has recently emerged as a promising setting for decentralized machine learning \citep{konevcny2016federated}.
% , and its effectiveness was first validated at a large scale by training a global model across all mobile devices via the Google Keyboard Android application \cite{konevcny2016federated}. 
%The term ``federated learning" was first introduced by \citet{mcmahan2017communication} with an emphasis on efficiently training deep models over mobile device applications. As significant amount of later works have applied federated learning to other applications, there may be variations in its meaning for different research communities. 
Since its debut in \citeyear{mcmahan2017communication}, there have been many variations for different applications \citep{yang2019federated}. However, most existing works study offline supervised learning problems \citep{li2019convergence,zhao2018federated}, which only concerns optimization convergence over a fixed dataset. How to perform federated bandit learning remains under-explored, and is the main focus of this paper. 

Analogous to its offline counterpart, the goal of federated bandit learning is to minimize the cumulative regret incurred by $N$ clients during their online interactions with the environment over time horizon $T$,
% $N$ clients in a learning system need to collaborate to minimize the overall cumulative regret over a finite time horizon $T$, 
while keeping each client's raw data local. Take recommender systems as an example, where the clients correspond to the edge devices that directly interact with user by making recommendations and receiving feedbacks. Unlike centralized setting where observations from all clients are immediately transmitted to the server to learn a single model, in federated bandit learning, each client makes recommendations based on its local model, with occasional communication for collaborative model estimation.

% In this paper, we follow the general definition by \citet{kairouz2019advances}: multiple clients collaborate in solving a machine learning problem under the coordination of a central server, while keeping each client's raw data local. 


%Though having potential for wide range of applications, online learning problems like linear bandit in federated learning setting, a.k.a. federated linear bandits \cite{dubey2020differentially}, have not attracted enough attention and still remain an open problem. 

% Therefore, it is a natural idea to study contextual linear bandit in a federated learning paradigm, which is also referred to as federated linear bandits \cite{dubey2020differentially}. In a federated learning paradigm, multiple clients collaborate in solving a machine learning problem, under the coordination of a central server, and each client's raw data is stored locally and not transferred to the server. 
% when linear bandit algorithms are applied to the federated learning paradigm, because these algorithms assume a traditional centralized machine learning system where all the data are collected together and all the computation happens in one machine or data center. 
Several new challenges arise in this problem setting. 
The first is the conflict between the need of timely data/model aggregation for \emph{regret minimization} and the need of \emph{communication efficiency}, since communication is the main bottleneck for many distributed application scenarios, e.g., communication in a network of mobile devices can be slower than local computation by several orders of magnitude \citep{huang2013depth}. A well-designed communication strategy becomes vital to strike the balance. 
In addition, 
% constraints from real-world applications should also be taken into consideration when designing the communication strategy. For example, 
the clients often have various response time and even occasional unavailability in reality, due to the differences in their computational and communication capacities.
% the clients may differ in their computational and communication capacities. This will lead to various response time and even occasional unavailability. 
This hampers global synchronization employed in existing federated bandit solutions \citep{wang2019distributed,dubey2020differentially}, which requires the server to first send a synchronization signal to all clients, wait and collect their returned local updates, and finally send the aggregated update back to every client.
Second, it is very restrictive to only assume homogeneous clients, i.e., they solve the same learning problem. 
% As bandit algorithms are mostly deployed to interact with individual users, studying heterogeneous clients with personalized learning problems has a greater potential.
Studying \emph{heterogeneous clients} with distinct learning problems has a greater potential in practice.
This is referred to as ``\emph{non-IIDness}" of data in the context of federated learning, e.g., the difference in $\mathcal{P}_{i}(\bx,y)=\mathcal{P}_{i}(\bx) \mathcal{P}_{i}(y|\bx)$ is caused by each client $i\in[N]$ serving a particular user or group of users, a particular geographic region, or a particular time period. Apparently, it is also unreasonable to assume every client has equal amount of new observations, which however is assumed in existing works. 

%To be more concrete, due to the time-varying arm set $\cA_{t}$ and the dependence on history data for arm selection in linear bandit, context vector $X$ is non-IID in nature and is not the main concern. 
% It is not a major concern since the performance metric, i.e. regret $r_{t}$, is defined against the best arm in $\cA_{t}$. 

% For example, internet connection and the different computation power of devices.
% \textcolor{red}{reasons we need async algo}

% This naturally leads to the question: how to balance between regret minimization and communication efficiency in the federated linear bandit problem.
To address the first challenge, we propose an asynchronous event-triggered communication framework for federated linear bandit. 
%Our event-triggering mechanism offers a flexible way to balance between the regret-minimization and communication-efficiency dilemma. 
Communication with a client happens only when the last communicated update to the client becomes irrelevant to the latest one; and we prove only by then effective regret reduction can be expected in this client because of the communication. 
Under this asynchronous communication, each client sends local update to and receives aggregated update from the server independently from other clients, with no need for global synchronization. This improves our method's robustness against possible delays and temporary unavailability of clients. It also brings in reduced communication cost when the clients have distinct availability of new observations, because global synchronization requires every client in the learning system to send its local update despite the fact that some clients can have very few new observations since last synchronization.
% make the proposed method more robust and practical against the infrastructure constraints, because the aggregated update sent to each client is asynchronous and  
% This makes our method more robust against possible delays in the communication, and we prove that the client enjoys the same benefit in regret reduction as long as it receives the update before its next interaction with the environment.

To address the second challenge, we design algorithms for federated linear bandit with both ``\emph{IIDness}" and ``\emph{non-IIDness}" based on the proposed communication framework. We consider two different assumptions on the reward functions. First, all the clients share a common reward function i.e., a single model is learned for all clients. Second, each client has a distinct reward function with mutual dependence captured by globally shared components in the unknown parameter, which resembles 
%so one model per client is learned during the interaction with the environment, which in essence is similar to the problem considered in
federated multi-task learning \citep{smith2017federated}.
We rigorously prove the upper bounds of accumulative regret and communication cost for the proposed algorithms in these two settings, and conduct extensive empirical evaluations to demonstrate the effectiveness of our proposed framework.
% especially its flexibility in balancing the trade-off between regret and communication cost.
\vspace{-0.3cm}
\section{Definition and Notation}
\vspace{-0.1cm}
\subsection{Atypical Samples and Memorization}\label{sec:def_atypical}
In this section, we start by introducing necessary concepts and definitions about the memorization effects. As well known, overparameterized DNNs have tremendous capacity to make them easy to perfectly fit the training dataset~\cite{zhang2016understanding}. In practice, on common benchmark classification tasks, such as CIFAR10~\cite{he2016deep}, CIFAR100 and ImageNet~\cite{krizhevsky2012imagenet}, we also tune the DNNs to achieve very high training accuracy even close to 100\%, and enjoy good test performance. While, this property of DNNs in practice cannot be well explained by standard theories about model generalization~\cite{evgeniou2000regularization, bartlett2002rademacher} from the regularization perspective. At a high level, the standard theories underline the importance of regularization on the model complexity, to make DNNs to avoid ``overfitting'' or ``memorizing'' the outliers and nonuseful samples in the training data. 

Fortunately, recent works~\cite{feldman2020does, feldman2020neural,bartlett2020benign, muthukumar2020harmless, chatterji2020finite} make significant progress to close this gap from both theoretical and empirical perspectives. They suggest that the memorization effect is one necessary property for DNNs to achieve optimal generalization performance. In detail, the empirical studies~\cite{feldman2020does, feldman2020neural} point out that the common benchmark datasets, such as CIFAR10, CIFAR100, ImageNet, contain a large portion of atypical samples (or namely, rare samples, sub-populations, etc.). These atypical samples look very different from the other samples in the main distribution of its labeled class (see Appendix~\ref{app:atypical}), and are statistically indistinguished from outliers or mislabeled samples. Because these atypical samples are deviated from the main distribution, DNNs can only fit these samples by memorizing their labels. Moreover, without memorizing these atypical samples during training, the DNNs can totally fail to predict the atypical samples appearing in the test set~\cite{feldman2020neural}.
%As the works~\cite{feldman2020neural} suggest, 





\noindent\textbf{Identify Atypical Samples.} To identify such atypical samples in common datasets in practice, the work~\cite{feldman2020neural} proposes to examine which training samples can only be fitted by memorization, and measure each training sample's \textit{``memorization value''}. Formally, for a training algorithm $\mathcal{A}$ (i.e., ERM), the memorization value ``$\text{mem}(\mathcal{A}, \mathcal{D}, x_i)$'' of a training sample $(x_i,y_i)\in \mathcal{D}$ in training set $\mathcal{D}$ is defined as:
\begin{align}\label{eq:mem}
    \text{mem}(\mathcal{A}, \mathcal{D}, x_i) = \underset{F\leftarrow \mathcal{A}(\mathcal{D})}{\text{Pr.}} (F(x_i) = y_i) - \underset{F\leftarrow \mathcal{A}(\mathcal{D}\backslash x_i)}{\text{Pr.}} (F(x_i) = y_i),
\end{align}
which calculates the difference between the model $F$'s accuracy on $x_i$ with and without $x_i$ removed from the training set $\mathcal{D}$ of algorithm $\mathcal{A}$. 
Note that for each sample $x_i$, if its memorization value is high, it means that removing $x_i$ from training data will cause the model with a high possibility to wrongly classify itself, so $x_i$ is very likely to be fitted only by memorization and be atypical. From the work~\cite{feldman2020neural}, these atypical samples have non-ignorable portion in common datasets. For example, there are around $40\%$ training samples in CIFAR100 having a large memorization value $>0.15$. 

The similar strategy can also facilitate to find atypical samples in the test set, which are the samples that are strongly influenced by atypical training samples.
In detail, by removing an atypical training sample $(x_i, y_i)$, we calculate its \textit{``influence value''} on each test sample $(x_j', y_j')\in\mathcal{D}'$ in test set $\mathcal{D}'$:
\begin{align}\label{eq:infl}
    \text{infl}(\mathcal{A}, \mathcal{D}, x_i, x'_j) = \underset{F\leftarrow \mathcal{A}(\mathcal{D})}{\text{Pr.}} (F(x_j') = y_j') - \underset{F\leftarrow \mathcal{A}(\mathcal{D}\backslash x_i)}{\text{Pr.}} (F(x_j') = y_j').
\end{align} 
If the sample pair $(x_i,x'_j)$ has a high influence value, removing the atypical sample $x_i$ will drastically decrease the model's accuracy on $x'_j$. It suggests that the model's prediction on $x'_j$ is mainly based on the memorization of $x_i$, thus, $x'_j$ is the corresponding atypical sample of $x_i$. The sample pair $x_i$ and $x'_j$ is called a \textit{high-influence pair}, if they have a high influence value and belong to the same class. In practice, the high-influence pairs of images are typically visually similar and have similar semantic features. The memorization benefits the model's performance especially on these test atypical samples, and hence boosts the overall test accuracy. 

\vspace{-0.2cm}
\subsection{Adversarial Training}
Similar to classification models trained via empirical risk minimization (ERM) algorithms, adversarial training methods
~\cite{madry2017towards, kurakin2016adversarial} are also devised to fit the whole dataset by training the model on manually generated adversarial examples. Formally, they are optimized to have the minimum adversarial loss:
\begin{align}
    \min_F \underset{x}{\E}~  \left[\max_{||\delta||\leq\epsilon} \mathcal{L}(F(x+\delta),y)\right],
    \label{eq:adv_training}
\end{align}
which is the model $F$'s average loss on the data $(x,y)$ that perturbed by adversarial noise $\delta$.
These adversarial training methods~\cite{kurakin2016adversarial, madry2017towards, zhang2019theoretically} have been shown to be one of the most effective approaches to improve the model robustness against adversarial attacks. Note that similar to traditional ERM, adversarially trained models can also achieve very high training performance. For example, under ResNet18~\cite{he2016deep}, PGD adversarial training~\cite{he2016deep} can achieve over 99\% clean accuracy and 85\% adversarial accuracy on the training set of CIFAR~100. Under a larger network with WideResNet28-10 (WRN28), it can achieve 100\% clean accuracy and 99\% adversarial accuracy. It suggests that these DNN models have sufficient capacity to memorize the labels of these atypical samples and their adversarial counterparts. However, different from ERM, adversarially trained models usually suffer from bad generalization performance on the test set. For example, the ResNet18 model can only have 57\% and 22\% test clean accuracy and adversarial accuracy ($\sim59\%$ and $24\%$ on WRN28). Moreover, the study~\cite{rice2020overfitting} suggests that during the adversarial training process, the model's test adversarial accuracy keeps dropping as more training data is fitted (after the first time learning rate decay). Thus, these facts indicate that the memorization in adversarial training is probably not always beneficial to the test performance and requires deep understanding. In the following sections, we will empirically draw a significant connection between these properties with the memorization effect.
\vspace{-0.4cm}
\section{Atypical Samples in Adversarial Training}\label{sec:pre}
\vspace{-0.2cm}
In this section, we attempt to understand adversarial training's behavior by studying its relation with the memorization effect. The discussions are mainly based on PGD-adversarial training~\cite{madry2017towards} on CIFAR~100. Implementation details are shown in Appendix~\ref{app:pre} where we also report the results in more datasets, i.e., CIFAR~10 and Tiny~ImageNet~\cite{le2015tiny}, and we make consistent observations.

\vspace{-0.2cm}
\subsection{Adversarial Robustness of Atypical Samples is Harder to Generalize}\label{sec:pre1}
\begin{figure}[t]
\centering
\hspace*{-1cm}
\subfloat[Clean (left) \& Adv Acc. (right) under ResNet18.]{
\label{fig:harder1}
\begin{minipage}[c]{0.55\textwidth}
\includegraphics[width = 0.5\textwidth]{figures/clean_rare_cifar.jpg}%
\hfill
\includegraphics[width = 0.5\textwidth]{figures/adv_rare_cifar100.png}
\end{minipage}
}
\hspace*{-0.4cm}
\subfloat[Clean (left) \& Adv Acc. (right) under WRN28.]{
\label{fig:harder3}
\begin{minipage}[c]{0.55\textwidth}
\includegraphics[width = 0.5\textwidth]{figures/wrn_clean_rare_cifar100.png}%
\hfill
\includegraphics[width = 0.5\textwidth]{figures/wrn_adv_rare_cifar100.png}
\end{minipage}
}
\caption{Clean Accuracy and Adversarial Accuracy on \textbf{Atypical} Set of CIFAR100}
\vspace{-0.5cm}
\label{fig:rare_benefit}
\end{figure}
In this subsection, we first check whether fitting atypical samples in adversarial training can effectively help the model correctly and robustly predict the atypical samples in the test set. We apply PGD adversarial training~\cite{madry2017towards} on original CIFAR~100 dataset for 200 epochs and evaluate the model's clean accuracy and adversarial accuracy on training atypical set $\mathcal{D}_\text{atyp}=\{x_i \in \mathcal{D}: \text{mem}(x_i)> 0.15\}$ and its corresponding test atypical set $\mathcal{D}_\text{atyp}' = \{x'_j \in \mathcal{D}': \text{infl}(x_i,x'_j)> 0.15, \text{for } \forall x_i\in \mathcal{D}_\text{atyp}\}$. In Fig.~\ref{fig:rare_benefit}, we report the algorithm's performance (clean \& adv. acc.) on these atypical sets along with the training process. From the results, we observe that both ResNet18 and WRN28 are capable to memorize all clean atypical samples and most adversarial atypical samples, since they both achieve $\approx100\%$ clean accuracy and high adversarial accuracy ($\approx80\%$ and $100\%$, respectively) on the training atypical set. 
As the training goes, the models' clean accuracy on the test atypical set gradually improves and finally approaches 40\%. 
However, their adversarial robustness keep constant around 10\% from the beginning epochs to the last ones, no matter how high the training performance is. 
These results suggest that the memorizing atypical samples in adversarial training may only improve their test clean accuracy, but hardly help their adversarial robustness. Recall that in CIFAR100, atypical set $\mathcal{D}_\text{atyp}$ (with memorization value > 0.15) covers 40\% samples of the whole dataset. Completely failing on the adversarial robustness of atypical samples could be one important reason that contributes to the poor robustness generalization of DNNs~\cite{rice2020overfitting}.

As the previous theoretical study~\cite{schmidt2018adversarially} states, for a model to have good robustness generalization performance, it always demands a training set with much larger amount of samples, than a model to have good clean accuracy generalization. In our case, the sub-population of each particular atypical sample has very low frequency to appear in the training set, and it is always deviated from the main sub-population. Thus, in the sub-population of this atypical sample, it is equivalent to a classification task based on an extremely small dataset, with one or a few training samples given. Therefore, the adversarial robustness of atypical samples can be extremely hard to generalize. 
%\han{Note that the sub-population of each particular atypical sample has very low frequency to appear in the training set and it is always deviated from the main sub-population. Thus, for DNNs to classify the samples in the sub-population of this atypical sample, it is equivalent to a claication }



\vspace{-0.2cm}
\subsection{Memorizing Atypical Samples Hurts Typical Samples' Performance}\label{sec:pre2}
\vspace{-0.2cm}

%\jt{for Figure 2, Because of the high performance of clean performance, currently the difference of adversarial performance with different portions of atypical examples is not such obvious. can we put all adversarial performance in one figure and the clean performance in the other? In this way, I think the difference will be much more obvious}

In this subsection, we further observe that fitting atypical samples will even bring negative effects on ``typical'' samples. Here, we define ``typical'' samples as the subset of training set $\mathcal{D}$ which have low memorization value: $\mathcal{D}_\text{typ} = \{x_i\in\mathcal{D}:\text{mem}(x_i)<0.02\}$. It means that they are not fitted by memorization and are from the main sub-population in their class. To define the test typical set $\mathcal{D}'_\text{typ}$, we exclude all test samples which have high influence values from any atypical training samples, and also exclude the samples that using ERM algorithm $\mathcal{A}$ has low success rate to predict (the samples which cannot be learned from $\mathcal{D}$): $\mathcal{D}'_\text{typ} = \mathcal{D}' - \{x'_j: \text{infl}(x_i,x'_j)> 0.02, \text{for } \forall x_i\in \mathcal{D}_\text{atyp}\} \cup \{x'_j: \text{Pr.}_{F\leftarrow\mathcal{A}(\mathcal{D})}(F(x'_j) = y_j) < 0.8\}$.

\begin{figure}[t]
\centering
\hspace*{-1cm}
\subfloat[Clean (left) \& Adv Acc. (right) under ResNet18.]{
\label{fig:hurt1}
\begin{minipage}[c]{0.55\textwidth}
\includegraphics[width = 0.5\textwidth]{figures/poison_clean_ResNet18.png}%
\hfill
\includegraphics[width = 0.5\textwidth]{figures/poison_adv_ResNet18.png}
\end{minipage}
}
\hspace*{-0.4cm}
\subfloat[Clean (left) \& Adv Acc. (right) under WRN28.]{
\label{fig:hurt2}
\begin{minipage}[c]{0.55\textwidth}
\includegraphics[width = 0.5\textwidth]{figures/poison_clean_WRN.png}%
\hfill
\includegraphics[width = 0.5\textwidth]{figures/poison_adv_WRN.png}
\end{minipage}
}
\caption{Clean Accuracy and Adversarial Accuracy on \textbf{Typical} Set of CIFAR100}
\vspace{-0.5cm}
\label{fig:hurt}
\end{figure}
To demonstrate the negative effect from fitting atypical samples, we conduct PGD adversarial training~\cite{madry2017towards} for several trails on resampled CIFAR100 datasets: each dataset is constructed with the whole training typical set $\mathcal{D}_\text{typ}$, and a part of the training atypical set $\mathcal{D}_\text{atyp}$ (randomly sample 0\%, 20\% and 100\% in $\mathcal{D}_\text{atyp}$). In Fig.~\ref{fig:hurt}, we report the adversarially trained model's clean and adversarial accuracy on the test typical set $\mathcal{D}_\text{typ}'$ and check the impact of atypical samples on the typical samples. From the results, we find that the amount of atypical samples makes a significant influence on the typical samples. For example, under ResNet18, an adversarially trained model without atypical samples has 92\% clean accuracy and 52\% adversarial accuracy on the test typical samples (on the last epochs). While, the model trained with all atypical samples included only has 85\% and 44\% clean \& adv. accuracy, respectively. 
These results suggest: the more atypical samples exist in training set, the poorer performance the model will have on $\mathcal{D}'_\text{typ}$. In other words, these atypical samples act more like ``poisoning'' samples~\cite{biggio2012poisoning, xu2019adversarial} which can deteriorate the model's performance on typical samples and consequently hurt the overall performance. 


\noindent\textbf{Poisoning Atypical Samples} A natural question is what kind of atypical samples are likely to ``poison'' model robustness and why? Different from previous literature about poisoning samples in traditional ERM, which assume that poisoning samples are most mis-labeled samples~\cite{li2020gradient}, CIFAR100 is a clean dataset with no or very few wrong labels. However, we hypothesize that the atypical samples which poison the model performance might pertain some features of a ``wrong'' class. Recall that atypical samples are always distinct from the main data distribution in their labeled class, it is likely that they are closer to the distribution of a ``wrong'' class. As shown in  Fig.~\ref{fig:atypical_samples}, an atypical ``plate'' is visually very similar to images in ``apple''. If the model memorizes this atypical ``plate'' and predicts any samples with similar features to be ``plate'', the model cannot distinguish between ``apple'' and ``plate''. As a simple verification to this hypothesis, in Table~\ref{Tab:dist}, we empirically show that atypical samples can cause adversarial training to produce ``less-discriminative'' representations among different classes. Under the same experimental setting and the models above, we measure the average \textit{Cosine Distance (CD)~\footnotemark} of the models' pen-ultimate layer representation output, for each pair of samples (in the training typical set $\mathcal{D}_\text{typ}$) from different classes. Table~\ref{Tab:dist} shows that in adversarial training, fitting more atypical samples will result in a smaller distance for the representations of samples in different classes. It suggests that with atypical samples, DNNs learn more similar and mixed representations for different classes, which can degrade the typical samples' test performance.
\begin{table}[h]
\vspace{-0.5cm}
\small
\centering
\setlength{\tabcolsep}{12pt}
\caption{Class-wise Cosine Distance of Representations of Typical Samples}
\begin{tabular}{c|ccc}
\hline
\# of Atypical Samples & 0\% &20\% &100\%\\
\hline
ResNet18 & 0.66 & 0.62  & 0.59\\
\hline
WRN28 &0.64 & 0.61 & 0.57\\
\hline
\end{tabular}
\vspace{-0.4cm}
\label{Tab:dist}
\end{table}

\footnotetext{Cosine Distance: $\E_{x_1,x_2} [\frac{h(x_1)\cdot h(x_2)}{||h(x_1)||_2\cdot||h(x_2)||_2}]$, where $h(\cdot)$ is the pen-ultimate layer output of DNN model $F(\cdot)$, and $x_1,  x_2\in \mathcal{D}_\text{typ}$ and from different classes.}


It is also worth to mention that this poisoning phenomenon does not appear in the traditional ERM algorithm (results in Appendix~\ref{app:pre}). In ERM, memorizing atypical samples will neither hurt typical sample's performance or degrade the feature space discrimination. A possible explanation is that adversarially trained models use more ``semantically meaningful'' features for prediction~\cite{tsipras2018robustness, ilyas2019adversarial}. During adversarial training, the models will not only memorize the labels of atypical samples, but also their semantic features. As a result, the adversarial trained DNNs will construct more mixed concepts of features, if the ``poisoning'' atypical samples exist.


\begin{figure}[t]
\subfloat{
\label{fig:poison_pair1}
\begin{minipage}[c]{0.5\textwidth}
\centering
\includegraphics[width = 0.8\textwidth]{figures/poison_pair1.pdf}
\end{minipage}
}
\subfloat{\label{fig:poison_pair2}
\begin{minipage}[c]{0.5\textwidth}
\centering
\includegraphics[width = 0.8\textwidth]{figures/poison_pair2.pdf}
\end{minipage}
}
\caption{Examples of Poisoning Atypical Samples}
\vspace{-0.4cm}
\label{fig:atypical_samples}
\end{figure}
\section{Experiment Setup}

\subsection{Model aspect ratio}\label{section:method:ratio}

For our Transformer models we fix the number of embedding features, sequence features, attention features, and the hidden layer dimension in the FFNs for each task.
We vary the number of layers, $L$, and the number of heads per layer, $H$, whilst keeping $L \times H$ constant.
Starting with typical values for $L$ \& $H$, we then move down to a single layer with one to two intermediate model aspect ratios, observing how test accuracy changes for trained models.
See \Cref{fig:wide} for an illustration of our deepest and widest models.

In all of our tasks we do not use pretrained embeddings or pretrained model parameters as this allows us to make a fairer comparisons.
Computing pretrained embeddings and weights that are optimised for each combination of attention and model aspect ratio would be computationally prohibitive, and using ones typically used for deep networks would introduce bias.


\subsection{Datasts and models}\label{section:method:datasets}

\begin{table}[!h]
    \caption{The different tasks and datasets used.}
    \label{table:tasks}
    \begin{center}
        \begin{tabular}{l | l l l l}
            \toprule
            \textbf{Task Name} & \textbf{Classification} & \textbf{Dataset} & \textbf{Input Type} & \textbf{Input Length} \\
            \midrule
            IMDb Token Level & Binary & IMDb Reviews & Review text tokens & 500 \\
            IMDb Byte Level & Binary & IMDb Reviews & Review text bytes & 1000 \\
            Listops & 10-way & LRA Listops & Listop bytes & 2000 \\
            Document Matching & Binary & ACL Anthology & Document bytes & 4000 \\
            \bottomrule
        \end{tabular}
    \end{center}
\end{table}

Primarily we investigate using 4 different text classification tasks, a vision based task is investigated in \Cref{sec:discussion:vit}.
The first two are sentiment analysis (binary classification) on the IMDb dataset.
One uses input embeddings at the token level with an input sequence length of 500, and the other uses input embeddings at the byte level and an input sequence length of 1k.
This second task is taken from LRA \citep{lra}, as are the final two.
The third task is Listops 10-way classification with a sequence length of 2k.
This task involves reasoning about sequences of hierarchical operations to determine a result, and the input is given at the byte level.
The final task used is byte level document matching, a binary classification task with a sequence length of 4k.
This uses the ACL anthology network for related article matching \citep{acl}.
We summarise each task in \Cref{table:tasks}, further details on them can be found in \citet{lra}.

For the text classification and Listops task we try four different model aspect ratios.
In terms of number of layers and heads per layer these are: 6 layers, 8 heads; 3 layers, 16 heads; 2 layers, 24 heads; and finally 1 layer, 48 heads.
As the matching task has an input sequence length of 4k, the models used are smaller to offset the computation size involved.
Thus the combinations we use are: 4 layers, 4 heads; 2 layers, 8 heads; 1 layer, 16 heads.

In order to investigate whether the type of the attention mechanism influences the effects of widening the attention layer, we test on 10 different types of Transformer attention, including the original dot-product attention \citep{tfm}.
The others are: Bigbird \citep{bigbird}, Linear Transformer \citep{linear_tfm}, Linformer \citep{linformer}, Local attention \citep{local_tfm}, Longformer \citep{longformer}, Performer \citep{performer}, Sinkhorn \citep{sinkhorn}, Sparse Transformer \citep{sparse_tfm}, and Synthesizer \citep{synthesizer}.
The implementations and hyper-parameter choices for each attention type are the same as used in LRA.
Unlike LRA, we do not test with Reformer \citep{reformer} due to it requiring the sequence features and attention features to have the same dimension. Training and other Transformer hyperparameters used for each task are given in \Cref{appendix:tasks}.


% \vspace*{-2mm}

\section{Experiments}
\label{sec: exp}
% \vspace*{-1mm}


% In this section, we present our experiment results on how model sparsity simplifies {\MU}, \textit{i.e.}, reducing the performance gap  between approximate unlearning and   {\retrain}.
%across different data-model setups,   unlearning scenarios, and evaluation metrics.

%In this section, we begin by introducing some essential experiment setups, and then empirically show the relationship between sparsity and MU across multiple datasets, various model architectures, diverse machine unlearning methods, and different MU settings. Compared to the `prune first, then unlearn' paradigm, we find that sparsity-aware MU can further boost the benefits of sparsity. Finally, we extend our methods to trojan model cleanse application to demonstrate the effectiveness of our methods further.  
%\SL{[A short paragraph to summarize the experiment section]} 


\subsection{Experiment setups}
\label{sec: exp_setup}
%\vspace*{-1mm}

\noindent \textbf{Datasets and models.}
Unless specified otherwise, our experiments will focus  on image classification under 
 CIFAR-10 \cite{krizhevsky2009learning}  using ResNet-18 \cite{he2016deep}. Yet,  
 experiments  on 
  additional datasets (CIFAR-100 \cite{krizhevsky2009learning}, SVHN \cite{netzer2011reading}, and ImageNet \cite{deng2009imagenet}) and an  alternative model architecture (VGG-16 \cite{simonyan2014very}) can  be found in Appendix\,\ref{appendix: additional results}.
  Across all the aforementioned datasets and model architectures, our experiments consistently show that model sparsification can effectively reduce the gap between approximate unlearning and exact unlearning. 
  %-\ref{appendix: additional results}
 
 % Additional datasets, such as CIFAR-100 \cite{krizhevsky2009learning}, SVHN \cite{netzer2011reading}, and ImageNet \cite{deng2009imagenet} will be  presented in the Appendix\,\ref{appendix: additional results}.  
 
 % Moreover, Appendix\,\ref{appendix: additional results} showcases the results of an alternative model architecture, namely VGG-16 \cite{simonyan2014very}, on CIFAR-10 alongside ResNet-18. 
% We will consider imagery    {datasets}     including CIFAR-10 \cite{krizhevsky2009learning}, CIFAR-100 \cite{krizhevsky2009learning}, SVHN \cite{netzer2011reading}, and ImageNet \cite{deng2009imagenet}. ResNet-18 \cite{he2016deep} or VGG-16 \cite{simonyan2014very} will give the corresponding image classifier.
% We default use CIFAR-10 as the dataset and ResNet-18 as the model architecture. 
%the image classification model by default is   ResNet-18 \cite{he2016deep}. Yet, VGG-16 \cite{simonyan2014very} will also be used for evaluating unlearning performance across   architectures. 


% Following previous work in machine unlearning and pruning, we consider four datasets, including CIFAR-10 \cite{krizhevsky2009learning}, CIFAR-100 \cite{krizhevsky2009learning}, SVHN \cite{netzer2011reading} and Tiny-ImageNet \cite{le2015tiny}, and two architecture types including ResNet-18 \cite{he2016deep} and VGG-16 \cite{simonyan2014very}. More datasets, model configurations, and setups are summarized in Table \ref{}. 

%\paragraph{{\MU} baselines and implementations.}


\noindent \textbf{Unlearning and pruning setups.}
% \SL{[Focus on implementation details of training methods in both 'prune first, then unlearn', and `sparsity-aware unlearning'. In the first category, you need to mention all implementation details of unlearning methods, and pruning methods. In the second category, you need to specify {\MUSparse} and {\MUAO}, etc.
% }
 %In our experiments, 
 We   focus on two unlearning scenarios mentioned in Sec.\,\ref{sec: primer_MU}, \textit{class-wise forgetting} and \textit{random data forgetting} ({$10\%$ of the whole training dataset} {together with 10 random trials}). 
 %Unless specified otherwise,  the class-wise forgetting will be the default setting.
 %that randomly removes a subset of data points from a single class. By default,  we will focus on the class-wise forgetting scenario. 
In the `\textit{prune first, then unlearn}' paradigm, we   focus on unlearning methods ({\FT}, {\GA}, {\FF},  and {\IU}) shown in Tab.\,\ref{tab: summary_MU_methods_metrics}
when applying to sparse models.
%including  fine-tuning (\FT),  gradient ascent (\GA),  Fisher forgetting (\FF), and   influence unlearning (\IU). 
We  implement these methods following their official repositories. However, it is worth noting that  the  implementation of {\FF} in \citet{golatkar2020eternal}
modifies the model architecture in class-wise forgetting, \textit{i.e.}, removes the  prediction head  of the class to be scrubbed. 
%As a result, it does not allow for unlearning random data points. 
By contrast, other   methods   keep the model architecture  intact during unlearning. 
Also, we choose {OMP} as the  default pruning method, 
as justified in Fig.\,\ref{fig: results_pruning_comparison}.
\iffalse 
If we relax such a condition for {\FF},  then the  unlearning performance would significantly degrade. Thus, even if it may lack fairness to compare  other methods with {\FF}, we cover the latter in class-wise forgetting for completeness. 
%Towards a fair comparison, we implement {\FF}  without modifying the model architecture. 
\fi
\iffalse
In the `\textit{sparsity-aware unlearning}' paradigm,  the sparsity-promoting regularization parameter $\gamma$  in   \eqref{eq: MUSparse} is set to   $\gamma = 5\times10^{-5}$ if it is given by a constant. This is found by a line search over $[10^{-5}, 10^{-1}]$ across   tradeoffs between testing accuracy and unlearning accuracy. 
\SL{We implement the linearly decaying scheduler by $\gamma_t = (1 - \frac{t}{T})\gamma$, where $t$ is the epoch index, and $T$ is the total number of  epochs. The linearly incr1easing scheduler is similarly given by $\gamma_t = \frac{t}{T} \gamma$ in Tab.\,\ref{tab: ablation_l1_scheduler}.}
\fi
\iffalse
\JH{In the `\textit{sparsity-aware unlearning}' paradigm,  the sparsity-promoting regularization parameter $\gamma$  in   \eqref{eq: MUSparse} is set to   $\gamma = 5\times10^{-5}$ if it is given by a constant, $\gamma=9\times10^{-5}$ for linear decaying scheduler, and $\gamma=1\times10^{-4}$ for linear increasing scheduler. They are found by a line search over $[10^{-5}, 10^{-1}]$ across  trade-offs between testing accuracy and unlearning accuracy. 
We implement the linearly decaying scheduler by $\gamma_t = (1 - \frac{t}{T})\gamma$, where $t$ is the epoch index, and $T$ is the total number of  epochs. The linearly increasing scheduler is similarly given by $\gamma_t = \frac{t}{T} \gamma$ in Tab.\,\ref{tab: ablation_l1_scheduler}.}
\JC{Need update @Jinghan}
\fi
In the `\textit{sparsity-aware unlearning}' paradigm, the sparsity-promoting regularization parameter $\gamma$ in \eqref{eq: MUSparse} is determined through the line search in the interval $[10^{-5}, 10^{-1}]$, with consideration for the trade-off between testing accuracy and unlearning accuracy. For all schedulers, $\gamma$ is set around to $5 \times 10^{-4}$. The linearly increasing and decaying schedulers are implemented as $\gamma_t = \frac{2t}{T} \gamma$ and $\gamma_t = (2 - \frac{2t}{T})\gamma$ respectively, where $t$ is the epoch index and $T$ is the total number of epochs. 
%as specified in Tab.\,\ref{tab: ablation_l1_scheduler}.
%}
% \JH{We employ a linear decay scheduler for $\gamma$, implemented as a function of epoch, denoted as $\gamma_t = (1 - \frac{t}{E})\gamma$. In this equation, $t$ represents the current epoch, $E$ stands for the total number of unlearning epochs, and $\gamma$ is the initial value.} \JC{@Jinghan add some description on scheduled parameters.}%Further, we implement {\MUAO} by choosing the pruning ratio $p\% = 20\%$ per iteration.
We refer readers to Appendix\,\ref{appendix: training and unlearning settings} for more details. 
%\JC{[Add further details to appendix]}

%following its recent benchmark in  \cite{ma2021sanity}. 

%As we have shown in Fig.\,\ref{fig: results_pruning_comparison}, OMP outperforms SynFlow \cite{tanaka2020pruning} in  generalization when the sparsity  increases.

%(\textit{e.g.},  $95\%$ sparsity in Fig.\,\ref{fig: results_pruning_comparison} \SL{[xxx]}). Meanwhile, OMP is more effective than IMP for {\MU} due to its  computation efficiency and less dependence on the training dataset. 


\iffalse 
In the `prune first, then unlearn' paradigm, we mainly focus on 3 SOTA pruning methods, \ding{172} SynFlow \cite{tanaka2020pruning}, \ding{173} OMP \cite{frankle2018lottery}, and \ding{174} IMP \cite{frankle2018lottery}, which follows the recent IMP benchmark's  \cite{ma2021sanity} setting. We chose four commonly used machine unlearning methods mentioned in section \ref{sec: primer_MU},  \ding{172} Fine-tuning (\FT), \ding{173} Gradient ascent (\GA), \ding{174} Fisher Forgetting (\FF), and \ding{175}  Influence Unlearning (\IU). We remark that {\FF} manually changes the last layer when forgetting the whole class. In other unlearning methods, we do not change the model architecture. For a fair comparison, we only conduct {\FF} on the CIFAR-10 dataset. Further experiments on {\FF} are listed in Appendix\ref{}. We choose OMP as our default pruning setting in the `prune first, then unlearn' experiment results and will focus on forgetting one class scenario. To make a fair comparison, we tune the hyperparameters carefully for {\IU} method at different sparsity levels. As for the `Sparsity-aware unlearning', we choose {\GA} and {\FT} these two methods to integrate with these proposed unlearning methods. We refer readers to Table \ref{} for more training and unlearning details, such as hyperparameters for unlearning methods and training epochs.
\fi 

\noindent \textbf{Evaluation metrics.}
We evaluate the unlearning performance following Tab.\,\ref{tab: summary_MU_methods_metrics}. 
Recall that {\UA}   and {\MIAF}   depict the \textit{efficacy} of {\MU}, {\RA}  reflects the \textit{fidelity} of {\MU}, and {\TA}   and {\RTE}  characterize the \textit{generalization ability} and the \textit{computation efficiency} of  an unlearning method. 
%It is worth noting that in the class-wise forgetting scenario, the class to be scrubbed will   be excluded in the testing dataset when evaluating {\TA} (testing accuracy). 
We implement MIA (membership inference attack) using the prediction confidence-based attack method \cite{song2019privacy,yeom2018privacy}, whose effectiveness has been justified in \citet{song2020systematic} compared to other   methods. We refer readers to Appendix\,\ref{appendix: metric settings} for more implementation details. {To more precisely gauge the proximity of each approximate {\MU} to {\retrain}, we introduce a metric termed `Disparity Average'. This metric quantifies the mean performance gap between each unlearning method and {\retrain} across all considered metrics. A lower value indicates closer performance to {\retrain}.}
%\SL{[variance, class-wise forgetting variance]}


% We remarked that test accuracy (\TA) is evaluated on the remaining classes in the test dataset in the unlearning one-class scenario. 

% As for MIA methods, we fulfill inference attacks based on prediction confidence \cite{song2019privacy,yeom2018privacy}, which effectiveness is demonstrated in the paper \cite{song2020systematic} compared to other MIA methods. T
% able \ref{} lists more details about our evaluation settings.  
% \SL{[you can refer to Table\,\ref{tab: summary_MU_methods_metrics} for unlearning baselines and metrics but include additional details. e.g. what is the specific MIA method used and why?]}


\subsection{Experiment results}
\label{sec: experiment_results}
\iffalse
\begin{table*}[htb]
\centering
\vspace*{1mm}
\caption{Performance overview of various MU methods  on dense and 95\%-sparse models considering different unlearning scenarios:
 class-wise forgetting, 
 and random data forgetting. The forgetting data of random data forgetting ratio is $10\%$ of the whole training dataset, 
 the sparse models are obtained using OMP \cite{ma2021sanity}, and the unlearning methods and evaluation metrics are summarized in Tab.\,\ref{tab: summary_MU_methods_metrics}. {Class-wise forgetting is conducted class-wise.}
The performance is reported in the form $a_{\pm b}$, with mean $a$ and standard deviation $b$ computed over $10$ independent trials. 
A performance gap  against \textcolor{blue}{{\retrain}} is provided 
in (\textcolor{blue}{$\bullet$}). Note that the better performance of approximate unlearning corresponds to the smaller performance gap with the gold-standard retrained model.
}
\label{tab: overall_performance}
% \vspace*{0.1in} % Requirements, do not delete.
% \vspace*{-1mm}
\resizebox{0.95\textwidth}{!}{
\begin{tabular}{c|cc|cc|cc|cc|c}
\toprule[1pt]
\midrule
  \multirow{2}{*}{\MU}& \multicolumn{2}{c|}{{\UA}} & \multicolumn{2}{c|}{{\MIAF}}& \multicolumn{2}{c|}{{\RA}} & \multicolumn{2}{c|}{{\TA}}&{\RTE}  \\ 
  & \multicolumn{1}{c|}{{\textsc{Dense}}}  & \multicolumn{1}{c|}{$\mathbf{95\%}$ \textbf{Sparsity}}
    & \multicolumn{1}{c|}{\textsc{Dense}}  & \multicolumn{1}{c|}{$\mathbf{95\%}$ \textbf{Sparsity}}
    & \multicolumn{1}{c|}{\textsc{Dense}}  & \multicolumn{1}{c|}{$\mathbf{95\%}$ \textbf{Sparsity}}
      & \multicolumn{1}{c|}{\textsc{Dense}}  & \multicolumn{1}{c|}{$\mathbf{95\%}$ \textbf{Sparsity}} & (min)
  \\
% \cline{3-10}

\midrule
\rowcolor{Gray}
\multicolumn{10}{c}{Class-wise forgetting} \\
\midrule
\retrain &\textcolor{blue}{$100.00_{\pm{0.00}}$}    & \textcolor{blue}{$100.00_{\pm{0.00}}$}
&\textcolor{blue}{$100.00_{\pm{0.00}}$}   & \textcolor{blue}{$100.00_{\pm{0.00}}$}
&\textcolor{blue}{$100.00_{\pm{0.00}}$}    & \textcolor{blue}{$99.99_{\pm{0.01}}$}
&\textcolor{blue}{$94.83_{\pm{0.11}}$}   & \textcolor{blue}{$91.80_{\pm{0.89}}$}
 &43.23\\
  \FT &$22.53_{\pm{8.16}}$ (\textcolor{blue}{$77.47$})&${73.64}_{\pm{9.46}}$  (\textcolor{blue}{${26.36}$})&$75.00_{\pm{14.68}}$ (\textcolor{blue}{${25.00}$})& ${83.02}_{\pm{16.33}}$ (\textcolor{blue}{${16.98}$}) 
  &$99.87_{\pm{0.04}}$ (\textcolor{blue}{$0.13$}) & ${99.87}_{\pm{0.05}}$ (\textcolor{blue}{${0.12}$})&$94.31_{\pm{0.19}}$ (\textcolor{blue}{$0.52$})
 &$94.32_{\pm{0.12}}$ (\textcolor{blue}{$2.52$})
&   2.52
  
  
  
  \\
 \GA &$93.08_{\pm{2.29}}$ (\textcolor{blue}{6.92}) &${98.09}_{\pm{1.11}}$ (\textcolor{blue}{${1.91}$})
& $94.03_{\pm{3.27}}$ (\textcolor{blue}{5.97})& ${97.74}_{\pm{2.24}}$ (\textcolor{blue}{${2.26}$})
& $92.60_{\pm{0.25}}$ (\textcolor{blue}{$7.40$})& $87.74_{\pm{0.27}}$ (\textcolor{blue}{$12.25$}) 
& $86.64_{\pm{0.28}}$ (\textcolor{blue}{$8.19$})& $82.58_{\pm{0.27}}$ (\textcolor{blue}{$9.22$}) 
&   0.33
 \\
  {\FF}  & $79.93_{\pm{8.92}}$ (\textcolor{blue}{$20.07$})& ${94.83}_{\pm{4.29}}$ (\textcolor{blue}{${5.17}$}) 
  & $100.00_{\pm{0.00}}$ (\textcolor{blue}{$0.00$})& ${100.00}_{\pm{0.00}}$ (\textcolor{blue}{$0.00$}) 
    & $99.45_{\pm{0.24}}$ (\textcolor{blue}{$0.55$})& ${99.48}_{\pm{0.33}}$ (\textcolor{blue}{${0.51}$})
        & $94.18_{\pm{0.08}}$ (\textcolor{blue}{$0.65$})& $94.04_{\pm{0.10}}$ (\textcolor{blue}{$0.28$})& 38.91
  \\
 \IU 
  &$87.82_{\pm{2.15}} $ (\textcolor{blue}{$12.18$})& ${99.47}_{\pm{0.15}}$ (\textcolor{blue}{${0.53}$})
 & $95.96_{\pm0.21}$ (\textcolor{blue}{$4.04$})
&${99.93}_{\pm{0.04}}$ (\textcolor{blue}{${0.07}$})
 &$97.98_{\pm{0.21}}$ (\textcolor{blue}{$2.02$}) 
 &$97.24_{\pm{0.13}}$ (\textcolor{blue}{$2.75$}) 
 &$91.42_{\pm{0.21}}$ (\textcolor{blue}{$3.41$})&${90.76_{\pm{0.18}}}$ (\textcolor{blue}{${1.04}$}) & 3.25
 \\
%  \FTSparse &$100.00_{\pm{0.00}}$  &$100.00_{\pm{0.00}}$  & $91.49_{\pm{1.21}}$&$87.17_{\pm1.31}$
%   &$100.00_{\pm{0.00}}$  &$100.00_{\pm{0.00}}$  & $91.69_{\pm{1.57}}$&$87.30_{\pm1.39}$
%   &$100.00_{\pm{0.00}}$  &$100.00_{\pm{0.00}}$  & $95.74_{\pm{0.54}}$&$88.97_{\pm1.00}$
% \\
% \FTAO  
%   & -&-  & -&-
% & $43.82_{\pm{11.68}}$& $98.64_{\pm{0.71}}$ & $99.96_{\pm{0.03}}$&$94.79_{\pm0.07}$
%   &$99.80_{\pm{0.19}}$  &$100.00_{\pm{0.00}}$ & $99.86_{\pm{0.05}}$&$94.55_{\pm0.11}$
% \\
% \midrule
% \rowcolor{Gray}
% \multicolumn{10}{c}{Random data forgetting (per class)} \\
% \midrule
% \retrain &\textcolor{blue}{$56.27_{\pm{0.07}}$}   & \textcolor{blue}{$57.86_{\pm{0.05}}$ }
% &\textcolor{blue}{$75.23_{\pm{0.14}}$}   & \textcolor{blue}{$76.14_{\pm{0.11}}$ }
% &\textcolor{blue}{$100.00_{\pm{0.00}}$}    & \textcolor{blue}{$99.99_{\pm{0.01}}$}
% &\textcolor{blue}{$89.54_{\pm{0.11}}$ }   & \textcolor{blue}{$88.41_{\pm{0.89}}$}
%  & 41.63 \\
%   \FT &$1.89_{\pm{0.79}}$(\textcolor{blue}{$54.38$})&${19.34}_{\pm{1.41}}$ (\textcolor{blue}{$38.52$})& $17.11_{\pm{2.21}}$ (\textcolor{blue}{$58.12$}) &${35.18}_{\pm{2.12}} $ (\textcolor{blue}{$40.96$})
%   &$99.92_{\pm{0.03}}$ (\textcolor{blue}{$0.08$}) & $99.21_{\pm{0.05}}$ (\textcolor{blue}{$0.78$})&$93.50_{\pm{0.52}}$ (\textcolor{blue}{$3.96$})
%  &${91.20}_{\pm{0.12}}$ (\textcolor{blue}{$2.79$})
% & 2.36
  
  
  
%   \\
%  \GA
% &$50.75_{\pm{3.29}} $ (\textcolor{blue}{$5.52$})& ${59.12}_{\pm{3.17}}$ (\textcolor{blue}{$1.26$})  
% &$69.27_{\pm{4.27}} $ (\textcolor{blue}{$5.96$})& ${74.06}_{\pm{3.15}}$ (\textcolor{blue}{$2.08$})
% &$98.43_{\pm{0.20}} $ (\textcolor{blue}{$1.57$})& ${98.59}_{\pm{0.17}}$ (\textcolor{blue}{$1.40$})
% &$87.56_{\pm{0.24}} $ (\textcolor{blue}{$1.98$})& ${87.12}_{\pm{0.31}}$ (\textcolor{blue}{$1.29$}) &0.29
%  \\
%   $\FF$  &$4.85_{\pm{4.20}} $ (\textcolor{blue}{$51.42$}) &${6.92}_{\pm{3.72}} $ (\textcolor{blue}{$50.94$})
%   &$11.29_{\pm{5.12}} $ (\textcolor{blue}{$63.94$}) &${12.37}_{\pm{4.54}} $ (\textcolor{blue}{$63.77$})
%   &$97.30_{\pm{0.52}} $ (\textcolor{blue}{$2.70$}) &$96.13_{\pm{0.41}} $ (\textcolor{blue}{$3.86$})
%   &$88.94_{\pm{0.21}} $ (\textcolor{blue}{$0.60$}) &$87.32_{\pm{0.15}} $ (\textcolor{blue}{$1.09$}) & 37.58
%   \\
%  \IU 
%   &$53.95_{\pm{1.24}} $ (\textcolor{blue}{$2.32$})& ${57.48}_{\pm{0.17}}$ (\textcolor{blue}{${0.38}$}) 
%  & $75.88_{\pm1.16}$ (\textcolor{blue}{$0.65$})
%  &${76.73}_{\pm{0.74}}$ (\textcolor{blue}{$0.59$})
%  &$99.68_{\pm{0.11}}$ (\textcolor{blue}{$0.32$})
%  &$99.67_{\pm{0.05}}$ (\textcolor{blue}{$0.32$})
%  &$88.93_{\pm{0.10}}$ (\textcolor{blue}{$0.61$})&$88.28_{\pm{0.14}}$ (\textcolor{blue}{$0.13$}) & 3.11 \\
\midrule
\rowcolor{Gray}
\multicolumn{10}{c}{Random data forgetting} \\
\midrule
 \retrain &\textcolor{blue}{$5.41_{\pm{0.11}}$}&\textcolor{blue}{$ 6.77_{\pm{0.23}}$}&\textcolor{blue}{$13.12_{\pm{0.14}}$}&\textcolor{blue}{$14.17_{\pm{0.18}}$}&\textcolor{blue}{$100.00_{\pm{0.00}}$}&\textcolor{blue}{$100.00_{\pm{0.00}}$}&\textcolor{blue}{$94.42_{\pm{0.09}}$}&\textcolor{blue}{$93.33_{\pm{0.12}}$} & 42.15 
\\
 \FT & $6.83_{\pm{0.51}}$ (\textcolor{blue}{$1.42$})& $5.97_{\pm{0.57}}$ (\textcolor{blue}{$0.80$})& $14.97_{\pm{0.62}}$ (\textcolor{blue}{$1.85$})& $13.36_{\pm{0.59}}$ (\textcolor{blue}{$0.81$})& $96.61_{\pm{0.25}}$ (\textcolor{blue}{$3.39$})& $96.99_{\pm{0.31}}$ (\textcolor{blue}{$3.01$})& $90.13_{\pm{0.26}}$ (\textcolor{blue}{$4.29$})& $90.29_{\pm{0.31}}$ (\textcolor{blue}{$1.51$}) & 2.33  
 \\
 \GA & $7.54_{\pm{0.29}}$ (\textcolor{blue}{$2.13$})& $5.62_{\pm{0.46}}$ (\textcolor{blue}{$1.15$})& $10.04_{\pm{0.31}}$ (\textcolor{blue}{$3.08$})& $11.76_{\pm{0.52}}$ (\textcolor{blue}{$2.41$})& $93.31_{\pm{0.04}}$ (\textcolor{blue}{$6.69$})& $95.44_{\pm{0.11}}$ (\textcolor{blue}{$4.56$})& $89.28_{\pm{0.07}}$ (\textcolor{blue}{$5.14$})& $89.26_{\pm{0.15}}$ (\textcolor{blue}{$4.07$}) & 0.31
 \\
  \FF & $7.84_{\pm{0.71}}$ (\textcolor{blue}{$2.43$})& $8.16_{\pm{0.67}}$ (\textcolor{blue}{$1.39$})& $9.52_{\pm{0.43}}$ (\textcolor{blue}{$3.60$})& $10.80_{\pm{0.37}}$ (\textcolor{blue}{$3.37$})& $92.05_{\pm{0.16}}$ (\textcolor{blue}{$7.95$})& $92.29_{\pm{0.24}}$ (\textcolor{blue}{$7.71$})& $88.10_{\pm{0.19}}$ (\textcolor{blue}{$6.32$})& $87.79_{\pm{0.23}}$ (\textcolor{blue}{$5.54$}) & 38.24
 \\
  \IU & $2.03_{\pm{0.43}}$ (\textcolor{blue}{$3.38$})& $6.51_{\pm{0.52}}$ (\textcolor{blue}{$0.26$})& $5.07_{\pm{0.74}}$ (\textcolor{blue}{$8.05$})& $11.93_{\pm{0.68}}$ (\textcolor{blue}{$2.41$})& $98.26_{\pm{0.29}}$ (\textcolor{blue}{$1.74$})& $94.94_{\pm{0.31}}$ (\textcolor{blue}{$5.06$})& $91.33_{\pm{0.22}}$ (\textcolor{blue}{$3.09$})& $88.74_{\pm{0.42}}$ (\textcolor{blue}{$4.59$}) & 3.22 \\
\midrule
\bottomrule[1pt]
\end{tabular}
}
\vspace*{-3mm}

\end{table*}
\fi
\begin{table*}[htb!]
\centering
\caption{ Performance overview of various MU methods  on dense and 95\%-sparse models considering different unlearning scenarios:
 class-wise forgetting, 
 and random data forgetting. The forgetting data of random data forgetting ratio is $10\%$ of the whole training dataset, 
 the sparse models are obtained using OMP \cite{ma2021sanity}, and the unlearning methods and evaluation metrics are summarized in Tab.\,\ref{tab: summary_MU_methods_metrics}. {Class-wise forgetting is conducted class-wise.}
The performance is reported in the form $a_{\pm b}$, with mean $a$ and standard deviation $b$ computed over $10$ independent trials. 
A performance gap  against \textcolor{blue}{{\retrain}} is provided 
in (\textcolor{blue}{$\bullet$}). Note that the better performance of approximate unlearning corresponds to the smaller performance gap with the gold-standard retrained model. {`Disparity Ave.' represents the average unlearning gaps across diverse metrics.}
%\revision{Disparity Ave. denotes the average disparity between the unlearned model and the retrained model, lower is better.}
%\JH{Should we change original table format to this table format?}
\iffalse 
Performance overview of various MU methods  on dense and 95\%-sparse models considering different unlearning scenarios:
%. ResNet-18 \cite{he2016deep} are used across different unlearning settings: 
 class-wise forgetting, 
 % random data forgetting (per class), 
 and random data forgetting.% The forgetting data of random data forgetting ratio is $10\%$ of the whole training dataset, 
 % the sparse models are obtained using OMP, and the unlearning methods and evaluation metrics are summarized in Tab.2. {Class-wise forgetting is conducted class-wise.}
%We carefully tune the hyperparameters for all machine unlearning methods to report the model which can achieve the best unlearning performance at different sparsity ratios. The results $a_{\pm{b}}$ represent mean $a$ and standard deviation $b$ over $10$ random trials.
The  performance is reported in the form $a_{\pm b}$, with mean $a$ and standard deviation $b$ computed over $10$ independent trials. 
A performance gap  against \textcolor{blue}{{\retrain}} is provided 
%. The relative drop or improvement represented 
in (\textcolor{blue}{$\bullet$}). Note that the better performance of    approximate unlearning corresponds to the smaller performance gap with the gold-standard retrained model. Disparity Ave. denotes the average disparity between the unlearned model and the retrained model, lower is better.
\fi 
%see  Table\,\ref{tab: summary_MU_methods_metrics} for used unlearning methods and evaluation metrics.
% \JC{Add {\MUSparse} to this table?}
% \SL{[update!]}
%\JH{[delete random data forgetting (one class)] and change the forgetting ratio.}
}
\vspace*{-2mm}
\label{tab: overall_performance}
% \vspace*{0.1in} % Requirements, do not delete.
% \vspace*{-1mm}
\resizebox{0.98\textwidth}{!}{
\begin{tabular}{c|cc|cc|cc|cc|cc|c}
\toprule[1pt]
\midrule
  \multirow{2}{*}{\MU}& \multicolumn{2}{c|}{{\UA}} & \multicolumn{2}{c|}{{\MIAF}}& \multicolumn{2}{c|}{{\RA}} & \multicolumn{2}{c|}{{\TA}}&\multicolumn{2}{c|}{{Disparity Ave. $\downarrow$}}& {\RTE}  \\ 
  & \multicolumn{1}{c|}{{\textsc{Dense}}}  & \multicolumn{1}{c|}{$\mathbf{95\%}$ \textbf{Sparsity}}
    & \multicolumn{1}{c|}{\textsc{Dense}}  & \multicolumn{1}{c|}{$\mathbf{95\%}$ \textbf{Sparsity}}
    & \multicolumn{1}{c|}{\textsc{Dense}}  & \multicolumn{1}{c|}{$\mathbf{95\%}$ \textbf{Sparsity}}
      & \multicolumn{1}{c|}{\textsc{Dense}}  & \multicolumn{1}{c|}{$\mathbf{95\%}$ \textbf{Sparsity}} & \multicolumn{1}{c|}{\textsc{Dense}}  & \multicolumn{1}{c|}{$\mathbf{95\%}$ \textbf{Sparsity}} & (min)
  \\
% \cline{3-10}

\midrule
\rowcolor{Gray}
\multicolumn{12}{c}{Class-wise forgetting} \\
\midrule
\retrain &\textcolor{blue}{$100.00_{\pm{0.00}}$}    & \textcolor{blue}{$100.00_{\pm{0.00}}$}
&\textcolor{blue}{$100.00_{\pm{0.00}}$}   & \textcolor{blue}{$100.00_{\pm{0.00}}$}
&\textcolor{blue}{$100.00_{\pm{0.00}}$}    & \textcolor{blue}{$99.99_{\pm{0.01}}$}
&\textcolor{blue}{$94.83_{\pm{0.11}}$}   & \textcolor{blue}{$91.80_{\pm{0.89}}$} & 0.00 & 0.00
 &43.23\\
  \FT &$22.53_{\pm{8.16}}$ (\textcolor{blue}{$77.47$})&${73.64}_{\pm{9.46}}$  (\textcolor{blue}{${26.36}$})&$75.00_{\pm{14.68}}$ (\textcolor{blue}{${25.00}$})& ${83.02}_{\pm{16.33}}$ (\textcolor{blue}{${16.98}$}) 
  &$99.87_{\pm{0.04}}$ (\textcolor{blue}{$0.13$}) & ${99.87}_{\pm{0.05}}$ (\textcolor{blue}{${0.12}$})&$94.31_{\pm{0.19}}$ (\textcolor{blue}{$0.52$})
 &$94.32_{\pm{0.12}}$ (\textcolor{blue}{$2.52$})
&  25.78&11.50& 2.52
  
  
  
  \\
 \GA &$93.08_{\pm{2.29}}$ (\textcolor{blue}{6.92}) &${98.09}_{\pm{1.11}}$ (\textcolor{blue}{${1.91}$})
& $94.03_{\pm{3.27}}$ (\textcolor{blue}{5.97})& ${97.74}_{\pm{2.24}}$ (\textcolor{blue}{${2.26}$})
& $92.60_{\pm{0.25}}$ (\textcolor{blue}{$7.40$})& $87.74_{\pm{0.27}}$ (\textcolor{blue}{$12.25$}) 
& $86.64_{\pm{0.28}}$ (\textcolor{blue}{$8.19$})& $82.58_{\pm{0.27}}$ (\textcolor{blue}{$9.22$}) 
&7.12 &6.41&  0.33
 \\
  {\FF}  & $79.93_{\pm{8.92}}$ (\textcolor{blue}{$20.07$})& ${94.83}_{\pm{4.29}}$ (\textcolor{blue}{${5.17}$}) 
  & $100.00_{\pm{0.00}}$ (\textcolor{blue}{$0.00$})& ${100.00}_{\pm{0.00}}$ (\textcolor{blue}{$0.00$}) 
    & $99.45_{\pm{0.24}}$ (\textcolor{blue}{$0.55$})& ${99.48}_{\pm{0.33}}$ (\textcolor{blue}{${0.51}$})
        & $94.18_{\pm{0.08}}$ (\textcolor{blue}{$0.65$})& $94.04_{\pm{0.10}}$ (\textcolor{blue}{$2.24$})&5.32 &1.98&38.91
  \\
 \IU 
  &$87.82_{\pm{2.15}} $ (\textcolor{blue}{$12.18$})& ${99.47}_{\pm{0.15}}$ (\textcolor{blue}{${0.53}$})
 & $95.96_{\pm0.21}$ (\textcolor{blue}{$4.04$})
&${99.93}_{\pm{0.04}}$ (\textcolor{blue}{${0.07}$})
 &$97.98_{\pm{0.21}}$ (\textcolor{blue}{$2.02$}) 
 &$97.24_{\pm{0.13}}$ (\textcolor{blue}{$2.75$}) 
 &$91.42_{\pm{0.21}}$ (\textcolor{blue}{$3.41$})&${90.76_{\pm{0.18}}}$ (\textcolor{blue}{${1.04}$}) &5.41&1.10& 3.25
 \\

% \textbf{\MUSparse} 
% &$100.00_{\pm{0.00}} $ (\textcolor{blue}{$0.00$})&n/a
%  & $100.00_{\pm0.00}$ (\textcolor{blue}{$0.00$})&n/a

%  &$98.99_{\pm{0.12}}$ (\textcolor{blue}{$1.01$}) &n/a
 
%  &$93.40_{\pm{0.43}}$ (\textcolor{blue}{$1.43$})&n/a&0.61&n/a
%  & 2.53 \\
%  \FTSparse &$100.00_{\pm{0.00}}$  &$100.00_{\pm{0.00}}$  & $91.49_{\pm{1.21}}$&$87.17_{\pm1.31}$
%   &$100.00_{\pm{0.00}}$  &$100.00_{\pm{0.00}}$  & $91.69_{\pm{1.57}}$&$87.30_{\pm1.39}$
%   &$100.00_{\pm{0.00}}$  &$100.00_{\pm{0.00}}$  & $95.74_{\pm{0.54}}$&$88.97_{\pm1.00}$
% \\
% \FTAO  
%   & -&-  & -&-
% & $43.82_{\pm{11.68}}$& $98.64_{\pm{0.71}}$ & $99.96_{\pm{0.03}}$&$94.79_{\pm0.07}$
%   &$99.80_{\pm{0.19}}$  &$100.00_{\pm{0.00}}$ & $99.86_{\pm{0.05}}$&$94.55_{\pm0.11}$
% \\
% \midrule
% \rowcolor{Gray}
% \multicolumn{10}{c}{Random data forgetting (per class)} \\
% \midrule
% \retrain &\textcolor{blue}{$56.27_{\pm{0.07}}$}   & \textcolor{blue}{$57.86_{\pm{0.05}}$ }
% &\textcolor{blue}{$75.23_{\pm{0.14}}$}   & \textcolor{blue}{$76.14_{\pm{0.11}}$ }
% &\textcolor{blue}{$100.00_{\pm{0.00}}$}    & \textcolor{blue}{$99.99_{\pm{0.01}}$}
% &\textcolor{blue}{$89.54_{\pm{0.11}}$ }   & \textcolor{blue}{$88.41_{\pm{0.89}}$}
%  & 41.63 \\
%   \FT &$1.89_{\pm{0.79}}$(\textcolor{blue}{$54.38$})&${19.34}_{\pm{1.41}}$ (\textcolor{blue}{$38.52$})& $17.11_{\pm{2.21}}$ (\textcolor{blue}{$58.12$}) &${35.18}_{\pm{2.12}} $ (\textcolor{blue}{$40.96$})
%   &$99.92_{\pm{0.03}}$ (\textcolor{blue}{$0.08$}) & $99.21_{\pm{0.05}}$ (\textcolor{blue}{$0.78$})&$93.50_{\pm{0.52}}$ (\textcolor{blue}{$3.96$})
%  &${91.20}_{\pm{0.12}}$ (\textcolor{blue}{$2.79$})
% & 2.36
  
  
  
%   \\
%  \GA
% &$50.75_{\pm{3.29}} $ (\textcolor{blue}{$5.52$})& ${59.12}_{\pm{3.17}}$ (\textcolor{blue}{$1.26$})  
% &$69.27_{\pm{4.27}} $ (\textcolor{blue}{$5.96$})& ${74.06}_{\pm{3.15}}$ (\textcolor{blue}{$2.08$})
% &$98.43_{\pm{0.20}} $ (\textcolor{blue}{$1.57$})& ${98.59}_{\pm{0.17}}$ (\textcolor{blue}{$1.40$})
% &$87.56_{\pm{0.24}} $ (\textcolor{blue}{$1.98$})& ${87.12}_{\pm{0.31}}$ (\textcolor{blue}{$1.29$}) &0.29
%  \\
%   $\FF$  &$4.85_{\pm{4.20}} $ (\textcolor{blue}{$51.42$}) &${6.92}_{\pm{3.72}} $ (\textcolor{blue}{$50.94$})
%   &$11.29_{\pm{5.12}} $ (\textcolor{blue}{$63.94$}) &${12.37}_{\pm{4.54}} $ (\textcolor{blue}{$63.77$})
%   &$97.30_{\pm{0.52}} $ (\textcolor{blue}{$2.70$}) &$96.13_{\pm{0.41}} $ (\textcolor{blue}{$3.86$})
%   &$88.94_{\pm{0.21}} $ (\textcolor{blue}{$0.60$}) &$87.32_{\pm{0.15}} $ (\textcolor{blue}{$1.09$}) & 37.58
%   \\
%  \IU 
%   &$53.95_{\pm{1.24}} $ (\textcolor{blue}{$2.32$})& ${57.48}_{\pm{0.17}}$ (\textcolor{blue}{${0.38}$}) 
%  & $75.88_{\pm1.16}$ (\textcolor{blue}{$0.65$})
%  &${76.73}_{\pm{0.74}}$ (\textcolor{blue}{$0.59$})
%  &$99.68_{\pm{0.11}}$ (\textcolor{blue}{$0.32$})
%  &$99.67_{\pm{0.05}}$ (\textcolor{blue}{$0.32$})
%  &$88.93_{\pm{0.10}}$ (\textcolor{blue}{$0.61$})&$88.28_{\pm{0.14}}$ (\textcolor{blue}{$0.13$}) & 3.11 \\
\midrule
\rowcolor{Gray}
\multicolumn{12}{c}{Random data forgetting} \\
\midrule
 \retrain &\textcolor{blue}{$5.41_{\pm{0.11}}$}&\textcolor{blue}{$ 6.77_{\pm{0.23}}$}&\textcolor{blue}{$13.12_{\pm{0.14}}$}&\textcolor{blue}{$14.17_{\pm{0.18}}$}&\textcolor{blue}{$100.00_{\pm{0.00}}$}&\textcolor{blue}{$100.00_{\pm{0.00}}$}&\textcolor{blue}{$94.42_{\pm{0.09}}$}&\textcolor{blue}{$93.33_{\pm{0.12}}$} & 0.00 & 0.00 & 42.15 
\\
 \FT & $6.83_{\pm{0.51}}$ (\textcolor{blue}{$1.42$})& $5.97_{\pm{0.57}}$ (\textcolor{blue}{$0.80$})& $14.97_{\pm{0.62}}$ (\textcolor{blue}{$1.85$})& $13.36_{\pm{0.59}}$ (\textcolor{blue}{$0.81$})& $96.61_{\pm{0.25}}$ (\textcolor{blue}{$3.39$})& $96.99_{\pm{0.31}}$ (\textcolor{blue}{$3.01$})& $90.13_{\pm{0.26}}$ (\textcolor{blue}{$4.29$})& $90.29_{\pm{0.31}}$ (\textcolor{blue}{$3.04$}) & 2.74 & 1.92 & 2.33  
 \\
 \GA & $7.54_{\pm{0.29}}$ (\textcolor{blue}{$2.13$})& $5.62_{\pm{0.46}}$ (\textcolor{blue}{$1.15$})& $10.04_{\pm{0.31}}$ (\textcolor{blue}{$3.08$})& $11.76_{\pm{0.52}}$ (\textcolor{blue}{$2.41$})& $93.31_{\pm{0.04}}$ (\textcolor{blue}{$6.69$})& $95.44_{\pm{0.11}}$ (\textcolor{blue}{$4.56$})& $89.28_{\pm{0.07}}$ (\textcolor{blue}{$5.14$})& $89.26_{\pm{0.15}}$ (\textcolor{blue}{$4.07$}) & 4.26 & 3.05 & 0.31
 \\
  \FF & $7.84_{\pm{0.71}}$ (\textcolor{blue}{$2.43$})& $8.16_{\pm{0.67}}$ (\textcolor{blue}{$1.39$})& $9.52_{\pm{0.43}}$ (\textcolor{blue}{$3.60$})& $10.80_{\pm{0.37}}$ (\textcolor{blue}{$3.37$})& $92.05_{\pm{0.16}}$ (\textcolor{blue}{$7.95$})& $92.29_{\pm{0.24}}$ (\textcolor{blue}{$7.71$})& $88.10_{\pm{0.19}}$ (\textcolor{blue}{$6.32$})& $87.79_{\pm{0.23}}$ (\textcolor{blue}{$5.54$}) &5.08 & 4.50 & 38.24
 \\
  \IU & $2.03_{\pm{0.43}}$ (\textcolor{blue}{$3.38$})& $6.51_{\pm{0.52}}$ (\textcolor{blue}{$0.26$})& $5.07_{\pm{0.74}}$ (\textcolor{blue}{$8.05$})& $11.93_{\pm{0.68}}$ (\textcolor{blue}{$2.24$})& $98.26_{\pm{0.29}}$ (\textcolor{blue}{$1.74$})& $94.94_{\pm{0.31}}$ (\textcolor{blue}{$5.06$})& $91.33_{\pm{0.22}}$ (\textcolor{blue}{$3.09$})& $88.74_{\pm{0.42}}$ (\textcolor{blue}{$4.59$}) &4.07 & 3.08& 3.22 \\
% \textbf{\MUSparse} & $5.35_{\pm{0.22}}$ (\textcolor{blue}{$0.06$})& n/a& $12.71_{\pm{0.31}}$ (\textcolor{blue}{$0.41$})&n/a& $97.39_{\pm{0.19}}$ (\textcolor{blue}{$2.61$})&n/a& $91.26_{\pm{0.20}}$ (\textcolor{blue}{$3.16$}) &n/a&1.56&n/a& 2.34
% \\
\midrule
\bottomrule[1pt]
\end{tabular}
}
\vspace*{-4mm}
\end{table*}
\iffalse 
We elaborate on our \textbf{experiment results} below.
\fi 
% \begin{table*}[htb]
% \centering
% \caption{\footnotesize{Performance overview of various machine unlearning methods  on dense and 95\%-sparse models considering two unlearning scenarios:
% %. ResNet-18 \cite{he2016deep} are used across different unlearning settings: 
% forgetting one class and forgetting random data points.  
% The overview of unlearning methods and evaluation metrics are provided in Table\,\ref{tab: summary_MU_methods_metrics}, and sparse models are obtained using OMP \cite{ma2021sanity}. 
% %We carefully tune the hyperparameters for all machine unlearning methods to report the model which can achieve the best unlearning performance at different sparsity ratios. The results $a_{\pm{b}}$ represent mean $a$ and standard deviation $b$ over $10$ random trials.
% The $a\%$ performance gap of an approximate unlearning method against Retrain is provided  
% %. The relative drop or improvement represented 
% in (\textcolor{blue}{$a$}).
% %$a$ or \textcolor{blue}{$\LARGE\uparrow$}$a$. 
% %The best performance of each unlearning method in each evaluation metric is in bold.
% }} 
% \label{tab: overall_performance}
% \vspace*{0.1in} % Requirements, do not delete.
% \resizebox{0.95\textwidth}{!}{
% \begin{tabular}{c|cc|cc|cc|cc|c}
% \toprule[1pt]
% \midrule
%   \multirow{2}{*}{\MU}& \multicolumn{2}{c|}{{\UA}} & \multicolumn{2}{c|}{{\MIAF}}& \multicolumn{2}{c|}{{\RA}} & \multicolumn{2}{c|}{{\TA}}&{\RTE}  \\ 
%   & \multicolumn{1}{c|}{{\textsc{Dense}}}  & \multicolumn{1}{c|}{$\mathbf{95\%}$ \textbf{Sparsity}}
%     & \multicolumn{1}{c|}{\textsc{Dense}}  & \multicolumn{1}{c|}{$\mathbf{95\%}$ \textbf{Sparsity}}
%     & \multicolumn{1}{c|}{\textsc{Dense}}  & \multicolumn{1}{c|}{$\mathbf{95\%}$ \textbf{Sparsity}}
%       & \multicolumn{1}{c|}{\textsc{Dense}}  & \multicolumn{1}{c|}{$\mathbf{95\%}$ \textbf{Sparsity}} & (min)
%   \\
% % \cline{3-10}

% \midrule
% \rowcolor{Gray}
% \multicolumn{10}{c}{\Large Class-wise Forgetting} \\
% \midrule
% \retrain &\textcolor{blue}{$100.00_{\pm{0.00}}$}    & \textcolor{blue}{$100.00_{\pm{0.00}}$}
% &\textcolor{blue}{$100.00_{\pm{0.00}}$}   & \textcolor{blue}{$100.00_{\pm{0.00}}$}
% &\textcolor{blue}{$100.00_{\pm{0.00}}$}    & \textcolor{blue}{$99.99_{\pm{0.01}}$}
% &\textcolor{blue}{$94.83_{\pm{0.11}}$}   & \textcolor{blue}{$91.80_{\pm{0.89}}$}
%  &64.48\\
%   \FT &$22.53_{\pm{8.16}}$ (\textcolor{blue}{$\LARGE\downarrow$}$77.47$)&$\mathbf{73.64}_{\pm{9.46}}$  (\textcolor{blue}{$\LARGE\downarrow$}${26.36}$)&$75.00_{\pm{14.68}}$ (\textcolor{blue}{$\LARGE\downarrow$}${25.00}$)& $\mathbf{83.02}_{\pm{16.33}}$ (\textcolor{blue}{$\LARGE\downarrow$}${16.98}$) 
%   &$99.87_{\pm{0.04}}$ (\textcolor{blue}{$\LARGE\downarrow$}0.13) & $\mathbf{99.87}_{\pm{0.05}}$ (\textcolor{blue}{$\LARGE\downarrow$}${0.12}$)&$94.31_{\pm{0.19}}$ (\textcolor{blue}{$\LARGE\downarrow$}0.52)
%  &$94.32_{\pm{0.12}}$ (\textcolor{blue}{$\LARGE\uparrow$}2.52)
% &   4.04
  
  
  
%   \\
%  \GA &$93.08_{\pm{0.29}}$ (\textcolor{blue}{$\LARGE\downarrow$}6.92) &$\mathbf{98.09}_{\pm{0.11}}$ (\textcolor{blue}{$\LARGE\downarrow$}${1.91}$)
% & $93.08_{\pm{0.31}}$ (\textcolor{blue}{$\LARGE\downarrow$}6.92)& $\mathbf{94.67}_{\pm{0.25}}$ (\textcolor{blue}{$\LARGE\downarrow$}${5.33}$)
% & $92.60_{\pm{0.25}}$ (\textcolor{blue}{$\LARGE\downarrow$}7.40)& $87.74_{\pm{0.27}}$ (\textcolor{blue}{$\LARGE\downarrow$}12.26) 
% & $86.64_{\pm{0.28}}$ (\textcolor{blue}{$\LARGE\downarrow$}8.19)& $82.58_{\pm{0.27}}$ (\textcolor{blue}{$\LARGE\downarrow$}9.22) 
% &   1.07
%  \\
%   {\FF}  & $79.93_{\pm{8.92}}$ (\textcolor{blue}{$\LARGE\downarrow$}20.07)& $\mathbf{94.83}_{\pm{4.29}}$ (\textcolor{blue}{$\LARGE\downarrow$}${5.17}$) 
%   & $100.00_{\pm{0.00}}$ (\textcolor{blue}{$\LARGE\downarrow$}0.00)& $\mathbf{100.00}_{\pm{0.00}}$ (\textcolor{blue}{$\LARGE\downarrow$}0.00) 
%     & $99.45_{\pm{0.24}}$ (\textcolor{blue}{$\LARGE\downarrow$}0.55)& $\mathbf{99.48}_{\pm{0.33}}$ (\textcolor{blue}{$\LARGE\downarrow$}${0.51}$)
%         & $94.18_{\pm{0.08}}$ (\textcolor{blue}{$\LARGE\downarrow$}0.65)& $94.04_{\pm{0.10}}$ (\textcolor{blue}{$\LARGE\uparrow$}2.24)& 58.67
%   \\
%  \IU 
%   &$87.82_{\pm{2.15}} $ (\textcolor{blue}{$\LARGE\downarrow$}12.18)& $\mathbf{99.47}_{\pm{0.15}}$ (\textcolor{blue}{$\LARGE\downarrow$}${0.53}$)
%  & $95.96_{\pm0.21}$ (\textcolor{blue}{$\LARGE\downarrow$}4.04)
% &$\mathbf{99.93}_{\pm{0.04}}$ (\textcolor{blue}{$\LARGE\downarrow$}${0.07}$)
%  &$97.98_{\pm{0.21}}$ (\textcolor{blue}{$\LARGE\downarrow$}2.02) 
%  &$97.24_{\pm{0.13}}$ (\textcolor{blue}{$\LARGE\downarrow$}2.76) 
%  &$91.42_{\pm{0.21}}$ (\textcolor{blue}{$\LARGE\downarrow$}3.41)&$\mathbf{90.76_{\pm{0.18}}}$ (\textcolor{blue}{$\LARGE\downarrow$}${1.04}$) & 5.23
%  \\
% %  \FTSparse &$100.00_{\pm{0.00}}$  &$100.00_{\pm{0.00}}$  & $91.49_{\pm{1.21}}$&$87.17_{\pm1.31}$
% %   &$100.00_{\pm{0.00}}$  &$100.00_{\pm{0.00}}$  & $91.69_{\pm{1.57}}$&$87.30_{\pm1.39}$
% %   &$100.00_{\pm{0.00}}$  &$100.00_{\pm{0.00}}$  & $95.74_{\pm{0.54}}$&$88.97_{\pm1.00}$
% % \\
% % \FTAO  
% %   & -&-  & -&-
% % & $43.82_{\pm{11.68}}$& $98.64_{\pm{0.71}}$ & $99.96_{\pm{0.03}}$&$94.79_{\pm0.07}$
% %   &$99.80_{\pm{0.19}}$  &$100.00_{\pm{0.00}}$ & $99.86_{\pm{0.05}}$&$94.55_{\pm0.11}$
% % \\
% \midrule
% \rowcolor{Gray}
% \multicolumn{10}{c}{\Large Random  Data Forgetting} \\
% \midrule
% \retrain &\textcolor{blue}{$56.27_{\pm{0.07}}$}   & \textcolor{blue}{$57.86_{\pm{0.05}}$ }
% &\textcolor{blue}{$75.23_{\pm{0.14}}$}   & \textcolor{blue}{$76.14_{\pm{0.11}}$ }
% &\textcolor{blue}{$100.00_{\pm{0.00}}$}    & \textcolor{blue}{$99.99_{\pm{0.01}}$}
% &\textcolor{blue}{$89.54_{\pm{0.11}}$ }   & \textcolor{blue}{$88.41_{\pm{0.89}}$}
%  & 66.40 \\
%   \FT &$1.89_{\pm{0.79}}$({\textcolor{blue}{$\LARGE\downarrow$}}54.38)&$\mathbf{19.34}_{\pm{1.41}}$ ({\textcolor{blue}{$\LARGE\downarrow$}}38.52)& $17.11_{\pm{2.21}}$ ({\textcolor{blue}{$\LARGE\downarrow$}}58.22) &$\mathbf{35.18}_{\pm{2.12}} $ ({\textcolor{blue}{$\LARGE\downarrow$}}40.96)
%   &$99.92_{\pm{0.03}}$ ({\textcolor{blue}{$\LARGE\downarrow$}}0.08) & $99.21_{\pm{0.05}}$ ({\textcolor{blue}{$\LARGE\downarrow$}}0.78)&$93.50_{\pm{0.52}}$ ({\textcolor{blue}{$\LARGE\uparrow$}}3.96)
%  &$\mathbf{91.20}_{\pm{0.12}}$ ({\textcolor{blue}{$\LARGE\uparrow$}}2.79)
% &  4.19 
  
  
  
%   \\
%  \GA
% &$50.75_{\pm{3.29}} $ ({\textcolor{blue}{$\LARGE\downarrow$}}5.52)& $\mathbf{59.12}_{\pm{3.17}}$ ({\textcolor{blue}{$\LARGE\downarrow$}}1.26)  
% &$69.27_{\pm{4.27}} $ ({\textcolor{blue}{$\LARGE\downarrow$}}5.96)& $\mathbf{74.06}_{\pm{3.15}}$ ({\textcolor{blue}{$\LARGE\downarrow$}}2.08)
% &$98.43_{\pm{0.20}} $ ({\textcolor{blue}{$\LARGE\downarrow$}}1.57)& $\mathbf{98.59}_{\pm{0.17}}$ ({\textcolor{blue}{$\LARGE\downarrow$}}1.41)
% &$87.56_{\pm{0.24}} $ ({\textcolor{blue}{$\LARGE\downarrow$}}1.98)& $\mathbf{87.12}_{\pm{0.31}}$ ({\textcolor{blue}{$\LARGE\downarrow$}}1.29) &1.03
%  \\
%   $\FF$  &$4.85_{\pm{4.20}} $ ({\textcolor{blue}{$\LARGE\downarrow$}}51.42) &$\mathbf{6.92}_{\pm{3.72}} $ ({\textcolor{blue}{$\LARGE\downarrow$}}50.94)
%   &$11.29_{\pm{5.12}} $ ({\textcolor{blue}{$\LARGE\downarrow$}}63.94) &$\mathbf{12.37}_{\pm{4.54}} $ ({\textcolor{blue}{$\LARGE\downarrow$}}63.77)
%   &$97.30_{\pm{0.52}} $ ({\textcolor{blue}{$\LARGE\downarrow$}}2.70) &$96.13_{\pm{0.41}} $ ({\textcolor{blue}{$\LARGE\downarrow$}}3.86)
%   &$88.94_{\pm{0.21}} $ ({\textcolor{blue}{$\LARGE\downarrow$}}0.60) &$87.32_{\pm{0.15}} $ ({\textcolor{blue}{$\LARGE\downarrow$}}1.09) & 60.17
%   \\
%  \IU 
%   &$53.95_{\pm{1.24}} $ ({\textcolor{blue}{$\LARGE\downarrow$}}$2.32$)& $\mathbf{57.48}_{\pm{0.17}}$ ({\textcolor{blue}{$\LARGE\downarrow$}}$\mathbf{0.38}$) 
%  & $75.88_{\pm1.16}$ ({\textcolor{blue}{$\LARGE\uparrow$}}0.65)
%  &$\mathbf{76.73}_{\pm{0.74}}$ ({\textcolor{blue}{$\LARGE\uparrow$}}0.59)
%  &$99.68_{\pm{0.11}}$ ({\textcolor{blue}{$\LARGE\downarrow$}}0.32)
%  &$99.67_{\pm{0.05}}$ ({\textcolor{blue}{$\LARGE\downarrow$}}0.33)
%  &$88.93_{\pm{0.10}}$ ({\textcolor{blue}{$\LARGE\downarrow$}}0.61)&$88.28_{\pm{0.14}}$ ({\textcolor{blue}{$\LARGE\downarrow$}}0.13) & 5.11 \\
% \midrule
% \bottomrule[1pt]
% \end{tabular}
% }
% \vspace*{-3mm}

% \end{table*}


% {$\LARGE\uparrow$}












\noindent \textbf{Model sparsity improves approximate unlearning.}
% \SL{[Remove 75\% results? Talk to me.]}
In \textbf{Tab.\,\ref{tab: overall_performance}}, we study the impact of model sparsity  on  the performance of various {\MU} methods 
%when applied to   class-wise forgetting   or  randomly  forgetting $9\%$ training set ($4400$ data points) 
in the `prune first, then unlearn' paradigm. 
The performance of the exact unlearning method  ({\retrain}) is also provided for comparison. 
Note that the better performance of    approximate unlearning  corresponds to the  smaller performance gap with  the gold-standard retrained model. 
%This also
%applies to other metrics.
%We summarize our   observations   below.
%Our key observations  and insights are illustrated below. 
%Two key insights can be drawn from Table\,\ref{tab: overall_performance}. 



\textit{First}, given an approximate unlearning  method ({\FT}, {\GA}, {\FF}, or {\IU}), we consistently observe that model sparsity improves {\UA} and {\MIAF} (\textit{i.e.}, the efficacy of approximate unlearning) without  much performance loss in  {\RA} (\textit{i.e.}, fidelity).
In particular, 
  the  performance gap between each approximate unlearning method and {\retrain} reduces as the model becomes sparser (see the `95\% sparsity' column vs. the `dense' column).
  %This also generally holds in other evaluation criteria, {\RA} and {\TA}. 
  Note that 
  the performance gap against {\retrain} is highlighted in $(\cdot)$ for each approximate unlearning.
  We also   observe that {\retrain}  on the 95\%-sparsity model encounters a  3\% {\TA} drop. Yet, from the perspective of approximate unlearning, this drop brings in 
  a more significant improvement  in    {\UA} and {\MIAF}  when model sparsity is promoted. Let us take {\FT} (the simplest unlearning method) for class-wise forgetting as an example. As the model sparsity   reaches   $95\%$, we obtain $51\%$ {\UA} improvement and $8\%$ {\MIAF} improvement. 
  Furthermore, {\FT} and {\IU} on the 95\%-sparsity model can better preserve  {\TA} compared to other  methods. 
Table\,\ref{tab: overall_performance} further indicates that sparsity reduces average disparity compared to a dense model across various approximate {\MU} methods and unlearning scenarios.
  %, and they can significantly improve their dense counterparts in {\UA} and {\MIAF}.



% and {\TA} (\textit{i.e.}, generalization). Let us take {\FT} (the simplest unlearning method) for class-wise forgetting as an example. As the model sparsity   reaches   $95\%$, we obtain $51\%$ {\UA} improvement and $8\%$ {\MIAF} improvement. %without any {\RA} and {\TA} drop. 

\iffalse 
  \textit{Second}, 
if we peer into the unlearning efficacy ({\UA} and {\MIAF}), 
  the  performance gap between each approximate unlearning method and {\retrain} reduces as the model becomes sparser (see the `95\% sparsity' column vs. the `dense' column). This also generally holds in other evaluation criteria, {\RA} and {\TA}. Note that the performance gap against {\retrain} is highlighted in $(\cdot)$ for each approximate unlearning.
On the other hand, we   observe that {\TA} of {\retrain}  on the 95\%-sparsity model may lead to  3\% accuracy drop. Yet, this drop allows a more significant improvement of approximate unlearning in    fidelity ({\UA} and {\MIAF})  on     sparse  models. Furthermore, {\FT} and {\IU} on the 95\%-sparsity model can better preserve  {\TA} compared to other approximate unlearning methods, and they can significantly improve their dense counterparts in {\UA} and {\MIAF}.
\fi 

  \iffalse 
\SL{The performance gap with {\retrain} is highlighted in $(\cdot)$ for each approximate unlearning method under each metric.  We also note that {\retrain} on the model of $95\%$ sparsity yields a slight {\TA} drop compared to the dense and $75\%$-sparse cases. This suggests that an extremely high sparsity may give rise to a tradeoff between unlearning efficacy and generalization.}
\YL{im thinking if this is a fair comparison. Sparser model does seem to have a significantly lower accuracy. I wonder if people can argue whether the reduced gap is due to the benefit of being sparse, or because of the drop of model performance and for example increasing uncertainty, which seem likely to be mediation factors.}
\fi 

\begin{wrapfigure}{r}{60mm}
\vspace*{-7.8mm}
\begin{center}
\includegraphics[width=60mm,height=!]{figs/Cifar_l1.pdf}

\vspace*{-1mm}
\begin{tabular}{cc}
{\scriptsize{\hspace*{-1.5mm}(a) Class-wise forgetting}} &{\scriptsize{\hspace*{3mm}(b) Random data forgetting}}
\end{tabular}
\end{center}
\vspace*{-2.5mm}


% \includegraphics[width=60mm,height=!]{figs/Cifar_l1.pdf}}
% \vspace*{-2mm}
\caption{\footnotesize{
 Performance   of sparsity-aware unlearning vs. {\FT} and {\retrain} on class-wise forgetting and random data forgetting under (CIFAR-10, ResNet-18). 
 Each   metric is normalized to $[0,1]$ based on the best result  across unlearning methods 
  for ease of visualization, while the actual best value  is provided (\textit{e.g.}, $2.52$  is the  least computation time for class-wise forgetting). 
% \SL{[talk to me.]}
 % Each performance metric is normalized to $[0,1]$ 
 % using $1-\frac{|M-R|}{R}$, where $M$ denotes the different metrics from approximate unlearning, and $R$ denotes those from {\retrain}. {\RTE} is normalized to $[0,1]$ by using ${\frac{M}{\FT}}$. \JC{need to clarify that RTE is based on FT running time} \JH{updated}
% \JC{Merge to one figure and use one caption}
% \JC{Add `prune-first-then-unlearn'?}
}
}
%\JC{move other datasets (except cifar10) to appendix}
  \label{fig: results_l1_sparse_unlearn}
\vspace*{-7mm}
\end{wrapfigure}



% \begin{figure}[ht]
%     \centering
%     \begin{minipage}[htb]{75mm}
% \begin{center}
% \raisebox{-0.5cm}{ % Adjust this value as needed
% \includegraphics[width=0.95\textwidth,height=!]{figs/Cifar_l1.pdf}
% }

% \vspace*{-1mm}
% \begin{tabular}{cc}
% {\scriptsize{\hspace*{-1.5mm}(a) Class-wise forgetting}} &{\scriptsize{\hspace*{3mm}(b) Random data forgetting}}
% \end{tabular}
% \end{center}
% \vspace*{-2.5mm}
% \caption{\footnotesize{
%  Performance   of sparsity-aware unlearning vs. {\FT} and {\retrain} on class-wise forgetting and all-class random data forgetting under (CIFAR-10, ResNet-18). 
%  Each   metric is normalized to $[0,1]$ based on the best result  across unlearning methods 
%   for ease of visualization, while the actual best value  is provided (\textit{e.g.}, $2.52$  is the  least computation time for class-wise forgetting). 
% }
% }
%   \label{fig: results_l1_sparse_unlearn}
%     \end{minipage}
%     \hfill % To ensure that the figures are placed side by side
% \begin{minipage}[htb]{60mm}
% \begin{center}
%   \begin{tabular}{cccc}
%     \hspace*{-2mm}  \includegraphics[width=30mm,height=!]{figs/rebuttal/backdoor_ASR.pdf} &
%     \hspace*{-5mm} \includegraphics[width=30mm,height=!]{figs/rebuttal/backdoor_SA.pdf} \\
% \end{tabular}  
% \end{center}
% \vspace*{-2mm}
% \caption{
% % One figure demonstrates that our methods can decrease attack success rates and maintain good generalization performance.
% Performance  of  Trojan model cleanse   via proposed unlearning vs. model sparsity, where `Original' refers to the original Trojan model.
% %the effectiveness of the `Unlearn first, then prune' paradigm on the trojan model
% %cleanse application. 
% \textbf{Left}: ASR vs. model sparsity. \textbf{Right}: SA vs. model sparsity. 
% %Each marker represents the mean value over $10$ independent trials. %\JC{Add {\MUSparse}}
% %Results  The line and shaded area of each plot represent the mean and variance   over $10$ independent trials. 
% %\JC{remove variance}
% }
%   \label{fig: results_MU_pruning_backdoor}
%     \end{minipage}
% \end{figure}

\textit{Second},  existing approximate unlearning methods have different pros and cons. Let us focus on the regime of $95\%$ sparsity.  We observe that {\FT}  typically yields the best {\RA} and {\TA}, which has a tradeoff with its unlearning efficacy ({\UA} and {\MIAF}). Moreover, {\GA} yields the worst {\RA} since it is most loosely connected with the remaining dataset $\Dr$.  {{\FF} becomes ineffective when scrubbing random data points compared to its class-wise unlearning performance. 
%This is not surprising since {\FF} is allowed to modify the model architecture when unlearning a class.} 
Furthermore,    {\IU} causes a {\TA} drop but yields the smallest gap with exact unlearning across diverse metrics under the $95\%$ model sparsity.
%We refer readers to Appendix\,\ref{appendix: additional results} for more dataset results. 
% \SL{[details.]}
In Appendix\,\ref{appendix: additional results}, we provide additional results on CIFAR-100 and SVHN datasets, as shown in Tab.\,\ref{tab: overall_performance_ext_datasets}, as well as on the ImageNet dataset, depicted in Tab.\,\ref{tab: overall_performance_ImageNet}. Other results pertaining to the VGG-16 architecture are  provided in Tab.\,\ref{tab: overall_performance_ext_archs}.
% Tab.\,\ref{tab: overall_performance_ext_datasets}
% Tab.\,\ref{tab: overall_performance_ext_archs}
% Tab.\,\ref{tab: overall_performance_ImageNet}




\iffalse 
% \paragraph{Overall performance: Sparsity reduces the gap between exact unlearning and approximate unlearning.}
We look at the relationship between machine learning and sparsity in what follows. There is \textit{one key observation} drawn from our overall results: Sparsity can reduce the gap between exact unlearning and approximate unlearning across all datasets and machine unlearning methods (as shown in \textbf{Table\,\ref{tab: overall_performance})}. Table\,\ref{tab: overall_performance} shows different evaluation metrics of different unlearning methods at different sparsity levels across different datasets. For comparison, we also present the performance of exact unlearning methods Retrain. As we can see, the gap between imperfect unlearning and perfect unlearning is decreasing with sparsity growing, especially in {\UA} and {\MIAF} these two metrics. For example, the gap between {\FT} and Retrain of $95\%$ sparsity model drops $51.11\%$  compared to the dense model in  the {\UA} on the CIFAR-10 dataset. This phenomenon is also justified on different datasets shown in Table\,\ref{tab: overall_performance}. Although sparsity can reduce the gap between exact unlearning and approximate unlearning, the improvement is different in the diverse machine unlearning methods. From Table\,\ref{tab: overall_performance}, {\FT} will benefit from sparsity most, where {\MIAF} reduced $13\%$-$20\%$ across all datasets.  
Besides, we also can observe in most cases that {\IU} outperforms other unlearning methods at different sparsity levels, which is the most competitive method in different evaluation metrics. However, this method needs to be tuned carefully to choose suitable hyperparameters.
\fi 



% \noindent \textbf{\JC{Ablation study on parameter scheduler in \MUSparse.}}
% % However, as \citep{bach2012optimization} mentioned, the downside of $\ell_1$ regulation term will affect the is its loss  in {\RA} and {\TA} compared to {\FT} and {\retrain}. Therefore, we conducted a comprehensive study of the scheduler of $\lambda$ in {\MUSparse}. In \textbf{Tab.\,\ref{tab: ablation_l1_scheduler}}, we present the results of unlearning performance on different parameter schedulers: constant scheduler, linear growing scheduler, and linear decaying scheduler.
% % It shows that the decaying $\gamma$ scheduler performs the best among all the schedulers. If we directly apply a constant $\lambda$ to {\MUSparse}, it will either get a low {\UA} with lower $\lambda$ (for $\lambda = 0$, the method reduces to {\FT}) or worse {\RA} and {\TA} 
% %  under higher $\lambda$. In Tab.\,\ref{tab: ablation_l1_scheduler}, we picked a sweet point to get a balance between them. If we use the linear growing scheduler, which means the method focuses on the unlearning term first then moves the focus to sparsity. If we use the linear decaying scheduler, the method focuses on the unlearning term first then moves the focus to sparsity.
% As Sec.\,\ref{sec: sparsity_MU_alg} pointed out, the downside of the $\ell_1$ regularization term is its suppression on {\RA} and {\TA} compared to {\FT} and {\retrain}. To facilitate this deficiency, we introduced a well-designed scheduler to $\gamma$, the parameter of the regularization term, and conducted a comprehensive ablation study on designing the scheduler. The results of unlearning performance with different parameter schedulers, constant, linearly increasing, and linearly decreasing schedulers, are presented in \textbf{Tab.\,\ref{tab: ablation_l1_scheduler}}.

% \begin{wrapfigure}{r}{60mm}
\vspace*{-7.8mm}
\begin{center}
\includegraphics[width=60mm,height=!]{figs/Cifar_l1.pdf}

\vspace*{-1mm}
\begin{tabular}{cc}
{\scriptsize{\hspace*{-1.5mm}(a) Class-wise forgetting}} &{\scriptsize{\hspace*{3mm}(b) Random data forgetting}}
\end{tabular}
\end{center}
\vspace*{-2.5mm}


% \includegraphics[width=60mm,height=!]{figs/Cifar_l1.pdf}}
% \vspace*{-2mm}
\caption{\footnotesize{
 Performance   of sparsity-aware unlearning vs. {\FT} and {\retrain} on class-wise forgetting and random data forgetting under (CIFAR-10, ResNet-18). 
 Each   metric is normalized to $[0,1]$ based on the best result  across unlearning methods 
  for ease of visualization, while the actual best value  is provided (\textit{e.g.}, $2.52$  is the  least computation time for class-wise forgetting). 
% \SL{[talk to me.]}
 % Each performance metric is normalized to $[0,1]$ 
 % using $1-\frac{|M-R|}{R}$, where $M$ denotes the different metrics from approximate unlearning, and $R$ denotes those from {\retrain}. {\RTE} is normalized to $[0,1]$ by using ${\frac{M}{\FT}}$. \JC{need to clarify that RTE is based on FT running time} \JH{updated}
% \JC{Merge to one figure and use one caption}
% \JC{Add `prune-first-then-unlearn'?}
}
}
%\JC{move other datasets (except cifar10) to appendix}
  \label{fig: results_l1_sparse_unlearn}
\vspace*{-7mm}
\end{wrapfigure}



% \begin{figure}[ht]
%     \centering
%     \begin{minipage}[htb]{75mm}
% \begin{center}
% \raisebox{-0.5cm}{ % Adjust this value as needed
% \includegraphics[width=0.95\textwidth,height=!]{figs/Cifar_l1.pdf}
% }

% \vspace*{-1mm}
% \begin{tabular}{cc}
% {\scriptsize{\hspace*{-1.5mm}(a) Class-wise forgetting}} &{\scriptsize{\hspace*{3mm}(b) Random data forgetting}}
% \end{tabular}
% \end{center}
% \vspace*{-2.5mm}
% \caption{\footnotesize{
%  Performance   of sparsity-aware unlearning vs. {\FT} and {\retrain} on class-wise forgetting and all-class random data forgetting under (CIFAR-10, ResNet-18). 
%  Each   metric is normalized to $[0,1]$ based on the best result  across unlearning methods 
%   for ease of visualization, while the actual best value  is provided (\textit{e.g.}, $2.52$  is the  least computation time for class-wise forgetting). 
% }
% }
%   \label{fig: results_l1_sparse_unlearn}
%     \end{minipage}
%     \hfill % To ensure that the figures are placed side by side
% \begin{minipage}[htb]{60mm}
% \begin{center}
%   \begin{tabular}{cccc}
%     \hspace*{-2mm}  \includegraphics[width=30mm,height=!]{figs/rebuttal/backdoor_ASR.pdf} &
%     \hspace*{-5mm} \includegraphics[width=30mm,height=!]{figs/rebuttal/backdoor_SA.pdf} \\
% \end{tabular}  
% \end{center}
% \vspace*{-2mm}
% \caption{
% % One figure demonstrates that our methods can decrease attack success rates and maintain good generalization performance.
% Performance  of  Trojan model cleanse   via proposed unlearning vs. model sparsity, where `Original' refers to the original Trojan model.
% %the effectiveness of the `Unlearn first, then prune' paradigm on the trojan model
% %cleanse application. 
% \textbf{Left}: ASR vs. model sparsity. \textbf{Right}: SA vs. model sparsity. 
% %Each marker represents the mean value over $10$ independent trials. %\JC{Add {\MUSparse}}
% %Results  The line and shaded area of each plot represent the mean and variance   over $10$ independent trials. 
% %\JC{remove variance}
% }
%   \label{fig: results_MU_pruning_backdoor}
%     \end{minipage}
% \end{figure}

% Directly applying a constant $\gamma$ to {\MUSparse} would either yield a low unlearning efficacy with lower $\gamma$ (for $\gamma = 0$, the method reduces to {\FT}) or degraded generalization performance under higher $\gamma$. In \textbf{Tab.\,\ref{tab: ablation_l1_scheduler}}, we have identified an optimal point that balances these metrics. Still, it cannot achieve as good performance as scheduled $\gamma$. The linearly increasing scheduler implies that the method initially emphasizes the unlearning term, then gradually shifts its focus towards sparsity. Conversely, the linearly decaying scheduler suggests that the focus initially lands on the sparsity, then gradually shifts to the unlearning term. The results reveal that the decaying scheduler outperforms all the others on both class-wise forgetting and random data forgetting, which aligned with the inspiration of the `prune first, then unlearn' paradigm. Therefore, the decaying scheduler will use in successive experiments except specified otherwise.



\noindent \textbf{{Effectiveness of sparsity-aware unlearning.}}
\iffalse
In  \textbf{Fig.\,\ref{fig: results_l1_sparse_unlearn}},
%and \textbf{Fig.\,\ref{fig: results_MU_pruning}},
%\textbf{Table\,\ref{tab: sparse_MU vs MU}} \SL{[or Fig.\,\ref{fig: results_MU_pruning}]}, 
we present the performance of proposed sparsity-aware unlearning methods (\textit{i.e.}, {\MUSparse}). 
%We      show the performance of {\retrain} and {\IU}-based approximate unlearning for comparison. 
To justify the effectiveness of our proposal, we implement  {\MUSparse} following the objective of {\FT},  the approximate unlearning method with the largest efficacy gap against exact unlearning on the original dense model as shown in Tab.\,\ref{tab: overall_performance}.

%As shown in Table\,\ref{tab: overall_performance}, {\IU} on sparse models can yield the    unlearning performance closest  to {\retrain}. 

%Moreover, to better justify the efficacy of our proposal, we adopt the  {\FT}-based unlearning objective and method to implement  {\MUSparse} and {\MUAO}. Recall that {\FT} is the simplest fine-tuning method with the worst unlearning efficacy on the dense model. 

As shown in \textbf{Fig.\,\ref{fig: results_l1_sparse_unlearn}}, {\MUSparse} improves the efficacy of unlearning (in terms of {\UA} and {\MIAF}) over {\IU}, and only has a quite small gap with {\retrain}  even if  the model considered for unlearning  is dense (without ever pruning). 
 This is because {\MUSparse} imposes a sparse regularization in  \eqref{eq: MUSparse} to penalize the model weights during unlearning, as shown in Fig.\,\ref{fig: l1_weight_magnitude}.
%outperforms {\IU}  in {\UA} and {\MIAF} and 
% Furthermore, 
% we note that {\MUAO} reduces to {\FT}  on dense model (\textit{i.e.}, there exists no alternating optimization between unlearning and pruning when $p\% = 0$). Thus, {\MUAO} only remains effective in the sparse regime  ($p >0$) and  outperforms {\MUSparse} in general, particularly in  {\RA} and {\TA}.
More results 
%vs. sparsity   
can be found in Fig.\,\ref{fig: results_l1_sparse_unlearn_others}. We also refer readers to Appendix \ref{appendix: additional results} for more unlearning scenarios. 
%in particular for the significant improvement in {\RA} and {\TA}.



% {\MUSparse} formulates unlearning as a sparsity-promoting optimization problem. 
\fi
In \textbf{Fig.\,\ref{fig: results_l1_sparse_unlearn}},
we   showcase the effectiveness of the proposed sparsity-aware unlearning method, \textit{i.e.}, {\MUSparse}. 
For ease of presentation, we focus on the comparison with  {\FT} and the optimal {\retrain}  strategy in both class-wise forgetting and random data forgetting  scenarios under (CIFAR-10, ResNet-18). As we can see, {\MUSparse}  outperforms {\FT} in  the unlearning efficacy ({\UA} and {\MIAF}), and closes the performance gap with {\retrain}  without losing the computation advantage of approximate unlearning. We refer readers to Appendix\,\ref{appendix: additional results} and Fig.\,\ref{fig: results_l1_sparse_unlearn_others} for further exploration of {\MUSparse} on additional datasets.
% \textbf{Fig.\,\ref{fig: results_l1_sparse_unlearn}} shows that {\MUSparse}  outperforms the conventional unlearning method such as {\FF}, and closes the performance gap with {\retrain} . 
% enhances the efficacy of unlearning (in terms of {\UA} and {\MIAF}) when compared to {\FF}, and the gap with {\retrain} is minimal, and the model considered for unlearning is dense (without any pruning). This is primarily attributed to {\MUSparse} imposing a sparse regularization, as per \eqref{eq: MUSparse}, to penalize the model weights during the unlearning process, as depicted in Fig.\,\ref{fig: l1_weight_magnitude}. Additional results are presented in Fig.\,\ref{fig: results_l1_sparse_unlearn_others}. 




\iffalse
\fi
%As a result, both {\IU} and {\GA} have advantages in privacy preservation of $\Dr$ after unlearning.




% \paragraph{Weight pruning gives rise to tradeoff in {\MU} between  efficacy and generalization.}
% One figure demonstrates that sparsity will bring degradation in the test accuracy, but improve efficacy.



% \iffalse 
% \paragraph{Prune first, then unlearn vs. Sparsity-infused MU.}
% We have demonstrated that sparsity will reduce the gap between exact unlearning and approximate unlearning in the previous results. Here we conduct experiments to verify the effectiveness of our proposed sparse-aware unlearning methods. \textbf{Table\,\ref{tab: sparse_MU vs MU}} shows the comparison between our proposed sparse-aware unlearning methods and the best `prune first, then unlearn' methods {\IU} on the CIFAR-10 dataset. We can observe that our proposed methods outperform the {\IU} in each metric.  
% \fi 



% \paragraph{Sparsification improves MU’s accuracy and efficacy across different models and unlearning scenarios}
% Table shows the relationship between sparsity and unlearning when using different arch and unlearning settings. 

%\SL{I stop here.}

\begin{wrapfigure}{r}{80mm}
\vspace*{-3mm}
\centerline{
\begin{tabular}{cccc}
    \hspace*{-2mm}  \includegraphics[width=40mm,height=!]{figs/rebuttal/backdoor_ASR.pdf} &
    \hspace*{-5mm} \includegraphics[width=40mm,height=!]{figs/rebuttal/backdoor_SA.pdf} \\

    % \hspace*{-5mm} \includegraphics[width=.25\textwidth,height=!]{figs/placeholder_JC/backdoor_ours_attack_acc.pdf} &
    % \hspace*{-5mm}  \includegraphics[width=.25\textwidth,height=!]{figs/placeholder_JC/backdoor_ours_test_acc.pdf} 
\end{tabular}
}
\vspace*{-2mm}
\caption{
% One figure demonstrates that our methods can decrease attack success rates and maintain good generalization performance.
Performance  of  Trojan model cleanse   via proposed unlearning vs. model sparsity, where `Original' refers to the original Trojan model.
%the effectiveness of the `Unlearn first, then prune' paradigm on the trojan model
%cleanse application. 
\textbf{Left}: ASR vs. model sparsity. \textbf{Right}: SA vs. model sparsity. 
%Each marker represents the mean value over $10$ independent trials. %\JC{Add {\MUSparse}}
%Results  The line and shaded area of each plot represent the mean and variance   over $10$ independent trials. 
%\JC{remove variance}
}
  \label{fig: results_MU_pruning_backdoor}
 \vspace*{-4mm}
%\end{wrapfigure}
%\end{figure}
\end{wrapfigure}
\noindent \textbf{Application: {\MU} for Trojan model cleanse.}
We next present an application of {\MU} to remove the influence of poisoned backdoor data from a learned model,  following the backdoor attack setup   \cite{gu2017badnets}, where an adversary 
%The so-called backdoor (poisoning) attack \citep{gu2017badnets,goldblum2022dataset} 
manipulates a small portion of training data (\textit{a.k.a.}   poisoning ratio) by 
injecting a backdoor trigger (\textit{e.g.}, a small image patch) and modifying data labels towards a targeted incorrect label.  
%attack then serves as a ‘backdoor’ and enforces a spurious correlation between the Trojan trigger and the model training. 
The trained model is called \textit{Trojan model}, yielding the backdoor-designated incorrect prediction if the trigger is present at testing. Otherwise, it behaves normally. 
% That is, 
% training over the poisoned data set will enforce a spurious correlation between the Trojan trigger and the model prediction, so that the former  serves as a ‘backdoor’ for the trained model. 
%Backdoor Since To demonstrate the unlearning performance, we set up several backdoor attack experiments to


We then regard {\MU} as a defensive method to scrub the harmful influence of  poisoned training data in  the model's prediction, with a similar motivation as \citet{liu2022backdoor}.
%We assume that the set of poisoned data points is known \textit{a priori}, \textit{e.g.}, via Trojan trigger  detection \cite{wang2020practical}.
We evaluate the performance of the unlearned model from two perspectives, backdoor attack success rate (\textbf{ASR}) and standard accuracy (\textbf{SA}). 
\textbf{Fig.\,\ref{fig: results_MU_pruning_backdoor}} shows   ASR and {SA} of the   Trojan model (with poisoning ratio $10\%$)  and its unlearned version using the simplest {\FT} method against model sparsity. {Fig.\,\ref{fig: results_MU_pruning_backdoor} also includes the $\ell_1$-sparse {\MU} to demonstrate its effectiveness on  model cleanse. Since it is applied to a dense model (without using hard thresholding to force weight sparsity), it contributes just a single data point at the sparsity level 0\%.}
As we can see, the  original Trojan model maintains $100\%$ ASR and a similar SA across different model sparsity levels. By contrast, {\FT}-based unlearning can  reduce ASR without inducing much {SA} loss. Such a defensive advantage becomes more significant when sparsity reaches $90\%$. {Besides, $\ell_1$-sparse {\MU} can also effectively remove the backdoor effect while largely preserving the model’s generalization.} 
Thus, our proposed unlearning shows promise in  application of backdoor attack defense.

\noindent \textbf{Application: {\MU} to improve transfer learning.}
Further, we utilize the   {\MUSparse} method to mitigate the   impact of harmful data classes of   ImageNet    on transfer learning.  This approach is inspired by \citet{jain2022data}, which shows that  removing specific negatively-influenced ImageNet classes and retraining a source model  can enhance its transfer learning accuracy on    downstream   datasets  after finetuning. However, retraining the source model introduces additional computational overhead. {\MU} naturally addresses this limitation and offers a solution.
% a transfer influence score was proposed in \cite{jain2022data} to evaluate the usefulness of ImageNet data classes,  retraining the source model raises the computation overhead. 

% In the following section, we introduce an application of MU that removes the influence of harmful source classes in a pre-trained model, thereby enhancing the performance of transfer learning. Transfer learning, as it is well known, allows for the adaptation of a model trained on a \textit{source dataset} to optimize its performance on a \textit{downstream target task}.
% Recent research \cite{jain2022data} indicates that the exclusion of detrimental data from the source dataset can indeed augment the performance of transfer learning. Additionally, the same study provided a method capable of identifying beneficial subsets of the source dataset for different downstream tasks. However, the retraining of a model considering the optimal subset of the source dataset for each downstream task is computationally expensive and time-consuming.


\textbf{Tab.\,\ref{tab: transfer_results}} illustrates the transfer learning accuracy of the unlearned or retrained source model (ResNet-18) on ImageNet, with $n$ classes removed. The downstream target datasets used for evaluation are  SUN397 \cite{xiao2010sun} and OxfordPets \cite{parkhi2012cats}.
The  employed finetuning approach   is linear probing, which finetunes the classification head of the source model on target datasets while keeping the feature extraction network of the source model intact. 
% \JC{Our analysis focuses on fixed-weight transfer learning \cite{jain2022data}, which updates the linear classification head with the target domain data while freezing the feature extraction network.
% And we leverage transfer influences in \cite{jain2022data} to determine the classes to be unlearned. }%\SL{[Is the above true?]}
As we can see, removing data classes from the source ImageNet dataset    can lead to improved transfer learning accuracy compared to the conventional method of using the pre-trained model on the full ImageNet  (\textit{i.e.}, $n = 0$). Moreover,
our proposed 
\begin{wraptable}{r}{63mm}
\centering
\vspace*{-3.3mm}
\caption{
Transfer learning accuracy (Acc) and computation time (mins) of the unlearned   ImageNet model with $n \in \{ 100,200,300\}$ classes removed, where SUN397 and OxfordPets are downstream target datasets on linear probing transfer learning setting. When $n = 0$, transfer learning is performed using the pretrained model on the full ImageNet, serving as a baseline, together with the method in \cite{jain2022data}  for comparison. %\JC{[updated]}
% \JC{need to align SUN397 performance}
}
% \JC{Performance comparison of transfer learning on various datasets using the proposed unlearning method with different numbers of classes removal from the source dataset. {\acc} represents the post-fine-tuning model's generalization performance on the test set of downstream tasks, while {\TIME} signifies the time cost in minutes required to obtain a pre-trained model on the scrubbed ImageNet dataset via retraining or unlearning. Removing no class indicates that pre-training is performed on the full ImageNet dataset, which serves as the baseline for transfer learning.}
% \JC{Keep or change to Tab.\,\ref{tab: transfer_results_new}}
%\JC{Remove RTE, change RTE to Time, TA to ACC}
%Performance of transfer learning on various datasets via proposed unlearning vs. different numbers of classes in the source dataset be removed. \JC{{\acc} indicates the generalization performance on the testing set after fine-tuning on downstream tasks, {\TIME} indicates the time cost of getting a pre-trained on scrubbed ImageNet dataset by retraining or unlearning. Furthermore, removing zero class means pre-training is conducted on the full ImageNet dataset, which is the baseline of transfer learning.}
\label{tab: transfer_results}
\resizebox{63mm}{!}{
\begin{tabular}{c|c|cc|cc|cc}
\toprule[1pt]
\midrule
\multirow{2}{*}{Forgetting class \#}
  & 0 & \multicolumn{2}{c|}{100} & \multicolumn{2}{c|}{200} & \multicolumn{2}{c}{300}  \\ 
 % \midrule
  %Methods
  & \multicolumn{1}{c|}{{\acc}}  & 

\multicolumn{1}{c|}{{\acc}}  & \multicolumn{1}{c|}{{\TIME}} &  
\multicolumn{1}{c|}{{\acc}}  & \multicolumn{1}{c|}{{\TIME}} & 
\multicolumn{1}{c|}{{\acc}}  & \multicolumn{1}{c}{{\TIME}} 
  \\
% \cline{3-10}

\midrule
\rowcolor{Gray}
\multicolumn{8}{c}{OxfordPets} \\
\midrule
% {\retrain}
Method \cite{jain2022data} 
 & \multirow{2}{*}{85.70}  & 85.79	& 71.84 &86.10 & 61.53  &86.32 & 54.53
 \\
 \MUSparse & & 85.83&35.47&	86.12&30.19& 86.26& 26.49
 \\
\midrule
\rowcolor{Gray}
\multicolumn{8}{c}{SUN397} \\
\midrule
 Method \cite{jain2022data}   & \multirow{2}{*}{46.55} & 46.97	& 73.26 &47.14& 61.43 &47.31 & 55.24
 % Re-pretrain \cite{jain2022data}   & \multirow{2}{*}{46.55} & 46.62	& 73.26 &46.85& 61.43 &47.06 & 55.24
 				
 \\
 \MUSparse & & 47.20& 36.69 &	47.25& 30.96 &	47.37& 27.12	
 \\
\midrule
\bottomrule[1pt]
\end{tabular}
}
\vspace*{-8mm}
\end{wraptable}%
{\MUSparse} method achieves comparable or even slightly better 
transfer learning accuracy than the retraining-based approach \citep{jain2022data}.  Importantly, {\MUSparse} offers the advantage of computational efficiency 2$\times$ speed up over previous method \citep{jain2022data} across all cases, making it an appealing choice for transfer learning using large-scale models.
Here we remark that in order to align with previous method \cite{jain2022data}, we employed a fast-forward computer vision training pipeline  (FFCV) \citep{leclerc2022ffcv}
to accelerate our ImageNet training on GPUs.
% \SL{[Missing FFCV discussion and citation. E.g., We remark that   a fast forward computer vision (FFCV) [refs] training pipeline is used to accelerate our ImageNet training on GPUs.]}
% demonstrates the performance of transfer learning on SUN397 \cite{xiao2010sun} and OxfordPets \cite{parkhi2012cats}, considering different numbers of excluded ImageNet classes as identified in \cite{jain2022data}. 
% As the table shown, {\FT}-based {\MU} consistently surpasses the performance of the full ImageNet pre-trained model, regardless of the number of ImageNet classes removed. For instance, {\FT}-based {\MU} enhances the accuracy of target tasks by $0.54\%$ compared to the full ImageNet pre-trained model when the number of ImageNet classes removed is $300$ on the OxfordPets with almost half time of retrain methods. Therefore, we propose that our unlearning method presents an effective strategy for enhancing transfer learning.


% To address this challenge, we employ {\FT}-based {\MU}  to neutralize the impact of detrimental source classes in the full-ImageNet pre-trained model, thereby improving transfer learning. The proposed methodology effectively eliminates the need for retraining the model with each optimal subset of the source dataset, making the process more efficient and scalable. 
% Here we operate under the assumption that the subset requiring removal is already known a priori, following the method detailed in the previous work \cite{jain2022data}. We adhere to the same training configuration for the full ImageNet pre-trained model as set out in \cite{jain2022data}, electing to use the linear probe (LP) as our transfer training protocol.

% \textbf{Tab. \ref{tab: transfer_results}} demonstrates the performance of transfer learning on SUN397 \cite{xiao2010sun} and OxfordPets \cite{parkhi2012cats}, considering different numbers of excluded ImageNet classes as identified in \cite{jain2022data}. 
% As the table shown, {\FT}-based {\MU} consistently surpasses the performance of the full ImageNet pre-trained model, regardless of the number of ImageNet classes removed. For instance, {\FT}-based {\MU} enhances the accuracy of target tasks by $0.54\%$ compared to the full ImageNet pre-trained model when the number of ImageNet classes removed is $300$ on the OxfordPets with almost half time of retrain methods. Therefore, we propose that our unlearning method presents an effective strategy for enhancing transfer learning.

% Here we assume that the subset should be scrubbed as a known prior via the method mentioned in previous work \cite{jain2022data}. We followed the same training setting for the full ImageNet pre-trained model in \cite{jain2022data}, and chose linear probe (LP) as our transfer training protocol. \textbf{Fig}\,\ref{fig: results_MU_transfer} shows the performance of transfer learning on SUN397 \cite{xiao2010sun} and OxfordPets \cite{parkhi2012cats} with different size of excluding ImageNet classes identified from \cite{jain2022data}. As we can see, the accuracy on the downstream task will increase first, then decrease with the number of ImageNet classes removed increasing. Besides, {\FT}-based {\MU} can outperform the full ImageNet pre-trained model at different numbers of ImageNet classes removed. For example, {\FT}-based {\MU} can boost $2\%$ target tasks accuracy compared to that of the full ImageNet pre-trained model. Thus, the proposed unlearning gives an effective method to improve transfer learning. 







% Since MU's purpose is to remove the effect of specific data, there is a straightforward application for MU. Backdoored neural network \citep{gu2017badnets}, or \textit{BadNet}, shows that when attackers poison part of the training dataset in a given pattern, the neural network, trained at the poisoned dataset, will misbehave on the images with the same pattern. Assuming we already know which part of the dataset was poisoned, we can apply MU methods to neutralize the poisoned images' influence to defend against data poisoning attacks on models. An effective MU method will get a relatively lower attack success rate (ASR) and be more stable in forgetting the poisoned data. To demonstrate the performance of the MU methods, we performed several experiments. 



% We follow the \textit{All-to-all attack} in \citep{gu2017badnets} to poison part of the training dataset. The model was trained and pruned over the same poisoned dataset in different sparsity. Then we deemed the poisoned dataset as the forgetting dataset, and applied both the `\textit{prune first, then unlearn}' methods and the `\textit{sparsity-aware unlearning}' methods. Additionally, In the `prune first, then unlearn' settings, we chose the OMP and Finetune as the backbones of pruning and unlearning. Fig \ref{fig: results_MU_pruning_backdoor} shows the results of experiments under backdoor attack settings. The curve indicates that under either unlearn method, the ASR decreased rapidly when the model became sparser, with a negligible trade-off in standard accuracy (SA).

% TODO: add results and analysis
% \vspace*{-4mm}
\noindent \textbf{Additional results.} 
{%We include more results in Appendix\,\ref{appendix: additional results}. In particular, 
We found that model sparsity also enhances the privacy of the unlearned model, as evidenced by a lower {\MIAR}. Refer to Appendix\,\ref{appendix: additional results} and Fig.\,\ref{fig: results_privacy} for more results. In addition, we have expanded our experimental scope to encompass the `prune first, then unlearn' approach across various datasets and architectures. The results can be found in Tab.\,\ref{tab: overall_performance_ext_datasets}, Tab.\,\ref{tab: overall_performance_ext_archs}, and Tab.\,\ref{tab: overall_performance_ImageNet}. Furthermore, we conducted experiments on the $\ell_1$-sparse {\MU} across different datasets, the Swin-Transformer architecture, and varying model sizes within the ResNet family. The corresponding findings are presented in Fig.\,\ref{fig: results_l1_sparse_unlearn_others} and Tab.\,\ref{tab: sparse_MU vs MU}, \ref{tab: vit}, \ref{tab: overall_perfoamnce_arch_20} and \ref{tab: overall_perfoamnce_arch_50}.}





% \paragraph{Overall performance: Sparsity reduces the gap between exact unlearning and approximate unlearning.}
% We look at the relationship between machine learning and sparsity in what follows. There is \textit{one key observation} drawn from our overall results: Sparsity can reduce the gap between exact unlearning and approximate unlearning across all datasets and machine unlearning methods (as shown in \textbf{Table\,\ref{tab: overall_perfoamnce})}. Table\,\ref{tab: overall_perfoamnce} shows different evaluation metrics of different unlearning methods at different sparsity levels across different datasets. For comparison, we also present the performance of exact unlearning methods Retrain. As we can see, the gap between imperfect unlearning and perfect unlearning is decreasing with sparsity growing, especially in {\UA} and {\MIAF} these two metrics. For example, the gap between {\FT} and Retrain of $95\%$ sparsity model drops $51.11\%$  compared to the dense model in  the {\UA} on the CIFAR-10 dataset. This phenomenon is also justified on different datasets shown in Table\,\ref{tab: overall_perfoamnce}. Although sparsity can reduce the gap between exact unlearning and approximate unlearning, the improvement is different in the diverse machine unlearning methods. From Table\,\ref{tab: overall_perfoamnce}, {\FT} will benefit from sparsity most, where {\MIAF} reduced $13\%$-$20\%$ across all datasets.  
% Besides, we also can observe in most cases that {\IU} outperforms other unlearning methods at different sparsity levels, which is the most competitive method in different evaluation metrics. However, this method needs to be tuned carefully to choose suitable hyperparameters.
% \begin{table*}[h]
% \centering
% \caption{\footnotesize{Performance overview of MU VS. Sparsity. ResNet-18 \cite{he2016deep} are used across different datasets. All sparse models are obtained from OMP  \cite{frankle2018lottery}. We carefully tune the hyperparameters for all machine unlearning methods to report the model which can achieve the best unlearning performance at different sparsity ratios. The results $a_{\pm{b}}$ represent mean $a$ and standard deviation $b$ over $10$ random trials. We also reported the performance gap between Retrain and other approximate unlearning.The relative drop or improvement represented by $a$\textcolor{red}{$\LARGE\downarrow$} or $a$\textcolor{green}{$\LARGE\uparrow$}.}}
% \label{tab: overall_perfoamnce}
% \vspace*{0.1in} % Requirements, do not delete.
% \resizebox{0.95\textwidth}{!}{
% \begin{tabular}{c|ccc|ccc|ccc|ccc|c}
% \toprule[1pt]
% \midrule
%   \multirow{2}{*}{\MU}& \multicolumn{3}{c|}{{\UA}} & \multicolumn{3}{c|}{{\MIAF}}& \multicolumn{3}{c|}{{\RA}} & \multicolumn{3}{c|}{{\TA}}&\multirow{2}{*}{\RTE}  \\ 
%   & \multicolumn{1}{c|}{\textsc{Dense}} & \multicolumn{1}{c|}{\textsc{0.75}} & \multicolumn{1}{c|}{\textsc{0.95}}
%     & \multicolumn{1}{c|}{\textsc{Dense}} & \multicolumn{1}{c|}{\textsc{0.75}} & \multicolumn{1}{c|}{\textsc{0.95}}
%     & \multicolumn{1}{c|}{\textsc{Dense}} & \multicolumn{1}{c|}{\textsc{0.75}} & \multicolumn{1}{c|}{\textsc{0.95}}
%       & \multicolumn{1}{c|}{\textsc{Dense}} & \multicolumn{1}{c|}{\textsc{0.75}} & \multicolumn{1}{c|}{\textsc{0.95}}
%   \\
% % \cline{3-10}

% \midrule
% \rowcolor{Gray}
% \multicolumn{14}{c}{\Large Cifar10} \\
% \midrule
% \retrain &$100.00_{\pm{0.00}}$   &  $100.00_{\pm{0.00}}$ & $100.00_{\pm{0.00}}$ 
% &$100.00_{\pm{0.00}}$   &  $100.00_{\pm{0.00}}$ & $100.00_{\pm{0.00}}$ 
% &$100.00_{\pm{0.00}}$   &  $100.00_{\pm{0.00}}$ & $99.99_{\pm{0.01}}$
% &$94.83_{\pm{0.11}}$   &  $94.71_{\pm{0.13}}$ & $91.80_{\pm{0.89}}$
%  &\\
%   \FT &$22.53_{\pm{8.16}}$ (77)&$28.00_{\pm{9.46}}$  (\textcolor{green}{\ding{116}}72)&$73.64_{\pm{6.44}}$ (\textcolor{green}{\ding{116}}26)&$75.00_{\pm{14.68}}$ (\textcolor{green}{\ding{116}}25)& $83.02_{\pm{16.33}}$ (\textcolor{green}{\ding{116}}16) &$96.92_{\pm{1.27}} $ (\textcolor{green}{\ding{116}}3)
%   &$99.87_{\pm{0.04}}$ (\textcolor{green}{\ding{116}}0.13)&\cellcolor{LightCyan} $99.92_{\pm{0.04}}$ (\textcolor{green}{\ding{116}}0.08) & $99.87_{\pm{0.05}}$ (\textcolor{green}{\ding{116}}0.12)&$94.31_{\pm{0.19}}$ (\textcolor{green}{\ding{116}}0.52)
  
%     &\cellcolor{LightCyan}$94.70_{\pm{0.08}}$ (\textcolor{green}{\ding{116}}0.01)

%  &$94.32_{\pm{0.12}}$ (\textcolor{green}{\ding{115}}2.52)
% &   
  
  
  
%   \\
%  \GA &$93.08_{\pm{0.29}}$ (6.92) & $93.55_{\pm{0.31}}$ (\textcolor{green}{\ding{116}}6.45)  &$98.09_{\pm{0.11}}$ (\textcolor{green}{\ding{116}}1.91)
% & $93.08_{\pm{0.31}}$ (\textcolor{green}{\ding{116}}6.92)& $94.03_{\pm{0.27}}$ (\textcolor{green}{\ding{116}}5.97)& $94.67_{\pm{0.25}}$ (\textcolor{green}{\ding{116}}5.33)
% & $92.60_{\pm{0.25}}$ (7\textcolor{red}{$\LARGE\downarrow$})& $92.90_{\pm{0.25}}$ (7\textcolor{red}{$\LARGE\downarrow$})& $87.74_{\pm{0.27}}$ (11\textcolor{red}{$\LARGE\downarrow$}) 
% & $86.64_{\pm{0.28}}$ (8\textcolor{red}{$\LARGE\downarrow$})& $84.07_{\pm{0.25}}$ (10\textcolor{red}{$\LARGE\downarrow$})& $82.58_{\pm{0.27}}$ (9\textcolor{red}{$\LARGE\downarrow$}) 
% &   
%  \\
%   $\FF$  & $79.93_{\pm{8.92}}$ (20\textcolor{red}{$\LARGE\downarrow$})& $87.66{\pm{7.03}}$ (13\textcolor{red}{$\LARGE\downarrow$})& $94.83_{\pm{4.29}}$ (5\textcolor{red}{$\LARGE\downarrow$}) 
%   & $100.00_{\pm{0.00}}$ (0\textcolor{red}{$\LARGE\downarrow$})& $100.00_{\pm{0.00}}$ (0\textcolor{red}{$\LARGE\downarrow$})& $100.00_{\pm{0.00}}$ (0\textcolor{red}{$\LARGE\downarrow$}) 
%     & $99.45_{\pm{0.24}}$ (0\textcolor{red}{$\LARGE\downarrow$})& $99.55_{\pm{0.19}}$ (0\textcolor{red}{$\LARGE\downarrow$})& $99.48_{\pm{0.23}}$ (0\textcolor{red}{$\LARGE\downarrow$})
%         & $94.18_{\pm{0.08}}$ (0\textcolor{red}{$\LARGE\downarrow$})& $94.47_{\pm{0.15}}$ (0\textcolor{red}{$\LARGE\downarrow$})& $94.04_{\pm{0.10}}$ (2\textcolor{green}{$\LARGE\uparrow$})& 
%   \\
%  \IU 
%   &$87.82_{\pm{2.15}} $ (12\textcolor{red}{$\LARGE\downarrow$})&$98.63_{\pm{0.22}}$  (1\textcolor{red}{$\LARGE\downarrow$})& \cellcolor{LightCyan}$99.47_{\pm{0.15}}$ (0\textcolor{red}{$\LARGE\downarrow$})
%  & $95.96_{\pm0.21}$ (4\textcolor{red}{$\LARGE\downarrow$})
%  &$99.82_{\pm{0.13}}$ (0\textcolor{red}{$\LARGE\downarrow$}) &\cellcolor{LightCyan}$99.93_{\pm{0.04}}$ (0\textcolor{red}{$\LARGE\downarrow$})
%  &$97.98_{\pm{0.21}}$ (2\textcolor{red}{$\LARGE\downarrow$}) &$94.50_{\pm{0.19}}$ (5\textcolor{red}{$\LARGE\downarrow$})
%  &$97.24_{\pm{0.13}}$ (2\textcolor{red}{$\LARGE\downarrow$}) 
%  &$91.42_{\pm{0.21}}$ (3\textcolor{red}{$\LARGE\downarrow$})&$88.04_{\pm{0.22}}$ (6\textcolor{red}{$\LARGE\downarrow$})&$90.76_{\pm{0.18}}$ (1\textcolor{red}{$\LARGE\downarrow$}) &
%  \\
% %  \FTSparse &$100.00_{\pm{0.00}}$  &$100.00_{\pm{0.00}}$  & $91.49_{\pm{1.21}}$&$87.17_{\pm1.31}$
% %   &$100.00_{\pm{0.00}}$  &$100.00_{\pm{0.00}}$  & $91.69_{\pm{1.57}}$&$87.30_{\pm1.39}$
% %   &$100.00_{\pm{0.00}}$  &$100.00_{\pm{0.00}}$  & $95.74_{\pm{0.54}}$&$88.97_{\pm1.00}$
% % \\
% % \FTAO  
% %   & -&-  & -&-
% % & $43.82_{\pm{11.68}}$& $98.64_{\pm{0.71}}$ & $99.96_{\pm{0.03}}$&$94.79_{\pm0.07}$
% %   &$99.80_{\pm{0.19}}$  &$100.00_{\pm{0.00}}$ & $99.86_{\pm{0.05}}$&$94.55_{\pm0.11}$
% % \\
% \midrule
% \rowcolor{Gray}
% \multicolumn{14}{c}{\Large Cifar100} \\
% \midrule
%  \retrain &$100.00_{\pm{0.00}}$  &  $100.00_{\pm{0.00}}$ & $100.00_{\pm{0.00}}$ 
%   &$100.00_{\pm{0.00}}$  &  $100.00_{\pm{0.00}}$ & $100.00_{\pm{0.00}}$
%    &$99.97_{\pm{0.01}}$  &  $99.96_{\pm{0.01}}$ & $96.68_{\pm{0.15}}$
%    &$73.74_{\pm{0.19}}$  &  $73.23_{\pm{0.17}}$ & $69.49_{\pm{0.41}}$ 
% & 
%  \\
%  \FT &$14.80_{\pm{6.29}}$  &  $17.20_{\pm{5.50}}$ & $42.22_{\pm{5.06}}$
%   &$69.82_{\pm{5.93}}$  &  $72.40_{\pm{9.98}}$ & $84.40_{\pm{4.32}}$
%    &$99.86_{\pm{0.04}}$  &  $99.87_{\pm{0.05}}$ & $97.72_{\pm{0.47}}$
%    &$72.16_{\pm{0.22}}$  &  $72.28_{\pm{0.13}}$ & $70.44_{\pm{0.11}}$
%  \\
%  \GA  &$81.47_{\pm{0.32}}$ &$87.38_{\pm{0.41}}$ & $99.01_{\pm{0.01}}$
%  & $93.47_{\pm{4.56}}$ &$97.42_{\pm{0.11}}$ &$100.00_{\pm{0.00}}$ 
%  & $90.33_{\pm{1.71}}$& $91.27_{\pm{1.02}}$ &$80.45_{\pm{0.78}}$ 
%  & $64.94_{\pm{0.74}}$&$65.36_{\pm{0.21}}$& $60.99_{\pm{0.14}}$ 
% &
%  \\
%  % \FF & \\
% \IU &$98.00_{\pm{0.34}}$ &$97.88_{\pm{0.21}}$ &$99.78_{\pm{0.01}}$
% & $100.00_{\pm{0.00}}$ & $100.00_{\pm{0.00}}$&$100.00_{\pm{0.00}}$  
% & $99.43_{\pm{0.02}}$&$99.60_{\pm{1.02}}$ & $97.68_{\pm{0.17}}$
% &$72.16_{\pm{0.22}}$ &$72.28_{\pm{0.13}}$ & $70.44_{\pm{0.11}}$
% & 
% \\
% % \FTSparse & 
% % $94.55_{\pm{2.82}}$& $97.02_{\pm{2.60}}$ & $88.06_{\pm{3.14}}$&$63.61_{\pm2.25}$
% % $94.55_{\pm{2.82}}$& $97.02_{\pm{2.60}}$ & $88.06_{\pm{3.14}}$&$63.61_{\pm2.25}$
% % & -&-  & -&-
% % \\
% % \FTAO   & -&-  & -&-
% % &$65.82_{\pm{12.66}}$ & $86.18_{\pm{7.08}}$ & $98.79_{\pm{0.21}}$&$73.54_{\pm0.08}$
% % &$74.27_{\pm{5.34}}$& $93.42_{\pm{2.82}}$&  $97.70_{\pm{0.21}}$ &$72.54_{\pm0.32}$
% % \\
% \midrule
% \rowcolor{Gray}
% \multicolumn{14}{c}{\Large SVHN} \\
% \midrule
%  \retrain &$100.00_{\pm{0.00}}$  &  $100.00_{\pm{0.00}}$ & $100.00_{\pm{0.00}}$ 
%  & $100.00_{\pm{0.00}}$ &$100.00_{\pm{0.00}}$  &  $100.00_{\pm{0.00}}$ 
% & $100.00_{\pm{0.00}}$ &$100.00_{\pm{0.00}}$  &  $100.00_{\pm{0.00}}$  
%  &  $95.71_{\pm{0.12}}$ & $95.72_{\pm{0.12}}$ & $94.95_{\pm{0.05}}$ 
%  \\
%  \FT & {$11.48_{\pm{8.12}}$ } &  $21.98_{\pm{14.87}}$ & $51.93_{\pm{19.62}}$ 
% & $86.12_{\pm{9.62}}$ & $87.49_{\pm{8.93}}$&$99.42_{\pm{0.51}}$  
% & $100.00_{\pm{0.00}}$ &$100.00_{\pm{0.00}}$  &  $99.00_{\pm{0.00}}$ 
% & $95.99_{\pm{0.07}}$ &$95.95_{\pm{0.09}}$  &  $95.89_{\pm{0.02}}$ 
%  \\
%   \GA &$83.87_{\pm{0.19}}$  &  $84.89{\pm{0.12}}$ & $86.52_{\pm{0.11}}$ 
%   & $99.97_{\pm{0.02}}$ &$100.00_{\pm{0.00}}$  &  $100.00_{\pm{0.00}}$ 
%   & $99.60_{\pm{0.15}}$ & $99.51_{\pm{0.13}}$ &$98.37_{\pm{0.11}}$  
%   &  $95.27_{\pm{0.02}}$ & $95.08_{\pm{0.01}}$ & $93.42_{\pm{0.04}}$ 
  
%   \\
% % \FF & & & & & & & & &   \\
% \IU &$95.11_{\pm{0.02}}$  &  $100.00_{\pm{0.00}}$ & $100.00_{\pm{0.00}}$ 
% &$99.89_{\pm{0.04}}$  &  $100.00_{\pm{0.00}}$ & $100.00_{\pm{0.00}}$
% &$100.00_{\pm{0.00}}$  &  $99.99{\pm{0.01}}$ & $99.85_{\pm{0.02}}$
% &$95.70_{\pm{0.07}}$  &  $95.19_{\pm{0.04}}$ & $94.90_{\pm{0.02}}$
% \\
% % \FTSparse & 
% % \\
% % \FTAO  & 
% % \\
% \midrule
% \rowcolor{Gray}
% \multicolumn{14}{c}{\Large TinyImagenet} \\
% \midrule
%  \retrain &$100.00_{\pm{0.00}}$  &  $100.00_{\pm{0.00}}$ & $100.00_{\pm{0.00}}$ 
%  &$100.00_{\pm{0.00}}$  &  $100.00_{\pm{0.00}}$ & $100.00_{\pm{0.00}}$ 
%  &$99.98_{\pm{0.01}}$  &  $99.98_{\pm{0.01}}$ & $90.89_{\pm{0.03}}$ 
%   &$65.01_{\pm{0.13}}$  &  $62.56_{\pm{0.22}}$ & $58.46_{\pm{0.28}}$ 
%  \\
% \FT &$25.13_{\pm{1.20}}$  &  $50.80_{\pm{2.59}}$ & $76.33_{\pm{3.52}}$ 
% & $76.87_{\pm{0.47}}$ &$87.07_{\pm{0.51}}$  &  $97.13_{\pm{0.68}}$ 
% & $99.98_{\pm{0.01}}$ & $97.94_{\pm{0.05}}$ &$89.18_{\pm{0.40}}$  
% &  $65.55_{\pm{0.18}}$ & $64.27_{\pm{0.32}}$ & $59.74_{\pm{0.12}}$  \\
% \GA &$83.87_{\pm{0.19}}$  &  $86.67_{\pm{0.34}}$ & $92.27_{\pm{0.09}}$
%  &$90.20_{\pm{0.02}}$  &  $92.87_{\pm{0.09}}$ & $97.00_{\pm{0.04}}$ & $98.44_{\pm{0.01}}$
%   &$85.16_{\pm{0.02}}$  &  $80.78_{\pm{0.03}}$ & $59.84_{\pm{0.03}}$ & $58.68_{\pm{0.02}}$  & $55.74_{\pm{0.03}}$ & \\
% % \FF & & & & & & & & &  \\
% \IU   &$89.60_{\pm{0.24}}$  &  $94.00_{\pm{0.15}}$ & $95.81_{\pm{0.07}}$ 

% &$100_{\pm{0.00}}$  &  $100.00_{\pm{0.00}}$ & $100.00_{\pm{0.00}}$
% &$96.78_{\pm{0.03}}$  &  $84.53_{\pm{0.21}}$ & $82.11_{\pm{0.13}}$
% &$63.19_{\pm{0.05}}$  &  $61.41_{\pm{0.01}}$ & $58.73_{\pm{0.06}}$
% \\
% % \FTSparse & 
% % \\
% % \FTAO  & 
% % \\
% \midrule
% \bottomrule[1pt]
% \end{tabular}
% }
% \vspace*{-3mm}

% \end{table*}




% \paragraph{Model sparsity benefits privacy of unlearning for `free'.}


% One figure to demonstrate the {\MIAR} decreasing. 
% \begin{figure}[htb]
% %\begin{wrapfigure}{r}{80mm}
% %\vspace*{-6mm}
% \centerline{
% %\begin{tabular}{cc}
% %\hspace*{0mm}\includegraphics[width=.3\textwidth,height=!]{figure/performance_comparison.pdf}  
% %&
% %\hspace*{-4mm}
% \includegraphics[width=.31\textwidth,height=!]{figs/SVC_MIA_training_privacy_confidence_vs_methods.pdf}
% % \\
% % \hspace*{2mm}\footnotesize{(a) Test accuracy vs. pruning ratio.} &   \footnotesize{(b) Runtime of pruning.}
% %\end{tabular}
% }
% \vspace*{-3mm}
% \caption{\footnotesize{
% Here we used (\MIAR) of different unlearning methods on the OMP to show that privacy will increase with sparsity growth.  [retrain,fisher,FT, GA, IU]
% }}
%   \label{fig: results_privacy}
% %  \vspace*{-3.8mm}
% %\end{wrapfigure}
% %\end{figure}
% \end{figure}

% % \paragraph{Weight pruning gives rise to tradeoff in {\MU} between  efficacy and generalization.}
% % One figure demonstrates that sparsity will bring degradation in the test accuracy, but improve efficacy.



% % \iffalse 
% % \paragraph{Prune first, then unlearn vs. Sparsity-infused MU.}
% % We have demonstrated that sparsity will reduce the gap between exact unlearning and approximate unlearning in the previous results. Here we conduct experiments to verify the effectiveness of our proposed sparse-aware unlearning methods. \textbf{Table\,\ref{tab: sparse_MU vs MU}} shows the comparison between our proposed sparse-aware unlearning methods and the best `prune first, then unlearn' methods {\IU} on the CIFAR-10 dataset. We can observe that our proposed methods outperform the {\IU} in each metric.  
% % \fi 


% \begin{table*}[htb]
% \centering
% \caption{\footnotesize{Performance comparison between `first prune, then unlearn' and `sparsity-aware MU'. The results $a_{\pm{b}}$ represent mean $a$ and standard deviation $b$ over $10$ random trials. We also reported the performance gap between Retrain and other approximate unlearning.The relative drop or improvement represented by $a$\textcolor{red}{$\LARGE\downarrow$} or $a$\textcolor{green}{$\LARGE\uparrow$}. }}
% \label{tab: sparse_MU vs MU}
% \vspace*{0.1in} % Requirements, do not delete.
% \resizebox{0.95\textwidth}{!}{
% \begin{tabular}{c|ccc|ccc|ccc|ccc|c}
% \toprule[1pt]
% \midrule
%   \multirow{2}{*}{\MU}& \multicolumn{3}{c|}{{\UA}} & \multicolumn{3}{c|}{{\MIAF}}& \multicolumn{3}{c|}{{\RA}} & \multicolumn{3}{c|}{{\TA}}&\multirow{2}{*}{\RTE}  \\ 
%   & \multicolumn{1}{c|}{\textsc{Dense}} & \multicolumn{1}{c|}{\textsc{0.75}} & \multicolumn{1}{c|}{\textsc{0.95}}
%     & \multicolumn{1}{c|}{\textsc{Dense}} & \multicolumn{1}{c|}{\textsc{0.75}} & \multicolumn{1}{c|}{\textsc{0.95}}
%     & \multicolumn{1}{c|}{\textsc{Dense}} & \multicolumn{1}{c|}{\textsc{0.75}} & \multicolumn{1}{c|}{\textsc{0.95}}
%       & \multicolumn{1}{c|}{\textsc{Dense}} & \multicolumn{1}{c|}{\textsc{0.75}} & \multicolumn{1}{c|}{\textsc{0.95}}
%   \\
% % \cline{3-10}

% \midrule
% \rowcolor{Gray}
% \multicolumn{14}{c}{\Large Cifar10} \\
% \midrule
% \retrain &$100.00_{\pm{0.00}}$   &  $100.00_{\pm{0.00}}$ & $100.00_{\pm{0.00}}$ 
% &$100.00_{\pm{0.00}}$   &  $100.00_{\pm{0.00}}$ & $100.00_{\pm{0.00}}$ 
% &$100.00_{\pm{0.00}}$   &  $100.00_{\pm{0.00}}$ & $99.99_{\pm{0.01}}$
% &$94.83_{\pm{0.11}}$   &  $94.71_{\pm{0.13}}$ & $91.80_{\pm{0.89}}$
%  &\\
% %   \FT &$28.00_{\pm{8.16}}$ (72\textcolor{red}{$\LARGE\downarrow$})& $83.02_{\pm{14.68}}$ (16\textcolor{red}{$\LARGE\downarrow$})&$99.87_{\pm{0.04}}$ (0\textcolor{red}{$\LARGE\downarrow$}) &$94.31_{\pm{0.19}}$ (0\textcolor{red}{$\LARGE\downarrow$})
% %   &$22.53_{\pm{9.46}}$  (77\textcolor{red}{$\LARGE\downarrow$}) &$75.00_{\pm{16.33}}$ (25\textcolor{red}{$\LARGE\downarrow$}) & $99.92_{\pm{0.04}}$ (0\textcolor{red}{$\LARGE\downarrow$})&$94.70_{\pm{0.08}}$ (0\textcolor{red}{$\LARGE\downarrow$})
% % &$73.64_{\pm{6.44}}$ (26\textcolor{red}{$\LARGE\downarrow$})&$96.92_{\pm{1.27}} $ (3\textcolor{red}{$\LARGE\downarrow$})& $99.87_{\pm{0.05}}$ (0\textcolor{red}{$\LARGE\downarrow$}) &$94.32_{\pm{0.12}}$ (2\textcolor{green}{$\LARGE\uparrow$})
% % &   
  
  
  
% %   \\
% %  \GA &$93.55_{\pm{0.29}}$ (6\textcolor{red}{$\LARGE\downarrow$})& $94.03_{\pm{0.27}}$ (6\textcolor{red}{$\LARGE\downarrow$})& $92.60_{\pm{0.25}}$ (8\textcolor{red}{$\LARGE\downarrow$}) & $86.64_{\pm{0.28}}$ (8\textcolor{red}{$\LARGE\downarrow$})
% %  & $93.08_{\pm{0.31}}$ (6\textcolor{red}{$\LARGE\downarrow$}) &$93.08_{\pm{0.31}}$ (6\textcolor{red}{$\LARGE\downarrow$})& $93.12_{\pm{0.23}}$ (7\textcolor{red}{$\LARGE\downarrow$}) & $87.40_{\pm{0.12}} $ (7\textcolor{red}{$\LARGE\downarrow$})
 
% %  &$98.09_{\pm{0.11}}$ (1\textcolor{red}{$\LARGE\downarrow$})& $97.74_{\pm{0.22}}$ (2\textcolor{red}{$\LARGE\downarrow$})& $87.74_{\pm{0.21}}$ (12\textcolor{red}{$\LARGE\downarrow$}) &$82.58_{\pm0.26}$ (9\textcolor{red}{$\LARGE\downarrow$})
% % &
% %  \\
% %   $\FF^\text{*}$  &$79.93_{\pm{8.92}} $ (20\textcolor{red}{$\LARGE\downarrow$})&$100.00_{\pm{0.00}}$  (0\textcolor{red}{$\LARGE\downarrow$})& $99.45_{\pm{0.24}}$ (0\textcolor{red}{$\LARGE\downarrow$})& $94.18_{\pm0.08}$ (2\textcolor{red}{$\LARGE\downarrow$})
% % &$87.66_{\pm{7.03}} $ (12\textcolor{red}{$\LARGE\downarrow$})&$100.00_{\pm{0.00}}$  (0\textcolor{red}{$\LARGE\downarrow$})& $99.55_{\pm{0.19}}$ (0\textcolor{red}{$\LARGE\downarrow$})& $94.47_{\pm0.15}$ (2\textcolor{red}{$\LARGE\downarrow$})
% % &$94.83_{\pm{4.29}} $ (5\textcolor{red}{$\LARGE\downarrow$})&$100.00_{\pm{0.00}}$  (0\textcolor{red}{$\LARGE\downarrow$})& $99.48_{\pm{0.23}}$ (0\textcolor{red}{$\LARGE\downarrow$})& $94.04_{\pm0.10}$ (2\textcolor{green}{$\LARGE\uparrow$})
  
% %   \\
% \IU 
%   &$87.82_{\pm{2.15}} $ (12\textcolor{red}{$\LARGE\downarrow$})&$98.63_{\pm{0.22}}$  (1\textcolor{red}{$\LARGE\downarrow$})& $99.47_{\pm{0.15}}$ (0\textcolor{red}{$\LARGE\downarrow$})
%  & $95.96_{\pm0.21}$ (4\textcolor{red}{$\LARGE\downarrow$})
%  &$99.82_{\pm{0.13}}$ (0\textcolor{red}{$\LARGE\downarrow$}) &$99.93_{\pm{0.04}}$ (0\textcolor{red}{$\LARGE\downarrow$})
%  &$97.98_{\pm{0.21}}$ (5\textcolor{red}{$\LARGE\downarrow$}) &$94.50_{\pm{0.19}}$ (6\textcolor{red}{$\LARGE\downarrow$})
%  &$97.24_{\pm{0.13}}$ (0\textcolor{red}{$\LARGE\downarrow$}) 
%  &$91.42_{\pm{0.21}}$ (3\textcolor{red}{$\LARGE\downarrow$})&$88.04_{\pm{0.22}}$ (6\textcolor{red}{$\LARGE\downarrow$})&$90.76_{\pm{0.18}}$ (1\textcolor{red}{$\LARGE\downarrow$}) &
%  \\
%  \FTSparse &$100.00_{\pm{0.00}}$ (0\textcolor{red}{$\LARGE\downarrow$})  &$100.00_{\pm{0.00}}$  (0\textcolor{red}{$\LARGE\downarrow$})&\cellcolor{LightCyan} ${100.00_{\pm{0.00}}}$ ($0$\textcolor{red}{$\LARGE\downarrow$})
%  &$100.00_{\pm0.00}$ (0\textcolor{red}{$\LARGE\downarrow$}) &$100.00_{\pm{0.00}}$ (0\textcolor{red}{$\LARGE\downarrow$})  &\cellcolor{LightCyan}$100.00_{\pm{0.00}}$  (0\textcolor{red}{$\LARGE\downarrow$}) 
 
%   & $91.49_{\pm{1.21}}$ (8\textcolor{red}{$\LARGE\downarrow$}) &$91.69_{\pm1.57}$ (8\textcolor{red}{$\LARGE\downarrow$})
%   &$95.74_{\pm{0.13}}$ (4\textcolor{red}{$\LARGE\downarrow$}) 
%   &$87.17_{\pm{1.31}}$ (7\textcolor{red}{$\LARGE\downarrow$}) & $87.30_{\pm{1.39}}$ (7\textcolor{red}{$\LARGE\downarrow$})&$88.97_{\pm1.00}$ (3\textcolor{red}{$\LARGE\downarrow$})
% \\
% \FTAO  & N/A &  $43.82_{\pm{11.68}}$ (56\textcolor{red}{$\LARGE\downarrow$}) & $99.80_{\pm{0.19}}$(0\textcolor{red}{$\LARGE\downarrow$})
% &N/A & $98.64_{\pm{0.71}}$ (1\textcolor{red}{$\LARGE\downarrow$}) & \cellcolor{LightCyan}$100.00_{\pm{0.00}}$(0\textcolor{red}{$\LARGE\downarrow$}) &N/A& $\cellcolor{LightCyan}99.96_{\pm{0.03}}$ (0\textcolor{red}{$\LARGE\downarrow$})&\cellcolor{LightCyan}$99.86_{\pm0.05}$ (0\textcolor{red}{$\LARGE\downarrow$})
%   &N/A  & \cellcolor{LightCyan}$94.79_{\pm{0.07}}$(0\textcolor{green}{$\LARGE\uparrow$}) &\cellcolor{LightCyan}$94.55_{\pm0.11}$(3\textcolor{green}{$\LARGE\uparrow$})
% \\
% \midrule
% \bottomrule[1pt]
% \end{tabular}
% }
% \vspace*{-3mm}

% \end{table*}


% % \paragraph{Sparsification improves MU’s accuracy and efficacy across different models and unlearning scenarios}
% % Table show the relationship between sparsity and unlearning when using different arch and unlearning settings. 

% \paragraph{A use case study: {\MU} for Trojan model cleanse.}

% \begin{figure}[htb]
% %\begin{wrapfigure}{r}{80mm}
% %\vspace*{-6mm}
% \centerline{

% \begin{tabular}{cccc}
%     \hspace*{-5mm}  \includegraphics[width=.25\textwidth,height=!]{figs/placeholder_JC/backdoor_FT_attack_acc.pdf} &
%     \hspace*{-5mm} \includegraphics[width=.25\textwidth,height=!]{figs/placeholder_JC/backdoor_FT_test_acc.pdf} \\

%     \hspace*{-5mm} \includegraphics[width=.25\textwidth,height=!]{figs/placeholder_JC/backdoor_ours_attack_acc.pdf} &
%     \hspace*{-5mm}  \includegraphics[width=.25\textwidth,height=!]{figs/placeholder_JC/backdoor_ours_test_acc.pdf} 
% \end{tabular}
% }
% \vspace*{-3mm}
% \caption{\footnotesize{
% % One figure demonstrates that our methods can decrease attack success rates and maintain good generalization performance.
% Left ASR, right SA. \JC{Just placeholders that need to be improved}
% }}
%   \label{fig: results_MU_pruning_backdoor}
% %  \vspace*{-3.8mm}
% %\end{wrapfigure}
% %\end{figure}
% \end{figure}

% %Backdoor Since To demonstrate the unlearning performance, we set up several backdoor attack experiments to
% Since MU's purpose is to remove the effect of specific data, there is a straightforward application for MU. Backdoored neural network \citep{gu2017badnets}, or \textit{BadNet}, shows that when attackers poison part of the training dataset in a given pattern, the neural network, trained at the poisoned dataset, will misbehave on the images with the same pattern. Assuming we already know which part of the dataset was poisoned, we can apply MU methods to neutralize the poisoned images' influence to defend against data poisoning attacks on models. An effective MU method will get a relatively lower attack success rate (ASR) and be more stable in forgetting the poisoned data. To demonstrate the performance of the MU methods, we performed several experiments.

% We follow the \textit{All-to-all attack} in \citep{gu2017badnets} to poison part of the training dataset. The model was trained and pruned over the same poisoned dataset in different sparsity. Then we deemed the poisoned dataset as the forgetting dataset, and applied both the `\textit{prune first, then unlearn}' methods and the `\textit{sparsity-aware unlearning}' methods. Additionally, In the `prune first, then unlearn' settings, we chose the OMP and Finetune as the backbones of pruning and unlearning. Fig \ref{fig: results_MU_pruning_backdoor} shows the results of experiments under backdoor attack settings. The curve indicates that under either unlearn method, the ASR decreased rapidly when the model became sparser, with a negligible trade-off in standard accuracy (SA).

% % TODO: add results and analysis
% % \SL{Please see Appendix\,xxx.}
% Please refer to Appendix \ref{appendix: additional results}.
% \SL{[what is this? do we need it? Also, the index is wrong!]}


%\SL{[Please add this following Bi-prune paper last paragraph in experiments.]}
% \SL{[remove the following paragraph.]}
% We include more experiment results in Appendix\,\ref{appendix: additional results and details}. In particular, we show more results of one class forgetting on various datasets in Table\,\ref{tab: overall_perfoamnce_datasets}. We also explore more model architectures shown in Table\,\ref{tab: overall_perfoamnce_arch}.  
\vspace{-0.4cm}
\section{Related Works}
\vspace{-0.3cm}
\textbf{Memorization and atypical samples.} The memorization effect of overparameterized DNNs have been extensively studied both empirically~\cite{zhang2016understanding, nakkiran2019deep} and theoretically~\cite{bartlett2002rademacher}. From traditional views, the memorization can be harmful to the model generalization, because it makes DNN models easily fit those outliers and noisy labels.  However, recent studies point out the concept of ``benign overfitting''~\cite{bartlett2020benign, feldman2020does, feldman2020neural}, which suggests the memorization effect necessary for DNNs to have extraordinary performance on modern machine learning tasks. 
Especially, the recent work~\cite{feldman2020neural} empirically figures out those atypical/rare samples in benchmark datasets and show the contribution from memorizing atypical samples to the DNN's performance. Besides the work~\cite{feldman2020neural}, there are also other strategies~\cite{carlini2019distribution} to find atypical samples in training dataset.  Notably, our work is not the first effort to study the influence of memorization on DNN's adversarial robustness. A previous study~\cite{sanyal2020benign} illustrates that memorizing the mis-labeled samples might be a reason to cause the DNNs' adversarial vulnerability. In our paper, we focus on atypical samples, which appear much more frequently in common datasets, and we study their impacts especially on adversarial training algorithms~\cite{madry2017towards, zhang2019theoretically,chatterji2020finite, muthukumar2020harmless}.\\
\textbf{Adversarial robustness. } 
Adversarial training methods~\cite{madry2017towards, zhang2019theoretically, wang2019improving, zhang2016understanding, rice2020overfitting} are considered as one of the most reliable and effective methods to protect DNN models against adversarial attacks~\cite{goodfellow2014explaining, xu2019adversarial}. However, there are several intrinsic properties of adversarial training which requires deeper understandings. 
For example, they always suffer from poor robustness generalization~\cite{ schmidt2018adversarially, rice2020overfitting}, and they always present strong trade-off relation between clean accuracy vs. robustness~\cite{tsipras2018robustness, zhang2019theoretically}. Our work aims to study these properties from the data perspective and demonstrate the significant connection of the memorization effect with these properties.

\vspace{-0.4cm}
\section{Conclusion}
\vspace{-0.2cm}
In this paper, we draw significant connections of the memorization effect of deep neural networks with the behaviors of adversarial training algorithms. Based on the findings, we devise a novel algorithm BAT to enhance the performance of adversarial training. The findings of the paper can motivate the futures studies in building robust DNNs with more attention on the data perspective.
\bibliographystyle{unsrt}
\bibliography{sample}
%\input{sections/checklist}



\appendix
\newpage
\begin{center}
% \large{\bf{Appendix for Graph Neural Networks}}
% \large{\bf{Supplementary Material}}
\large{\bf{Appendix}}
\end{center}

In the supplementary materials, we provide more details about atypical samples and the proposed algorithm, as well as the full experimental results of the preliminary study and the proposed method. 
\begin{itemize}
    \item Additional Introduction of Atypical Samples\hfill Appendix~\ref{app:atypical}
    \item Additional Results of Preliminary Study (in Section~\ref{sec:pre1})\hfill Appendix~\ref{app:pre1}
    \subitem Traditional ERM \& Adversarial Training
    \item Additional Results of Preliminary Study (in Section~\ref{sec:pre2})\hfill Appendix~\ref{app:pre2}
    \subitem Traditional ERM \& Adversarial Training
    \item Full Training Scheme of BAT\hfill Appendix~\ref{app:algorithm}
    \item Additional Results of BAT\hfill Appendix~\ref{app:exp}
    \item Boarder Impacts of this Paper \hfill Appendix~\ref{app:board}
\end{itemize}

\section{Additional Introduction of Atypical Samples}\label{app:atypical}

In this section, we provide additional introductions about the atypical samples in common datasets.
In Fig.~\ref{fig:show_mem}, we provide several examples of images from CIFAR10, CIFAR100~\cite{krizhevsky2009learning} and Tiny ImageNet~\cite{le2015tiny} respectively, with different memorization value (as defined in Section~\ref{sec:def_atypical}) around $0.0, 0.5, 1.0$. These examples suggest that if the memorization value of an image is large, this image is very likely to be ``atypical'', as it presents very distinct semantic features with the images in the main distribution (with memorization value 0.0). The detailed introduction about how to estimate the memorization value in practice can be found in the  work~\cite{feldman2020neural}.
\begin{figure}[h]
    \centering
    \includegraphics[width = 0.9\linewidth]{figures/atypical_examples.pdf}
    \caption{Examples of Images with Different Memorization Values}
    \label{fig:show_mem}
\end{figure}

\newpage
In Fig.~\ref{fig:hist}, we provide histograms to show the distribution of the estimated memorization values of all training samples from CIFAR10, CIFAR100 and Tiny~ImageNet. From Fig.~\ref{fig:hist}, we can observe that atypical samples (with high memorization value > 0.15) consist of a significant fraction (over 40\% \& 50\% respectively) in CIFAR100 and Tiny~ImageNet. In CIFAR10, they also consist of a non-ignorable fraction which is over 10\%.
\begin{figure}[h]
\centering
\subfloat[CIFAR100.]{
\begin{minipage}[c]{0.3\textwidth}
\includegraphics[width = 1.0\textwidth]{figures/hist_cifar100.png}
\end{minipage}
}
\hspace*{0.0cm}
\subfloat[CIFAR10.]{
\begin{minipage}[c]{0.3\textwidth}
\includegraphics[width = 1.0\textwidth]{figures/hist_cifar10.png}%
\end{minipage}
}
\hspace*{0.0cm}
\subfloat[Tiny~ImageNet.]{
\begin{minipage}[c]{0.3\textwidth}
\includegraphics[width = 1.0\textwidth]{figures/hist_imagenet.png}%
\end{minipage}
}
\caption{Frequencies of Training samples with Different \textbf{Memorization Values} in Various Datasets}
\label{fig:hist}
\end{figure}


In Fig.~\ref{fig:show_pair}, we provide several pairs of images with high influence value (as defined in Section~\ref{sec:def_atypical}) which is over 0.15. In each pair, the training sample also has a high memorization value over 0.15. These examples suggest that there exist atypical samples in both training \& test sets of CIFAR10, CIFAR100 and Tiny~ImageNet. A pair of atypical samples (in the training set and test set) with a high influence value are visually very similar. Moreover, since they have high influence values, removing the atypical samples in the training set is very likely to cause the model to fail on the test atypical samples. Therefore, without memorizing the atypical sample in the training set, the model can hardly predict the atypical samples in the test set.
\begin{figure}[h]
    \centering
    \includegraphics[width = 0.75\linewidth]{figures/pair.pdf}
    \caption{High Influence Pairs with Influence Value > 0.15}
    \label{fig:show_pair}
\end{figure}






\section{Additional Results for Preliminary Study}\label{app:pre}

In this section, we provide the full results of the preliminary study in Section~\ref{sec:pre} 
on CIFAR10, CIFAR100 and Tiny~ImageNet, to illustrate the distinct behaviors of the memorization effect between traditional ERMs and adversarial training. In both ERM and adversarial training, we train the models under ResNet18 and WideResNet28-10 (WRN28) architectures. In the experiments, we train the models for 200 epochs with learning rate 0.1, momentum 0.9, weight decay 5e-4, and decay the learning rate by 0.1 at the epoch 150 and 200. For adversarial training, we conduct experiments using PGD adversarial training~\cite{madry2017towards} by default to defense against $l_\infty$-8/255 adversarial attack, with the exception on Tiny~ImageNet, which is against $l_\infty$-4/255 attack. For robustness evaluation, we conduct a 20-step PGD attack.


\subsection{Additional Results for Preliminary Study - Section~\ref{sec:pre1}}\label{app:pre1}

In this subsection, we provide more experimental results to validate the statement 
in Section~\ref{sec:pre1}, where we state that fitting atypical samples in adversarial training can only improve the clean accuracy of test atypical samples, but hardly help their adversarial robustness. We provide full empirical results to show that: \textbf{(i)} In traditional ERM, fitting atypical samples improves the clean accuracy of test atypical samples. \textbf{(ii)} In adversarial training, fitting (adversarial) atypical samples improves the clean accuracy of test atypical samples but can hardly improve the adversarial robustness of them. The experimental setting follows Section~\ref{sec:pre1}, where we apply traditional ERM and adversarial training on original CIFAR10, CIFAR100, Tiny~ImageNet datasets. We evaluate the model's clean accuracy and adversarial accuracy on training atypical set $\mathcal{D}_\text{atyp}=\{x_i \in \mathcal{D}: \text{mem}(x_i)> 0.15\}$ and its corresponding test atypical set $\mathcal{D}_\text{atyp}' = \{x'_j \in \mathcal{D}': \text{infl}(x_i,x'_j)> 0.15, \text{for } \forall x_i\in \mathcal{D}_\text{atyp}\}$.


\textbf{(i) Additional Results for Preliminary Study - Section~\ref{sec:pre1} In Traditional ERM}

Fig.~\ref{fig:app_1_11}, Fig.~\ref{fig:app_1_12} and Fig.~\ref{fig:app_1_13} report the performance (clean accuracy) of traditional ERM, which is evaluated on atypical sets under ResNet18 (left) and WRN28 (right) on CIFAR100, CIFAR10 and Tiny~ImageNet. From the figures, we can obverse that fitting atypical samples during training can effectively help the models to achieve good clean accuracy on test atypical samples in all datasets. Note that here we only report clean accuracy as they are not robust against adversarial attacks.

\begin{figure}[h]
\centering
\hspace*{-1cm}
\subfloat[ResNet18.]{
\begin{minipage}[c]{0.3\textwidth}
\includegraphics[width = 1.0\textwidth]{figures/pre1_clean_cifar100_ResNet18.png}
\end{minipage}
}
\hspace*{0.4cm}
\subfloat[ WRN28.]{
\begin{minipage}[c]{0.3\textwidth}
\includegraphics[width = 1.0\textwidth]{figures/pre1_clean_cifar100_WRN28.png}%
\end{minipage}
}
\caption{Clean Accuracy on \textbf{Atypical} Set of CIFAR100}
\label{fig:app_1_11}
\vspace{-0.5cm}
\end{figure}

\begin{figure}[h]
\centering
\hspace*{-1cm}
\subfloat[ ResNet18.]{
\begin{minipage}[c]{0.3\textwidth}
\includegraphics[width = 1.0\textwidth]{figures/pre1_clean_cifar10_ResNet18.png}
\end{minipage}
}
\hspace*{0.4cm}
\subfloat[WRN28.]{
\begin{minipage}[c]{0.3\textwidth}
\includegraphics[width = 1.0\textwidth]{figures/pre1_clean_cifar10_WRN28.png}%
\end{minipage}
}
\caption{Clean Accuracy on \textbf{Atypical} Set of CIFAR10}
\label{fig:app_1_12}
\vspace{-0.5cm}
\end{figure}

\begin{figure}[h!]
\centering
\hspace*{-1cm}
\subfloat[ResNet32.]{
\begin{minipage}[c]{0.3\textwidth}
\includegraphics[width = 1.0\textwidth]{figures/pre1_clean_imagenet_ResNet18.png}
\end{minipage}
}
\hspace*{0.4cm}
\subfloat[WRN28.]{
\begin{minipage}[c]{0.3\textwidth}
\includegraphics[width = 1.0\textwidth]{figures/pre1_clean_imagenet_WRN28.png}%
\end{minipage}
}
\caption{Clean Accuracy on \textbf{Atypical} Set of Tiny~ImageNet}
\label{fig:app_1_13}
\end{figure}

%%%%%%%%%%%%%%%%%%%%%%%%%%%%%%%%%%%%%%%%%%%%%%%%%%%%%%%%%%%%%%%%%%%%%%%%%%%%%%%%%%%%%%%%%%%%%%%%%%%%%%%%%%%%%%%%%%%%%%%%%%%%%%%%%%%%%%%%%%%%%%%%%%%%%%%%%%%%%%%%%%%%%%%%%%%%%%%%%%%%%%%%%%%%%%%%%%%%%%%%%%%%%%%%%%%%%%%%%%%%%%%%%%%%%%%%%%%%%%%%%%%%%%%%%%%%%%%%%%%%%%%%5

\newpage
\textbf{(ii) Additional Results for Preliminary Study - Section~\ref{sec:pre1} In Adversarial Training}

Fig.~\ref{fig:app_1_21}, Fig.~\ref{fig:app_1_22} and Fig.~\ref{fig:app_1_23} report the performance of adversarially trained models. We evaluate the clean accuracy and adversarial accuracy on the training atypical set $\mathcal{D}_\text{atyp}$ and test atypical set $\mathcal{D}'_\text{atyp}$. From the results, we can observe that although fitting atypical samples can help the model to have modest clean accuracy on test atypical samples, the adversarial robustness of them is constantly low and can hardly be improved during the whole training process.


\begin{figure}[h]
\centering
\hspace*{-1cm}
\subfloat[Clean (left) \& Adv Acc. (right) under ResNet18.]{
\begin{minipage}[h]{0.55\textwidth}
\includegraphics[width = 0.5\textwidth]{figures/clean_rare_cifar.jpg}%
\hfill
\includegraphics[width = 0.5\textwidth]{figures/adv_rare_cifar100.png}
\end{minipage}
}
\hspace*{-0.4cm}
\subfloat[Clean (left) \& Adv Acc. (right) under WRN28.]{
\begin{minipage}[c]{0.55\textwidth}
\includegraphics[width = 0.5\textwidth]{figures/wrn_clean_rare_cifar100.png}%
\hfill
\includegraphics[width = 0.5\textwidth]{figures/wrn_adv_rare_cifar100.png}
\end{minipage}
}
\caption{Clean Accuracy and Adversarial Accuracy on \textbf{Atypical} Set of CIFAR100}
\label{fig:app_1_21}
\vspace{-0.5cm}
\end{figure}

\begin{figure}[h]
\centering
\hspace*{-1cm}
\subfloat[Clean (left) \& Adv Acc. (right) under ResNet18.]{
\begin{minipage}[h]{0.55\textwidth}
\includegraphics[width = 0.5\textwidth]{figures/pre1_cifar10_adv1_resnet18.png}%
\hfill
\includegraphics[width = 0.5\textwidth]{figures/pre1_cifar10_adv2_resnet18.png}
\end{minipage}
}
\hspace*{-0.4cm}
\subfloat[Clean (left) \& Adv Acc. (right) under WRN28.]{
\begin{minipage}[c]{0.55\textwidth}
\includegraphics[width = 0.5\textwidth]{figures/pre1_cifar10_adv1_wrn28.png}%
\hfill
\includegraphics[width = 0.5\textwidth]{figures/pre1_cifar10_adv2_wrn28.png}
\end{minipage}
}
\caption{Clean Accuracy and Adversarial Accuracy on \textbf{Atypical} Set of CIFAR10}
\label{fig:app_1_22}
\vspace{-0.5cm}
\end{figure}

\begin{figure}[h!]
\centering
\hspace*{-1cm}
\subfloat[Clean (left) \& Adv Acc. (right) under ResNet32.]{
\begin{minipage}[h]{0.55\textwidth}
\includegraphics[width = 0.5\textwidth]{figures/pre1_imagenet_adv1_resnet18.png}%
\hfill
\includegraphics[width = 0.5\textwidth]{figures/pre1_imagenet_adv2_resnet18.png}
\end{minipage}
}
\hspace*{-0.4cm}
\subfloat[Clean (left) \& Adv Acc. (right) under WRN28.]{
\begin{minipage}[c]{0.55\textwidth}
\includegraphics[width = 0.5\textwidth]{figures/pre1_imagenet_adv1_wrn28.png}%
\hfill
\includegraphics[width = 0.5\textwidth]{figures/pre1_imagenet_adv2_wrn28.png}
\end{minipage}
}
\caption{Clean Accuracy and Adversarial Accuracy on \textbf{Atypical} Set of Tiny~ImageNet}
\label{fig:app_1_23}
\end{figure}



%%%%%%%%%%%%%%%%%%%%%%%%%%%%%%%%%%%%%%%%%%%%%%%%%%%%%%%%%%%%%%%%%%%%%%%%%%%%%%%%%%%%%%%%%%%%%%%%%%%%%%%%%%%%%
\subsection{Additional Results for Preliminary Study - Section~\ref{sec:pre2}}\label{app:pre2}

In this subsection, we provide more experimental results to validate the statement 
in Section~\ref{sec:pre2}, where we state that fitting atypical samples in adversarial training can hurt the performance (clean \& adversarial accuracy) of typical samples.
We provide full empirical results to show that: \textbf{(i)} In traditional ERM, fitting atypical samples will not hurt the models' performance (clean accuracy) of typical samples. \textbf{(ii)} In adversarial training, fitting atypical samples can degrade the clean \& adversarial accuracy of typical samples. \textbf{(iii)} In adversarial training, fitting atypical samples can degrade the quality of learned representations, especially the models' discrimination between classes.
The experimental setting follows Section~\ref{sec:pre2}, where we train the models for several trails on resampled (CIFAR100, CIFAR10, Tiny~ImageNet) datasets: each dataset is constructed with the whole training typical set $\mathcal{D}_\text{typ}$, and a part of the training atypical set $\mathcal{D}_\text{atyp}$ (randomly sample 0\%, 20\% and 100\% in $\mathcal{D}_\text{atyp}$). We evaluate the models' performance on test ``typical'' set $\mathcal{D}'_\text{typ}$ which is defined in Section~\ref{sec:pre2}.

\newpage
\textbf{(i) Additional Results for Preliminary Study - Section~\ref{sec:pre2} In Traditional ERM}

Fig.~\ref{fig:app2_11}, Fig.~\ref{fig:app2_12} and Fig.~\ref{fig:app2_13} report the performance of traditional ERM, trained on different resampled datasets with different amount of atypical samples existed. The figures report the clean accuracy on test typical set of CIFAR100, CIFAR10 and Tiny~ImageNet. We also leave the robustness performance out here as the models are not robust to adversarial attacks. From the results, we can conclude that in traditional ERM, fitting atypical samples will not degrade the models' performance on typical samples. For example, in CIFAR100 dataset, with 100\% atypical samples included (Atypical-100\%)., the accuracy on the test typical set is even slightly higher than the model trained without atypical samples (Atypical-0\%). This conclusion is consistent for all three datasets and model architectures.


\begin{figure}[h]
\centering
\hspace*{-1cm}
\subfloat[ResNet18.]{
\begin{minipage}[c]{0.3\textwidth}
\includegraphics[width = 1.0\textwidth]{figures/pre2_clean_cifar100_ResNet18.png}
\end{minipage}
}
\hspace*{0.4cm}
\subfloat[WRN28.]{
\begin{minipage}[c]{0.3\textwidth}
\includegraphics[width = 1.0\textwidth]{figures/pre2_clean_cifar100_WRN28.png}%
\end{minipage}
}
\caption{Clean Accuracy on \textbf{Typical} Set of CIFAR100}
\label{fig:app2_11}
\vspace{-0.5cm}
\end{figure}
%%%%%%%%%%%%%%%%%%%%%%%%%%%%%%%%%%%
\begin{figure}[h]
\centering
\hspace*{-1cm}
\subfloat[ResNet18.]{
\begin{minipage}[c]{0.3\textwidth}
\includegraphics[width = 1.0\textwidth]{figures/pre2_clean_cifar10_ResNet18.png}
\end{minipage}
}
\hspace*{0.4cm}
\subfloat[WRN28.]{
\begin{minipage}[c]{0.3\textwidth}
\includegraphics[width = 1.0\textwidth]{figures/pre2_clean_cifar10_WRN28.png}%
\end{minipage}
}
\caption{Clean Accuracy on \textbf{Typical} Set of CIFAR10}
\label{fig:app2_12}
\vspace{-0.5cm}
\end{figure}
%%%%%%%%%%%%%%%%%%%%%%%%%%%%%%%%%%%
\begin{figure}[h!]
\centering
\hspace*{-1cm}
\subfloat[ResNet32.]{
\begin{minipage}[c]{0.3\textwidth}
\includegraphics[width = 1.0\textwidth]{figures/pre2_clean_imagenet_ResNet18.png}
\end{minipage}
}
\hspace*{0.4cm}
\subfloat[WRN28.]{
\begin{minipage}[c]{0.3\textwidth}
\includegraphics[width = 1.0\textwidth]{figures/pre2_clean_imagenet_WRN28.png}%
\end{minipage}
}
\caption{Clean Accuracy on \textbf{Typical} Set of Tiny~ImageNet}
\label{fig:app2_13}
\end{figure}





%%%%%%%%%%%%%%%%%%%%%%%%%%%%%%%%%%%%%%%%%%%%%%%%%%%%%%%%%%%%%%%%%%%%%%%%%%%%%%%%%%%%%%%%%%%%%%%%%%%%%%%%%%%%%%%%%%%%%%%%%%%%%%%%%%%%%%%%%%%%%%%%%%%%%%%%%%%%%%%%%%%%%%%%%%%%%%%%%%%%%%%%%%
\textbf{(ii) Additional Results for Preliminary Study - Section~\ref{sec:pre2} In Adversarial Training}

Fig.~\ref{fig:pre2_21}, Fig.~\ref{fig:pre2_22} and Fig.~\ref{fig:pre2_23} report the performance of adversarial training, on different resampled datasets with different amount of atypical samples existed. The figures report both clean and adversarial accuracy on test atypical sets of CIFAR100, CIFAR10 and Tiny~ImageNet. Based on the experimental results, we find that including more atypical samples can cause the model have worse performance on typical samples in all three datasets. In datasets with a large portion of atypical samples, such as CIFAR100, the negative effects of atypical samples are more obvious. In CIFAR100, training on datasets with 100\% atypical samples (Atypical 100\%) can cause the clean \& adversarial accuracy drop by $\sim7\%$ and $8\%$, respectively.

\begin{figure}[h]
\centering
\hspace*{-1cm}
\subfloat[Clean (left) \& Adv Acc. (right) under ResNet18.]{
\begin{minipage}[h]{0.55\textwidth}
\includegraphics[width = 0.5\textwidth]{figures/poison_clean_ResNet18.png}%
\hfill
\includegraphics[width = 0.5\textwidth]{figures/poison_adv_ResNet18.png}
\end{minipage}
}
\hspace*{-0.4cm}
\subfloat[Clean (left) \& Adv Acc. (right) under WRN28.]{
\begin{minipage}[c]{0.55\textwidth}
\includegraphics[width = 0.5\textwidth]{figures/poison_clean_WRN.png}%
\hfill
\includegraphics[width = 0.5\textwidth]{figures/poison_adv_WRN.png}
\end{minipage}
}
\caption{Clean Accuracy and Adversarial Accuracy on \textbf{Typical} Set of CIFAR100}
\label{fig:pre2_21}
\vspace{-0.3cm}
\end{figure}
%%%%%%%%%%%%%%%%%%%%%%%%%%%%%%%%%%%%%%%%%%
\begin{figure}[h]
\centering
\hspace*{-1cm}
\subfloat[Clean (left) \& Adv Acc. (right) under ResNet18.]{
\begin{minipage}[h]{0.55\textwidth}
\includegraphics[width = 0.5\textwidth]{figures/pre3_cifar10_adv1_resnet18.png}%
\hfill
\includegraphics[width = 0.5\textwidth]{figures/pre3_cifar10_adv2_resnet18.png}
\end{minipage}
}
\hspace*{-0.4cm}
\subfloat[Clean (left) \& Adv Acc. (right) under WRN28.]{
\begin{minipage}[c]{0.55\textwidth}
\includegraphics[width = 0.5\textwidth]{figures/pre3_cifar10_adv1_wrn28.png}%
\hfill
\includegraphics[width = 0.5\textwidth]{figures/pre3_cifar10_adv2_wrn28.png}
\end{minipage}
}
\caption{Clean Accuracy and Adversarial Accuracy on \textbf{Typical} Set of CIFAR10}
\label{fig:pre2_22}
\vspace{-0.3cm}
\end{figure}
%%%%%%%%%%%%%%%%%%%%%%%%%%%%%%%%%%%%%%%%%%
\begin{figure}[h!]
\centering
\hspace*{-1cm}
\subfloat[Clean (left) \& Adv Acc. (right) under ResNet32.]{
\begin{minipage}[h]{0.55\textwidth}
\includegraphics[width = 0.5\textwidth]{figures/pre3_imagenet_adv1_resnet18.png}%
\hfill
\includegraphics[width = 0.5\textwidth]{figures/pre3_imagenet_adv2_resnet18.png}
\end{minipage}
}
\hspace*{-0.4cm}
\subfloat[Clean (left) \& Adv Acc. (right) under WRN28.]{
\begin{minipage}[c]{0.55\textwidth}
\includegraphics[width = 0.5\textwidth]{figures/pre3_imagenet_adv1_resnet18.png}%
\hfill
\includegraphics[width = 0.5\textwidth]{figures/pre3_imagenet_adv2_wrn28.png}
\end{minipage}
}
\caption{Clean Accuracy and Adversarial Accuracy on \textbf{Typical} Set of CIFAR100}
\label{fig:pre2_23}
\end{figure}


\vspace{2cm}
\textbf{(iii) Additional Results for Preliminary Study - Section~\ref{sec:pre2} Classwise Representation Distance}

Here, we provide additional evidence to show that in adversarial training, fitting atypical samples can degrade the quality of DNN's learned representations (as proposed in Section~\ref{sec:pre2}). In Fig.~\ref{fig:dist1}, Fig.~\ref{fig:dist2} and Fig.~\ref{fig:dist3}, we measure the Cosine Distance (defined in Section~\ref{sec:pre2}) of the representations for (typical) samples from different classes. In these figures, we provide detailed results of both traditional ERM and adversarial training under ResNet18 and WRN28. From the results, we can see that in adversarial training, more atypical samples can cause the smaller classwise Cosine Distance of the models (on their last epochs). Moreover, during the training process of adversarial training, the classwise Cosine Distance first increases and then starts to decrease, especially when there are more atypical samples are fitted. For example, in CIFAR100 under ResNet18, the Cosine Distance starts to decrease at around Epoch 100, which is when many more atypical samples are fitted. These results suggest that, in adversarial training, fitting atypical samples can be an important reason to cause the models to produce poor representations. As a comparison, in Traditional ERM, the classwise Cosine Distance keeps increasing from the first epoch to the last one. Although given the same epoch, the models trained with more atypical samples also have smaller Cosine Distance, they would not harm the model's final performance. It is because the models trained with more atypical samples can also have relatively large classwise Cosine Distance in their last epochs.


%%%%%%%%%%%%%%%%%%%%%%%%%%%%
\begin{figure}[h]
\centering
\hspace*{-1cm}
\subfloat[Traditional ERM under ResNet18 (left), WRN28 (right).]{
\begin{minipage}[h]{0.55\textwidth}
\includegraphics[width = 0.5\textwidth]{figures/dist_cifar100_ResNet18.png}%
\hfill
\includegraphics[width = 0.5\textwidth]{figures/dist_cifar100_wrn28.png}
\end{minipage}
}
\hspace*{-0.4cm}
\subfloat[Adv. Training under ResNet18 (left), WRN28 (right).]{
\begin{minipage}[c]{0.55\textwidth}
\includegraphics[width = 0.5\textwidth]{figures/dist_adv_cifar100_ResNet18.png}%
\hfill
\includegraphics[width = 0.5\textwidth]{figures/dist_adv_cifar100_WRN28.png}
\end{minipage}
}
\caption{Classwise Cosine Distance of Output Representation of \textbf{Typical} Set of CIFAR100}
\label{fig:dist1}
\end{figure}

%%%%%%%%%%%%%%%%%%%%%%%%%%%%
\begin{figure}[h]
\centering
\hspace*{-1cm}
\subfloat[Traditional ERM under ResNet18 (left), WRN28 (right).]{
\begin{minipage}[h]{0.55\textwidth}
\includegraphics[width = 0.5\textwidth]{figures/dist_cifar10_ResNet18.png}%
\hfill
\includegraphics[width = 0.5\textwidth]{figures/dist_cifar10_WRN28.png}
\end{minipage}
}
\hspace*{-0.4cm}
\subfloat[Adv. Training under ResNet18 (left), WRN28 (right).]{
\begin{minipage}[c]{0.55\textwidth}
\includegraphics[width = 0.5\textwidth]{figures/dist_adv_cifar10_ResNet18.png}%
\hfill
\includegraphics[width = 0.5\textwidth]{figures/dist_adv_cifar10_WRN28.png}
\end{minipage}
}
\caption{Classwise Cosine Distance of Output Representation of \textbf{Typical} Set of CIFAR10}
\label{fig:dist2}
\end{figure}


%%%%%%%%%%%%%%%%%%%%%%%%%%%%
\begin{figure}[h!]
\centering
\hspace*{-1cm}
\subfloat[Traditional ERM under ResNet32 (left), WRN28 (right).]{
\begin{minipage}[h]{0.55\textwidth}
\includegraphics[width = 0.5\textwidth]{figures/dist_imagenet_ResNet18.png}%
\hfill
\includegraphics[width = 0.5\textwidth]{figures/dist_imagenet_WRN28.png}
\end{minipage}
}
\hspace*{-0.4cm}
\subfloat[Adv. Training under ResNet32 (left), WRN28 (right).]{
\begin{minipage}[c]{0.55\textwidth}
\includegraphics[width = 0.5\textwidth]{figures/dist_adv_imagenet_ResNet18.png}%
\hfill
\includegraphics[width = 0.5\textwidth]{figures/dist_adv_imagenet_WRN28.png}
\end{minipage}
}
\caption{Classwise Cosine Distance of Output Representation of \textbf{Typical} Set of Tiny~ImageNet}
\label{fig:dist3}
\end{figure}




\section{The Detailed Training Scheme of BAT}\label{app:algorithm}

In this section, we provide the detailed training scheme of BAT in Algorithm~\ref{alg:bat}. In particular, BAT algorithm starts from a randomly initialized neural network. On each mini-batch, it applies PGD attack to generate (training) adversarial examples (Step 5). Following the Eq~\ref{eq:poisoning_score} and Eq~\ref{eq:reweight}, BAT calculates which samples are likely to be \textit{Poisoning Atypical Samples} and their corresponding weight values (Step 6). Under the current mini-batch, next, BAT calculates the \textit{Discrimination Loss} of the typical samples $\mathcal{D}_\text{typ}$ (Step 8). Finally, BAT uses SGD to update the model parameter to minimize the reweighted adversarial loss regularized by $\beta$ times discrimination loss (Step 8).

\begin{algorithm}[h!]
\begin{algorithmic}[1]
\setstretch{1}
\STATE \textbf{Input:} Training dataset $\mathcal{D}$, with typical set $\mathcal{D}_\text{typ} = \{x \in\mathcal{D}; \text{mem}(x) \leq \sigma\}$, atypical set $\mathcal{D}_\text{atyp} = \{x \in\mathcal{D}; \text{mem}(x) > \sigma\}$. Targeted type of adversarial attack: $l_\infty$-$\epsilon$ attack.
Hyperparameters $\alpha, \beta \in \R^+$. \\
\STATE Randomly initialize the network $F$ \\
\REPEAT
\STATE Fetch mini-batch data $\{(x_i,y_i)\}$ at current epoch\\
\STATE Using PGD to generate adversarial training sample  $\{(x_i^\text{adv},y_i)\}$\\
\STATE Calculate Poisoning Score $\textbf{q}(x^\text{adv}_i)$ and weight $w_i$ as in Equation~\ref{eq:poisoning_score} and Equation~\ref{eq:reweight}.
\STATE Calculate Discrimination Loss $\mathcal{L}_{DL}(F)$ within the current mini-batch, as in Equation~\ref{eq:dl}.
\STATE Update $F$ by SGD on the objective: $\mathcal{L}_\text{BAT} = \frac{1}{\sum_i w_i }\sum_i \left[w_i \cdot \mathcal{L}(F(x_i^\text{adv}), y_i)\right] + \beta\cdot\mathcal{L}_{DL}(F)$.
\UNTIL{End of Training}
\caption{The Benign Adversarial Training (BAT) Algorithm}
\label{alg:bat}
\end{algorithmic}
\end{algorithm}



\section{Additional Results for Experiment}\label{app:exp}

In this section, we provide additional experimental results to validate the effectiveness of BAT method. In Table~\ref{tab:resnet18_cifar100} and Table~\ref{tab:wrn28_cifar100}, we provide the results of BAT and baseline models on CIFAR100 dataset under ResNet18 and WRN28 architectures. In Table~\ref{tab:resnet18_imagenet} and Table~\ref{tab:wrn28_imagenet}, we provide the results of BAT and baseline models on Tiny~ImageNet dataset under ResNet32 and WRN28 architectures. In the experiments, we train the models for 160 epochs with learning rate 0.1, momentum 0.9, weight decay 5e-4, and decay the learning rate by 0.1 at the epoch 80 and 120. To have a more comprehensive and reliable adversarial robustness, in addition to PGD adversarial attack~\cite{madry2017towards}, we also measure the model's adversarial accuracy via other attack algorithms, including FGSM attack~\cite{goodfellow2014explaining}, CW attack~\cite{carlini2017towards} and Auto Attack~\cite{croce2020reliable}. The results show that the BAT method can consistently outperform baseline models, as BAT has better clean accuracy vs. adversarial accuracy trade-off. Under ResNet18, the BAT method achieves comparable adversarial accuracy to the best baseline methods, and BATs have the highest clean accuracy. Under WRN28, the BAT method has both higher clean \& adversarial accuracy than the baseline methods.



\begin{table}[h]
\small
\centering
\caption{Performance of BAT vs. Baselines on CIFAR100 Under ResNet18}
\label{tab:resnet18_cifar100}
\begin{tabular}{c|c|ccccc}
\hline
Method & Clean Acc. & FGSM & PGD & CW & AA. \\
\hline
\hline
PGD Train (Best Adv.) & 56.9 & 36.0 & 27.4 & 25.4 & 23.6 \\
PGD Train (Best Clean) & 57.8 & 33.5 & 21.9 & 22.5 & 20.2 \\
TRADES ($1/\lambda = 5$) & 56.6 & 36.5 & 26.9 & 25.3 & 23.9 \\
MART~\cite{wang2019improving} & 51.8 & 36.1 & \textbf{30.4} & 25.8 & \textbf{24.4} \\
GAIRAT~\cite{zhang2020geometry} & 58.2 &36.5 & 27.8 & 25.9 & 23.8\\
\hline
BAT ($\alpha = 1, \beta = 0.2$) & \textbf{59.5} & \textbf{37.3} & 27.3 & \textbf{26.6} & 24.3\\
BAT ($\alpha = 2, \beta = 0.2$) &59.3 & 37.1 & 27.4 & 26.5 & 24.0\\
\hline
\hline
\end{tabular}
\end{table}

\begin{table}[h]
\small
\centering
\caption{Performance of BAT vs. Baselines on CIFAR100 Under WRN28}
\label{tab:wrn28_cifar100}
\begin{tabular}{c|c|ccccc}
\hline
Method & Clean Acc. & FGSM & PGD & CW & AA. \\
\hline
\hline
PGD Train (Best Adv.) & 59.7 & 34.9 & 24.7 & 24.2 & 22.5 \\
PGD Train (Best Clean.) & 59.7 & 34.9 & 24.7 & 24.2 & 22.5 \\
%TRADES ($1/\lambda = 1$) & \textbf{61.3} & 21.9 & \textbf{93.0} & 50.0 & 44.7 & 7.9\\
TRADES ($1/\lambda = 5$) & 57.3 & 34.5 & 24.9 & 24.6 & 22.9 \\
MART~\cite{wang2019improving} & 56.5 & 36.1 & 26.8 & 25.3 & 23.8 \\
GAIRAT~\cite{zhang2020geometry} & 60.2 & 34.8 & 24.4 & 24.8 & 22.9 \\
\hline
BAT ($\alpha = 1, \beta = 0.2$) & \textbf{62.0} & 38.6 & \textbf{28.5} & \textbf{26.5} & \textbf{24.8} \\
BAT ($\alpha = 2, \beta = 0.2$) & 61.4 & \textbf{38.9} & 28.2 & 26.3 & \textbf{24.8} \\
\hline
\hline
\end{tabular}
\end{table}

\begin{table}[h]
\small
\centering
\caption{Performance of BAT vs. Baselines on Tiny~ImageNet Under ResNet32}
\label{tab:resnet18_imagenet}
\begin{tabular}{c|c|ccccc}
\hline
Method & Clean Acc. & FGSM & PGD & CW & AA. \\
\hline
\hline
Adv. Train (Best Adv.) & 56.3 & 37.5 & 32.3 & 29.8 & 29.8\\
Adv. Train (Best Clean) & 58.2 & 36.8 & 30.5 & 28.8 & 28.4\\
TRADES ($1/\lambda = 5$) & 55.4 & 35.2 & 28.8 & 27.0 & 27.0\\
MART~\cite{wang2019improving} & 56.2 & 38.1& \textbf{34.5} & \textbf{31.8} & 32.0\\
GAIRAT~\cite{zhang2020geometry} & 58.4 & 37.3& 30.4 & 28.9 & 29.0\\
\hline
BAT ($\alpha = 1, \beta = 0.2$) & \textbf{59.4} &40.4 &32.0 & 31.5 & 32.0\\
BAT ($\alpha = 2, \beta = 0.2$) & \textbf{59.4} & \textbf{41.3}  &32.9 & \textbf{31.8} & \textbf{32.4}\\
\hline
\hline
\end{tabular}
\end{table}
\begin{table}[h!]
\small
\centering
\caption{Performance of BAT vs. Baselines on Tiny~ImageNet Under WRN28}
\label{tab:wrn28_imagenet}
\begin{tabular}{c|c|ccccc}
\hline
Method & Clean Acc. & FGSM & PGD & CW & AA. \\
\hline
\hline
Adv. Train (Best Adv.) & 58.9 & 35.7 & 31.7 & 30.0 & 30.0  \\
Adv. Train (Best Clean) & 60.0 & 35.1 & 30.1 & 28.8 & 28.5\\
TRADES ($1/\lambda = 5$) & 59.7 & 37.4 & 32.0 & 31.8 & 32.0\\
MART~\cite{wang2019improving} & 58.2 & 41.0 & 35.6 & 33.9 & 34.1 \\
GAIRAT~\cite{zhang2020geometry} & 59.9 & 38.3 & 32.4 & 31.0 & 31.1\\
\hline
BAT ($\alpha = 1, \beta = 0.2$) & 62.3 & 41.6 & 35.8 & 33.7 & 34.3 \\
BAT ($\alpha = 2, \beta = 0.2$) & \textbf{62.4} & \textbf{43.1} &\textbf{37.4} & \textbf{35.3} & \textbf{35.6}\\
\hline
\hline
\end{tabular}
\end{table}



\section{Boarder Impact}\label{app:board}

Nowadays, deep neural networks (DNNs) have been widely applied to solve various machine learning tasks, especially on many safety-critical tasks such as autonomous vehicle~\cite{fagnant2015preparing}, AI healthcare~\cite{jiang2017artificial} and ID authentication~\cite{mohammed2011human}, etc. However, the existence of adversarial attacks~\cite{xu2019adversarial} brings huge threats to the safety of these DNNs' applications. As one of the most successful approaches to defend DNNs against adversarial attacks, adversarial training methods~\cite{madry2017towards, zhang2016understanding} still suffer from several disadvantages. For example, for a DNN model to achieve good adversarial robustness, it usually sacrifices its clean accuracy. Adversarially trained DNNs also present strong overfitting effects. In our study, we draw important findings to explain these drawbacks of adversarial training from the data perspective and empirically validate these issues' relation to atypical samples in the data distribution. Moreover, the currently existed adversarially trained models can only achieve satisfactory performance on simple datasets such as MNIST and CIFAR10. Motivated from our findings, we propose a new method to improve the performance of adversarial training, especially on more complex datasets. We anticipate that our findings and method are helpful for further studies to improve the effectiveness and feasibility of adversarial training and eventually build safer DNNs.


\end{document}
